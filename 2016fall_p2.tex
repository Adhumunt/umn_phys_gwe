%%%%%%%%%%%%%%%%%%%%%%%%%%%%%%%%%%%%%%%%%%%%%%%%%%%%%%%%%%%%%%%%%%%%%%%%%%%%%%%
%%%% Problem 1
%%%%%%%%%%%%%%%%%%%%%%%%%%%%%%%%%%%%%%%%%%%%%%%%%%%%%%%%%%%%%%%%%%%%%%%%%%%%%%%
%\subsection{Problem 1}
\problem{1}
\subsubsection{Question}
% Keywords
	\index{unsolved!Fall 2016 II.P1}
A simple pendulum of length $L$ and mass $M$ is suspended from the middle of a horizontal, rigid rod in such a way that the pendulum can swing only in the plane perpendicular to the plane of the support frame, as shown in the figure. The pendulum is a massless, rigid rod with the mass $M$ at the end. The horizontal rod is rotated counterclockwise (when viewed from above) at rotation rate $\boldsymbol{\Omega}$. Let $\theta$ be the angle between the pendulum and the vertical.
\begin{enumerate}
	\item Find the angles of equilibrium (whether stable or unstable). Under what conditions are the equilibrium angles stable points?
	\item Using the reference frame that rotates with the support (i.e., in which the support frame is stationary), find the oscillation frequency for small amplitude oscillations about $\theta=0$. For what rotation rates $\boldsymbol{\Omega}$ is $\theta=0$ a stable equilibrium point?
	\item If there are other stable equilibrium angles(s) $\theta$, find the frequency of small amplitude oscillations about these equilibrium point(s).
\end{enumerate}
\subsubsection{Answer}


%%%%%%%%%%%%%%%%%%%%%%%%%%%%%%%%%%%%%%%%%%%%%%%%%%%%%%%%%%%%%%%%%%%%%%%%%%%%%%%
%%%% Problem 2
%%%%%%%%%%%%%%%%%%%%%%%%%%%%%%%%%%%%%%%%%%%%%%%%%%%%%%%%%%%%%%%%%%%%%%%%%%%%%%%
%\subsection{Problem 2}
\problem{2}
\subsubsection{Question}
% Keywords
	\index{unsolved!Fall 2016 II.P2}
A Helmholtz coil consists of two parallel circular current loops of identical radius $R$ and separated by a distance $d$ from each other. Each loop carries a uniform current $I$ along the same direction. Define the $z$ axis to be the common axis that crosses the centers of the two coils, such that $z=0$ is at the point midway between them.
\begin{enumerate}
	\item What is the value of $d$ for which the amplitude of the magnetic field $\boldsymbol{B}$ produced by the Helmholtz coil is such that $z\pdv{B}{z}=0$ and $\pdv[2]{B}{z}=0$ at $z=0$.
	\item For the value of d that you obtained above, calculate $\boldsymbol{B}(\boldsymbol{z}=\boldsymbol{0})$
\end{enumerate}
\subsubsection{Answer}



%%%%%%%%%%%%%%%%%%%%%%%%%%%%%%%%%%%%%%%%%%%%%%%%%%%%%%%%%%%%%%%%%%%%%%%%%%%%%%%
%%%% Problem 3
%%%%%%%%%%%%%%%%%%%%%%%%%%%%%%%%%%%%%%%%%%%%%%%%%%%%%%%%%%%%%%%%%%%%%%%%%%%%%%%
%\subsection{Problem 3}
\problem{3}
\subsubsection{Question}
% Keywords
	\index{unsolved!Fall 2016 II.P3}

A futuristic starship with the mass one million metric tons departs from a base in the outer space. The cruising speed of the starship corresponds to the time dilation (as observed from the base) 10 times. The starship accelerates in a straight line and reaches the cruising speed and then decelerates (also in straight line) reaching its destination near a star in a distant galaxy, which moves very slowly with respect to the ship’s home base. On the way back the ship again accelerates to its cruising speed and then decelerates returning to the base. Assuming that all the starships in the future are propelled by converting their fuel into light with 100\% efficiency and perfectly directing all the generated light in the direction opposite to the thrust, find the mass of the starship after it returns to the base. Ignore gravitational effects.
\subsubsection{Answer}



%%%%%%%%%%%%%%%%%%%%%%%%%%%%%%%%%%%%%%%%%%%%%%%%%%%%%%%%%%%%%%%%%%%%%%%%%%%%%%%
%%%% Problem 4
%%%%%%%%%%%%%%%%%%%%%%%%%%%%%%%%%%%%%%%%%%%%%%%%%%%%%%%%%%%%%%%%%%%%%%%%%%%%%%%
%\subsection{Problem 4}
\problem{4}
\subsubsection{Question}
% Keywords
	\index{unsolved!Fall 2016 II.P4}
A one-dimensional simple harmonic oscillator of angular frequency $\omega$ is acted upon by a spatially uniform but time-dependent perturbation force: $$F(t) = \frac{F_0\tau/\omega}{(\tau^2+t^2)}\hspace{.5in} -\infty<t<+\infty.$$ At $t=-\infty$, the oscillator is known to be in the ground state. Calculate, to leading order in the perturbation potential derived from the force above, the probability that the oscillator is in the first excited state at $t=+\infty$. You may find useful the following relationships between the position and momentum of the harmonic oscillator and the creation and annihilation operators: $$x=\sqrt{\frac{\hbar}{2m\omega}}(a+a^\dagger)\hspace{.5in}p = i\sqrt{\frac{m\omega\hbar}{2}}(-a+a^\dagger).$$
\subsubsection{Answer}


%%%%%%%%%%%%%%%%%%%%%%%%%%%%%%%%%%%%%%%%%%%%%%%%%%%%%%%%%%%%%%%%%%%%%%%%%%%%%%%
%%%% Problem 5
%%%%%%%%%%%%%%%%%%%%%%%%%%%%%%%%%%%%%%%%%%%%%%%%%%%%%%%%%%%%%%%%%%%%%%%%%%%%%%%
%\subsection{Problem 5}
\problem{5}
\subsubsection{Question}
% Keywords
	\index{unsolved!Fall 2016 II.P5}
Consider a classical statistical mechanics system consisting of $N$ subsystems labeled by $i=1,\cdots,N$, each of which can exists in two states $s_i=\pm 1$. Call $n_+$ the number of $+$ values and $n_-$ the number of $-$ values. Let the total energy of the system be given by
\begin{equation*}
	E = -J\sum_{i=1}^{N}s_i
\end{equation*}
\begin{enumerate}
	\item Let the energy of the system be fixed at $E_0 = N\epsilon$ where $-J<\epsilon<0$ (microcanonical ensemble). Express the number of configuration in terms of $N$ and $x=n_+/N$. Express $x$ in terms of $E_0, N$ and $J$.
	\item Focus on a particular subsystem, $i=1$. Compute the probability $r$ (as a function of $x$) that $s_1=+1$ divided by the probability that $s_1=-1$, directly in the microcanonical ensemble, again assuming that $N$ is large.
	\item Now suppose that the entire system is used as a heat bath for the subsystem considered in part $(b)$. What temperature does the entire system have as a function of $J$ and $x$?
	\item Repeat part $(b)$ above, treating the subsystems $s_2,\cdots,s_N$ as a heat bath for system $s_1$ and then working in the canonical ensemble. Are your answers consistent?
	\item Give a qualitative sketch of the temperature as a function of energy. Note a peculiarity in the system for $E_0>0$.
\end{enumerate}
You may find Sterling's approximation useful: $log(N!)\approx N\log(N)-N$.


\subsubsection{Answer}
