%%%%%%%%%%%%%%%%%%%%%%%%%%%%%%%%%%%%%%%%%%%%%%%%%%%%%%%%%%%%%%%%%%%%%%%%%%%%%%%
%%%% Problem 1
%%%%%%%%%%%%%%%%%%%%%%%%%%%%%%%%%%%%%%%%%%%%%%%%%%%%%%%%%%%%%%%%%%%%%%%%%%%%%%%
%\subsection{Problem 1}
\problem{1}
\subsubsection{Question}
% Keywords
	\index{unsolved!Fall 2013 II.P1}
A free particle of mass $m$ and spin $s$ is initially (at $t = 0$) in a state corresponding to the wave function
\begin{equation*}
	\psi(r) = \qty(\frac{\gamma}{\pi})^{3/4}e^{-\gamma r^2/2}
\end{equation*}
\begin{enumerate}
	\item Calculate the probability density of finding the particle with momentum $\hbar\kappa$ at any time $t$. Is it isotropic?
	\item What is the probability of finding the particle with energy E?
	\item Examine whether the particle is in an eigenstate of the square of the angular momentum $\mathbf{L}$, and of its $z$-component $\mathbf{L}_z$, for any time $t$.
\end{enumerate}
\subsubsection{Answer}


%%%%%%%%%%%%%%%%%%%%%%%%%%%%%%%%%%%%%%%%%%%%%%%%%%%%%%%%%%%%%%%%%%%%%%%%%%%%%%%
%%%% Problem 2
%%%%%%%%%%%%%%%%%%%%%%%%%%%%%%%%%%%%%%%%%%%%%%%%%%%%%%%%%%%%%%%%%%%%%%%%%%%%%%%
%\subsection{Problem 2}
\problem{2}
\subsubsection{Question}
% Keywords
	\index{unsolved!Fall 2013 II.P2}
Consider an elastic film (2D square lattice of $N$ atoms) stretched over a rigid square with side $L$. Assume that only transverse modes, i.e. corresponding to displacements orthogonal to the film, can be excited and that sound velocity $u$ is frequency independent. Find:
\begin{enumerate}
	\item The dispersion relationship between $\omega$ and $k$ (wave number of the sound)? 
	\item The maximum energy of an atom (often referred to as the Debye energy $\theta_D = k_BT_D$, where $T_D$ is the Debye temperature).
	\item The energy $U$ for extreme temperatures, $T\gg T_D$ and $T\ll T_D$.
	\item The heat capacity $C_V$ , for $T\gg T_D$ and $T\ll T_D$.
\end{enumerate}
\subsubsection{Answer}



%%%%%%%%%%%%%%%%%%%%%%%%%%%%%%%%%%%%%%%%%%%%%%%%%%%%%%%%%%%%%%%%%%%%%%%%%%%%%%%
%%%% Problem 3
%%%%%%%%%%%%%%%%%%%%%%%%%%%%%%%%%%%%%%%%%%%%%%%%%%%%%%%%%%%%%%%%%%%%%%%%%%%%%%%
%\subsection{Problem 3}
\problem{3}
\subsubsection{Question}
% Keywords
	\index{unsolved!Fall 2013 II.P3}
Two identical pendulae $(\omega_0 = g/l)$ are connected by a light coupling spring. With the coupling spring connected, one pendulum is clamped and the period of the other is found to be T seconds. With neither of the connected pendulum clamped, that is, free to oscillate, what are the periods of the two normal modes?
\subsubsection{Answer}



%%%%%%%%%%%%%%%%%%%%%%%%%%%%%%%%%%%%%%%%%%%%%%%%%%%%%%%%%%%%%%%%%%%%%%%%%%%%%%%
%%%% Problem 4
%%%%%%%%%%%%%%%%%%%%%%%%%%%%%%%%%%%%%%%%%%%%%%%%%%%%%%%%%%%%%%%%%%%%%%%%%%%%%%%
%\subsection{Problem 4}
\problem{4}
\subsubsection{Question}
% Keywords
	\index{unsolved!Fall 2013 II.P4}
Two long coaxial cylindrical metal tubes (inner radius $a$, outer radius $b$) stand vertically in a tank of dielectric oil (with dielectric constant $\epsilon$ and mass density $\rho$). The inner cylinder is maintained at potential $V$, while the outer one is grounded. To what height h does the oil rise in the space between the tubes? For simplicity, you can assume that the top surface of the oil between the cylinders is flat and horizontal.
\subsubsection{Answer}


%%%%%%%%%%%%%%%%%%%%%%%%%%%%%%%%%%%%%%%%%%%%%%%%%%%%%%%%%%%%%%%%%%%%%%%%%%%%%%%
%%%% Problem 5
%%%%%%%%%%%%%%%%%%%%%%%%%%%%%%%%%%%%%%%%%%%%%%%%%%%%%%%%%%%%%%%%%%%%%%%%%%%%%%%
%\subsection{Problem 5}
\problem{5}
\subsubsection{Question}
% Keywords
	\index{unsolved!Fall 2013 II.P5}
At Brookhaven National Laboratory, a beam of heavy nuclei smash into a target composed of the same, identical nuclei. The kinetic energy of the beam particles is $14.5$ GeV per nucleon. What is the Lorentz factor $\gamma$ of each nucleus as seen by an observer in the center of mass frame?

Assume that these nuclei stop each other and form one composite nuclear system, what nucleon density results? Compare this density with the average density of a neutron star with mass $1.4$ solar masses and radius $R = 10$ km. Atomic nuclei have a normal density of $n_0 = 0.15$ nucleons/fm$^3$ , $m_N = 1.7\times10^{-27}$ kg and the solar mass, $M_O = 2\times10^{30}$ kg.
\subsubsection{Answer}

