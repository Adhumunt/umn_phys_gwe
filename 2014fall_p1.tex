%%%%%%%%%%%%%%%%%%%%%%%%%%%%%%%%%%%%%%%%%%%%%%%%%%%%%%%%%%%%%%%%%%%%%%%%%%%%%%%
%%%% Problem 1
%%%%%%%%%%%%%%%%%%%%%%%%%%%%%%%%%%%%%%%%%%%%%%%%%%%%%%%%%%%%%%%%%%%%%%%%%%%%%%%
%\subsection{Problem 1}
\problem{1}
\subsubsection{Question}
% Keywords
	\index{unsolved!Fall 2014 I.P1}
A photon source is at the focus of a parabolic mirror and both are attached to a rocket, see Figure 1. Upon reflection the photons form a parallel beam. Find the final velocity of a rocket if it starts from rest with mass $m_1$ at and its final rest mass is $m_2$. (Be sure to use relativistic expressions throughout.)
\subsubsection{Answer}


%%%%%%%%%%%%%%%%%%%%%%%%%%%%%%%%%%%%%%%%%%%%%%%%%%%%%%%%%%%%%%%%%%%%%%%%%%%%%%%
%%%% Problem 2
%%%%%%%%%%%%%%%%%%%%%%%%%%%%%%%%%%%%%%%%%%%%%%%%%%%%%%%%%%%%%%%%%%%%%%%%%%%%%%%
%\subsection{Problem 2}
\problem{2}
\subsubsection{Question}
% Keywords
	\index{unsolved!Fall 2014 I.P2}
In a laboratory on Earth mass $m_1$ is on a table and connected by a string, which runs over a frictionless pulley, to mass $m_2$ that is hanging from the side of the table. The table moves horizontally with acceleration a such that $m_2$ is tilted at a constant angle away from the table and $m_1$ is sliding. Find the tension in the string if the coefficient of kinetic friction between $m_1$ and the table is $\mu$.
\subsubsection{Answer}



%%%%%%%%%%%%%%%%%%%%%%%%%%%%%%%%%%%%%%%%%%%%%%%%%%%%%%%%%%%%%%%%%%%%%%%%%%%%%%%
%%%% Problem 3
%%%%%%%%%%%%%%%%%%%%%%%%%%%%%%%%%%%%%%%%%%%%%%%%%%%%%%%%%%%%%%%%%%%%%%%%%%%%%%%
%\subsection{Problem 3}
\problem{3}
\subsubsection{Question}
% Keywords
	\index{unsolved!Fall 2014 I.P3}
	\index{oscillator!Anisotropic}
	\index{mathematics!Lissajous Curves}
Consider a two-dimensional anisotropic oscillator where the potential energy is given by
\begin{equation*}
	V= \frac{1}{2}\qty(k_1x^2+k_2y^2)
\end{equation*}
\begin{enumerate}
	\item Find the position and velocity as functions of time if the initial position and velocity are given by $r = (x_0, y_0)$ and $v = (0, 0)$.
	\item What is the condition on $k_1$ and $k_2$ for the particle’s trajectory to be a closed Lissajous figure?
\end{enumerate}
\subsubsection{Answer}



%%%%%%%%%%%%%%%%%%%%%%%%%%%%%%%%%%%%%%%%%%%%%%%%%%%%%%%%%%%%%%%%%%%%%%%%%%%%%%%
%%%% Problem 4
%%%%%%%%%%%%%%%%%%%%%%%%%%%%%%%%%%%%%%%%%%%%%%%%%%%%%%%%%%%%%%%%%%%%%%%%%%%%%%%
%\subsection{Problem 4}
\problem{4}
\subsubsection{Question}
% Keywords
	\index{unsolved!Fall 2014 I.P4}
	\index{optics!Index of Refraction}
The refractive index of glass can be represented approximately by the empirical relation
\begin{equation}
	n = A + B\lambda^{-2}
\end{equation}
where $\lambda$ is the wavelength of light in vacuum. What are the corresponding phase and group velocities of light in glass? Do your formulae reduce to what you expect if there is no dispersion?
\subsubsection{Answer}


%%%%%%%%%%%%%%%%%%%%%%%%%%%%%%%%%%%%%%%%%%%%%%%%%%%%%%%%%%%%%%%%%%%%%%%%%%%%%%%
%%%% Problem 5
%%%%%%%%%%%%%%%%%%%%%%%%%%%%%%%%%%%%%%%%%%%%%%%%%%%%%%%%%%%%%%%%%%%%%%%%%%%%%%%
%\subsection{Problem 5}
\problem{5}
\subsubsection{Question}
% Keywords
	\index{unsolved!Fall 2014 I.P5}
Two identical bodies, each characterized by a heat capacity at constant pressure $C$ which is independent of temperature, are used as heat reservoirs for a heat engine. The bodies remain at constant pressure and undergo no change of phase. Initially, their temperature are $T_1$ and $T_2$, $T_1 > T_2$. At the final state, as a result of the operation of the heat engine, the bodies will attain a common final temperature $T_f$.
\begin{enumerate}
	\item What is the total amount of work $W$ done by the engine? Express the answer in terms of $C,T_1,T_2,$ and $T_f.$
	\item Use arguments based on entropy considerations to derive an inequality relating $T_f$ to the initial temperatures.
	\item For given initial temperatures, what is the maximum amount of work obtainable from the engine?
\end{enumerate}
\subsubsection{Answer}



%%%%%%%%%%%%%%%%%%%%%%%%%%%%%%%%%%%%%%%%%%%%%%%%%%%%%%%%%%%%%%%%%%%%%%%%%%%%%%%
%%%% Problem 6
%%%%%%%%%%%%%%%%%%%%%%%%%%%%%%%%%%%%%%%%%%%%%%%%%%%%%%%%%%%%%%%%%%%%%%%%%%%%%%%
%\subsection{Problem 6}
\problem{6}
\subsubsection{Question}
% Keywords
	\index{unsolved!Fall 2014 I.P6}
	\index{electrostatics!Hydrogen Atom}
	\index{quantum!Hydrogen Atom}
	Consider a model for the hydrogen atom in which the proton is assumed to be a point charge located at the origin and the electron is described by a continuous charge distribution with spherical symmetry:
	\begin{equation}
		\rho(r) = -\frac{e}{\pi a^3}\exp(-\frac{2r}{a})
	\end{equation}
	where $a$ is the Bohr radius and $r$ is the distance from the origin. Here $e$ is the absolute value of the charge of the electron. In the presence of an electric field $\mathbf{E}$, the proton is displaced from the origin to a new equilibrium position a distance $d\ll a$ from the origin. Calculate $d$ and the induced dipole moment $p$ to find an expression for the polarizability of the hydrogen atom. Hint: use the approximation $d\ll a$ before evaluating the integral.

\subsubsection{Answer}

%%%%%%%%%%%%%%%%%%%%%%%%%%%%%%%%%%%%%%%%%%%%%%%%%%%%%%%%%%%%%%%%%%%%%%%%%%%%%%%
%%%% Problem 7
%%%%%%%%%%%%%%%%%%%%%%%%%%%%%%%%%%%%%%%%%%%%%%%%%%%%%%%%%%%%%%%%%%%%%%%%%%%%%%%
%\subsection{Problem 7}
\problem{7}
\subsubsection{Question}
% Keywords
	\index{unsolved!Fall 2014 I.P7}
	\index{thermodynamics!Specific Heat of Superconductors}
At low enough temperatures, the thermodynamic properties of a two-dimensional $d$-wave superconductor can be described in terms of a gas of non-interacting fermions that follow the $p_2$ dispersion relation $E(\mathbf{k}) = \sqrt{a^2 k_x^2 + b^2 k_y^2}$, with $a, b$ denoting positive constants. The total number of these fermions is not conserved. Determine how the specific heat of this system depends on the temperature $T$ in this low-temperature regime. Hint: you do not need to evaluate the pre-factors and you can ignore the spin degeneracy.
\subsubsection{Answer}



%%%%%%%%%%%%%%%%%%%%%%%%%%%%%%%%%%%%%%%%%%%%%%%%%%%%%%%%%%%%%%%%%%%%%%%%%%%%%%%
%%%% Problem 8
%%%%%%%%%%%%%%%%%%%%%%%%%%%%%%%%%%%%%%%%%%%%%%%%%%%%%%%%%%%%%%%%%%%%%%%%%%%%%%%
%\subsection{Problem 8}
\problem{8}
\subsubsection{Question}
% Keywords
	\index{unsolved!Fall 2014 I.P8}
	\index{quantum!Wavefunction Probabilities}
The wavefunction of a particle in the ground state of a one-dimensional oscillator with potential energy $U(x) = m^2\omega^2 x^2 /2$ is
\begin{equation*}
	\psi(x) = \frac{e^{-x^2/(2\ell^2)}}{(\pi \ell^2)^{1/4}}
\end{equation*}
where $\ell=\sqrt{\hbar/(m\omega)}$. The oscillator potential is abruptly shifted by distance $a$ so that the potential energy becomes $U(x) = m^2 \omega^2 (x - a)^2/2.$ What is the probability that the particle will stay in the ground state?
\subsubsection{Answer}
After the shift, the wave function should read 
\begin{equation*}
	\psi(x) = \frac{e^{-(x-a)^2/(2\ell^2)}}{(\pi\ell^2)^{1/4}}.
\end{equation*}
To find the probability of finding the new wave function in the old ground state, we need to calculate
\begin{align}
	P = \abs{\braket{\psi_{g,o}}{\psi_{g,n}}}
\end{align}


%%%%%%%%%%%%%%%%%%%%%%%%%%%%%%%%%%%%%%%%%%%%%%%%%%%%%%%%%%%%%%%%%%%%%%%%%%%%%%%
%%%% Problem 9
%%%%%%%%%%%%%%%%%%%%%%%%%%%%%%%%%%%%%%%%%%%%%%%%%%%%%%%%%%%%%%%%%%%%%%%%%%%%%%%
%\subsection{Problem 9}
\problem{9}
\subsubsection{Question}
% Keywords
	\index{unsolved!Fall 2014 I.P9}
	
\subsubsection{Answer}


%%%%%%%%%%%%%%%%%%%%%%%%%%%%%%%%%%%%%%%%%%%%%%%%%%%%%%%%%%%%%%%%%%%%%%%%%%%%%%%
%%%% Problem 10
%%%%%%%%%%%%%%%%%%%%%%%%%%%%%%%%%%%%%%%%%%%%%%%%%%%%%%%%%%%%%%%%%%%%%%%%%%%%%%%
%\subsection{Problem 10}
\problem{10}
\subsubsection{Question}
% Keywords
	\index{unsolved!Fall 2014 I.P10}

The magnetic field inside a large piece of magnetic material is $\mathbf{B}_0$, so that $\mathbf{H}_0 = (1/\mu_0)\mathbf{B}_0 - \mathbf{M},$ where $\mathbf{M}$ is a magnetization that is frozen in the material.
\begin{enumerate}
	\item A long narrow cylinder with the long axis parallel to $\mathbf{M}$ is hollowed out of the material. Find the field $\mathbf{B}$ at the center of the cavity in terms of $\mathbf{B}_0$, and $\mathbf{M}$. Also find $\mathbf{H}$ at the center of the cavity in terms of $\mathbf{H}_0$, and $\mathbf{M}$.
	\item Do the same assuming that the cavity is a thin disc with the symmetry axis parallel to $\mathbf{M}$.
	\item Do the same assuming that a small spherical cavity is hollowed out of the material.
\end{enumerate}
Note: the magnetic field inside a magnetized sphere with frozen-in magnetization $\mathbf{M}$ is $\mathbf{B}=\frac{2}{3}\mu_0\mathbf{M}.$
\subsubsection{Answer}

