%%%%%%%%%%%%%%%%%%%%%%%%%%%%%%%%%%%%%%%%%%%%%%%%%%%%%%%%%%%%%%%%%%%%%%%%%%%%%%%
%%%% Problem 1: Aditya
%%%%%%%%%%%%%%%%%%%%%%%%%%%%%%%%%%%%%%%%%%%%%%%%%%%%%%%%%%%%%%%%%%%%%%%%%%%%%%%
%\subsection{Problem 1}
\problem{1}
\subsubsection{Question}
% Keywords 
	\index{particle!Decay}
A positively charge particle $\Sigma^+$ decays into a neutron $n$ and a pion $\pi^+$. Both the neutron and pion are observed to move in the same direction as the $\Sigma^+$ was originally moving, with momentum $\abs{\mathbf{p}_n} = 4702\text{MeV/c}$, and $\abs{\mathbf{p}_{\pi^+}}=171\text{MeV/c}$. What is the mass of $\Sigma^+$? The rest energy of the neutron and the charged pion are, respectively, $E_{0n}=940\text{MeV}$, and $E_{0\pi^+}=140\text{MeV}$.

\subsubsection{Answer}
We need to use the invariance of the relativistic four momentum for this problem. Let $p^\mu_{\Sigma^+} = p^\mu_{n} + p^\mu_{\pi^+}$, then the relativistic invariant is
\begin{equation*}
	p^2_{\Sigma^+} = m^2_{\Sigma^+} = m^2_n + m^2_{\pi^+} + 2p_{n}\cdot p_{\pi^+},
\end{equation*}
where $p_n\cdot p_{\pi^+} = E_{n} E_{\pi^+} - \abs{\mathbf{p}_n}\abs{\mathbf{p}_{\pi^+}}$. Plugging in for the above values, and taking the positive square root, one finds
\begin{equation*}
	m_{\Sigma^+} = 1189.3 \text{ MeV/c}.
\end{equation*}


%%%%%%%%%%%%%%%%%%%%%%%%%%%%%%%%%%%%%%%%%%%%%%%%%%%%%%%%%%%%%%%%%%%%%%%%%%%%%%%
%%%% Problem 2
%%%%%%%%%%%%%%%%%%%%%%%%%%%%%%%%%%%%%%%%%%%%%%%%%%%%%%%%%%%%%%%%%%%%%%%%%%%%%%%
%\subsection{Problem 2}
\problem{2}
\subsubsection{Question}
% Keywords
	\index{unsolved!Fall 2010 I.P2}
	\index{quantum!Time Independent Perturbation Theory}
The spin-orbit interaction for an electron in a hydrogen atom is governed by the Hamiltonian
\begin{equation*}
	H_{SO} = \frac{1}{4\pi \epsilon_0}\frac{e^2}{2m^2c^2r^3}\mathbf{L}\cdot\mathbf{S} 
\end{equation*}
where $\epsilon_0$ is the vacuum permitivity, $e$ and $m$ are the electron's electric charge and mass, respectively, $c$ is the speed of light, $r$ is the radial distance of the electron from the proton, and $\mathbf{L}$ and $\mathbf{S}$ are the orbital and spin angular momentum of the electron.
\begin{enumerate}
	\item Compute the energy correction due to the spin-orbit term. You may need the following identity:
	\begin{equation*}
		\expval{\frac{1}{r^3}} = \frac{2}{a^3n^3l(l+1)(2l+1)}
	\end{equation*}
	where $a$ is the Bohr radius and $n$ and $l$ are the principle and orbital quantum numbers, respectively (notice the identity holds for $l\ne 0$).
	\item Describe the energy level splitting for $n=2$ (use spectroscopic notation).
\end{enumerate}
\subsubsection{Answer}



%%%%%%%%%%%%%%%%%%%%%%%%%%%%%%%%%%%%%%%%%%%%%%%%%%%%%%%%%%%%%%%%%%%%%%%%%%%%%%%
%%%% Problem 3: Aditya
%%%%%%%%%%%%%%%%%%%%%%%%%%%%%%%%%%%%%%%%%%%%%%%%%%%%%%%%%%%%%%%%%%%%%%%%%%%%%%%
%\subsection{Problem 3}
\problem{3}
\subsubsection{Question}
% Keywords
	\index{quantum!Bohr radius}
Consider the positronium, namely a bound state of a positron and an electron. What is the corresponding Bohr radius? What is the energy corresponding to the $(n=2)\to(n=1)$ transition?

\subsubsection{Answer}
In the hydrogen atom, the reduced mass may be neglected in the calculations since $m_{e^-}/m_p$. In this case, the Bohr radius is instead
\begin{equation*}
	a = \frac{4\pi \epsilon_0\hbar^2}{\mu e^2}
\end{equation*}
where $\mu = m_{e^-}m_{e^+}/(m_{e^{-}} + m_{e^{+}}) = \frac{1}{2}m_{e^-}$ is the reduced mass. The energy corresponding to the $(n=2)\to(n=1)$ transition is given by 
\begin{align*}
	\Delta E = E_2 - E_1 = E_1\qty[\frac{1}{2^2} - \frac{1}{1^2}] = - \frac{3}{4} E_1
\end{align*}
where the ground state binding energy is $E_1 = - \frac{\mu}{2\hbar^2}\qty(\frac{e^2}{4\pi \epsilon_0})^2 = - \frac{m_e}{4\hbar}\qty(\frac{e^2}{4\pi \epsilon_0})^2 = -6.8$ eV. Thus
\begin{equation}
	\boxed{\Delta E = - \frac{3E_1}{4} = 5.1 \text{ eV}}
\end{equation}


%%%%%%%%%%%%%%%%%%%%%%%%%%%%%%%%%%%%%%%%%%%%%%%%%%%%%%%%%%%%%%%%%%%%%%%%%%%%%%%
%%%% Problem 4
%%%%%%%%%%%%%%%%%%%%%%%%%%%%%%%%%%%%%%%%%%%%%%%%%%%%%%%%%%%%%%%%%%%%%%%%%%%%%%%
%\subsection{Problem 4}
\problem{4}
\subsubsection{Question}
% Keywords
	\index{unsolved!Fall 2010 I.P4}
	\index{electrodynamics!Dielectrics}
Two conductors of arbitrary shape are placed (without touching each other) in a liquid with a uniform conductivity $\sigma$. At $t = 0$ a total charge of $+Q_0$ is placed on one of the conductors, and $-Q_0$ on the other. Derive the time dependence of the charge on the conductors as a function of time.

\subsubsection{Answer}


%%%%%%%%%%%%%%%%%%%%%%%%%%%%%%%%%%%%%%%%%%%%%%%%%%%%%%%%%%%%%%%%%%%%%%%%%%%%%%%
%%%% Problem 5
%%%%%%%%%%%%%%%%%%%%%%%%%%%%%%%%%%%%%%%%%%%%%%%%%%%%%%%%%%%%%%%%%%%%%%%%%%%%%%%
%\subsection{Problem 5}
\problem{5}
\subsubsection{Question}
% Keywords
	\index{unsolved!Fall 2010 I.P5}
	\index{mechanics!Ekman spiral}
	\index{dimensional analysis!Ekman spiral}
Wind-driven currents in a body of water on the Earth spiral down due to a combination of viscous and Coriolis forces. The pitch of this spiral (called Ekman spiral in oceanography) is a length $\lambda$ which depends on the water density $\rho$, the viscosity $\eta$, and the angular speed of the Earth rotation $\omega$. Assuming that $\lambda\propto\rho^a\eta^b\omega^c$, use dimensional analysis to find the exponents $a$, $b$, and $c$.

\subsubsection{Answer}



%%%%%%%%%%%%%%%%%%%%%%%%%%%%%%%%%%%%%%%%%%%%%%%%%%%%%%%%%%%%%%%%%%%%%%%%%%%%%%%
%%%% Problem 6
%%%%%%%%%%%%%%%%%%%%%%%%%%%%%%%%%%%%%%%%%%%%%%%%%%%%%%%%%%%%%%%%%%%%%%%%%%%%%%%
%\subsection{Problem 6}
\problem{6}
\subsubsection{Question}
% Keywords
	\index{unsolved!Fall 2010 I.P6}
	\index{thermodynamics!Volume Occupation of Ideal Gases}
A container of volume $V$ is divided into two parts by a sliding partition. On one side there is one mole of an ideal gas made of spin $1/2$ particles, and on the other side one mole of an ideal gas consisting of spin zero particles. The two gases have the same temperature $T$. Find the ratio of the volumes occupied by the two gases $(i)$ at $T = 0$ and $(ii)$ at very high $T$.

\subsubsection{Answer}

%%%%%%%%%%%%%%%%%%%%%%%%%%%%%%%%%%%%%%%%%%%%%%%%%%%%%%%%%%%%%%%%%%%%%%%%%%%%%%%
%%%% Problem 7
%%%%%%%%%%%%%%%%%%%%%%%%%%%%%%%%%%%%%%%%%%%%%%%%%%%%%%%%%%%%%%%%%%%%%%%%%%%%%%%
%\subsection{Problem 7}
\problem{7}
\subsubsection{Question}
% Keywords
	\index{unsolved!Fall 2010 I.P7}
	\index{thermodynamics!Equilibrium of Ideal Gases}
	\index{statistical mechanics!Equilibrium of Ideal Gases}
A cylindrical container of total volume $4V$, thermally insulated, is separated into two compartments of volumes $V$ and $3V$ by a non-insulated partition. The partition is initially fixed: the smaller compartment holds one mole of an ideal gas, and the larger one six moles of a different ideal gas. The system has the temperature $T_0$. The partition is then allowed to slide inside the container (see Figure 2) until the system reaches equilibrium.
\begin{enumerate}
	\item What is the final temperature?
	\item What are the final volumes?
	\item What is the change in entropy in this process? 
\end{enumerate}
\subsubsection{Answer}



%%%%%%%%%%%%%%%%%%%%%%%%%%%%%%%%%%%%%%%%%%%%%%%%%%%%%%%%%%%%%%%%%%%%%%%%%%%%%%%
%%%% Problem 8
%%%%%%%%%%%%%%%%%%%%%%%%%%%%%%%%%%%%%%%%%%%%%%%%%%%%%%%%%%%%%%%%%%%%%%%%%%%%%%%
%\subsection{Problem 8}
\problem{8}
\subsubsection{Question}
% Keywords
	\index{statistical mechanics!Fermi Lattice}
Consider a monovalent simple cubic metal in which the interactions between the electrons and the lattice are so weak that the electrons can be treated as free. $(i)$ Calculate the Fermi wavevector $k_F$ in terms of the lattice spacing $a$. $(ii)$ Show that the minimum energy that a photon must have to be absorbed by this metal is (approximately) $0.063$ times the Fermi energy. Hint: You may use free electron bands in the reduced zone scheme.

\subsubsection{(Partial) Answer}
The definition of the Fermi wavevector is given by $k_F = \qty(3\rho \pi^2)^{1/3}$ where $\rho = Nq/V$ is the free electron density; $N$ is the number of atoms, $q$ is the number of electrons (typically $1$ or $2$), and $V$ is the lattice volume. Since the simple cubic metal is monovalent, $q =1$ and $V=a^3$, thus
\begin{equation}
	k_F = \qty(\frac{3N\pi^2}{a^3})^{1/3}.
\end{equation}


%%%%%%%%%%%%%%%%%%%%%%%%%%%%%%%%%%%%%%%%%%%%%%%%%%%%%%%%%%%%%%%%%%%%%%%%%%%%%%%
%%%% Problem 9
%%%%%%%%%%%%%%%%%%%%%%%%%%%%%%%%%%%%%%%%%%%%%%%%%%%%%%%%%%%%%%%%%%%%%%%%%%%%%%%
%\subsection{Problem 9}
\problem{9}
\subsubsection{Question}
% Keywords
	\index{unsolved!Fall 2010 I.P9}
	\index{mechanics!Escape velocity}
An astronaut is on the surface of a spherical asteroid, with a uniform density equal to the average density of Earth. Estimate the condition on the radius of the asteroid, for which the astronaut can escape from the asteroid with a jump (give a numerical answer).

\subsubsection{Answer}



%%%%%%%%%%%%%%%%%%%%%%%%%%%%%%%%%%%%%%%%%%%%%%%%%%%%%%%%%%%%%%%%%%%%%%%%%%%%%%%
%%%% Problem 10
%%%%%%%%%%%%%%%%%%%%%%%%%%%%%%%%%%%%%%%%%%%%%%%%%%%%%%%%%%%%%%%%%%%%%%%%%%%%%%%
%\subsection{Problem 10}
\problem{10}
\subsubsection{Question}
% Keywords
	\index{unsolved!Fall 2010 I.P10}
	\index{mechanics!Yo-yo}
	\index{Lagrangian!Yo-yo}
A yo-yo consists of two disks of radius $R_2$ connected by an axle of radius $R_1 < R_2$ (see Figure 3). The yo-yo descends under the influence of gravity by the unwinding of a string wrapped around its axle (the top end of the string is kept fixed). The total mass of the yo-yo is $m$, and the axle and the string have negligible mass. Compute the downward acceleration of the yo-yo's center of mass, and the tension in the string.

\subsubsection{Answer}

