%%%%%%%%%%%%%%%%%%%%%%%%%%%%%%%%%%%%%%%%%%%%%%%%%%%%%%%%%%%%%%%%%%%%%%%%%%%%%%%
%%%% Problem 1
%%%%%%%%%%%%%%%%%%%%%%%%%%%%%%%%%%%%%%%%%%%%%%%%%%%%%%%%%%%%%%%%%%%%%%%%%%%%%%%
%\subsection{Problem 1}
\problem{1}
\subsubsection{Question}
% Keywords
	\index{unsolved!Spring 2018 I.P1}
The electron in a hydrogen atom is in a state described by the following superposition of normalized energy eignestates $u$, with real $A > 0$,
\begin{equation*}
	\psi(r,\theta,\phi) = \frac{1}{5}\qty(3u_{100} + Au_{211} - 2u_{21-1}+3u_{321})
\end{equation*}
where the subscripts represent the quantum numbers $\{n, l, m_l \}$.
\begin{enumerate}
	\item Calculate $A$ such that this wavefunction is normalized.
	\item Find the expectation value of the energy in this state, in terms of the ground state energy of hydrogen $E_1$.
	\item Find the expectation values of $L^2$ and $L_z$ in this state. 
\end{enumerate}
\subsubsection{Answer}


%%%%%%%%%%%%%%%%%%%%%%%%%%%%%%%%%%%%%%%%%%%%%%%%%%%%%%%%%%%%%%%%%%%%%%%%%%%%%%%
%%%% Problem 2
%%%%%%%%%%%%%%%%%%%%%%%%%%%%%%%%%%%%%%%%%%%%%%%%%%%%%%%%%%%%%%%%%%%%%%%%%%%%%%%
%\subsection{Problem 2}
\problem{2}
\subsubsection{Question}
% Keywords
	\index{unsolved!Spring 2018 I.P2}
An electron is confined to the interior of a hollow spherical cavity of radius $R$ with impenetrable walls. Find an expression for the pressure exerted on the walls of the cavity by the electron in its ground state, recalling that the Laplacian in spherical polar coordinates is given by
\begin{equation*}
 	\grad^2 = \frac{1}{r^2}\pdv[2]{r}r^2 + (angular\ part)
\end{equation*}
\subsubsection{Answer}



%%%%%%%%%%%%%%%%%%%%%%%%%%%%%%%%%%%%%%%%%%%%%%%%%%%%%%%%%%%%%%%%%%%%%%%%%%%%%%%
%%%% Problem 3
%%%%%%%%%%%%%%%%%%%%%%%%%%%%%%%%%%%%%%%%%%%%%%%%%%%%%%%%%%%%%%%%%%%%%%%%%%%%%%%
%\subsection{Problem 3}
\problem{3}
\subsubsection{Question}
% Keywords
	\index{unsolved!Spring 2018 I.P3}
Consider a ``ballistic pendulum'' made up of a heavy uniform rod $1$ meter long suspended vertically from one end. A bullet traveling horizontally hits this rod at a distance of $75$cm from the pivot point and becomes embedded in the rod, causing the rod to deflect by an angle of $30^\circ$ . Assuming that the mass of the bullet is 1\% of the mass of the rod, find the speed of the bullet. (Hint: you can neglect the mass of the bullet compared to the rod when determining the moment of inertia).
\subsubsection{Answer}



%%%%%%%%%%%%%%%%%%%%%%%%%%%%%%%%%%%%%%%%%%%%%%%%%%%%%%%%%%%%%%%%%%%%%%%%%%%%%%%
%%%% Problem 4
%%%%%%%%%%%%%%%%%%%%%%%%%%%%%%%%%%%%%%%%%%%%%%%%%%%%%%%%%%%%%%%%%%%%%%%%%%%%%%%
%\subsection{Problem 4}
\problem{4}
\subsubsection{Question}
% Keywords
	\index{unsolved!Spring 2018 I.P4}
Find the value of $c$ for which the force given below is conservative. Determine the potential energy function for this force.
\subsubsection{Answer}


%%%%%%%%%%%%%%%%%%%%%%%%%%%%%%%%%%%%%%%%%%%%%%%%%%%%%%%%%%%%%%%%%%%%%%%%%%%%%%%
%%%% Problem 5
%%%%%%%%%%%%%%%%%%%%%%%%%%%%%%%%%%%%%%%%%%%%%%%%%%%%%%%%%%%%%%%%%%%%%%%%%%%%%%%
%\subsection{Problem 5}
\problem{5}
\subsubsection{Question}
% Keywords
	\index{unsolved!Spring 2018 I.P5}
Consider a spherical shell of radius R that has an imposed potential given by
\begin{equation*}
	V(\theta) = V_0\cos^2(\theta)
\end{equation*}
\begin{enumerate}
	\item Determine the potential inside and outside of the sphere in terms of a Legendre polynomial expansion.
	\item Calculate the radial electric field inside and outside, and find the surface charge density.
	\item What is the total charge on the spherical shell?
\end{enumerate}
\subsubsection{Answer}



%%%%%%%%%%%%%%%%%%%%%%%%%%%%%%%%%%%%%%%%%%%%%%%%%%%%%%%%%%%%%%%%%%%%%%%%%%%%%%%
%%%% Problem 6
%%%%%%%%%%%%%%%%%%%%%%%%%%%%%%%%%%%%%%%%%%%%%%%%%%%%%%%%%%%%%%%%%%%%%%%%%%%%%%%
%\subsection{Problem 6}
\problem{6}
\subsubsection{Question}
% Keywords
	\index{unsolved!Spring 2018 I.P6}
Consider a co-axial cable made up of an inner conductor of radius $a$ and an outer conductor of radius $b$. The inner conductor is solid and contains a constant current density $j$. The outer conductor is a thin shell and contains a surface current that exactly balances the current of the inner conductor.
\begin{enumerate}
	\item Determine the surface current density on the outer conductor.
	\item Determine the magnetic field everywhere in space and calculate the total magnetic energy per unit length.
	\item Find the inductance per unit length of this cable.
\end{enumerate}
\subsubsection{Answer}

%%%%%%%%%%%%%%%%%%%%%%%%%%%%%%%%%%%%%%%%%%%%%%%%%%%%%%%%%%%%%%%%%%%%%%%%%%%%%%%
%%%% Problem 7
%%%%%%%%%%%%%%%%%%%%%%%%%%%%%%%%%%%%%%%%%%%%%%%%%%%%%%%%%%%%%%%%%%%%%%%%%%%%%%%
%\subsection{Problem 7}
\problem{7}
\subsubsection{Question}
% Keywords
	\index{unsolved!Spring 2018 I.P7}
A system with an equidistant energy spectrum ($\epsilon_n = n\cdot\Delta,n=0,1,2,\cdots$) is populated with identical bosons of spin $s = 0$. If $N \gg 1$ is the number of bosons occupying the lowest two orbitals and the occupation of the ground state is twice the occupation of the lowest excited state, find the:
\begin{enumerate}
	\item Temperature.
	\item Chemical Potential.
	\item Occupancy o the second excited orbital $(n=2)$. 
\end{enumerate}
\subsubsection{Answer}



%%%%%%%%%%%%%%%%%%%%%%%%%%%%%%%%%%%%%%%%%%%%%%%%%%%%%%%%%%%%%%%%%%%%%%%%%%%%%%%
%%%% Problem 8
%%%%%%%%%%%%%%%%%%%%%%%%%%%%%%%%%%%%%%%%%%%%%%%%%%%%%%%%%%%%%%%%%%%%%%%%%%%%%%%
%\subsection{Problem 8}
\problem{8}
\subsubsection{Question}
% Keywords
	\index{unsolved!Spring 2018 I.P8}
Two identical classical monoatomic ideal gases with the same temperature $\tau$ and the same number of atoms $N$ are contained in two vessels of volumes $V_1$ and $V_2$ which are then connected. The combined system is isolated from the environment. After the system has reached equilibrium, what is the:
\begin{enumerate}
	\item Total energy.
	\item Pressure.
	\item Change in entropy.
	\item Change in temperature. 
\end{enumerate}
\subsubsection{Answer}


%%%%%%%%%%%%%%%%%%%%%%%%%%%%%%%%%%%%%%%%%%%%%%%%%%%%%%%%%%%%%%%%%%%%%%%%%%%%%%%
%%%% Problem 9
%%%%%%%%%%%%%%%%%%%%%%%%%%%%%%%%%%%%%%%%%%%%%%%%%%%%%%%%%%%%%%%%%%%%%%%%%%%%%%%
%\subsection{Problem 9}
\problem{9}
\subsubsection{Question}
% Keywords
	\index{unsolved!Spring 2018 I.P9}
A particle of mass $m$ is given a kinetic energy equal to $n$ times its rest energy. What is its speed and momentum?
\subsubsection{Answer}


%%%%%%%%%%%%%%%%%%%%%%%%%%%%%%%%%%%%%%%%%%%%%%%%%%%%%%%%%%%%%%%%%%%%%%%%%%%%%%%
%%%% Problem 10
%%%%%%%%%%%%%%%%%%%%%%%%%%%%%%%%%%%%%%%%%%%%%%%%%%%%%%%%%%%%%%%%%%%%%%%%%%%%%%%
%\subsection{Problem 10}
\problem{10}
\subsubsection{Question}
% Keywords
	\index{unsolved!Spring 2018 I.P10}
A typical neutron star has approximately the same mass as the sun but is as dense as a proton. Estimate the radius of a neutron star and the gravitational binding energy released in its formation.
\subsubsection{Answer}

