%%%%%%%%%%%%%%%%%%%%%%%%%%%%%%%%%%%%%%%%%%%%%%%%%%%%%%%%%%%%%%%%%%%%%%%%%%%%%%%
%%%% Problem 1
%%%%%%%%%%%%%%%%%%%%%%%%%%%%%%%%%%%%%%%%%%%%%%%%%%%%%%%%%%%%%%%%%%%%%%%%%%%%%%%
%\subsection{Problem 1}
\problem{1}
\subsubsection{Question}
% Keywords
	\index{unsolved!Spring 2018 II.P1}
Consider the infinite square well in one dimension, extending from $-a/2$ to $+a/2$. A particle of mass $m$ is sitting in its ground state. At time $t = 0$ the size of the well doubles, so that instead of going from $-a/2$ to $a/2$, it goes from $-a$ to $a$. If one then measures the energy, what is the most probable result, and what is the probability of getting it?

\subsubsection{Answer}


%%%%%%%%%%%%%%%%%%%%%%%%%%%%%%%%%%%%%%%%%%%%%%%%%%%%%%%%%%%%%%%%%%%%%%%%%%%%%%%
%%%% Problem 2
%%%%%%%%%%%%%%%%%%%%%%%%%%%%%%%%%%%%%%%%%%%%%%%%%%%%%%%%%%%%%%%%%%%%%%%%%%%%%%%
%\subsection{Problem 2}
\problem{2}
\subsubsection{Question}
% Keywords
	\index{unsolved!Spring 2018 II.P2}
Consider a system as shown in the figure with three equal masses m connected with identical springs of spring constant $k$, with the whole system attached to two immovable walls. The masses are constrained to move only in the horizontal direction.
\begin{enumerate}
	\item Write down the equations of motion for each mass and find the normal mode frequencies (Hint: this gives a cubic equation, but you should be able to find an obvious common factor).
	\item Determine the eigenvectors (i.e., the ratio between the amplitudes of the three mases in each mode). Assume that the first mass has an amplitude of 1 for each mode. Sketch the motion in each mode. 
\end{enumerate}
\subsubsection{Answer}

%%%%%%%%%%%%%%%%%%%%%%%%%%%%%%%%%%%%%%%%%%%%%%%%%%%%%%%%%%%%%%%%%%%%%%%%%%%%%%%
%%%% Problem 3
%%%%%%%%%%%%%%%%%%%%%%%%%%%%%%%%%%%%%%%%%%%%%%%%%%%%%%%%%%%%%%%%%%%%%%%%%%%%%%%
%\subsection{Problem 3}
\problem{3}
\subsubsection{Question}
% Keywords
	\index{unsolved!Spring 2018 II.P3}
Consider a ``sliding bar'' generator made up of a U-shaped wire with width $w$ and having a resistance $R$ with the current closed by a movable bar with initial velocity $v_0$ as in the figure. A magnetic field $\mathbf{B} = B_0\hat{\boldsymbol{z}}$ comes out of the page.
\begin{enumerate}
	\item Find the emf generated and the current that flows through the circuit. Draw a diagram showing the direction of the current flow.
	\item Determine the magnetic force on the bar, and solve for its motion, assuming the bar has mass $M.$
	\item Show that the energy dissipated by the resistor is equal to the kinetic energy lost by the bar. 
\end{enumerate}
\subsubsection{Answer}



%%%%%%%%%%%%%%%%%%%%%%%%%%%%%%%%%%%%%%%%%%%%%%%%%%%%%%%%%%%%%%%%%%%%%%%%%%%%%%%
%%%% Problem 4
%%%%%%%%%%%%%%%%%%%%%%%%%%%%%%%%%%%%%%%%%%%%%%%%%%%%%%%%%%%%%%%%%%%%%%%%%%%%%%%
%\subsection{Problem 4}
\problem{4}
\subsubsection{Question}
% Keywords
	\index{unsolved!Spring 2018 II.P4}
The energy of a photon gas contained to a volume $V$ and in thermal equilibrium with a reservoir at temperature $\tau$ is given by a Stefan-Boltzmann law, $U = \alpha V\tau^4$.
\begin{enumerate}
	\item Find the heat capacity $C_V$ and the entropy $\sigma$.
	\item Find the free energy.
	\item Derive the ``photon gas law'', i.e., find $p(V,\tau)$.
	\item Calculate the work performed by the photon gas in isothermal expansion from $V_1=V$ to $V_2=2V$.
	\item Calculate the heat transferred to the photon gas when it expands at constant pressure from $V_1 =V$ to $V_2=V$.
\end{enumerate}
\subsubsection{Answer}


%%%%%%%%%%%%%%%%%%%%%%%%%%%%%%%%%%%%%%%%%%%%%%%%%%%%%%%%%%%%%%%%%%%%%%%%%%%%%%%
%%%% Problem 5
%%%%%%%%%%%%%%%%%%%%%%%%%%%%%%%%%%%%%%%%%%%%%%%%%%%%%%%%%%%%%%%%%%%%%%%%%%%%%%%
%\subsection{Problem 5}
\problem{5}
\subsubsection{Question}
% Keywords
	\index{unsolved!Spring 2018 II.P5}
A radioactive isotope of bismuth, ${}^{210}_{83}Bi$ undergoes beta-decay into polonium with a mean lifetime $\tau_1$ of $7.2$ days. In turn, the polonium alpha-decays into lead with mean lifetime $\tau_2$ of $200$ days. Denote the number of bismuth and polonium nuclei at time $t$ respectively by $N_1(t)$ and $N_2(t)$.
\begin{enumerate}
	\item Write out the nuclear reactions corresponding to both decays, carefully accounting for the atomic and mass numbers $Z$ and $A$ of the nuclei involved.
	\item The number of parent bismuth nuclei evolves with time according to the differential equation 
	\begin{equation*}
		\dv{N_1}{t} = -\lambda_1N_1
	\end{equation*}
	where $\lambda_1 = 1/\tau_1$ is the decay constant of bismuth. Write down the corresponding equation for the number of polonium nuclei $N_2(t)$ produced by bismuth decay, with $\lambda_2$ as the polonium decay constant.
	\item Solve these equations for $N_2(t)$, given the initial conditions $N_1(0)=N$ and $N_2(0)=0$. Hint: First solve for $N_1(t)$ and use this result together with a simple integrating factor in the equation for $\dv{N_2}{t}.$
	\item Since we start with no polonium at $t=0$, and that after a long enough time all the polonium produced will have decayed, there will be a time $t^*$ at which the number of polonium nuclei and correspondingly, the rate of $\alpha-$particle emission will reach a maximum. What is that time (in days)?
\end{enumerate}
\subsubsection{Answer}

