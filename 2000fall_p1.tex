%%%%%%%%%%%%%%%%%%%%%%%%%%%%%%%%%%%%%%%%%%%%%%%%%%%%%%%%%%%%%%%%%%%%%%%%%%%%%%%
%%%% Problem 1: Aditya
%%%%%%%%%%%%%%%%%%%%%%%%%%%%%%%%%%%%%%%%%%%%%%%%%%%%%%%%%%%%%%%%%%%%%%%%%%%%%%%
\problem{1}
\subsubsection{Question}
% Keywords
	\index{mechanics!Circular Motion}

A circle of rope of total mass $M$ and radius $R$ is spinning at angular velocity $\omega$ about an axis through the center of the circle. What is the tension $T$ in the rope?

\subsubsection{Answer}
An angular segment $\text{d}\phi$ of the spinning rope has mass $\dd m = \rho r \dd \phi$. Note that $\omega = \dot{\phi}$. The inward radial force $T\dd\phi$ on this angular segment obeys 
\begin{align*}
	T \dd\phi= \rho r a\dd\phi = \rho r \left(\omega r\right)\dd\phi = \rho \omega^2 r^2\dd\phi \implies \boxed{T = \rho \omega^2 r^2 }
\end{align*}
If the rope has a uniform density, then $M = \rho r$, thus $T = M\omega^2 r$. 

%%%%%%%%%%%%%%%%%%%%%%%%%%%%%%%%%%%%%%%%%%%%%%%%%%%%%%%%%%%%%%%%%%%%%%%%%%%%%%%
%%%% Problem 2: Aditya
%%%%%%%%%%%%%%%%%%%%%%%%%%%%%%%%%%%%%%%%%%%%%%%%%%%%%%%%%%%%%%%%%%%%%%%%%%%%%%%
\problem{2}
\subsubsection{Question}
% Keywords
    \index{electrostatics!Potential of the Sphere}
    \index{electrostatics!Method of Images}
A point charge $Q$ is placed at a distance $D$ from the center of an uncharged, solid 
metal sphere of radius $R$, thereby polarizing it. What is the potential $V$ of the sphere?

\subsubsection{Answer}
We may use the method of images to solve this problem. Assume that the point charge is placed a distance $D$ from the center of the sphere and is outside the sphere. The point charge causes negative charge to move to the side of the sphere closest to the point charge, thereby polarizing it. The potential is found by removing the sphere and placing a second point charge $-Q$ at a location $C$ from the center of the sphere such that the total potential $V_T=0$ at infinity and at $R$.

Given the above, the potential may be written as
\begin{equation*}
  	V_T(x,y,z) = \frac{1}{4\pi \epsilon_0}\qty(\frac{q}{r} + \frac{q^\prime}{r^\prime})
\end{equation*}
where $q,\ (q^\prime)$ is the real (image) charge and $r,\ (r^\prime)$ is the distance between the field point and the image charge. Choose an axis which contains both the field and the image charge. Since the potential is supposed to be zero on the surface of the sphere, $V(R,0,0)=0$. Solving for when this occurs will lead us to
\begin{equation*}
	\boxed{q^\prime = -\frac{R}{D}q}\hspace{.5in} \boxed{C = \frac{R^2}{D}}
\end{equation*}
This completes the problem.
%%%%%%%%%%%%%%%%%%%%%%%%%%%%%%%%%%%%%%%%%%%%%%%%%%%%%%%%%%%%%%%%%%%%%%%%%%%%%%%
%%%% Problem 8
%%%%%%%%%%%%%%%%%%%%%%%%%%%%%%%%%%%%%%%%%%%%%%%%%%%%%%%%%%%%%%%%%%%%%%%%%%%%%%%
\problem{8}
\subsubsection{Question}
% Keywords
	\index{thermodynamics!Mean Free Path of $N_2$ and Particle Velocity}
	\index{statistical mechanics!Mean Free Path of $N_2$ and Particle Velocity}

Estimate (a) the average speed (in \si{\m\per\s}) and (b) the mean free path
(in \si{\m}) of a nitrogen molecule in this room.

\subsubsection{Answer}

\begin{enumerate}[(a)]
	\item
		We relate the kinetic energy of an $N_2$ molecule with the thermal
		energy by the equipartition theorem. Since there are 3 translational
		degrees of freedom,
		\begin{align*}
			\frac 12 mv^2 &= \frac 32 k_B T \\
			v &= \sqrt{\frac{3 k_B T}{m}}
		\end{align*}
		The mass of the molecule is twice that of a single nitrogen atom
		which is itself about 14 proton masses. Therefore
		\begin{align}
			\boxed{
			v \approx \sqrt{\frac{3 k_B T}{28 m_p}} \approx \SI{515}{\m\per\s}
			}
		\end{align}
	\item
		Two particles collide if they come within $2r_0$ of each other where
		$r_0$ is the typical radius of the particle. For diatomic nitrogen,
		we assume $r_0 \approx 2a_0$ where $a_0$ is the Bohr radius. Then in the time
		$\tau$ that the particle is moving at velocity $\langle v \rangle$, the particle can
		collide with any other particle within the swept-out volume
		\begin{align*}
			\mathcal V &= \pi(2r_0)^2\cdot \langle v \rangle\tau
		\end{align*}
		Since there are $n$ particles per unit volume, there are $\mathcal{N}$
		atoms to collide with:
		\begin{align*}
			\mathcal{N} &= n\mathcal{V} = 4\pi n{r_0}^2 \langle v \rangle\tau
		\end{align*}
		On average then, there are $\mathcal{N}$ collisions per length $\langle v \rangle\tau$
		traversed, or in its reciprocal form, the mean free path $\lambda$ is
		\begin{align*}
			\lambda &= \frac{1}{4\pi n{r_0}^2}
		\end{align*}
		To estimate the particle density, consider the ideal gas law
		$PV=Nk_BT$. We can assume atmospheric pressure at room temperature,
		so the density is
		\begin{align*}
			n &= \frac N V = \frac{P}{k_B T}
		\end{align*}
		Putting it all together,
		\begin{align}
			\boxed{
			\lambda \approx \frac{k_B T}{4\pi{r_0}^2P} \approx \SI{2.90e-7}{\m}
			}
		\end{align}
\end{enumerate}
