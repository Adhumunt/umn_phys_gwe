%%%%%%%%%%%%%%%%%%%%%%%%%%%%%%%%%%%%%%%%%%%%%%%%%%%%%%%%%%%%%%%%%%%%%%%%%%%%%%%
%%%% Problem 2
%%%%%%%%%%%%%%%%%%%%%%%%%%%%%%%%%%%%%%%%%%%%%%%%%%%%%%%%%%%%%%%%%%%%%%%%%%%%%%%
\problem{2}
\subsubsection{Question}
% Keywords
	\index{quantum!Infinite square-well periodicity}

Show that a particle in a one-dimensional infinite square well initially in a
state $Ψ(x,0)$ will always return to that state after a time $T = 4ma²/π\hbar$
where $a$ is the width of the well.

\subsubsection{Answer}
Use the standard time independent Schödinger equation
\begin{align*}
	Ψ(x,t) &= ψ(x) e^{iEt/\hbar}
\end{align*}
with associated differential equation
\begin{align*}
	-\frac{\hbar²}{2m} \frac{d²ψ}{dx²} + V(x)ψ &= Eψ
\end{align*}

For an infinite square well, the potential has the form
\begin{align*}
	V(x) &=
		\begin{cases}
			0	&	|x| < \frac{a}{2} \\
			∞	&	\text{otherwise}
		\end{cases}
\end{align*}
so that the only region to consider is $-\frac{a}{2} < x < \frac{a}{2}$. In this
region, the differential equation takes the form of a harmonic oscillator
\begin{align*}
	\frac{d²ψ}{dx²} &= -\frac{2mE}{\hbar²}ψ
\end{align*}
leading to solutions
\begin{align*}
	ψ(x) ={}& A\cos kx + B\sin kx \\
	{}& \text{where } k² = \frac{2mE}{\hbar²}
\end{align*}

The boundary conditions $ψ(-\frac{a}{2}) = 0$ and $ψ(\frac{a}{2}) = 0$ impose
\begin{align*}
	ψ\left(-\frac{a}{2}\right) &= 0 = A\cos \frac{ka}{2} - B\sin \frac{ka}{2} \\
	ψ\left(\frac{a}{2}\right)  &= 0 = A\cos \frac{ka}{2} + B\sin \frac{ka}{2}
\end{align*}
so that $B = 0$ and
\begin{align*}
	0 &= 2A \cos \frac{ka}{2} \\
	\frac{(2n+1)π}{2} &= \frac{ka}{2} \\
	k &= \frac{(2n+1)π}{a}
\end{align*}

We already had a relation for $k$ defined, so substitute and solve for the
energies $E_n$.
\begin{align*}
	\frac{(2n+1)²π²}{a²} &= \frac{2mE}{\hbar²} \\
	E_n &= \frac{(2n+1)²π²\hbar²}{2ma²}
\end{align*}

Then considering $Ψ(x,t)$, the complex exponential is periodic in time with
period
\begin{align*}
	T_n &= \frac{2π\hbar}{E}
\end{align*}
where $n = 0$ will be the case with the longest periodicity, so
\begin{align*}
	T &= \frac{2π\hbar · 2ma²}{π²\hbar²} \\
	{} &= \frac{4ma²}{π\hbar}
\end{align*}

Therefore, the function is periodic in time with a periodicity
\begin{empheq}[box=\fbox]{align}
	T &= \frac{4ma²}{π\hbar}
\end{empheq}

%%%%%%%%%%%%%%%%%%%%%%%%%%%%%%%%%%%%%%%%%%%%%%%%%%%%%%%%%%%%%%%%%%%%%%%%%%%%%%%
%%%% Problem 4
%%%%%%%%%%%%%%%%%%%%%%%%%%%%%%%%%%%%%%%%%%%%%%%%%%%%%%%%%%%%%%%%%%%%%%%%%%%%%%%
\problem{4}
\subsubsection{Question}
% Keywords
	\index{particle!Compton scattering}
	\index{relativity!Compton scattering}

A photon collides with a stationary electron. If the photon scatters at an
angle $θ$, show that the resulting wavelength $λ'$ is given in terms of the
original wavelength $λ$ by
\begin{align*}
	λ' &= λ + \frac{h}{mc} (1 - \cos θ)
\end{align*}
where $m$ is the mass of the electron.

\subsubsection{Answer}
Start by considering conservation of momentum for the system. The initial values
are
\begin{align*}
	p_{γx} &= \frac{h}{λ}		&		p'_{γx} &= \frac{h}{λ'} \cos θ \\
	p_{γy} &= 0					&		p'_{γy} &= \frac{h}{λ'} \sin θ \\
	p_{ex} &= 0					&		p'_{ex} &= ? \\
	p_{ey} &= 0					&		p'_{ey} &= ?
\end{align*}
and considering each component in turn:
\begin{align*}
	\frac{h}{λ} + 0 &= \frac{h}{λ'}\cos θ + p'_{ex}
		& 0 &= \frac{h}{λ'}\sin θ + p'_{ey}
	\\
	p'_{ex} &= \frac{h}{λ} - \frac{h}{λ'}\cos θ
		& p'_{ey} &= -\frac{h}{λ'}\sin θ
\end{align*}
The total momentum of the electron is then
\begin{align}
	p²_e &= \left( \frac{h}{λ} - \frac{h}{λ'}\cos θ \right)² +
		\left( -\frac{h}{λ'}\sin θ \right)²
		\nonumber
	\\
	{} &= \frac{h²}{λ²} - \frac{2h²}{λλ'}\cos θ + \frac{h²}{λ'^2}\cos ²θ +
		\frac{h²}{λ'^2}\sin ²θ
		\nonumber
	\\
	p²_e &= h² \left( \frac{1}{λ²} + \frac{1}{λ'^2} \right) -
		\frac{2h²}{λλ'}\cos θ
\end{align}

Then consider energy conservation, with initial values
\begin{align*}
	E_γ &= \frac{hc}{λ}			&		E'_γ &= \frac{hc}{λ'} \\
	E_e &= mc²					&		E'_e &= \frac{{p'_e}²}{2m} + mc²
\end{align*}
leading to the equation
\begin{align*}
	\frac{hc}{λ} + mc² &= \frac{hc}{λ'} + \frac{{p'_e}²}{2m} + mc²
	\\
	\frac{hc}{λ} &= \frac{hc}{λ'} + \frac{h²}{2m} \left( \frac{1}{λ²} +
		\frac{1}{λ'^2} \right) - \frac{2h²}{2mλλ'}\cos θ
	\\
	\frac{hc}{λ} - \frac{hc}{λ'} &= \frac{h²}{2m} \left( \frac{1}{λ²} +
		\frac{1}{λ'^2} \right) - \frac{2h²}{2mλλ'}\cos θ
	\\
	\frac{λ' - λ}{λλ'} &= \frac{h}{2mc} \frac{λ'^2 + λ²}{λ²{λ'}^2} -
		\frac{h}{mcλλ'}\cos θ
	\\
	λ' - λ &= \frac{h}{2mc} \left( \frac{(λ' - λ)² + 2λλ'}{λλ'} \right) -
		\frac{h}{mc}\cos θ
	\\
	λ' - λ &= \frac{h}{2mc} \left( \frac{(λ' - λ)²}{λλ'} + 2 \right) -
		\frac{h}{mc}\cos θ
	\\
	λ' - λ &= \frac{h}{2mc} \frac{(λ' - λ)²}{λλ'} + \frac{h}{mc}(1 - \cos θ)
\end{align*}
The difference in the wavelengths is small, so
\begin{align*}
	\frac{(λ' - λ)²}{λλ'} &≈ 0
\end{align*}
leading to the final Compton scattering equation
\begin{empheq}[box=\fbox]{align}
	λ' &= λ + \frac{h}{mc}(1 - \cos θ)
\end{empheq}
