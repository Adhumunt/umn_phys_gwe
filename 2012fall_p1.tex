%%%%%%%%%%%%%%%%%%%%%%%%%%%%%%%%%%%%%%%%%%%%%%%%%%%%%%%%%%%%%%%%%%%%%%%%%%%%%%%
%%%% Problem 2
%%%%%%%%%%%%%%%%%%%%%%%%%%%%%%%%%%%%%%%%%%%%%%%%%%%%%%%%%%%%%%%%%%%%%%%%%%%%%%%
\problem{1}
\subsubsection{Question}
% Keywords
	\index{unsolved!Fall 2012 P1}

Assume you have three identical particles and three single particle states $|a\rangle$, $|b\rangle$, and $|c\rangle$ available for them. Count how many different three-particle states there can be if the particles are $(a)$ fermions, and $(b)$ bosons.

\subsubsection{Answer}
Fermionic statistics exclude two fermions living in the same energy state, thus there are $3!$ different particle states for fermions. Bosonic statistics allow an arbitrary number of bosons to live in the same energy, thus there are $3^3$ different particle states for bosons.

%%%%%%%%%%%%%%%%%%%%%%%%%%%%%%%%%%%%%%%%%%%%%%%%%%%%%%%%%%%%%%%%%%%%%%%%%%%%%%%
%%%% Problem 2
%%%%%%%%%%%%%%%%%%%%%%%%%%%%%%%%%%%%%%%%%%%%%%%%%%%%%%%%%%%%%%%%%%%%%%%%%%%%%%%
\problem{2}
\subsubsection{Question}
% Keywords
	\index{quantum!Infinite square-well periodicity}

Show that a particle in a one-dimensional infinite square well initially in a
state $\psi (x,0)$ will always return to that state after a time $T = 4ma^2/{\pi}{\hbar}$
where $a$ is the width of the well.

\subsubsection{Answer}
Use the standard time independent Sch\"odinger equation
\begin{align*}
	\psi (x,t) &= \psi (x) e^{iEt/{\hbar}}
\end{align*}
with associated differential equation
\begin{align*}
	-\frac{{\hbar}^2}{2m} \frac{d^2\psi }{dx^2} + V(x)\psi  &= E\psi 
\end{align*}

For an infinite square well, the potential has the form
\begin{align*}
	V(x) &=
		\begin{cases}
			0	&	|x| < \frac{a}{2} \\
			\infty 	&	\text{otherwise}
		\end{cases}
\end{align*}
so that the only region to consider is $-\frac{a}{2} < x < \frac{a}{2}$. In this
region, the differential equation takes the form of a harmonic oscillator
\begin{align*}
	\frac{d^2\psi }{dx^2} &= -\frac{2mE}{{\hbar}^2}\psi 
\end{align*}
leading to solutions
\begin{align*}
	\psi (x) ={}& A\cos kx + B\sin kx \\
	{}& \text{where } k^2 = \frac{2mE}{{\hbar}^2}
\end{align*}

The boundary conditions $\psi (-\frac{a}{2}) = 0$ and $\psi (\frac{a}{2}) = 0$ impose
\begin{align*}
	\psi \left(-\frac{a}{2}\right) &= 0 = A\cos \frac{ka}{2} - B\sin \frac{ka}{2} \\
	\psi \left(\frac{a}{2}\right)  &= 0 = A\cos \frac{ka}{2} + B\sin \frac{ka}{2}
\end{align*}
so that $B = 0$ and
\begin{align*}
	0 &= 2A \cos \frac{ka}{2} \\
	\frac{(2n+1){\pi}}{2} &= \frac{ka}{2} \\
	k &= \frac{(2n+1){\pi}}{a}
\end{align*}

We already had a relation for $k$ defined, so substitute and solve for the
energies $E_n$.
\begin{align*}
	\frac{(2n+1)^2{\pi}^2}{a^2} &= \frac{2mE}{{\hbar}^2} \\
	E_n &= \frac{(2n+1)^2{\pi}^2{\hbar}^2}{2ma^2}
\end{align*}

Then considering $\psi (x,t)$, the complex exponential is periodic in time with
period
\begin{align*}
	T_n &= \frac{2{\pi}{\hbar}}{E}
\end{align*}
where $n = 0$ will be the case with the longest periodicity, so
\begin{align*}
	T &= \frac{2{\pi}{\hbar}\cdot 2ma^2}{{\pi}^2{\hbar}^2} \\
	{} &= \frac{4ma^2}{{\pi}{\hbar}}
\end{align*}

Therefore, the function is periodic in time with a periodicity
\begin{align}
	\boxed{
	T = \frac{4ma^2}{{\pi}{\hbar}}
	}
\end{align}


%%%%%%%%%%%%%%%%%%%%%%%%%%%%%%%%%%%%%%%%%%%%%%%%%%%%%%%%%%%%%%%%%%%%%%%%%%%%%%%
%%%% Problem 3
%%%%%%%%%%%%%%%%%%%%%%%%%%%%%%%%%%%%%%%%%%%%%%%%%%%%%%%%%%%%%%%%%%%%%%%%%%%%%%%
%\subsection{Problem 3}
\problem{3}
\subsubsection{Question}
% Keywords
	\index{unsolved!Fall 2012 I.P3}
	\index{statistical mechanics!1D Ising Model}
	\index{thermodynamics!1D Ising Model}

Consider a system of $N$ independent classical molecules, each at a fixed position, with magnetic moment $\mathbf{\mu}$ in an external magnetic field $B$. Determine the partition function, and hence find the free energy and the magnetization at temperature $T$, when the molecules can only be oriented parallel or antiparallel to the external magnetic field.

\subsubsection{Answer}

%%%%%%%%%%%%%%%%%%%%%%%%%%%%%%%%%%%%%%%%%%%%%%%%%%%%%%%%%%%%%%%%%%%%%%%%%%%%%%%
%%%% Problem 4
%%%%%%%%%%%%%%%%%%%%%%%%%%%%%%%%%%%%%%%%%%%%%%%%%%%%%%%%%%%%%%%%%%%%%%%%%%%%%%%
\problem{4}
\subsubsection{Question}
% Keywords
	\index{particle!Compton scattering}
	\index{relativity!Compton scattering}

A photon collides with a stationary electron. If the photon scatters at an
angle $\theta $, show that the resulting wavelength ${\lambda}'$ is given in terms of the
original wavelength ${\lambda}$ by
\begin{align*}
	{\lambda}' &= {\lambda} + \frac{h}{mc} (1 - \cos \theta )
\end{align*}
where $m$ is the mass of the electron.

\subsubsection{Answer}
Start by considering conservation of momentum for the system. The initial values
are
\begin{align*}
	p_{\gamma x} &= \frac{h}{{\lambda}}		&		p'_{\gamma x} &= \frac{h}{{\lambda}'} \cos \theta  \\
	p_{\gamma y} &= 0					&		p'_{\gamma y} &= \frac{h}{{\lambda}'} \sin \theta  \\
	p_{ex} &= 0					&		p'_{ex} &= ? \\
	p_{ey} &= 0					&		p'_{ey} &= ?
\end{align*}
and considering each component in turn:
\begin{align*}
	\frac{h}{{\lambda}} + 0 &= \frac{h}{{\lambda}'}\cos \theta  + p'_{ex}
		& 0 &= \frac{h}{{\lambda}'}\sin \theta  + p'_{ey}
	\\
	p'_{ex} &= \frac{h}{{\lambda}} - \frac{h}{{\lambda}'}\cos \theta 
		& p'_{ey} &= -\frac{h}{{\lambda}'}\sin \theta 
\end{align*}
The total momentum of the electron is then
\begin{align}
	p^2_e &= \left( \frac{h}{{\lambda}} - \frac{h}{{\lambda}'}\cos \theta  \right)^2 +
		\left( -\frac{h}{{\lambda}'}\sin \theta  \right)^2
		\nonumber
	\\
	{} &= \frac{h^2}{{\lambda}^2} - \frac{2h^2}{{\lambda}{\lambda}'}\cos \theta  + \frac{h^2}{{\lambda}'^2}\cos ^2\theta  +
		\frac{h^2}{{\lambda}'^2}\sin ^2\theta 
		\nonumber
	\\
	p^2_e &= h^2 \left( \frac{1}{{\lambda}^2} + \frac{1}{{\lambda}'^2} \right) -
		\frac{2h^2}{{\lambda}{\lambda}'}\cos \theta 
\end{align}

Then consider energy conservation, with initial values
\begin{align*}
	E_\gamma  &= \frac{hc}{{\lambda}}			&		E'_\gamma  &= \frac{hc}{{\lambda}'} \\
	E_e &= mc^2					&		E'_e &= \frac{{p'_e}^2}{2m} + mc^2
\end{align*}
leading to the equation
\begin{align*}
	\frac{hc}{{\lambda}} + mc^2 &= \frac{hc}{{\lambda}'} + \frac{{p'_e}^2}{2m} + mc^2
	\\
	\frac{hc}{{\lambda}} &= \frac{hc}{{\lambda}'} + \frac{h^2}{2m} \left( \frac{1}{{\lambda}^2} +
		\frac{1}{{\lambda}'^2} \right) - \frac{2h^2}{2m{\lambda}{\lambda}'}\cos \theta 
	\\
	\frac{hc}{{\lambda}} - \frac{hc}{{\lambda}'} &= \frac{h^2}{2m} \left( \frac{1}{{\lambda}^2} +
		\frac{1}{{\lambda}'^2} \right) - \frac{2h^2}{2m{\lambda}{\lambda}'}\cos \theta 
	\\
	\frac{{\lambda}' - {\lambda}}{{\lambda}{\lambda}'} &= \frac{h}{2mc} \frac{{\lambda}'^2 + {\lambda}^2}{{\lambda}^2{{\lambda}'}^2} -
		\frac{h}{mc{\lambda}{\lambda}'}\cos \theta 
	\\
	{\lambda}' - {\lambda} &= \frac{h}{2mc} \left( \frac{({\lambda}' - {\lambda})^2 + 2{\lambda}{\lambda}'}{{\lambda}{\lambda}'} \right) -
		\frac{h}{mc}\cos \theta 
	\\
	{\lambda}' - {\lambda} &= \frac{h}{2mc} \left( \frac{({\lambda}' - {\lambda})^2}{{\lambda}{\lambda}'} + 2 \right) -
		\frac{h}{mc}\cos \theta 
	\\
	{\lambda}' - {\lambda} &= \frac{h}{2mc} \frac{({\lambda}' - {\lambda})^2}{{\lambda}{\lambda}'} + \frac{h}{mc}(1 - \cos \theta )
\end{align*}
The difference in the wavelengths is small, so
\begin{align*}
	\frac{({\lambda}' - {\lambda})^2}{{\lambda}{\lambda}'} &\approx 0
\end{align*}
leading to the final Compton scattering equation
\begin{align}
	\boxed{
	{\lambda}' = {\lambda} + \frac{h}{mc}(1 - \cos \theta )
	}
\end{align}

%%%%%%%%%%%%%%%%%%%%%%%%%%%%%%%%%%%%%%%%%%%%%%%%%%%%%%%%%%%%%%%%%%%%%%%%%%%%%%%
%%%% Problem 5
%%%%%%%%%%%%%%%%%%%%%%%%%%%%%%%%%%%%%%%%%%%%%%%%%%%%%%%%%%%%%%%%%%%%%%%%%%%%%%%
%\subsection{Problem 5}
\problem{5}
\subsubsection{Question}
% Keywords
	\index{unsolved!Fall 2012 I.P5}
	\index{optics!Index of Refraction}
The composition of a glass block varies as a function of the distance $x$ from the top surface. Thus the index of refraction $n(x)$ increases as a function of $x$ according to the relationship $n(x) = 1.50 - (0.20) /(x +1)^2 $, where $x$ is expressed in centimeters. A beam of light strikes the surface with an angle of incidence $\theta_i$ , measured from the vertical, as shown in the figure opposite. What will be the direction of the beam deep inside the block?
\subsubsection{Answer}


%%%%%%%%%%%%%%%%%%%%%%%%%%%%%%%%%%%%%%%%%%%%%%%%%%%%%%%%%%%%%%%%%%%%%%%%%%%%%%%
%%%% Problem 6
%%%%%%%%%%%%%%%%%%%%%%%%%%%%%%%%%%%%%%%%%%%%%%%%%%%%%%%%%%%%%%%%%%%%%%%%%%%%%%%
%\subsection{Problem 6}
\problem{6}
\subsubsection{Question}
% Keywords
	\index{unsolved!Fall 2012 I.P6}
You want to measure a current when using a voltmeter and a precision resistor. You measure a voltage of $0.85$ V across a $10,000\Omega$ resistor and $-0.05$ V when the input to the voltmeter is short-circuited. The precision of the voltmeter is $0.001$ V and the resistor is rated at $0.1\%$. What is the value of the current and the precision of the measurement?
\subsubsection{Answer}


%%%%%%%%%%%%%%%%%%%%%%%%%%%%%%%%%%%%%%%%%%%%%%%%%%%%%%%%%%%%%%%%%%%%%%%%%%%%%%%
%%%% Problem 7
%%%%%%%%%%%%%%%%%%%%%%%%%%%%%%%%%%%%%%%%%%%%%%%%%%%%%%%%%%%%%%%%%%%%%%%%%%%%%%%
%\subsection{Problem 7}
\problem{7}
\subsubsection{Question}
% Keywords
	\index{unsolved!Fall 2012 I.P7}

Find the capacitance per unit length of two coaxial conducting cylindrical tubes, of radii $a$ and $b$.

\subsubsection{Answer}


%%%%%%%%%%%%%%%%%%%%%%%%%%%%%%%%%%%%%%%%%%%%%%%%%%%%%%%%%%%%%%%%%%%%%%%%%%%%%%%
%%%% Problem 8
%%%%%%%%%%%%%%%%%%%%%%%%%%%%%%%%%%%%%%%%%%%%%%%%%%%%%%%%%%%%%%%%%%%%%%%%%%%%%%%
%\subsection{Problem 8}
\problem{8}
\subsubsection{Question}
% Keywords
	\index{unsolved!Fall 2012 I.P8}
Suppose that the radius of the Earth were to gravitationally collapse uniformly by one percent, with its mass remaining the same. What would happen to the Earth’s kinetic energy of rotation? If it changes, how does it change and by how much? Assume that the Earth is a uniform sphere.
\subsubsection{Answer}


%%%%%%%%%%%%%%%%%%%%%%%%%%%%%%%%%%%%%%%%%%%%%%%%%%%%%%%%%%%%%%%%%%%%%%%%%%%%%%%
%%%% Problem 9
%%%%%%%%%%%%%%%%%%%%%%%%%%%%%%%%%%%%%%%%%%%%%%%%%%%%%%%%%%%%%%%%%%%%%%%%%%%%%%%
%\subsection{Problem 9}
\problem{9}
\subsubsection{Question}
% Keywords
	\index{unsolved!Fall 2012 I.P9}
A neutron star has a radius of $10$ km, a mass of $3.0\times10^{30}$ kg. Find the nearest distance to the surface that a person $2$ m tall could approach the pulsar without being pulled apart. Assume a uniform mass distribution, feet toward the pulsar and that a person starts to come apart when the force that each half of the body exerts on the other exceeds ten times the body weight on Earth. What is the period of revolution in a circular orbit about the pulsar at this distance?
\subsubsection{Answer}


%%%%%%%%%%%%%%%%%%%%%%%%%%%%%%%%%%%%%%%%%%%%%%%%%%%%%%%%%%%%%%%%%%%%%%%%%%%%%%%
%%%% Problem 10
%%%%%%%%%%%%%%%%%%%%%%%%%%%%%%%%%%%%%%%%%%%%%%%%%%%%%%%%%%%%%%%%%%%%%%%%%%%%%%%
%\subsection{Problem 10}
\problem{10}
\subsubsection{Question}
% Keywords
	\index{unsolved!Fall 2012 I.P10}
Given that the latent heat of fusion for water is $334$ kJ/kg and the density of water and ice are $1000$ kg/m$^3$ and $917.0$ kg/m$^3$, respectively. At what pressure will ice melt at $-0.1^{\circ}$ C? It may be useful to remember that the Gibbs free energy is the same across a phase boundary.
\subsubsection{Answer}



