%%%%%%%%%%%%%%%%%%%%%%%%%%%%%%%%%%%%%%%%%%%%%%%%%%%%%%%%%%%%%%%%%%%%%%%%%%%%%%%
%%%% Problem 1
%%%%%%%%%%%%%%%%%%%%%%%%%%%%%%%%%%%%%%%%%%%%%%%%%%%%%%%%%%%%%%%%%%%%%%%%%%%%%%%
\subsection{Problem 1}
\subsubsection{Question}
% Keywords
	\index{mechanics!Pendulum in Elevator}
	\index{pendulum!Pendulum in Elevator}

An elevator operator in a skyscraper, being a very meticulous person, put a
pendulum clock on the wall of the elevator to make sure that he spends exactly 8
hours a day at his work place. Over the course of his work day, he records that
the time during which the elevator has acceleration $a$ is exactly equal to the
time during which it has acceleration $-a$. Does the elevator operator work, in
actual time, (1) more than 8 hours, (2) exactly 8 hours, or (3) less than 8
hours? Why?

\subsubsection{Answer}
The nominal period of a pendulum is
\begin{align*}
	T_{nom} &= 2π \sqrt\frac{ℓ}{g}
\end{align*}
but within the elevator, the acceleration $g$ is not going to be constant and
will rather depend on the acceleration of the elevator. Therefore,
\begin{align*}
	T_↑ &= 2π\sqrt\frac{ℓ}{g+a} & T_↓ &= 2π\sqrt\frac{ℓ}{g-a}
\end{align*}
for the upward and downward cases, respectively.

Since the elevator operator observed that equal time was spent going up as was
spent going down, so he must have observed $N$ oscillations in both cases. In
order to compare to the actual time, we simply compare the elevator's total time
measurement with that of a stationary clock.
\begin{align*}
	NT_↑ + NT_↓ \stackrel{?}{=} 2NT_{nom}
\end{align*}

\begin{align*}
	2πN\sqrt\frac{ℓ}{g+a} + 2πN\sqrt\frac{ℓ}{g-a}
		&\stackrel{?}{=} 4πN\sqrt\frac{ℓ}{g}
		\\
	\sqrt\frac{1}{g+a} + \sqrt\frac{1}{g-a} &\stackrel{?}{=} 2\sqrt\frac{1}{g}\\
	\sqrt\frac{g}{g+a} + \sqrt\frac{g}{g-a} &\stackrel{?}{=} 2 \\
\end{align*}
Use the test value $a=5$ for comparison (with $g = 10$)
\begin{empheq}[box=\fbox]{align}
	2.23 > 2
\end{empheq}
Therefore the elevator operator actually spends more than 8 hours in the
elevator during his shift.

%%%%%%%%%%%%%%%%%%%%%%%%%%%%%%%%%%%%%%%%%%%%%%%%%%%%%%%%%%%%%%%%%%%%%%%%%%%%%%%
%%%% Problem 2
%%%%%%%%%%%%%%%%%%%%%%%%%%%%%%%%%%%%%%%%%%%%%%%%%%%%%%%%%%%%%%%%%%%%%%%%%%%%%%%
\clearpage
\subsection{Problem 2}
\subsubsection{Question}
% Keywords
	\index{mechanics!Central Forces}
	\index{Lagrangian!Central Forces}
	\index{orbits!Central Forces}
	
A classical particle is subject to an attractive centra force proportional to
$r^α$, where $r$ is the radius and $α$ is a constant. Show by perturbation
analysis what is required of $α$ in order for the particle to have a stable
circular orbit.

\subsubsection{Answer}
Construct the Lagrangian for the system in order to determine the equations of
motion for the given central force (noting that we were given the \emph{force}
so we need to make an appropriate potential).
\begin{align*}
	T &= \frac{1}{2}m ( \dot r² + r²\dot θ² )
		& V &= \frac{k}{α+1}r^{α+1}
\end{align*}
\begin{align*}
	\sL &= \frac{1}{2}m\dot r² + \frac{1}{2}mr²\dot θ² - \frac{k}{α+1}r^{α+1}
\end{align*}
Conservation of angular momentum is a consequence of the $θ$ and $\dot θ$
coordinates:
\begin{align*}
	0 &= \frac{∂\sL}{∂θ} - \frac{d}{dt} \left[ \frac{∂\sL}{∂\dot θ} \right] \\
	0 &= \frac{d}{dt} \left[ mr²\dot θ \right] \\
\intertext{Nothing that}
	ℓ &= \left|\frac{\vec r × \vec p}{m}\right| = r²\dot θ
\intertext{we can say that}
	\dot θ &= \frac{ℓ}{r²}
\end{align*}

Then returning to the $r$ and $\dot r$ coordinates in the Lagrangian,
\begin{align*}
	\frac{∂\sL}{∂r} &= mr\dot θ² - kr^α &
		\frac{∂\sL}{∂\dot r} &= m\dot r
	\\
	{}&{}&
	\frac{d}{dt}\left[ \frac{∂\sL}{∂\dot φ} \right]
		&= m \ddot r
\end{align*}
Putting the differential equation together and substituting for the angular
momentum per unit mass gives
\begin{align}
	m\ddot r &= \frac{mℓ²}{r³} - kr^α
\end{align}

In the case that the orbit is circular, $r$ must be a constant, so let $r = a$
and note that $\ddot r = 0$ necessarily.
\begin{align*}
	\frac{mℓ²}{a³} &= ka^α
\end{align*}

Returning to the differential equation, let the actual distance $r$ be a
perturbation from a circular orbit, and Taylor expand in $x$ where $x = r - a$.
\begin{align*}
	m\ddot x &= \frac{mℓ²}{a³} (1 + \frac{x}{a} )^{-3} -
		ka^α \left(1 + \frac{x}{a} \right)^α \\
	m\ddot x &≈ \frac{mℓ²}{a³} (1 - 3\frac{x}{a} + \ldots ) -
		ka^α (1 + α\frac{x}{a} + \ldots ) \\
	m\ddot x &≈ ka^α (1 - 3\frac{x}{a} ) -
		ka^α (1 + α\frac{x}{a} ) \\
	m\ddot x &≈ -3ka^α \frac{x}{a} - αka^α\frac{x}{a} \\
	m\ddot x &≈ -ka^{α-1} (3+α)x \\	
\end{align*}

To form a stable orbit, the coefficient on $x$ must be negative, giving a simple
harmonic solution. Therefore $3+α > 0$ to keep the coefficient negative and
\begin{empheq}[box=\fbox]{align}
	a > -3
\end{empheq}

%%%%%%%%%%%%%%%%%%%%%%%%%%%%%%%%%%%%%%%%%%%%%%%%%%%%%%%%%%%%%%%%%%%%%%%%%%%%%%%
%%%% Problem 3
%%%%%%%%%%%%%%%%%%%%%%%%%%%%%%%%%%%%%%%%%%%%%%%%%%%%%%%%%%%%%%%%%%%%%%%%%%%%%%%
\clearpage
\subsection{Problem 3}
\subsubsection{Question}
% Keywords
	\index{electrostatics!Charges in Conductor Cavities}
	\index{Gauss' Law!Charges in Conductor Cavities}

A neutral conductor A with a spherical outer surface of radius $R$ contains
three cavities B, C, and D, but is solid otherwise. B and C are spherical, and
D is hemispherical. Without touching A, positive charges $q_B$ and $q_C$ are
introduced at the centers of B and C, respectively.
\begin{enumerate}
	\item
		Give the amount and the distribution of the induced charges on the
		surfaces of A, B, C, and D.
	\item
		Now another positive charge $q_E$ is introduced at a distance $r > R$
		from the center of A. Describe qualitatively the distribution of
		induced charges on the surfaces of A, B, C, and D.
	\item
		Give the amount of the induced charges on the surfaces of A, B, C, and
		D for the situation in (2).
\end{enumerate}

\subsubsection{Answer (1)}
An ideal conductor will not support an electric field inside the solid, so each
of cavities B and C will have a surface charge to cancel the electric fields
emminating from $q_B$ and $q_C$ respectively.
\begin{itemize}
	\item
		Cavity B will have a uniform surface charge density of $-q_B/4πr²_B$,
		where $r_B$ is the radius of cavity B, with total induced charge $-q_B$
		(because of symmetry and use of a Gaussian surface).
	\item
		Cavity C will have a uniform surface charge density of $-q_C/4πr²_C$,
		where $r_C$ is the radius of cavity B, with total induced charge $-q_C$
		(because of symmetry and use of a Gaussian surface).
\end{itemize}
Cavity D will not have a surface charge since a Guassian surface coincident with
its boundary contains no charge.

The surface A will have total charge $q_B + q_C$ with uniform surface charge
density of $(q_B + q_C) / 4πR²$ in accordance with the symmetry of a Gaussian
surface containing the sphere as well as properties of an ideal conductor.

\subsubsection{Answer (2)}
The surfaces B, C, and D will remain unaffected since the surrounding conductor
shields the cavities from electric fields produced by charge $q_E$. The
distribution on surface A will shift so that the negative charge concentration
is greatest on the side nearest to $q_E$ with an increasingly positive
distribution towards the opposite side.

\subsubsection{Answer (3)}
The surface of A will still contain the same total charge $q_B + q_C$ since only
a redistribution of induced charges occurred along the surface. Similarly,
because surface B, C, and D are shielded from the electric field of $q_E$ by
conductor A, the total charges along their surfaces remains unchanged as well.

%%%%%%%%%%%%%%%%%%%%%%%%%%%%%%%%%%%%%%%%%%%%%%%%%%%%%%%%%%%%%%%%%%%%%%%%%%%%%%%
%%%% Problem 4
%%%%%%%%%%%%%%%%%%%%%%%%%%%%%%%%%%%%%%%%%%%%%%%%%%%%%%%%%%%%%%%%%%%%%%%%%%%%%%%
\clearpage
\subsection{Problem 4}
\subsubsection{Question}
% Keywords
	\index{electrostatics!Dielectric Breakdown of Air}

The dielectric strength of air at standard temperature and pressure is
\SI{3e6}{\V\per\m}. What is the maximum intensity in units of \si{\W\per\m^2}
for a monocromatic laser that can be used in the laboratory?

\subsubsection{Answer}
Failure of a dielectric occurs when the energy density in the dielectric is
great enough to overcome the ionization energy of the constituent atoms. This
suggests that an electric field of greater than \SI{3e6}{\V\per\m} would cause
this ionization to occur.

Starting here, We can calculate the energy density of the electric field at any
point in space by
\begin{align*}
	U_{em} = \frac{ε₀}{2}E²
\end{align*}
(where we've used the vacuum energy density since air differs very little from
the vacuum permittivity).

Then the power transmitted by the laser is $P = cU_{em}$, so plugging in the
numbers,
\begin{align*}
	P &= \frac{1}{2}
		\left( \SI[per-mode=fraction]{8.854e-12}
			{\coulomb\squared\per\N\per\m\squared} \right)
		\left( \SI{3e8}{\m\per\s} \vphantom{\frac{V}{V}} \right)
		\left( \SI{3e6}{\V\per\m} \right)² \\
	P &= \SI{1.19e10}{\W\per\m\squared}
\end{align*}

\begin{center}
	\fbox{The maximum power of a laser usable in the lab is \SI{1.19e10}
	{\W\per\m\squared}.}
\end{center}

%%%%%%%%%%%%%%%%%%%%%%%%%%%%%%%%%%%%%%%%%%%%%%%%%%%%%%%%%%%%%%%%%%%%%%%%%%%%%%%
%%%% Problem 5
%%%%%%%%%%%%%%%%%%%%%%%%%%%%%%%%%%%%%%%%%%%%%%%%%%%%%%%%%%%%%%%%%%%%%%%%%%%%%%%
\clearpage
\subsection{Problem 5}
\subsubsection{Question}
% Keywords
	\index{particle!Proton Collision}
	\index{relativity!Proton Collision}

What is the minimum energy of the projectile proton required to induce the
reaction $p + p \rightarrow p + p + p + \bar p$ if the target proton is at rest?

\subsubsection{Answer}
Energy and momentum must be conserved. At the minimum allowed energy, the
resultant 4 proton/anti-protons will be colinear with no relative momentum with
respect to one another, so the momenum equation in the lab frame is simply
\begin{align}
	p_i = 4p_f
\end{align}
Similarly, the resultant (anti-)protons are indistinguishable, so they will
all have equivalent energy $E_f$. The initial protons have different energies
since one is at rest in the lab frame while the other is moving, leading to
the energy equation
\begin{align*}
	\sqrt{p²_i c² + m²_p c⁴} + m_p c² &= 4\sqrt{p²_f c² + m²_p c⁴}
\end{align*}
Substituting the momentum relation into the equation, squaring, and simplifying,
\begin{align*}
	\sqrt{16p²_f c² + m²_p c⁴} + m_p c² &= 4\sqrt{p²_f c² + m²_p c⁴} \\
	16p²_f c² + m²_p c⁴ + m²_p c⁴ + 2\sqrt{m²_p c⁴(16p²_f c² + m²_p c⁴)}
		&= 16p²_f c² + 16m²_p c⁴ \\
	2\sqrt{m²_p c⁴(16p²_f c² + m²_p c⁴)} &= 14m²_p c⁴ \\
	16p²_f c² + m²_p c⁴ &= 49m²_p c⁴ \\
	p²_f &= 3m²_p c⁴
\end{align*}
Therefore,
\begin{align*}
	p²_i &= 48m²_p c⁴ \\
\intertext{and}
	E₁ &= \sqrt{49m²_p c⁴}
\end{align*}
\begin{empheq}[box=\fbox]{align}
	E₁ &≈ \SI{6.567}{\GeV\per c\squared}
\end{empheq}

%%%%%%%%%%%%%%%%%%%%%%%%%%%%%%%%%%%%%%%%%%%%%%%%%%%%%%%%%%%%%%%%%%%%%%%%%%%%%%%
%%%% Problem 6
%%%%%%%%%%%%%%%%%%%%%%%%%%%%%%%%%%%%%%%%%%%%%%%%%%%%%%%%%%%%%%%%%%%%%%%%%%%%%%%
\clearpage
\subsection{Problem 6}
\subsubsection{Question}
% Keywords
	\index{particle!Particle Decay}

A subatomic particle has spin 1 and negative parity. It decays at rest into an
$e^+ e^-$ pair, which is produced in the $s$ and $d$ waves. From these data
determine (1) the total spin of the $e^+ e^-$ pair and (2) the intrinsic parity
of $e^+$ relative to $e^-$.

\subsubsection{Answer}

\begingroup\color{red}

Let $p$ denote the original particle which decayed.

The $s$ state indicates an angular momentum $ℓ = 0$ while $d$ indicates $ℓ = 2$,
so the relative angular momentum between the $e^+$ and $e^-$ is $Δℓ = 2$.
To conserve total angular momentum (including spin) which was originally 1,
both electron and positron must be in the spin down state.

\begin{empheq}[box=\fbox]{align}
	\begin{split}
		(1)_p = (2)_{Δℓ} + (-\tfrac{1}{2})_{e^+} + (-\tfrac{1}{2})_{e^-}
	\end{split}
\end{empheq}


Parity is a multiplicative quantum number, so because the original particle
has negative parity, the electron and positron must have opposite relative
parities (the electron is odd with respect to the positron and vice versa).
\begin{empheq}[box=\fbox]{align}
	\begin{split}
		(-1)_p ={}& (-1)_{e^+} ⋅ (+1)_{e^-} \\
		{}&{}\text{or}\\
		(-1)_p ={}& (+1)_{e^+} ⋅ (-1)_{e^-}
	\end{split}
\end{empheq}

\endgroup

%%%%%%%%%%%%%%%%%%%%%%%%%%%%%%%%%%%%%%%%%%%%%%%%%%%%%%%%%%%%%%%%%%%%%%%%%%%%%%%
%%%% Problem 7
%%%%%%%%%%%%%%%%%%%%%%%%%%%%%%%%%%%%%%%%%%%%%%%%%%%%%%%%%%%%%%%%%%%%%%%%%%%%%%%
\clearpage
\subsection{Problem 7}
\subsubsection{Question}
% Keywords
	\index{quantum!Particle in Gravity with Elastic Boundary}

There is a uniform, vertical gravitation field with a downward acceleration of
gravity $g$ above a horizontal, perfectly elastic surface. A particle of mass
$m$ can only move above the surface. Give a rough estimate of the energy
eigenvalue for the ground state of the particle.

%\subsubsection{Answer}
%Let $x$ denote the height of the particle above the elastic surface, and assume
%that the elastic surface is spring-like with a potential function of $V_e(x) =
%-kx$.

%The potential energy function of the particle is a combination of the constant
%gravitational field and the piecewise potential barrier imposed by the elastic
%surface.
%\begin{align*}
%	V(x) &= \begin{cases}
%			mgx & x > 0 \\
%			mgx - kx & x < 0
%		\end{cases}
%\end{align*}

%%%%%%%%%%%%%%%%%%%%%%%%%%%%%%%%%%%%%%%%%%%%%%%%%%%%%%%%%%%%%%%%%%%%%%%%%%%%%%%
%%%% Problem 8
%%%%%%%%%%%%%%%%%%%%%%%%%%%%%%%%%%%%%%%%%%%%%%%%%%%%%%%%%%%%%%%%%%%%%%%%%%%%%%%
\clearpage
\subsection{Problem 8}
\subsubsection{Question}
% Keywords
	\index{thermodynamics!Atmospheric Scale Height (Pressure)}

Assume that the atmosphere near the earth's surface is in approximate
hydrostatic equilibrium, where any movement of air parcels is gentle and
adiabatic. Find an expression for the pressure $P$ of the atmosphere as a
function of the height $z$.

\subsubsection{Answer}
Note that the pressure at a given point is due to the mass of air above the
given point. Then by moving an infinitesimal distance vertically, the total
mass is changed by the density of the air (which is affected by the
gravitational force). This leads to the differential equation
\begin{align*}
	\frac{dP}{dz} &= ρg
\end{align*}
Then using the ideal gas equation
\begin{align*}
	PV &= N k_B T \\
\intertext{multiply and divide by the average molecular mass $m$ of the air (in
\si{\kg}) which combined with the number of molecules $N$ gives the total mass}
	PV &= (Nm) \frac{1}{m} k_B T \\
\intertext{and then divide by the volume to get the ideal gas equation in terms
of the mass density}
	P &= \frac{Nm}{V} \frac{1}{m} k_B T \\
	P &= ρ \frac{k_B T}{m} \\
	ρ &= \frac{P m}{k_B T}
\end{align*}
Finally, substitute this into the differential equation above and solve to 
get the atmospheric scale height equation.
\begin{align*}
	\frac{dP}{dz} ={}& \frac{P m}{k_B T}g \\
	\frac{dP}{P} ={}& \frac{mg}{k_B T} dz
\end{align*}
\begin{empheq}[box=\fbox]{align}
	\begin{split}
		P(z) ={}& P₀ e^{z/ξ} \\
			{}&\text{where }ξ = \frac{k_B T}{mg}
	\end{split}
\end{empheq}


%%%%%%%%%%%%%%%%%%%%%%%%%%%%%%%%%%%%%%%%%%%%%%%%%%%%%%%%%%%%%%%%%%%%%%%%%%%%%%%
%%%% Problem 9
%%%%%%%%%%%%%%%%%%%%%%%%%%%%%%%%%%%%%%%%%%%%%%%%%%%%%%%%%%%%%%%%%%%%%%%%%%%%%%%
\clearpage
\subsection{Problem 9}
\subsubsection{Question}
% Keywords
	\index{thermodynamics!Spin Statistics}

Consider a gas of atoms in a magnetic field of \SI{10}{\tesla}. The nucleus of
the atom has spin $\frac{1}{2}$, magnetic moment $μ ≈ \SI{e-26}{\J\per\tesla}$,
and mass $m ≈ \SI{5e-27}{\kg}$. The electrons in the atom of zero total angular
momentum. What is the maximum number density of the gas for which the nuclei
are completely polarized by the magnetic field at zero temperature?

\subsubsection{Answer}
Since the atoms have $\frac{1}{2}$ nuclear spin and zero total angular momentum
due to the electrons, the atoms are Fermi particles and must obey Fermi-Dirac
statistics.
