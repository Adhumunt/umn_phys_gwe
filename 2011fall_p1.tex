%%%%%%%%%%%%%%%%%%%%%%%%%%%%%%%%%%%%%%%%%%%%%%%%%%%%%%%%%%%%%%%%%%%%%%%%%%%%%%%
%%%% Problem 1
%%%%%%%%%%%%%%%%%%%%%%%%%%%%%%%%%%%%%%%%%%%%%%%%%%%%%%%%%%%%%%%%%%%%%%%%%%%%%%%
%\subsection{Problem 1}
\problem{1}
\subsubsection{Question}
% Keywords
	\index{mechanics!Pendulum in Elevator}
	\index{pendulum!Pendulum in Elevator}

An elevator operator in a skyscraper, being a very meticulous person, put a
pendulum clock on the wall of the elevator to make sure that he spends exactly 8
hours a day at his work place. Over the course of his work day, he records that
the time during which the elevator has acceleration $a$ is exactly equal to the
time during which it has acceleration $-a$. Does the elevator operator work, in
actual time, (1) more than 8 hours, (2) exactly 8 hours, or (3) less than 8
hours? Why?

\subsubsection{Answer}
The nominal period of a pendulum is
\begin{align*}
	T_{nom} &= 2{\pi} \sqrt\frac{\ell }{g}
\end{align*}
but within the elevator, the acceleration $g$ is not going to be constant and
will rather depend on the acceleration of the elevator. Therefore,
\begin{align*}
	T_\uparrow &= 2{\pi}\sqrt\frac{\ell }{g+a} & T_\downarrow &= 2{\pi}\sqrt\frac{\ell }{g-a}
\end{align*}
for the upward and downward cases, respectively.

Since the elevator operator observed that equal time was spent going up as was
spent going down, so he must have observed $N$ oscillations in both cases. In
order to compare to the actual time, we simply compare the elevator's total time
measurement with that of a stationary clock.
\begin{align*}
	NT_\uparrow + NT_\downarrow \stackrel{?}{=} 2NT_{nom}
\end{align*}

\begin{align*}
	2{\pi} N\sqrt\frac{\ell }{g+a} + 2{\pi} N\sqrt\frac{\ell }{g-a}
		&\stackrel{?}{=} 4{\pi} N\sqrt\frac{\ell }{g}
		\\
	\sqrt\frac{1}{g+a} + \sqrt\frac{1}{g-a} &\stackrel{?}{=} 2\sqrt\frac{1}{g}\\
	\sqrt\frac{g}{g+a} + \sqrt\frac{g}{g-a} &\stackrel{?}{=} 2 \\
\end{align*}
Use the test value $a=5$ for comparison (with $g = 10$)
\begin{align}
	\boxed{
	2.23 > 2
	}
\end{align}
Therefore the elevator operator actually spends more than 8 hours in the
elevator during his shift.

%%%%%%%%%%%%%%%%%%%%%%%%%%%%%%%%%%%%%%%%%%%%%%%%%%%%%%%%%%%%%%%%%%%%%%%%%%%%%%%
%%%% Problem 2
%%%%%%%%%%%%%%%%%%%%%%%%%%%%%%%%%%%%%%%%%%%%%%%%%%%%%%%%%%%%%%%%%%%%%%%%%%%%%%%
\problem{2}
\subsubsection{Question}
% Keywords
	\index{mechanics!Central Forces}
	\index{Lagrangian!Central Forces}
	\index{orbits!Central Forces}

A classical particle is subject to an attractive central force proportional to
$r^{\alpha}$, where $r$ is the radius and ${\alpha}$ is a constant. Show by perturbation
analysis what is required of ${\alpha}$ in order for the particle to have a stable
circular orbit.

\subsubsection{Answer}
Construct the Lagrangian for the system in order to determine the equations of
motion for the given central force (noting that we were given the \emph{force}
so we need to make an appropriate potential).
\begin{align*}
	T &= \frac{1}{2}m ( \dot r^2 + r^2\dot \theta ^2 )
		& V &= \frac{k}{{\alpha}+1}r^{{\alpha}+1}
\end{align*}
\begin{align*}
	\mathcal{L} &= \frac{1}{2}m\dot r^2 + \frac{1}{2}mr^2\dot \theta ^2 - \frac{k}{{\alpha}+1}r^{{\alpha}+1}
\end{align*}
Conservation of angular momentum is a consequence of the $\theta $ and $\dot \theta $
coordinates:
\begin{align*}
	0 &= \frac{\partial \mathcal{L}}{\partial \theta } - \frac{d}{dt} \left[ \frac{\partial \mathcal{L}}{\partial \dot \theta } \right] \\
	0 &= \frac{d}{dt} \left[ mr^2\dot \theta  \right] \\
\intertext{Noting that}
	L  &= \left|{\vec r \times  \vec p}\right| = mr^2\dot \theta 
\intertext{we can say that}
	\dot \theta  &= \frac{L}{m r^2}
\end{align*}

Then returning to the $r$ and $\dot r$ coordinates in the Lagrangian,
\begin{align*}
	\frac{\partial \mathcal{L}}{\partial r} &= mr\dot \theta ^2 - kr^{\alpha} &
		\frac{\partial \mathcal{L}}{\partial \dot r} &= m\dot r
	\\
	{}&{}&
	\frac{d}{dt}\left[ \frac{\partial \mathcal{L}}{\partial \dot \phi } \right]
		&= m \ddot r
\end{align*}
Putting the differential equation together and substituting for the angular
momentum gives
\begin{align}
	m\ddot r &= \frac{L ^2}{mr^3 } - kr^{\alpha}
\end{align}

In the case that the orbit is circular, $r$ must be a constant, so let $r = a$
and note that $\ddot r = 0$ necessarily.
\begin{align*}
	\frac{L^2}{ma^3 } &= ka^{\alpha}
\end{align*}
For simplicity, we introduce $\ell = L/m$ for the following calculation.

Returning to the differential equation, let the actual distance $r$ be a
perturbation from a circular orbit, and Taylor expand in $x$ where $x = r - a$.
\begin{align*}
	m\ddot x &= \frac{m\ell ^2}{a^3 } (1 + \frac{x}{a} )^{-3} -
		ka^{\alpha} \left(1 + \frac{x}{a} \right)^{\alpha} \\
	m\ddot x &\approx \frac{m\ell ^2}{a^3 } (1 - 3\frac{x}{a} + \ldots ) -
		ka^{\alpha} (1 + {\alpha}\frac{x}{a} + \ldots ) \\
	m\ddot x &\approx ka^{\alpha} (1 - 3\frac{x}{a} ) -
		ka^{\alpha} (1 + {\alpha}\frac{x}{a} ) \\
	m\ddot x &\approx -3ka^{\alpha} \frac{x}{a} - {\alpha}ka^{\alpha}\frac{x}{a} \\
	m\ddot x &\approx -ka^{{\alpha}-1} (3+{\alpha})x 
\end{align*}
To form a stable orbit, the coefficient on $x$ must be negative, giving a simple
harmonic solution. Therefore $3+{\alpha} > 0$ to keep the coefficient negative and
\begin{align}
	\boxed{
	a > -3
	}
\end{align}

%%%%%%%%%%%%%%%%%%%%%%%%%%%%%%%%%%%%%%%%%%%%%%%%%%%%%%%%%%%%%%%%%%%%%%%%%%%%%%%
%%%% Problem 3
%%%%%%%%%%%%%%%%%%%%%%%%%%%%%%%%%%%%%%%%%%%%%%%%%%%%%%%%%%%%%%%%%%%%%%%%%%%%%%%
\problem{3}
\subsubsection{Question}
% Keywords
	\index{electrostatics!Charges in Conductor Cavities}
	\index{Gauss' Law!Charges in Conductor Cavities}

A neutral conductor A with a spherical outer surface of radius $R$ contains
three cavities B, C, and D, but is solid otherwise. B and C are spherical, and
D is hemispherical. Without touching A, positive charges $q_B$ and $q_C$ are
introduced at the centers of B and C, respectively.
\begin{enumerate}
	\item
		Give the amount and the distribution of the induced charges on the
		surfaces of A, B, C, and D.
	\item
		Now another positive charge $q_E$ is introduced at a distance $r > R$
		from the center of A. Describe qualitatively the distribution of
		induced charges on the surfaces of A, B, C, and D.
	\item
		Give the amount of the induced charges on the surfaces of A, B, C, and
		D for the situation in (2).
\end{enumerate}

\subsubsection{Answer (1)}
An ideal conductor will not support an electric field inside the solid, so each
of cavities B and C will have a surface charge to cancel the electric fields
emminating from $q_B$ and $q_C$ respectively.
\begin{itemize}
	\item
		Cavity B will have a uniform surface charge density of $-q_B/4{\pi} r^2_B$,
		where $r_B$ is the radius of cavity B, with total induced charge $-q_B$
		(because of symmetry and use of a Gaussian surface).
	\item
		Cavity C will have a uniform surface charge density of $-q_C/4{\pi} r^2_C$,
		where $r_C$ is the radius of cavity B, with total induced charge $-q_C$
		(because of symmetry and use of a Gaussian surface).
\end{itemize}
Cavity D will not have a surface charge since a Gaussian surface coincident with
its boundary contains no charge.

The surface A will have total charge $q_B + q_C$ with uniform surface charge
density of $(q_B + q_C) / 4{\pi} r^2$ in accordance with the symmetry of a Gaussian
surface containing the sphere as well as properties of an ideal conductor.

\subsubsection{Answer (2)}
The surfaces B, C, and D will remain unaffected since the surrounding conductor
shields the cavities from electric fields produced by charge $q_E$. The
distribution on surface A will shift so that the negative charge concentration
is greatest on the side nearest to $q_E$ with an increasingly positive
distribution towards the opposite side.

\subsubsection{Answer (3)}
The surface of A will still contain the same total charge $q_B + q_C$ since only
a redistribution of induced charges occurred along the surface. Similarly,
because surface B, C, and D are shielded from the electric field of $q_E$ by
conductor A, the total charges along their surfaces remains unchanged as well.

%%%%%%%%%%%%%%%%%%%%%%%%%%%%%%%%%%%%%%%%%%%%%%%%%%%%%%%%%%%%%%%%%%%%%%%%%%%%%%%
%%%% Problem 4
%%%%%%%%%%%%%%%%%%%%%%%%%%%%%%%%%%%%%%%%%%%%%%%%%%%%%%%%%%%%%%%%%%%%%%%%%%%%%%%
\problem{4}
\subsubsection{Question}
% Keywords
	\index{electrostatics!Dielectric Breakdown of Air}

The dielectric strength of air at standard temperature and pressure is
\SI{3e6}{\V\per\m}. What is the maximum intensity in units of \si{\W\per\m^2}
for a monochromatic laser that can be used in the laboratory?

\subsubsection{Answer}
Failure of a dielectric occurs when the energy density in the dielectric is
great enough to overcome the ionization energy of the constituent atoms. This
suggests that an electric field of greater than \SI{3e6}{\V\per\m} would cause
this ionization to occur.

Starting here, We can calculate the energy density of the electric field at any
point in space by
\begin{align*}
	U_{em} = \frac{\varepsilon _0}{2}E^2
\end{align*}
(where we've used the vacuum energy density since air differs very little from
the vacuum permitivity).

Then the power transmitted by the laser is $P = cU_{em}$, so plugging in the
numbers,
\begin{align*}
	P &= \frac{1}{2}
		\left( \SI[per-mode=fraction]{8.854e-12}
			{\coulomb\squared\per\N\per\m\squared} \right)
		\left( \SI{3e8}{\m\per\s} \vphantom{\frac{V}{V}} \right)
		\left( \SI{3e6}{\V\per\m} \right)^2 \\
	P &= \SI{1.19e10}{\W\per\m\squared}
\end{align*}

\begin{center}
	\fbox{The maximum power of a laser usable in the lab is \SI{1.19e10}
	{\W\per\m\squared}.}
\end{center}

%%%%%%%%%%%%%%%%%%%%%%%%%%%%%%%%%%%%%%%%%%%%%%%%%%%%%%%%%%%%%%%%%%%%%%%%%%%%%%%
%%%% Problem 5
%%%%%%%%%%%%%%%%%%%%%%%%%%%%%%%%%%%%%%%%%%%%%%%%%%%%%%%%%%%%%%%%%%%%%%%%%%%%%%%
\problem{5}
\subsubsection{Question}
% Keywords
	\index{particle!Proton Collision}
	\index{relativity!Proton Collision}

What is the minimum energy of the projectile proton required to induce the
reaction $p + p \rightarrow p + p + p + \bar p$ if the target proton is at rest?

\subsubsection{Answer}
Energy and momentum must be conserved. At the minimum allowed energy, the
resultant 4 proton/anti-protons will be collinear with no relative momentum with
respect to one another, so the momentum equation in the lab frame is simply
\begin{align}
	p_i = 4p_f
\end{align}
Similarly, the resultant (anti-)protons are indistinguishable, so they will
all have equivalent energy $E_f$. The initial protons have different energies
since one is at rest in the lab frame while the other is moving, leading to
the energy equation
\begin{align*}
	\sqrt{p^2_i c^2 + m^2_p c^4} + m_p c^2 &= 4\sqrt{p^2_f c^2 + m^2_p c^4}
\end{align*}
Substituting the momentum relation into the equation, squaring, and simplifying,
\begin{align*}
	\sqrt{16p^2_f c^2 + m^2_p c^4} + m_p c^2 &= 4\sqrt{p^2_f c^2 + m^2_p c^4} \\
	16p^2_f c^2 + m^2_p c^4 + m^2_p c^4 + 2\sqrt{m^2_p c^4(16p^2_f c^2 + m^2_p c^4)}
		&= 16p^2_f c^2 + 16m^2_p c^4 \\
	2\sqrt{m^2_p c^4(16p^2_f c^2 + m^2_p c^4)} &= 14m^2_p c^4 \\
	16p^2_f c^2 + m^2_p c^4 &= 49m^2_p c^4 \\
	p^2_f &= 3m^2_p c^4
\end{align*}
Therefore,
\begin{align*}
	p^2_i &= 48m^2_p c^4 \\
\intertext{and}
	E_1 &= \sqrt{49m^2_p c^4}
\end{align*}
\begin{align}
	\boxed{
	E_1 \approx \SI{6.567}{\GeV\per c\squared}
	}
\end{align}


%%%%%%%%%%%%%%%%%%%%%%%%%%%%%%%%%%%%%%%%%%%%%%%%%%%%%%%%%%%%%%%%%%%%%%%%%%%%%%%
%%%% Problem 6
%%%%%%%%%%%%%%%%%%%%%%%%%%%%%%%%%%%%%%%%%%%%%%%%%%%%%%%%%%%%%%%%%%%%%%%%%%%%%%%
%\subsection{Problem 6}
\problem{6}
\subsubsection{Question}
% Keywords
	\index{unsolved!Fall 2011 I.P6}
A subatomic particle has spin 1 and negative parity. It decays at rest into an $e^+ e^-$ pair, which is produced in the $s$ and $d$ waves. From these data determine (1) the total spin of the $e^+ e^-$ pair and (2) the intrinsic parity of $e^+$ relative to $e^-$.
\subsubsection{Answer}



%%%%%%%%%%%%%%%%%%%%%%%%%%%%%%%%%%%%%%%%%%%%%%%%%%%%%%%%%%%%%%%%%%%%%%%%%%%%%%%
%%%% Problem 7
%%%%%%%%%%%%%%%%%%%%%%%%%%%%%%%%%%%%%%%%%%%%%%%%%%%%%%%%%%%%%%%%%%%%%%%%%%%%%%%
%\subsection{Problem 7}
\problem{7}
\subsubsection{Question}
% Keywords
	\index{unsolved!Fall 2011 I.P7}

There is a uniform, vertical gravitational field with a downward acceleration of gravity g above a horizontal, perfectly elastic surface. A particle of mass m can only move above the surface. Give a rough estimate of the energy eigenvalue for the ground state of the particle.

\subsubsection{Answer}





%%%%%%%%%%%%%%%%%%%%%%%%%%%%%%%%%%%%%%%%%%%%%%%%%%%%%%%%%%%%%%%%%%%%%%%%%%%%%%%
%%%% Problem 8
%%%%%%%%%%%%%%%%%%%%%%%%%%%%%%%%%%%%%%%%%%%%%%%%%%%%%%%%%%%%%%%%%%%%%%%%%%%%%%%
\problem{8}
\subsubsection{Question}
% Keywords
	\index{thermodynamics!Atmospheric Scale Height (Pressure)}

Assume that the atmosphere near the earth's surface is in approximate
hydrostatic equilibrium, where any movement of air parcels is gentle and
adiabatic. Find an expression for the pressure $P$ of the atmosphere as a
function of the height $z$.

\subsubsection{Answer}
Note that the pressure at a given point is due to the mass of air above the
given point. Then by moving an infinitesimal distance vertically, the total
mass is changed by the density of the air (which is affected by the
gravitational force). This leads to the differential equation
\begin{align*}
	\frac{dP}{dz} &= \rho g
\end{align*}
Then using the ideal gas equation
\begin{align*}
	PV &= N k_B T \\
\intertext{multiply and divide by the average molecular mass $m$ of the air (in
\si{\kg}) which combined with the number of molecules $N$ gives the total mass}
	PV &= (Nm) \frac{1}{m} k_B T \\
\intertext{and then divide by the volume to get the ideal gas equation in terms
of the mass density}
	P &= \frac{Nm}{V} \frac{1}{m} k_B T \\
	P &= \rho  \frac{k_B T}{m} \\
	\rho  &= \frac{P m}{k_B T}
\end{align*}
Finally, substitute this into the differential equation above and solve to
get the atmospheric scale height equation.
\begin{align*}
	\frac{dP}{dz} ={}& \frac{P m}{k_B T}g \\
	\frac{dP}{P} ={}& \frac{mg}{k_B T} dz
\end{align*}
\begin{align}
	\boxed{
	P(z) = P_0 e^{z/{\xi}}
		\quad\quad\text{where }{\xi} = \frac{k_B T}{mg}
	}
\end{align}


%%%%%%%%%%%%%%%%%%%%%%%%%%%%%%%%%%%%%%%%%%%%%%%%%%%%%%%%%%%%%%%%%%%%%%%%%%%%%%%
%%%% Problem 9
%%%%%%%%%%%%%%%%%%%%%%%%%%%%%%%%%%%%%%%%%%%%%%%%%%%%%%%%%%%%%%%%%%%%%%%%%%%%%%%
%\subsection{Problem 9}
\problem{9}
\subsubsection{Question}
% Keywords
	\index{unsolved!Fall 2011 I.P9}

Consider a gas of atoms in a magnetic field of 10 Tesla. The nucleus of the atom has spin 1/2, magnetic moment $\mu\approx10^{-26}$J/Tesla, and mass $m\approx5\times10^{-27}$kg. The electrons in the atom have zero total angular momentum. What is the maximum number density of the gas for which the nuclei are completely polarized by the magnetic field at zero temperature?

\subsubsection{Answer}



%%%%%%%%%%%%%%%%%%%%%%%%%%%%%%%%%%%%%%%%%%%%%%%%%%%%%%%%%%%%%%%%%%%%%%%%%%%%%%%
%%%% Problem 10
%%%%%%%%%%%%%%%%%%%%%%%%%%%%%%%%%%%%%%%%%%%%%%%%%%%%%%%%%%%%%%%%%%%%%%%%%%%%%%%
%\subsection{Problem 10}
\problem{10}
\subsubsection{Question}
% Keywords
	\index{unsolved!Fall 2011 I.P10}

A new long-lived particle $X$ is observed to decay via $X \to K^+ + K^-$ . The mass of $X$ is about 1.2 GeV/$c^2$ . You wish to determine this mass to within 1\% using the momenta of the $K^+$ and $K^-$ from decay of $X$ at rest. What should be the maximum relative error on your momentum measurements if you use only a single decay event? You know that the mass of $K^+$ and $K^-$ is $493.677\pm0.013$MeV/$c^2$ .

\subsubsection{Answer}

