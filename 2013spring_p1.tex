%%%%%%%%%%%%%%%%%%%%%%%%%%%%%%%%%%%%%%%%%%%%%%%%%%%%%%%%%%%%%%%%%%%%%%%%%%%%%%%
%%%% Problem 1
%%%%%%%%%%%%%%%%%%%%%%%%%%%%%%%%%%%%%%%%%%%%%%%%%%%%%%%%%%%%%%%%%%%%%%%%%%%%%%%
%\subsection{Problem 1}
\problem{1}
\subsubsection{Question}
% Keywords
	\index{unsolved!Spring 2013 I.P1}

Three perfectly cylindrical frictionless oil pipes are packed inside an inclined railway car as shown on the figure. Find the range of angles $\theta$ where such a packing is stable.

\subsubsection{Answer}


%%%%%%%%%%%%%%%%%%%%%%%%%%%%%%%%%%%%%%%%%%%%%%%%%%%%%%%%%%%%%%%%%%%%%%%%%%%%%%%
%%%% Problem 2
%%%%%%%%%%%%%%%%%%%%%%%%%%%%%%%%%%%%%%%%%%%%%%%%%%%%%%%%%%%%%%%%%%%%%%%%%%%%%%%
%\subsection{Problem 2}
\problem{2}
\subsubsection{Question}
% Keywords
	\index{unsolved!Spring 2013 I.P2}
An ``air molecule'' at room temperature and atmospheric pressure is moving with the average speed of $450$ m/sec. It travels about $7\times10^{-6}$ cm between collisions. Estimate how much time it takes for a molecule to travel $1$cm?
\subsubsection{Answer}



%%%%%%%%%%%%%%%%%%%%%%%%%%%%%%%%%%%%%%%%%%%%%%%%%%%%%%%%%%%%%%%%%%%%%%%%%%%%%%%
%%%% Problem 3
%%%%%%%%%%%%%%%%%%%%%%%%%%%%%%%%%%%%%%%%%%%%%%%%%%%%%%%%%%%%%%%%%%%%%%%%%%%%%%%
%\subsection{Problem 3}
\problem{3}
\subsubsection{Question}
% Keywords
	\index{unsolved!Spring 2013 I.P3}
A particle of mass $m$, moving with velocity $v$, crosses a boundary between the region where its potential energy is equal to $U_1$ to a region where its potential energy is equal to $U_2$. Derive a ``Snell’s law'' of refraction for this boundary.
\subsubsection{Answer}



%%%%%%%%%%%%%%%%%%%%%%%%%%%%%%%%%%%%%%%%%%%%%%%%%%%%%%%%%%%%%%%%%%%%%%%%%%%%%%%
%%%% Problem 4
%%%%%%%%%%%%%%%%%%%%%%%%%%%%%%%%%%%%%%%%%%%%%%%%%%%%%%%%%%%%%%%%%%%%%%%%%%%%%%%
%\subsection{Problem 4}
\problem{4}
\subsubsection{Question}
% Keywords
	\index{unsolved!Spring 2013 I.P4}
A coaxial cable made of ideal conductor (no resistance) has an inner cylindrical conductor of radius a, and air gap between $r = a$ and $r = b$, and another conductor between $r = b$ and $r = c$, as shown. The inner conductor is connected to a voltage source at one of its ends such that its voltage with respect to the outer conductor is $V_0$ . The inner conductor carries a current $I_0$ in the $–z$ direction (into the page) which is returned along the outer conductor via a resister which connects the inner and outer conductors at the other end. Calculate the magnitude and direction of the electric field $\mathbf{E}$ and magnetic field $\mathbf{B}$ in the region in the air gap where $a < r < b.$
\subsubsection{Answer}


%%%%%%%%%%%%%%%%%%%%%%%%%%%%%%%%%%%%%%%%%%%%%%%%%%%%%%%%%%%%%%%%%%%%%%%%%%%%%%%
%%%% Problem 5
%%%%%%%%%%%%%%%%%%%%%%%%%%%%%%%%%%%%%%%%%%%%%%%%%%%%%%%%%%%%%%%%%%%%%%%%%%%%%%%
%\subsection{Problem 5}
\problem{5}
\subsubsection{Question}
% Keywords
	\index{unsolved!Spring 2013 I.P5}
An ideal voltage generator with output voltage $V=V_0\sin(\omega t)$ where $V_0 = 100$ Volts is connected to the circuit shown below. Calculate the time-averaged power dissipated in the 100 $\Omega$ resistor as a function of the generator frequency $\Omega$.
\subsubsection{Answer}



%%%%%%%%%%%%%%%%%%%%%%%%%%%%%%%%%%%%%%%%%%%%%%%%%%%%%%%%%%%%%%%%%%%%%%%%%%%%%%%
%%%% Problem 6
%%%%%%%%%%%%%%%%%%%%%%%%%%%%%%%%%%%%%%%%%%%%%%%%%%%%%%%%%%%%%%%%%%%%%%%%%%%%%%%
%\subsection{Problem 6}
\problem{6}
\subsubsection{Question}
% Keywords
	\index{unsolved!Spring 2013 I.P6}
A beam of monoenergetic $\pi^+$ of kinetic energy $T = 140$ MeV and rest mass $m_0 = 140$ MeV/$c^2$ is to traverse a total flight path of length $D = 20$ m. Calculate the fraction of Pions that survive the 20 m trip, provided that the mean lifetime of charged Pions is $\tau= 2.6\times10^{-8}$ sec (in the rest frame of the pion).
\subsubsection{Answer}

%%%%%%%%%%%%%%%%%%%%%%%%%%%%%%%%%%%%%%%%%%%%%%%%%%%%%%%%%%%%%%%%%%%%%%%%%%%%%%%
%%%% Problem 7
%%%%%%%%%%%%%%%%%%%%%%%%%%%%%%%%%%%%%%%%%%%%%%%%%%%%%%%%%%%%%%%%%%%%%%%%%%%%%%%
%\subsection{Problem 7}
\problem{7}
\subsubsection{Question}
% Keywords
	\index{unsolved!Spring 2013 I.P7}
Some organic molecules have a spin triplet $(S = 1)$ excited state at an energy $k_B\Delta$ above a singlet $(S = 0)$ ground state. Find an expression for the magnetic moment, $\expval{\mu}$, in a field $\mathbf{B}$. What is the susceptibility at high temperature?
\subsubsection{Answer}



%%%%%%%%%%%%%%%%%%%%%%%%%%%%%%%%%%%%%%%%%%%%%%%%%%%%%%%%%%%%%%%%%%%%%%%%%%%%%%%
%%%% Problem 8
%%%%%%%%%%%%%%%%%%%%%%%%%%%%%%%%%%%%%%%%%%%%%%%%%%%%%%%%%%%%%%%%%%%%%%%%%%%%%%%
%\subsection{Problem 8}
\problem{8}
\subsubsection{Question}
% Keywords
	\index{unsolved!Spring 2013 I.P8}
Two identical perfect gases with the same pressure $P$ and the same number of particles $N$, but with different temperatures, $T_1$ and $T_2$ are confined in two vessels, of volume $V_1$ and $V_1$ that are then connected. Find the change in entropy after the system has reached equilibrium.
\subsubsection{Answer}



%%%%%%%%%%%%%%%%%%%%%%%%%%%%%%%%%%%%%%%%%%%%%%%%%%%%%%%%%%%%%%%%%%%%%%%%%%%%%%%
%%%% Problem 9
%%%%%%%%%%%%%%%%%%%%%%%%%%%%%%%%%%%%%%%%%%%%%%%%%%%%%%%%%%%%%%%%%%%%%%%%%%%%%%%
%\subsection{Problem 9}
\problem{9}
\subsubsection{Question}
% Keywords
	\index{unsolved!Spring 2013 I.P9}
A charged particle is residing in an infinite one-dimensional square well potential.
\begin{enumerate}
	\item Write down an expression for the matrix element of the electric dipole moment for the particle when making a transition from one quantized level to another.
	\item From consideration of your answer in part (a), what would be the corresponding selection rules governing the allowed transitions for this particle? Provide a physical justification for your answer.
\end{enumerate} 
\subsubsection{Answer}



%%%%%%%%%%%%%%%%%%%%%%%%%%%%%%%%%%%%%%%%%%%%%%%%%%%%%%%%%%%%%%%%%%%%%%%%%%%%%%%
%%%% Problem 10
%%%%%%%%%%%%%%%%%%%%%%%%%%%%%%%%%%%%%%%%%%%%%%%%%%%%%%%%%%%%%%%%%%%%%%%%%%%%%%%
%\subsection{Problem 10}
\problem{10}
\subsubsection{Question}
% Keywords
	\index{unsolved!Spring 2013 I.P10}
Consider a diatomic molecule of two dissimilar nuclei that have the following properties: a reduced mass $\mu = 20 m_p$ (where $m_p$ is the mass of a proton), interatomic spacing, $r_0 = 3.0\times10^{-10}$ m, and the force constant, $C = 8\times10^{18}$ eV/$cm^2$.
\begin{enumerate}
	\item What is the energy difference, in $eV$, between the $r = 0$ and $r = 1$ rotational levels for this molecule in the vibrational ground state?
	\item What is the energy difference, in $eV$, between the $\nu = 0$ and $\nu = 1$ vibrational levels for this molecule? [Assume that the rotational state does not change.]
\end{enumerate}
\subsubsection{Answer}

