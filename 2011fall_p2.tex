%%%%%%%%%%%%%%%%%%%%%%%%%%%%%%%%%%%%%%%%%%%%%%%%%%%%%%%%%%%%%%%%%%%%%%%%%%%%%%%
%%%% Problem 1
%%%%%%%%%%%%%%%%%%%%%%%%%%%%%%%%%%%%%%%%%%%%%%%%%%%%%%%%%%%%%%%%%%%%%%%%%%%%%%%
\problem{1}
\subsubsection{Question}
% Keywords
	\index{Lagrangian!Pendulum from free horizontal support}
	\index{mechanics!Pendulum from free horizontal support}

Mass $m_1$ moves freely along a fixed, long, horizontal rod. The position of
$m_1$ on the rod is $x$. A massless string of length $\ell $ is attached to $m_1$
at the end and to mass $m_2$ at the other. Mass $m_2$ executes pendulum motion
in the vertical plane containing the rod.
\begin{enumerate}
	\item
		Find the Lagrangian of the system.
	\item
		Derive the equations of motion and the corresponding conservation laws.
	\item
		Assume that $x(0)=x_0$, $\dot x(0)=0$, $\phi (0)=\phi_0$ $(|\phi_0| {\ll} 1)$,
		and $\dot \phi (0)=0$. Find $x(t)$ and $\phi (t)$ for $t > 0$.
\end{enumerate}


\subsubsection{Answer (1)}
For the sliding support mass $m_1$:
\begin{align*}
	T_1 &= \frac{1}{2}m_1 \dot{x}^2 \\
	V_1 &= 0
\end{align*}
For the pendulum mass $m_2$:
\begin{align*}
	T_2 &= \frac{1}{2}m_2 \dot{y}^2 + \frac{1}{2}m_2(\dot x + \dot x_2)^2 \\
	V_2 &= -m_2gy_2 \\
\intertext{Then using $x_2 = \ell \sin \phi $ and $y_2 = -\ell \cos \phi $,}
	T_2 &= \frac{1}{2}m_2 \left( \ell ^2\dot \phi ^2 + \dot x^2 + 2\ell \dot \phi \dot x \cos \phi 
		\right) \\
	V_2 &= -m_2 g y_2
\end{align*}
Putting the Lagrangian together equals the first line. Applying the small
angle approximation gives the second line where the kinetic energy term
involving $\cos \phi $ can be simply expanded as $\cos \phi  \approx 1$, but the potential
energy term must be expanded to second order so that $\cos \phi  \approx 1 -
\frac{1}{2}\phi ^2$.

\begin{align}
	\boxed{
	\mathcal{L} = \frac{1}{2} (m_1 + m_2)\dot x^2 + \frac{1}{2} m_2 \left(
		\ell ^2\dot \phi ^2 + 2\ell \dot \phi \dot x\cos \phi  \right) + m_2g\ell \cos \phi 
	}
	\\
	\boxed{
	\mathcal{L} \approx \frac{1}{2} (m_1 + m_2)\dot x^2 + \frac{1}{2} m_2 \left(
		\ell ^2\dot \phi ^2 + 2\ell \dot \phi \dot x \right) + m_2g\ell  - \frac{1}{2}m_2g\ell \phi ^2
	}
\end{align}

\subsubsection{Answer (2)}
Constructing the Euler-Lagrange equations for $x$ and $\dot x$:
\begin{align*}
	\frac{\partial \mathcal{L}}{\partial x} &= 0 &
		\frac{\partial \mathcal{L}}{\partial \dot x} &= (m_1 + m_2)\dot x + m_2\ell \dot \phi 
	\\
	{}&{}&
	\frac{d}{dt}\left[ \frac{\partial \mathcal{L}}{\partial \dot x} \right]
		&= (m_1 + m_2)\ddot x + m_2\ell \ddot \phi 
\end{align*}
\begin{align}
	(m_1 + m_2)\ddot x + m_2\ell \ddot \phi  = 0
\end{align}
and for $\phi $ and $\dot \phi $:
\begin{align*}
	\frac{\partial \mathcal{L}}{\partial \phi } &= -m_2g\ell \phi  &
		\frac{\partial \mathcal{L}}{\partial \dot \phi } &= m_2\ell ^2\dot \phi  + m_2\ell \dot x
	\\
	{}&{}&
	\frac{d}{dt}\left[ \frac{\partial \mathcal{L}}{\partial \dot \phi } \right]
		&= m_2\ell ^2\ddot \phi  + m_2\ell \ddot x
\end{align*}
\begin{align}
	-m_2g\ell \phi  - m_2\ell ^2\ddot \phi  + m_2\ell \ddot x = 0
\end{align}

The equations of motion are:
\begin{align}
	\boxed{
	\ddot x + \frac{m_2}{m_1+m_2} \ell  \ddot \phi  = 0
	} \\
	\boxed{
	\ddot \phi  + \frac{1}{\ell }\ddot x + \frac{g}{\ell }\phi  = 0
	}
\end{align}

\subsubsection{Answer (3)}
Solve for $\ddot x$ and substitute into the other differential equation
\begin{align}
	\ddot \phi  - \frac{1}{\ell }\frac{m_2}{m_1+m_2} \ell  \ddot \phi  + \frac{g}{\ell }\phi  &= 0\nonumber
	\\
	\frac{m_1}{m_1+m_2} \ddot \phi  + \frac{g}{\ell }\phi  &= 0\nonumber
	\\
	\ddot \phi  + \frac{g}{\ell }\frac{m_1+m_2}{m_1} \phi  &= 0
\end{align}
This is just the differential equation for a simple harmonic oscillator, so
considering the given boundary conditions,
\begin{align}
	\boxed{
	\phi (t) = \phi _0\cos({\omega} t)
		\quad\quad\text{where } {\omega}^2 = \frac{g}{\ell }\frac{m_1+m_2}{m_1}
	}
\end{align}

Then differentiating $\phi (t)$ twice and substituting into the first equation,
\begin{align*}
	\ddot x &= \ell \phi _0{\omega}^2\frac{m_2}{m_1+m_2}\cos({\omega} t)
\end{align*}
Then integrating twice and applying the boundary conditions,
\begin{align}
	\boxed{
	x(t) = x_0 - \frac{g}{{\omega}^2}\frac{m_2}{m_1}\cos({\omega} t)
	}
\end{align}

%%%%%%%%%%%%%%%%%%%%%%%%%%%%%%%%%%%%%%%%%%%%%%%%%%%%%%%%%%%%%%%%%%%%%%%%%%%%%%%
%%%% Problem 2
%%%%%%%%%%%%%%%%%%%%%%%%%%%%%%%%%%%%%%%%%%%%%%%%%%%%%%%%%%%%%%%%%%%%%%%%%%%%%%%
%\subsection{Problem 2}
\problem{2}
\subsubsection{Question}
% Keywords
	\index{unsolved!Spring 2011 II.P2}
A light bulb has a tungsten filament formed into a coil of 60 turns. The coil is a straight column of 3 mm in diameter and 20 mm in length. The bulb is rated 75 W for an AC source of 110 V with a frequency of 60 Hz. Find the current in the bulb as a function of time $t$ after it is connected to the AC source at $t = 0$. Assume that the voltage of the source is of the form $V(t) = V_0 \cos(\omega t)$ and the resistance of the filament stays constant.
\subsubsection{Answer}



%%%%%%%%%%%%%%%%%%%%%%%%%%%%%%%%%%%%%%%%%%%%%%%%%%%%%%%%%%%%%%%%%%%%%%%%%%%%%%%
%%%% Problem 3
%%%%%%%%%%%%%%%%%%%%%%%%%%%%%%%%%%%%%%%%%%%%%%%%%%%%%%%%%%%%%%%%%%%%%%%%%%%%%%%
%\subsection{Problem 3}
\problem{3}
\subsubsection{Question}
% Keywords
	\index{unsolved!Spring 2011 II.P3}

A particle of mass $m$ is in the potential
\begin{equation}
V(x)= \begin{cases}
		-\Omega_0\delta(x) &-a/2<x<a/2\\
		\infty, &\text{otherwise}
\end{cases}
\end{equation}
where $\delta(x)$ is the Dirac delta function and $\Omega_0$ and $a$ are positive parameters.
\begin{enumerate}
	\item For the energy eigenstates that have wave functions with odd parity, find these wave functions and the corresponding eigenvalues.
	\item For the rest of the energy eigenstates, find the approximate eigenvalues by treating $-\Omega_0\delta(x)$ as a perturbation. What is required of $\Omega_0$ for the approximation to be valid? 
	\item For a special value of $\Omega_0$, the energy of the ground state is exactly zero. Find this special value of $\Omega_0$
\end{enumerate}
\subsubsection{Answer}



%%%%%%%%%%%%%%%%%%%%%%%%%%%%%%%%%%%%%%%%%%%%%%%%%%%%%%%%%%%%%%%%%%%%%%%%%%%%%%%
%%%% Problem 4
%%%%%%%%%%%%%%%%%%%%%%%%%%%%%%%%%%%%%%%%%%%%%%%%%%%%%%%%%%%%%%%%%%%%%%%%%%%%%%%
%\subsection{Problem 4}
\problem{4}
\subsubsection{Question}
% Keywords
	\index{unsolved!Spring 2011 II.P4}

For the simple harmonic oscillator with the Hamiltonian
\begin{equation*}
	H_{\text{sho}} = \frac{P^2}{2m} + \frac{1}{2}m\omega^2 X^2 
\end{equation*}
where $P$ and $X$ are the momentum and position operators, the wave functions for the energy eigenstates are $\phi_n(x/a)$, where $n$ is a non-negative integer, $a$ is a constant with the dimension R of length, and $\int_{-\infty}^{\infty}\abs{\psi_{n}(x/a)}^2\dd x =1$. Now consider two distinguishable particles of mass m with the total Hamiltonian
\begin{equation*}
	H = \frac{P_1^2}{2m}+\frac{P_2^2}{2m} + \frac{1}{2}m\omega^2\qty[X_1^2+X_2^2+(X_1-X_2)^2].
\end{equation*}
\begin{enumerate}
	\item Find the normalized wave functions for the eigenstates of $H$ in terms of $\psi_{n^\prime}(\eta)$ and $\psi_{n^{\prime\prime}}$ by choosing two appropriate dimensionless position variables $\eta$ and $\xi$.
	\item Find the eigenalues of $H$.
	\item How would the answers to the above two problems change if the two particles are identical and have spin 0?
\end{enumerate}

\subsubsection{Answer}


%%%%%%%%%%%%%%%%%%%%%%%%%%%%%%%%%%%%%%%%%%%%%%%%%%%%%%%%%%%%%%%%%%%%%%%%%%%%%%%
%%%% Problem 5
%%%%%%%%%%%%%%%%%%%%%%%%%%%%%%%%%%%%%%%%%%%%%%%%%%%%%%%%%%%%%%%%%%%%%%%%%%%%%%%
%\subsection{Problem 5}
\problem{5}
\subsubsection{Question}
% Keywords
	\index{unsolved!Spring 2011 II.P5}

Consider only the rotational motion of a diatomic molecule with moment of inertia $I$.
\begin{enumerate}
	\item What is the specific heat for a classical system of $N$ such molecules?
	\item In the quantum mechanical case, the energy levels for an individual molecule are
	\begin{equation*}
		E_{lm} = \frac{l(l+1)\hbar^2}{2I}
	\end{equation*}
	where $l = 0, 1, 2, \cdots$ and for each $l, m = -l, -l+1,\cdots, l$. For a system of $N$ such molecules, express the partition function $Z$ and the energy $E$ as sums of well-defined quantities.
	\item Calculate the specific heat in the low-temperature and high-temperature limits for the quantum mechanical case.
	\item For what range of temperature is the classical result in (1) valid?
\end{enumerate}

\subsubsection{Answer}

