%%%%%%%%%%%%%%%%%%%%%%%%%%%%%%%%%%%%%%%%%%%%%%%%%%%%%%%%%%%%%%%%%%%%%%%%%%%%%%%
%%%% Problem 1
%%%%%%%%%%%%%%%%%%%%%%%%%%%%%%%%%%%%%%%%%%%%%%%%%%%%%%%%%%%%%%%%%%%%%%%%%%%%%%%
\problem{1}
\subsubsection{Question}
% Keywords
	\index{Lagrangian!Pendulum from free horizontal support}
	\index{mechanics!Pendulum from free horizontal support}

Mass $m₁$ moves freely along a fixed, long, horizontal rod. The position of
$m₁$ on the rod is $x$. A massless string of length $ℓ$ is attached to $m₁$
at the end and to mass $m₂$ at the other. Mass $m₂$ executes pendulum motion
in the vertical plane containing the rod.
\begin{enumerate}
	\item
		Find the Lagrangian of the system.
	\item
		Derive the equations of motion and the corresponding conservation laws.
	\item
		Assume that $x(0)=x₀$, $\dot x(0)=0$, $φ(0)=φ₀$ $(|φ₀| ≪ 1)$,
		and $\dot φ(0)=0$. Find $x(t)$ and $φ(t)$ for $t > 0$.
\end{enumerate}


\subsubsection{Answer (1)}
For the sliding support mass $m₁$:
\begin{align*}
	T₁ &= \frac{1}{2}m₁ \dot{x}² \\
	V₁ &= 0
\end{align*}
For the pendulum mass $m₂$:
\begin{align*}
	T₂ &= \frac{1}{2}m₂ \dot{y}² + \frac{1}{2}m₂(\dot x + \dot x₂)² \\
	V₂ &= -m₂gy₂ \\
\intertext{Then using $x₂ = ℓ\sin φ$ and $y₂ = -ℓ\cos φ$,}
	T₂ &= \frac{1}{2}m₂ \left( ℓ²\dot φ² + \dot x² + 2ℓ\dot φ\dot x \cos φ
		\right) \\
	V₂ &= -m₂ g y₂
\end{align*}
Putting the Lagrangian together equals the first line. Applying the small
angle approximation gives the second line where the kinetic energy term
involving $\cos φ$ can be simply expanded as $\cos φ ≈ 1$, but the potential
energy term must be expanded to second order so that $\cos φ ≈ 1 -
\frac{1}{2}φ²$.

\begin{align}
	\boxed{
	\sL = \frac{1}{2} (m₁ + m₂)\dot x² + \frac{1}{2} m₂ \left(
		ℓ²\dot φ² + 2ℓ\dot φ\dot x\cos φ \right) + m₂gℓ\cos φ
	}
	\\
	\boxed{
	\sL ≈ \frac{1}{2} (m₁ + m₂)\dot x² + \frac{1}{2} m₂ \left(
		ℓ²\dot φ² + 2ℓ\dot φ\dot x \right) + m₂gℓ - \frac{1}{2}m₂gℓφ²
	}
\end{align}

\subsubsection{Answer (2)}
Constructing the Euler-Lagrange equations for $x$ and $\dot x$:
\begin{align*}
	\frac{∂\sL}{∂x} &= 0 &
		\frac{∂\sL}{∂\dot x} &= (m₁ + m₂)\dot x + m₂ℓ\dot φ
	\\
	{}&{}&
	\frac{d}{dt}\left[ \frac{∂\sL}{∂\dot x} \right]
		&= (m₁ + m₂)\ddot x + m₂ℓ\ddot φ
\end{align*}
\begin{align}
	(m₁ + m₂)\ddot x + m₂ℓ\ddot φ = 0
\end{align}
and for $φ$ and $\dot φ$:
\begin{align*}
	\frac{∂\sL}{∂φ} &= -m₂gℓφ &
		\frac{∂\sL}{∂\dot φ} &= m₂ℓ²\dot φ + m₂ℓ\dot x
	\\
	{}&{}&
	\frac{d}{dt}\left[ \frac{∂\sL}{∂\dot φ} \right]
		&= m₂ℓ²\ddot φ + m₂ℓ\ddot x
\end{align*}
\begin{align}
	-m₂gℓφ - m₂ℓ²\ddot φ + m₂ℓ\ddot x = 0
\end{align}

The equations of motion are:
\begin{align}
	\boxed{
	\ddot x + \frac{m₂}{m₁+m₂} ℓ \ddot φ = 0
	} \\
	\boxed{
	\ddot φ + \frac{1}{ℓ}\ddot x + \frac{g}{ℓ}φ = 0
	}
\end{align}

\subsubsection{Answer (3)}
Solve for $\ddot x$ and substitute into the other differential equation
\begin{align}
	\ddot φ - \frac{1}{ℓ}\frac{m₂}{m₁+m₂} ℓ \ddot φ + \frac{g}{ℓ}φ &= 0\nonumber
	\\
	\frac{m₁}{m₁+m₂} \ddot φ + \frac{g}{ℓ}φ &= 0\nonumber
	\\
	\ddot φ + \frac{g}{ℓ}\frac{m₁+m₂}{m₁} φ &= 0
\end{align}
This is just the differential equation for a simple harmonic oscillator, so
considering the given boundary conditions,
\begin{align}
	\boxed{
	φ(t) = φ₀\cos(ωt)
		\quad\quad\text{where } ω² = \frac{g}{ℓ}\frac{m₁+m₂}{m₁}
	}
\end{align}

Then differentiating $φ(t)$ twice and substituting into the first equation,
\begin{align*}
	\ddot x &= ℓφ₀ω²\frac{m₂}{m₁+m₂}\cos(ωt)
\end{align*}
Then integrating twice and applying the boundary conditions,
\begin{align}
	\boxed{
	x(t) = x₀ - \frac{g}{ω²}\frac{m₂}{m₁}\cos(ωt)
	}
\end{align}

