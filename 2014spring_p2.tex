%%%%%%%%%%%%%%%%%%%%%%%%%%%%%%%%%%%%%%%%%%%%%%%%%%%%%%%%%%%%%%%%%%%%%%%%%%%%%%%
%%%% Problem 1
%%%%%%%%%%%%%%%%%%%%%%%%%%%%%%%%%%%%%%%%%%%%%%%%%%%%%%%%%%%%%%%%%%%%%%%%%%%%%%%
%\subsection{Problem 1}
\problem{1}
\subsubsection{Question}
% Keywords
	\index{unsolved!Spring 2014 II.P1}
	\index{Lagrangian!Bead on a Wire}
A bead of mass $m$ is constrained to move on a frictionless hoop (see Figure 1). The hoop of radius $R$ is forced to rotate about the vertical diameter with angular speed $\omega.$ The position of the bead is specified by coordinate $\theta$. Write the Lagrangian for the system and derive the differential equation of motion for $\theta$. Include the gravitational force in your analysis. Determine the equilibrium solutions for this problem, that is, the values of constant $\theta$ that solve the differential equation of motion. For each solution, discuss the conditions under which this solution exists and indicate whether it is a stable or an unstable equilibrium solution.
\subsubsection{Answer}


%%%%%%%%%%%%%%%%%%%%%%%%%%%%%%%%%%%%%%%%%%%%%%%%%%%%%%%%%%%%%%%%%%%%%%%%%%%%%%%
%%%% Problem 2
%%%%%%%%%%%%%%%%%%%%%%%%%%%%%%%%%%%%%%%%%%%%%%%%%%%%%%%%%%%%%%%%%%%%%%%%%%%%%%%
%\subsection{Problem 2}
\problem{2}
\subsubsection{Question}
% Keywords
	\index{unsolved!Spring 2014 II.P2}
	\index{electrostatics!Conducting Sphere}
A charge $q$ is uniformly distributed throughout a spherical volume of radius $a$. Surrounding and concentric with this charge distribution is an uncharged, conducting spherical shell of inner radius $b$ and outer radius $c (c > b > a)$. The space between the charge distribution and the conducting shell is filled with a linear isotropic, homogeneous dielectric, having dielectric constant ${\kappa}$. Find the polarization vector $\mathbf{P}$ in the dielectric, and find the electrostatic field $\mathbf{E}$ and potential $V$ everywhere. Take $V$ to approach zero at large distances from this configuration. Make two sketches highlighting the main features of the functional dependence of $\mathbf{E}$ and $V$ on the radial coordinate.
\subsubsection{Answer}



%%%%%%%%%%%%%%%%%%%%%%%%%%%%%%%%%%%%%%%%%%%%%%%%%%%%%%%%%%%%%%%%%%%%%%%%%%%%%%%
%%%% Problem 3
%%%%%%%%%%%%%%%%%%%%%%%%%%%%%%%%%%%%%%%%%%%%%%%%%%%%%%%%%%%%%%%%%%%%%%%%%%%%%%%
%\subsection{Problem 3}
\problem{3}
\subsubsection{Question}
% Keywords
	\index{unsolved!Spring 2014 II.P3}
An ideal gas engine is working in a reversible Joule cycle shown in the $T$ vs. $S$ diagram (figure 2). The gas is monatomic and has $n$ moles.

1. Draw the corresponding P-V diagram for the cycle.

2. Express the work done in terms of temperatures $T_1, T_2, T_3,\text{ and }T_4$.

3. Find the efficiency of the engine in terms of these temperatures.
\subsubsection{Answer}



%%%%%%%%%%%%%%%%%%%%%%%%%%%%%%%%%%%%%%%%%%%%%%%%%%%%%%%%%%%%%%%%%%%%%%%%%%%%%%%
%%%% Problem 4
%%%%%%%%%%%%%%%%%%%%%%%%%%%%%%%%%%%%%%%%%%%%%%%%%%%%%%%%%%%%%%%%%%%%%%%%%%%%%%%
%\subsection{Problem 4}
\problem{4}
\subsubsection{Question}
% Keywords
	\index{unsolved!Spring 2014 II.P4}
A particle is in an infinitely deep one-dimensional potential well of the width $a$ located at $0\le x\le a$
\begin{align*}
	V(x) &= 0, &0\le x\le a\\
	V(x) &= \infty, &x\le 0,\ x\ge a
\end{align*}
\begin{enumerate}
	\item Find the normalized wave functions that describe its energy states. 
	\item Find the first order corrections ${\Delta}E_n$ to the energy levels for a perturbation of the form
	\begin{align*}
		\Delta V(x) &= V_0 \frac{2x}{a}, && 0\le x\le \frac{a}{2},\\
		\Delta V(x) &= V_0\qty(2-\frac{2x}{a}), && \frac{a}{2}\le x\le a
	\end{align*}
\end{enumerate}
(Hint: to solve integrals of the form $\int x \cos(kx)\dd x$ you may use the technique of differentiating with respect to a parameter, as in $\int x \cos(kx)\dd x = \dv{k} \int \sin(kx) \dd x$.)
\subsubsection{Answer}


%%%%%%%%%%%%%%%%%%%%%%%%%%%%%%%%%%%%%%%%%%%%%%%%%%%%%%%%%%%%%%%%%%%%%%%%%%%%%%%
%%%% Problem 5
%%%%%%%%%%%%%%%%%%%%%%%%%%%%%%%%%%%%%%%%%%%%%%%%%%%%%%%%%%%%%%%%%%%%%%%%%%%%%%%
%\subsection{Problem 5}
\problem{5}
\subsubsection{Question}
% Keywords
	\index{unsolved!Spring 2014 II.P5}
	\index{relativity!Compton Scattering}
It has been proposed to create a terrestrial source of high energy photons by having low energy laser photons collide with accelerator-generated high energy electrons. Consider a laser photon with energy $\hbar\omega = 1$ eV that collides head-to-head with an electron and reflects backward. The energy of the electron is $E = 50$ GeV $(= 50 {\times} 10^9 \text{ eV})$. Find the energy of the photon after the collision.

[Hints: 1. note that you cannot neglect the electron mass because it is much larger than the photon energy; 2. to simplify calculations you may want to transform to (and back from) the center of mass frame. A different approach (with the same goal) is to make use of $(E + pc)(E - pc) = E^2 - p^2 c^2 = m^2 c^4$ ].
\subsubsection{Answer}























