%%%%%%%%%%%%%%%%%%%%%%%%%%%%%%%%%%%%%%%%%%%%%%%%%%%%%%%%%%%%%%%%%%%%%%%%%%%%%%%
%%%% Problem 1
%%%%%%%%%%%%%%%%%%%%%%%%%%%%%%%%%%%%%%%%%%%%%%%%%%%%%%%%%%%%%%%%%%%%%%%%%%%%%%%
%\subsection{Problem 1}
\problem{1}
\subsubsection{Question}
% Keywords
	\index{unsolved!Fall 2010 II.P1}

Consider the system shown in Figure 1: Two objects, of mass $m_1$ and $m_2$ , can be treated as point-like. Each of them is suspended from the ceiling by a wire of negligible mass, and of length $L$. The two objects are connected to each other by a spring, of spring constant $k$. The spring is relaxed when the two wires are vertical, as shown in the Figure. Denote by $\theta_1$ and $\theta_2$ the angles that the two wires form with the vertical, for an arbitrary position of the two masses ($\theta_1 = \theta_2 = 0$ in the figure). Consider small oscillations of the two masses in the plane of the Figure, about the equilibrium position shown in the Figure.

(i) Find the angular frequencies $\omega$ of the two normal modes of the system (in other terms, the eigenfrequencies of the system). (ii) Provide the physical explanation of why the eigenfrequencies are what they are. (iii) Find the time evolution $\theta_1 (t)$ and $\theta_2 (t)$ for small oscillations of the two objects, starting at rest from the initial values $\theta_1 = 0$ , and $\theta_2 = \epsilon$.
\subsubsection{Answer}


%%%%%%%%%%%%%%%%%%%%%%%%%%%%%%%%%%%%%%%%%%%%%%%%%%%%%%%%%%%%%%%%%%%%%%%%%%%%%%%
%%%% Problem 2
%%%%%%%%%%%%%%%%%%%%%%%%%%%%%%%%%%%%%%%%%%%%%%%%%%%%%%%%%%%%%%%%%%%%%%%%%%%%%%%
%\subsection{Problem 2}
\problem{2}
\subsubsection{Question}
% Keywords
	\index{unsolved!Fall 2010 II.P2}
The Hamiltonian for a rigid body is
\begin{equation*}
	H = \frac{1}{2}\qty(\frac{L_1^2}{I_1}+\frac{L_2^2}{I_2}+\frac{L_3^2}{I_3})
\end{equation*}
where $I_i$ are the principal moments of inertia and $(L_1, L_2, L_3) = (L_x, L_y, L_z)$ are the angular momentum operators. This Hamiltonian describes the rotational spectrum of molecules.
\begin{enumerate}
	\item First consider the case $I_1 = I_2 = I_3 = I$ - that is, a spherical top, such as methane. Write down a formula for the energy levels in terms of an appropriate quantum number.
	\item Next consider the case $I_1 = I_2 = I_0\ne I_3$ - that is, a symmetric top, such as ammonia. Show that $\comm{L_3}{H} = 0$ and $\comm{L_2}{H} = 0$, and write the Hamiltonian in terms of $L_3$ and $L_2$. Then write down a formula for the energy levels in terms of appropriate quantum numbers. Indicate the allowed ranges of the quantum numbers and indicate if there are any energy degeneracies.
	\item Now consider a slightly asymetric top, such as water molecule, for which we can write
	\begin{align*}
		I_1 &= I_0 -\alpha\\
		I_2 &= I_0 +\alpha
	\end{align*}
	where $\alpha$ is small. Write down the Hamiltonian for this system as a sum of the symmetric top Hamiltonian (from part (ii)) and a perturbation term. Find the shifts in energy levels relative to those of the symmetric top, using perturbation theory to first order in $\alpha$. Be complete and list the energy shifts for all values of the two quantum numbers in part (ii).
\end{enumerate}
\subsubsection{Answer}



%%%%%%%%%%%%%%%%%%%%%%%%%%%%%%%%%%%%%%%%%%%%%%%%%%%%%%%%%%%%%%%%%%%%%%%%%%%%%%%
%%%% Problem 3
%%%%%%%%%%%%%%%%%%%%%%%%%%%%%%%%%%%%%%%%%%%%%%%%%%%%%%%%%%%%%%%%%%%%%%%%%%%%%%%
%\subsection{Problem 3}
\problem{3}
\subsubsection{Question}
% Keywords
	\index{unsolved!Fall 2010 II.P3}

A nonrelativistic particle of mass m moves in one dimension in a square well potential with walls of infinite height at a distance $L$ apart (to be concrete, you may take $V = 0$ for $0 < x < L$ and $V = +\infty$ elsewhere). At time $t = 0$ the particle is in a state with equal admixtures (with zero relative phase) of the two lowest energy eigenstates of the system.
\begin{enumerate}
	\item Write down the wavefunction for this particle at $t = 0$ and at an arbitrary later time $t$.
	\item Calculate and plot as a function of time the probability that the particle will be found in the right-hand side of the well, i.e. at some $x > L/2$.
\end{enumerate}

\subsubsection{Answer}



%%%%%%%%%%%%%%%%%%%%%%%%%%%%%%%%%%%%%%%%%%%%%%%%%%%%%%%%%%%%%%%%%%%%%%%%%%%%%%%
%%%% Problem 4
%%%%%%%%%%%%%%%%%%%%%%%%%%%%%%%%%%%%%%%%%%%%%%%%%%%%%%%%%%%%%%%%%%%%%%%%%%%%%%%
%\subsection{Problem 4}
\problem{4}
\subsubsection{Question}
% Keywords
	\index{unsolved!Fall 2010 II.P4}

Consider a gas of (nonrelativistic) spin $1/2$ fermions, of mass $m$, in the volume $V$. Denote by $n_\pm$ the densities of the fermions with spin up and spin down, respectively.
\begin{enumerate}
	\item Assuming that the ground state for this system can be described as two Fermi spheres, one for spin up and the other for spin down, find the ground state kinetic energy in terms of $n_+$ and $n_-$.
	\item Assuming now that the average densities deviate slightly from average, $n_\pm\approx(n/2)(1\pm\delta)$, expand the kinetic energy to the lowest nontrivial order in $\delta$.
	\item Assume that the interactions between these fermions can be described in terms of a potential energy term $U$ of the form:
	\begin{equation}
		\frac{U}{V} = \alpha n_+n_-
	\end{equation}
	Add the potential energy to the kinetic energy obtained above, and find the total energy to lowest nontrivial order in $\delta$. Show that for small $\alpha$ one has $\delta= 0$ in the ground state, but that for $\alpha>\alpha_c$ where $\alpha_c$ is a critical value that you should find as a function of $n$, the ground state is ferromagnetic (that is, $\delta \ne 0$).
\end{enumerate}

\subsubsection{Answer}


%%%%%%%%%%%%%%%%%%%%%%%%%%%%%%%%%%%%%%%%%%%%%%%%%%%%%%%%%%%%%%%%%%%%%%%%%%%%%%%
%%%% Problem 5
%%%%%%%%%%%%%%%%%%%%%%%%%%%%%%%%%%%%%%%%%%%%%%%%%%%%%%%%%%%%%%%%%%%%%%%%%%%%%%%
%\subsection{Problem 5}
\problem{5}
\subsubsection{Question}
% Keywords
	\index{unsolved!Fall 2010 II.P5}

A solenoid, of radius a and length $L\gg a$, has n turns per unit length of wire carrying the current $I$. An insulating cylindrical shell with negligible thickness, radius $b > a$, and length $L \gg b$, is placed with the axis of the shell coinciding with the axis of the solenoid. The shell has a total mass $M$ and a total charge $Q$, distributed uniformly on it. The current through the solenoid is reduced from $I = I_0$ to $I = 0$ over some time interval.
\begin{enumerate}
	\item Discuss (very briefly) what happens to the shell during this time interval. Compute the velocity acquired by the shell after the current of the solenoid has been dropped to zero. Ignore any effects of the magnetic field produced by the shell (assume that the shell can move without any mechanical friction).
	\item Answer the same questions in part (i), including the magnetic field induced by the shell. Discuss under which condition the magnetic field produced by the shell can be ignored.
	\item Derive a relation between the initial vector potential of the magnetic field of the solenoid at the radius of the shell and the final velocity of the shell (the approximated one obtained in part (i)). Notice that the vector potential is not uniquely defined from the magnetic field; however, choose the vector potential for which the relation to be derived is as simple as possible.
\end{enumerate}

\subsubsection{Answer}

