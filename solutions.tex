\documentclass[10pt]{article}
%%
% Collection of LaTeX input files which provide many convenient changes to the
% default environment.
%
% See also:
%    defaults-beamer.tex
%    defaults-bib.tex
%    defaults-drawing.tex
%    defaults-fonts.tex
%    defaults-utilities.sty
%
% AUTHORS:      Justin Willmert ‹justin@jdjlab.com›
% SOURCES:      http://www.jdjlab.com/hg/latex/
%
% CONTRIBUTORS:
%   1. Ian-Mathew Hornburg ‹imhornburg@gmail.com›
%      (none — by personal correspondence)
%
%==============================================================================
%
% ********************
% CONFIGURABLE OPTIONS
% ********************
%
% There are a few configurable options that if set before the file is loaded
% will override defaults set here. In order to override a feature, simply
% declare a macro with the desired value. For example, to define the default
% font color used throughout the document, one should provide a command such
% as
%     \newcommand{\deftextcolor}{0,0,0,0.5}
%
% TODO: Use pgfkeys as a way to get and set options instead? Maybe something
%       like
%           \def\optdefaults{%
%               textcolor      = {0,0,0,1},
%               stretchyparens = 0
%               ...
%            }
%            %%
% Collection of LaTeX input files which provide many convenient changes to the
% default environment.
%
% See also:
%    defaults-beamer.tex
%    defaults-bib.tex
%    defaults-drawing.tex
%    defaults-fonts.tex
%    defaults-utilities.sty
%
% AUTHORS:      Justin Willmert ‹justin@jdjlab.com›
% SOURCES:      http://www.jdjlab.com/hg/latex/
%
% CONTRIBUTORS:
%   1. Ian-Mathew Hornburg ‹imhornburg@gmail.com›
%      (none — by personal correspondence)
%
%==============================================================================
%
% ********************
% CONFIGURABLE OPTIONS
% ********************
%
% There are a few configurable options that if set before the file is loaded
% will override defaults set here. In order to override a feature, simply
% declare a macro with the desired value. For example, to define the default
% font color used throughout the document, one should provide a command such
% as
%     \newcommand{\deftextcolor}{0,0,0,0.5}
%
% TODO: Use pgfkeys as a way to get and set options instead? Maybe something
%       like
%           \def\optdefaults{%
%               textcolor      = {0,0,0,1},
%               stretchyparens = 0
%               ...
%            }
%            %%
% Collection of LaTeX input files which provide many convenient changes to the
% default environment.
%
% See also:
%    defaults-beamer.tex
%    defaults-bib.tex
%    defaults-drawing.tex
%    defaults-fonts.tex
%    defaults-utilities.sty
%
% AUTHORS:      Justin Willmert ‹justin@jdjlab.com›
% SOURCES:      http://www.jdjlab.com/hg/latex/
%
% CONTRIBUTORS:
%   1. Ian-Mathew Hornburg ‹imhornburg@gmail.com›
%      (none — by personal correspondence)
%
%==============================================================================
%
% ********************
% CONFIGURABLE OPTIONS
% ********************
%
% There are a few configurable options that if set before the file is loaded
% will override defaults set here. In order to override a feature, simply
% declare a macro with the desired value. For example, to define the default
% font color used throughout the document, one should provide a command such
% as
%     \newcommand{\deftextcolor}{0,0,0,0.5}
%
% TODO: Use pgfkeys as a way to get and set options instead? Maybe something
%       like
%           \def\optdefaults{%
%               textcolor      = {0,0,0,1},
%               stretchyparens = 0
%               ...
%            }
%            \input{defaults}
%
% The following options are current recognized and supported:
%
% COMMAND              DEFAULT VALUE        DESCRIPTION
% -----------------------------------------------------------------------------
% \deftextcolor        0,0,0,1              The default text color is any valid
%                                           CMYK color value supported by the
%                                           xcolor package. Currently, no
%                                           support is available for setting
%                                           the default in other color spaces.
%
%
%==============================================================================
%
% *****************
% COMMANDS PROVIDED
% *****************
%
% This section documents some commands which may be necessary due to the
% modifications which are made by default, either overriding a previous change
% locally or restoring the LaTeX default functionality.
%
% COMMAND              DESCRIPTION
% -----------------------------------------------------------------------------
% \definetextcolor     Change the font color of the body text. Attempts are
%                      made to propagate the changes into all environments,
%                      but see the inline documentation for a list of the
%                      exceptions/caveats.
%
% \resetparents        By default, all parentheses in math mode made stretchy,
%                      but this may be undesirable in some context, so this
%                      command restores them to their default, inactive state.
%
%%

%% Provide a package-like environment where internal macros are accessible
\makeatletter

%% Load several commonly-assumed commands or macros to be defined
    \RequirePackage{defaults-utilities}



%% Load all commonly used packages
	\usepackage[
		margin=1in,
		twoside=true
	]{geometry}
	\usepackage{amsmath,amssymb}
	\usepackage[
		usenames,dvipsnames,svgnames,
		table,
		hyperref
	]{xcolor}
	\usepackage{float}
	\usepackage{verbatim}
	\usepackage{array}
	\usepackage{tabularx}
	\usepackage[
		exponent-product     = \cdot,
		load-configurations  = abbreviations,
		per-mode             = symbol-or-fraction,
		separate-uncertainty = true,
		math-micro           = \text{µ},
		text-micro           = µ
	]{siunitx}
	\usepackage[
		font      = footnotesize,
		labelfont = bf
	]{caption}



%% Make some amsmath fixes
%
% As of 2012-12-23 (XeTeX version 3.1415926-2.4-0.9998, TeXLive 2012), XeTeX
% does not properly handle parentheses when using the TeX primitives
%   \overwithdelims
%   \atopwithdelims
%   \abovewithdelims
% Because of this, the macro \binom which internally uses these these commands
% creates a binomial with the wrong size of brackets (too small). Patch around
% this issue by using \left( and \right) around a bracket-less binomial.
% LuaTeX doesn't seem to suffer from this.
\ifxetex
	\renewrobustcmd{\binom} [2]{\left(\genfrac{}{}{0pt}{}{#1}{#2}\right)}
	\renewcommand  {\dbinom}[2]{\left(\genfrac{}{}{0pt}{0}{#1}{#2}\right)}
	\renewcommand  {\tbinom}[2]{\left(\genfrac{}{}{0pt}{1}{#1}{#2}\right)}
\fi



%% Before loading advanced fonts, set up a default color scheme.
	% We choose black by default to correspond to norms, but this default can
	% be overridden by simply defining textcolor before loading this defaults
	% file.
	\providecommand{\deftextcolor}{0,0,0,1}
	\providecolor{textcolor}{cmyk}{\deftextcolor}
	% Now define a command which can be used to change the text colors mid-way
	% through a document. Right now, this doesn't work in all locations (e.g.
	% the body text will change color but section headings don't), but by
	% using a macro, we can potentially update the macro to also affect various
	% text environments.
	%
	% TODO: Colors do not apply to these environments even when the color is
	%       set within the preamble:
	%         · \hrule, \rule
	%
	% TODO: Colors currently will not propagate into the following modes if
	%       this macro is used after font selection/definition has occurred:
	%         · math mode
	%         · section headings
	%
	% \definetextcolor
	%    #1 -> xcolor model-list — See the xcolor documentation for more info
	%    #2 -> xcolor spec-list  — See the xcolor documentation for more info
	\newcommand{\definetextcolor}[2]{%
		% Define/Change the color which textcolor corresponds to
		\definecolor{textcolor}{#1}{#2}
		% If we're in a XeTeX/LuaTeX engine, use fontspec features to update
		% the font color.
		\ifxeluatex
			\addfontfeatures{Color=textcolor}
		\else
			% TODO: Extensive testing in non-XeTeX/LuaTeX environments.
			%       PDFTEX HAS BEEN COMPLETELY UNTESTED!
		\fi
	}

	% Taken from I.-M.:
	% “
	%   Many of these macro patches are gonna take work, since the color
	%   commands aren’t hooking well into the \rule family of commands. They’re
	%   pretty fragile to start, and occasionally \color commands “bleed” their
	%   color past their local contexts.
	%
	%   According to The LaTeX Companion, TeX’s \hrule command is actually used
	%   in the default LaTeX definition of \footnoterule instead of LaTeX’s
	%   \rule command in order to avoid strange effects of \rule, which
	%   actually starts a paragraph and related calculations. KOMA-Script
	%   modifies the footnote definition, but it still appears to be patchable.
	%
	%   If the below patch fails, the undefined macro \xpatchfailed will be
	%   “executed”, resulting in an undefined control sequence error, and will
	%   stop processing. A successful patch will be silent.
	%
	%   [TO-DO] Create a macro that writes a confirmation of the successful
	%           patch either to the terminal or a logfile during compilation.
	% ”
	\usepackage{xpatch}
	\xpretocmd{\footnoterule}{\color{textcolor}}{}{\xpatchfailed}
	% “
	%   Make table/array rules match the ambient text color...
	% ”
	\usepackage{colortbl}
	\AfterPreamble{\arrayrulecolor{textcolor}}



%% Now include advanced font features from defaults-fonts.tex
	\input{defaults-fonts}



%% Taken from I.-M.:
%
% Default glue settings under XeTeX for some fonts sometimes creates very
% cramped text.
%
% From http://tex.stackexchange.com/questions/49298/what-settings-for-xspaceskip-would-you-suggest-for-linux-libertine?rq=1
% Bringhurst suggests these values for body text: m/3 (even better if you can
% keep it at m/4) for interword space, m/2 for maximum space, and m/5 for
% minimum space. He also feels strongly against using the extra space after
% sentences, so assume \frenchspacing.
\appto\selectfont{%
		\fontdimen2\font=0.250em % M/4
		\fontdimen3\font=\dimexpr0.500em-\fontdimen2\font % M/2
		\fontdimen4\font=\dimexpr\fontdimen2\font-0.200em%
	}
    \AfterPreamble{\frenchspacing}



%% Make all parentheses stretchy
%
% By default, all parentheses are just normal parentheses in math mode. This
% to me doesn't seem to make sense since I almost always want them to be
% stretchy.
%
% Adapted from http://tex.stackexchange.com/questions/7359/how-to-make-a-real-backslash-escape-character/7372#7372
	% Start by defining a macro to quickly reset the default behavior if we
	% decide that a given environment shouldn't use the active definition. We
	% need the \edef so that the \the\mathcode is immediately expanded before
	% we change things.
	\edef\resetparens{%
			\mathcode`(=\the\mathcode`(
			\mathcode`)=\the\mathcode`)\relax
		}
	% Now we actually do the magic like this:
	%   1) Start a local group so that we can temporarily define a lowercase
	%      tilde to be the left paranetheses and not affect anything later in
	%      the document.
	%      a) We use the tilde because it is already defined to be an active
	%         character. This saves a step since to define a character to
	%         execute a macro sequence, it first needs to be active. The
	%         \lowercase command does for us since the lowercase tilde (i.e.
	%         paren) is made to have the same category code as its uppercase
	%         counterpart.
	%   2) \lccode assigns the lowercase tilde to be the left paren
	%   3) Then use \lowercase to cause us to define the macro of active left
	%      paren (without explicitly changing category codes).
	%      a) The \endgroup ends the local group we just started, but not
	%         before lowercase has already learned how to convert the tilde
	%         This lets the next \edef to be at the current global scope,
	%         avoiding use of \global
	%   4) Then define the active left paren to be the sequence \left( so that
	%      we invoke the stretchy code
	%   5) Finally change the mathcode of the left paren to be active within
	%      math environments, meaning we don't have to worry about the
	%      character being active in normal text mode.
	\begingroup%
			\lccode`~=`(%
			\lowercase{\endgroup\edef~}{\left(}
		\mathcode`(="8000
	% Do it again for the right paren
	\begingroup%
			\lccode`~=`)%
			\lowercase{\endgroup\edef~}{\right)}
		\mathcode`)="8000
	% Finally, this change of math codes breaks some changes the amsmath
	% package makes to the definitions of \left( and \right), so we fix this
	% by patching the internal amsmath macro which breaks to be prepended with
	% the \resetparens command
	\makeatletter
		\preto{\resetMathstrut@}{\resetparens}
		\makeatother



% Enable the microtypographic extensions
% · Requires that the 2.5 (beta-08) version of microtype be installed. This can
%   be included by loading the tlcontrib repo with tlmgr and upgrading
%   the TeXLive version of microtype
\usepackage[
    protrusion=true,
]{microtype}



%% Enable all the PDF features
	% First patch the author and title macros to let us access their values
	% more easily
	\let\dflttitle\title%
	\let\dfltauthor\author%
	\renewcommand{\title}[1]{\def\dfltthetitle{#1}\dflttitle{#1}}
	\renewcommand{\author}[1]{\def\dflttheauthor{#1}\dfltauthor{#1}}
	% Then make the inclusion of hyperref and hypcap the last things done
	% before the end of the preamble.
	%
	% IMPORTANT NOTE!
	%   It took much debugging to figure out that biblatex makes use of the
	%   \AtEndPreamble command to finalize much of its setup, especially with
	%   respect to knowing if hyperref has been loaded or not. For this reason,
	%   loading hyperref with \AtEndPreamble as well *after* we've already
	%   loaded biblatex means the the code below is executed after code that
	%   biblatex has loaded into the hook.
	%
	%   To guard against such mistakes (and assumptions that hyperref was
	%   loaded in the user-written preamble which will definitely occur
	%   before the hook), we can instead *prepend* our code to the hook that
	%   \AtEndPreamble normally appends to. We do this by patching a copy of
	%   the command to use \gpreto instead of \gappto.
	%
	\let\PreAtEndPreamble\AtEndPreamble%
	\patchcmd{\PreAtEndPreamble}{\gappto}{\gpreto}{}{%
		\PackageError{defaults}{%
			Prepending to the \verb|\AtEndPreamble| hook failed. Could not
			setup hyperref correctly.
		}{%
			Loading hyperref is a tricky business, and our attempt to have
			hyperref load at the correct time when also using biblatex failed.
			A guess would be that the format of the \verb|\AtEndPreamble|
			command has changed, which means our attempt to patch a copy for
			our purposes has failed.
			\MessageBreak\MessageBreak
			See comments in the source for umnthesis.cls for more information.`
		}
	}
	% Now prepend to the \AtEndPreamble hook.
	\PreAtEndPreamble{
		\usepackage[bookmarks,pagebackref=false]{hyperref}
			\usepackage[all]{hypcap}
			\hypersetup{
				,bookmarksopen  = false%
				,pdfborderstyle = {/S/U/W 1}%
				,ocgcolorlinks  = true%
				,colorlinks     = false%
				,setpagesize    = false%
				,pdfstartview   = {XYZ null null 1}%
				,pdfprintscaling= None%
				,pdfpagelayout  = TwoPageLeft%
				,verbose        = true%
			}

		% Taken from I.M.:
		% “
		%   When default colors are set by fontspec, hyperref isn’t able to
		%   color the links at all. The following hack allows hyperref colors
		%   to make it to the top of the stack.
		% ”
		\def\HyColor@@@@UseColor#1\@nil{\addfontfeatures{Color=#1}}

		% Check to see if the author and title macros have been set in the
		% preamble. If they have, then set the PDF attributes accordingly
		\ifdefined\dfltthetitle
			\hypersetup{ pdftitle = {\dfltthetitle} }
		\fi
		\ifdefined\dflttheauthor%
			% Sanitize the author string for use in the PDF properties:
			\begingroup
				% Separate authors with ampersands
				\def\and{ \& }
				\hypersetup{ pdfauthor = {\dflttheauthor} }%
			\endgroup
		\else
			\hypersetup{ pdfauthor = {Justin Willmert} }%
		\fi
	}

%% Hide internal macros again
\makeatletter

%
% The following options are current recognized and supported:
%
% COMMAND              DEFAULT VALUE        DESCRIPTION
% -----------------------------------------------------------------------------
% \deftextcolor        0,0,0,1              The default text color is any valid
%                                           CMYK color value supported by the
%                                           xcolor package. Currently, no
%                                           support is available for setting
%                                           the default in other color spaces.
%
%
%==============================================================================
%
% *****************
% COMMANDS PROVIDED
% *****************
%
% This section documents some commands which may be necessary due to the
% modifications which are made by default, either overriding a previous change
% locally or restoring the LaTeX default functionality.
%
% COMMAND              DESCRIPTION
% -----------------------------------------------------------------------------
% \definetextcolor     Change the font color of the body text. Attempts are
%                      made to propagate the changes into all environments,
%                      but see the inline documentation for a list of the
%                      exceptions/caveats.
%
% \resetparents        By default, all parentheses in math mode made stretchy,
%                      but this may be undesirable in some context, so this
%                      command restores them to their default, inactive state.
%
%%

%% Provide a package-like environment where internal macros are accessible
\makeatletter

%% Load several commonly-assumed commands or macros to be defined
    \RequirePackage{defaults-utilities}



%% Load all commonly used packages
	\usepackage[
		margin=1in,
		twoside=true
	]{geometry}
	\usepackage{amsmath,amssymb}
	\usepackage[
		usenames,dvipsnames,svgnames,
		table,
		hyperref
	]{xcolor}
	\usepackage{float}
	\usepackage{verbatim}
	\usepackage{array}
	\usepackage{tabularx}
	\usepackage[
		exponent-product     = \cdot,
		load-configurations  = abbreviations,
		per-mode             = symbol-or-fraction,
		separate-uncertainty = true,
		math-micro           = \text{µ},
		text-micro           = µ
	]{siunitx}
	\usepackage[
		font      = footnotesize,
		labelfont = bf
	]{caption}



%% Make some amsmath fixes
%
% As of 2012-12-23 (XeTeX version 3.1415926-2.4-0.9998, TeXLive 2012), XeTeX
% does not properly handle parentheses when using the TeX primitives
%   \overwithdelims
%   \atopwithdelims
%   \abovewithdelims
% Because of this, the macro \binom which internally uses these these commands
% creates a binomial with the wrong size of brackets (too small). Patch around
% this issue by using \left( and \right) around a bracket-less binomial.
% LuaTeX doesn't seem to suffer from this.
\ifxetex
	\renewrobustcmd{\binom} [2]{\left(\genfrac{}{}{0pt}{}{#1}{#2}\right)}
	\renewcommand  {\dbinom}[2]{\left(\genfrac{}{}{0pt}{0}{#1}{#2}\right)}
	\renewcommand  {\tbinom}[2]{\left(\genfrac{}{}{0pt}{1}{#1}{#2}\right)}
\fi



%% Before loading advanced fonts, set up a default color scheme.
	% We choose black by default to correspond to norms, but this default can
	% be overridden by simply defining textcolor before loading this defaults
	% file.
	\providecommand{\deftextcolor}{0,0,0,1}
	\providecolor{textcolor}{cmyk}{\deftextcolor}
	% Now define a command which can be used to change the text colors mid-way
	% through a document. Right now, this doesn't work in all locations (e.g.
	% the body text will change color but section headings don't), but by
	% using a macro, we can potentially update the macro to also affect various
	% text environments.
	%
	% TODO: Colors do not apply to these environments even when the color is
	%       set within the preamble:
	%         · \hrule, \rule
	%
	% TODO: Colors currently will not propagate into the following modes if
	%       this macro is used after font selection/definition has occurred:
	%         · math mode
	%         · section headings
	%
	% \definetextcolor
	%    #1 -> xcolor model-list — See the xcolor documentation for more info
	%    #2 -> xcolor spec-list  — See the xcolor documentation for more info
	\newcommand{\definetextcolor}[2]{%
		% Define/Change the color which textcolor corresponds to
		\definecolor{textcolor}{#1}{#2}
		% If we're in a XeTeX/LuaTeX engine, use fontspec features to update
		% the font color.
		\ifxeluatex
			\addfontfeatures{Color=textcolor}
		\else
			% TODO: Extensive testing in non-XeTeX/LuaTeX environments.
			%       PDFTEX HAS BEEN COMPLETELY UNTESTED!
		\fi
	}

	% Taken from I.-M.:
	% “
	%   Many of these macro patches are gonna take work, since the color
	%   commands aren’t hooking well into the \rule family of commands. They’re
	%   pretty fragile to start, and occasionally \color commands “bleed” their
	%   color past their local contexts.
	%
	%   According to The LaTeX Companion, TeX’s \hrule command is actually used
	%   in the default LaTeX definition of \footnoterule instead of LaTeX’s
	%   \rule command in order to avoid strange effects of \rule, which
	%   actually starts a paragraph and related calculations. KOMA-Script
	%   modifies the footnote definition, but it still appears to be patchable.
	%
	%   If the below patch fails, the undefined macro \xpatchfailed will be
	%   “executed”, resulting in an undefined control sequence error, and will
	%   stop processing. A successful patch will be silent.
	%
	%   [TO-DO] Create a macro that writes a confirmation of the successful
	%           patch either to the terminal or a logfile during compilation.
	% ”
	\usepackage{xpatch}
	\xpretocmd{\footnoterule}{\color{textcolor}}{}{\xpatchfailed}
	% “
	%   Make table/array rules match the ambient text color...
	% ”
	\usepackage{colortbl}
	\AfterPreamble{\arrayrulecolor{textcolor}}



%% Now include advanced font features from defaults-fonts.tex
	%%
% Package which provides some minimal common elements for use by all defaults-*
% files.
%
% See defaults.tex for more info.
%
% AUTHORS:      Justin Willmert ‹justin@jdjlab.com›
% SOURCES:      http://jmert.jdjlab.com/cgit/latex.git
%
% CONTRIBUTORS:
%   1. Ian-Mathew Hornburg ‹imhornburg@gmail.com›
%      (none — by personal correspondence)
%
%%
\NeedsTeXFormat{LaTeX2e}
\ProvidesPackage{defaults-fonts}
    [2016/04/09 J. Willmert's preample elements]
\RequirePackage{defaults-utilities}


%% Add fancy font features if using an advanced TeX engine.
\ifxeluatex\RequirePackage{fontspec}

    % Set font features which should be true for all fonts loaded by default
    \defaultfontfeatures{
        % Permit using TeX-style ligatures (i.e. --- for em-dash)
        Ligatures = TeX
    }

    % Also allow math to input using Unicode by default
    \RequirePackage{amsmath,amssymb}
    \RequirePackage{unicode-math}

    % Skip resetting the fonts if we're in draft mode
    \ifdraft{\relax}{
        % Use the XITS Math font by default
        \setmathfont[
                % Use italic fonts at all times
                math-style = ISO,
                bold-style = ISO,
                % Force \epsilon,\varepsilon and \phi,\varphi match the unicode
                % character ε,ϵ and φ,ϕ
                %vargreek-shape = unicode
            ]{XITS Math}
        
        % Additionally, use an alternate stylistic set for caligraphic letters
        % so that they are different from the script letters
        \setmathfont[range={\mathcal,\mathbfcal},StylisticSet=1]{XITS Math}
        
        % The author's opinion is that the STIX summation symbol is highly
        % inferior to those in Computer Modern/Latin Modern. Use CM since it
        % provides a symbol which is slightly larger than LM, and therefore
        % matches the rest of STIX a bit better.
        \setmathfont[range={"220F}]{Latin Modern Math}  % Product
        \setmathfont[range={"2210}]{Latin Modern Math}  % Coproduct
        \setmathfont[range={"2211}]{Latin Modern Math}  % Summation
        % Use the Asana Math for \top (Unicode Down Tack) since it's slightly
        % better when used as a transpose symbol.
        \setmathfont[range={"22A4}]{Asana Math}  % Down Tack
        % The beginner of the STIX \ell loop starts far too high up (close to
        % half-way up the character height), so replace it. Again, LM has the
        % right shape, but in this case, it's a bit too light compared to the
        % rest of STIX glyphs. Asana Math is a reasonable compromise.
        \setmathfont[range={"2113}]{Asana Math}  % Small Script L

        % Have the Linux Libertine family of fonts provide the default serif
        % and sans-serif fonts
        \setmainfont{Linux Libertine O}
        \setsansfont{Linux Biolinum O}
    } %\ifdraft

\else %\ifxeluatex

    % Allow pdfLaTeX to parse Unicode without choking. Note, though, that a
    % document *probably* still won't compile since the mapping from character
    % code to font glyph isn't setup by this. Additional work is needed yet
    % before this can happen automatically (if ever).
    %
    % Currently, my use is to allow \ifxeluatex ... \else ... \fi guards to be
    % used around tricky spots and not have pdflatex completely barf as it
    % parses those.
    \RequirePackage[T1]{fontenc}
    \RequirePackage[utf8]{inputenc}

    \ifdraft{\relax}{
        % Also use Linux Libertine fonts in pdflatex mode, but do this through
        % the libertine package.
        \RequirePackage{libertine}
        % Also choose to use use the STIX fonts like the XeLaTeX version
        \RequirePackage[notext]{stix}

        % Similar font substitutions as in the Xe/LuaLaTeX cases above.

        % Summation symbol substitution:
        %     Loads Computer Modern instead of Latin Modern
        \DeclareSymbolFont{cmlargesymbols}{OMX}{cmex}{m}{n}
        %     Map the summation operator to use the CM glyph
        \DeclareMathSymbol{\sumop}{\mathop}{cmlargesymbols}{"50}

        % Small Script L substitution:
        %     Loads Palatino instead of Asana Math
        \DeclareSymbolFont{plletters}{OML}{zplm}{m}{it}
        \DeclareMathSymbol{\ell}{\mathalpha}{plletters}{"60}

        % Inconsolata is sized much more appropriately for the other chosen
        % fonts
        \RequirePackage{inconsolata}
    }

\fi %\ifxeluatex

\endinput





%% Taken from I.-M.:
%
% Default glue settings under XeTeX for some fonts sometimes creates very
% cramped text.
%
% From http://tex.stackexchange.com/questions/49298/what-settings-for-xspaceskip-would-you-suggest-for-linux-libertine?rq=1
% Bringhurst suggests these values for body text: m/3 (even better if you can
% keep it at m/4) for interword space, m/2 for maximum space, and m/5 for
% minimum space. He also feels strongly against using the extra space after
% sentences, so assume \frenchspacing.
\appto\selectfont{%
		\fontdimen2\font=0.250em % M/4
		\fontdimen3\font=\dimexpr0.500em-\fontdimen2\font % M/2
		\fontdimen4\font=\dimexpr\fontdimen2\font-0.200em%
	}
    \AfterPreamble{\frenchspacing}



%% Make all parentheses stretchy
%
% By default, all parentheses are just normal parentheses in math mode. This
% to me doesn't seem to make sense since I almost always want them to be
% stretchy.
%
% Adapted from http://tex.stackexchange.com/questions/7359/how-to-make-a-real-backslash-escape-character/7372#7372
	% Start by defining a macro to quickly reset the default behavior if we
	% decide that a given environment shouldn't use the active definition. We
	% need the \edef so that the \the\mathcode is immediately expanded before
	% we change things.
	\edef\resetparens{%
			\mathcode`(=\the\mathcode`(
			\mathcode`)=\the\mathcode`)\relax
		}
	% Now we actually do the magic like this:
	%   1) Start a local group so that we can temporarily define a lowercase
	%      tilde to be the left paranetheses and not affect anything later in
	%      the document.
	%      a) We use the tilde because it is already defined to be an active
	%         character. This saves a step since to define a character to
	%         execute a macro sequence, it first needs to be active. The
	%         \lowercase command does for us since the lowercase tilde (i.e.
	%         paren) is made to have the same category code as its uppercase
	%         counterpart.
	%   2) \lccode assigns the lowercase tilde to be the left paren
	%   3) Then use \lowercase to cause us to define the macro of active left
	%      paren (without explicitly changing category codes).
	%      a) The \endgroup ends the local group we just started, but not
	%         before lowercase has already learned how to convert the tilde
	%         This lets the next \edef to be at the current global scope,
	%         avoiding use of \global
	%   4) Then define the active left paren to be the sequence \left( so that
	%      we invoke the stretchy code
	%   5) Finally change the mathcode of the left paren to be active within
	%      math environments, meaning we don't have to worry about the
	%      character being active in normal text mode.
	\begingroup%
			\lccode`~=`(%
			\lowercase{\endgroup\edef~}{\left(}
		\mathcode`(="8000
	% Do it again for the right paren
	\begingroup%
			\lccode`~=`)%
			\lowercase{\endgroup\edef~}{\right)}
		\mathcode`)="8000
	% Finally, this change of math codes breaks some changes the amsmath
	% package makes to the definitions of \left( and \right), so we fix this
	% by patching the internal amsmath macro which breaks to be prepended with
	% the \resetparens command
	\makeatletter
		\preto{\resetMathstrut@}{\resetparens}
		\makeatother



% Enable the microtypographic extensions
% · Requires that the 2.5 (beta-08) version of microtype be installed. This can
%   be included by loading the tlcontrib repo with tlmgr and upgrading
%   the TeXLive version of microtype
\usepackage[
    protrusion=true,
]{microtype}



%% Enable all the PDF features
	% First patch the author and title macros to let us access their values
	% more easily
	\let\dflttitle\title%
	\let\dfltauthor\author%
	\renewcommand{\title}[1]{\def\dfltthetitle{#1}\dflttitle{#1}}
	\renewcommand{\author}[1]{\def\dflttheauthor{#1}\dfltauthor{#1}}
	% Then make the inclusion of hyperref and hypcap the last things done
	% before the end of the preamble.
	%
	% IMPORTANT NOTE!
	%   It took much debugging to figure out that biblatex makes use of the
	%   \AtEndPreamble command to finalize much of its setup, especially with
	%   respect to knowing if hyperref has been loaded or not. For this reason,
	%   loading hyperref with \AtEndPreamble as well *after* we've already
	%   loaded biblatex means the the code below is executed after code that
	%   biblatex has loaded into the hook.
	%
	%   To guard against such mistakes (and assumptions that hyperref was
	%   loaded in the user-written preamble which will definitely occur
	%   before the hook), we can instead *prepend* our code to the hook that
	%   \AtEndPreamble normally appends to. We do this by patching a copy of
	%   the command to use \gpreto instead of \gappto.
	%
	\let\PreAtEndPreamble\AtEndPreamble%
	\patchcmd{\PreAtEndPreamble}{\gappto}{\gpreto}{}{%
		\PackageError{defaults}{%
			Prepending to the \verb|\AtEndPreamble| hook failed. Could not
			setup hyperref correctly.
		}{%
			Loading hyperref is a tricky business, and our attempt to have
			hyperref load at the correct time when also using biblatex failed.
			A guess would be that the format of the \verb|\AtEndPreamble|
			command has changed, which means our attempt to patch a copy for
			our purposes has failed.
			\MessageBreak\MessageBreak
			See comments in the source for umnthesis.cls for more information.`
		}
	}
	% Now prepend to the \AtEndPreamble hook.
	\PreAtEndPreamble{
		\usepackage[bookmarks,pagebackref=false]{hyperref}
			\usepackage[all]{hypcap}
			\hypersetup{
				,bookmarksopen  = false%
				,pdfborderstyle = {/S/U/W 1}%
				,ocgcolorlinks  = true%
				,colorlinks     = false%
				,setpagesize    = false%
				,pdfstartview   = {XYZ null null 1}%
				,pdfprintscaling= None%
				,pdfpagelayout  = TwoPageLeft%
				,verbose        = true%
			}

		% Taken from I.M.:
		% “
		%   When default colors are set by fontspec, hyperref isn’t able to
		%   color the links at all. The following hack allows hyperref colors
		%   to make it to the top of the stack.
		% ”
		\def\HyColor@@@@UseColor#1\@nil{\addfontfeatures{Color=#1}}

		% Check to see if the author and title macros have been set in the
		% preamble. If they have, then set the PDF attributes accordingly
		\ifdefined\dfltthetitle
			\hypersetup{ pdftitle = {\dfltthetitle} }
		\fi
		\ifdefined\dflttheauthor%
			% Sanitize the author string for use in the PDF properties:
			\begingroup
				% Separate authors with ampersands
				\def\and{ \& }
				\hypersetup{ pdfauthor = {\dflttheauthor} }%
			\endgroup
		\else
			\hypersetup{ pdfauthor = {Justin Willmert} }%
		\fi
	}

%% Hide internal macros again
\makeatletter

%
% The following options are current recognized and supported:
%
% COMMAND              DEFAULT VALUE        DESCRIPTION
% -----------------------------------------------------------------------------
% \deftextcolor        0,0,0,1              The default text color is any valid
%                                           CMYK color value supported by the
%                                           xcolor package. Currently, no
%                                           support is available for setting
%                                           the default in other color spaces.
%
%
%==============================================================================
%
% *****************
% COMMANDS PROVIDED
% *****************
%
% This section documents some commands which may be necessary due to the
% modifications which are made by default, either overriding a previous change
% locally or restoring the LaTeX default functionality.
%
% COMMAND              DESCRIPTION
% -----------------------------------------------------------------------------
% \definetextcolor     Change the font color of the body text. Attempts are
%                      made to propagate the changes into all environments,
%                      but see the inline documentation for a list of the
%                      exceptions/caveats.
%
% \resetparents        By default, all parentheses in math mode made stretchy,
%                      but this may be undesirable in some context, so this
%                      command restores them to their default, inactive state.
%
%%

%% Provide a package-like environment where internal macros are accessible
\makeatletter

%% Load several commonly-assumed commands or macros to be defined
    \RequirePackage{defaults-utilities}



%% Load all commonly used packages
	\usepackage[
		margin=1in,
		twoside=true
	]{geometry}
	\usepackage{amsmath,amssymb}
	\usepackage[
		usenames,dvipsnames,svgnames,
		table,
		hyperref
	]{xcolor}
	\usepackage{float}
	\usepackage{verbatim}
	\usepackage{array}
	\usepackage{tabularx}
	\usepackage[
		exponent-product     = \cdot,
		load-configurations  = abbreviations,
		per-mode             = symbol-or-fraction,
		separate-uncertainty = true,
		math-micro           = \text{µ},
		text-micro           = µ
	]{siunitx}
	\usepackage[
		font      = footnotesize,
		labelfont = bf
	]{caption}



%% Make some amsmath fixes
%
% As of 2012-12-23 (XeTeX version 3.1415926-2.4-0.9998, TeXLive 2012), XeTeX
% does not properly handle parentheses when using the TeX primitives
%   \overwithdelims
%   \atopwithdelims
%   \abovewithdelims
% Because of this, the macro \binom which internally uses these these commands
% creates a binomial with the wrong size of brackets (too small). Patch around
% this issue by using \left( and \right) around a bracket-less binomial.
% LuaTeX doesn't seem to suffer from this.
\ifxetex
	\renewrobustcmd{\binom} [2]{\left(\genfrac{}{}{0pt}{}{#1}{#2}\right)}
	\renewcommand  {\dbinom}[2]{\left(\genfrac{}{}{0pt}{0}{#1}{#2}\right)}
	\renewcommand  {\tbinom}[2]{\left(\genfrac{}{}{0pt}{1}{#1}{#2}\right)}
\fi



%% Before loading advanced fonts, set up a default color scheme.
	% We choose black by default to correspond to norms, but this default can
	% be overridden by simply defining textcolor before loading this defaults
	% file.
	\providecommand{\deftextcolor}{0,0,0,1}
	\providecolor{textcolor}{cmyk}{\deftextcolor}
	% Now define a command which can be used to change the text colors mid-way
	% through a document. Right now, this doesn't work in all locations (e.g.
	% the body text will change color but section headings don't), but by
	% using a macro, we can potentially update the macro to also affect various
	% text environments.
	%
	% TODO: Colors do not apply to these environments even when the color is
	%       set within the preamble:
	%         · \hrule, \rule
	%
	% TODO: Colors currently will not propagate into the following modes if
	%       this macro is used after font selection/definition has occurred:
	%         · math mode
	%         · section headings
	%
	% \definetextcolor
	%    #1 -> xcolor model-list — See the xcolor documentation for more info
	%    #2 -> xcolor spec-list  — See the xcolor documentation for more info
	\newcommand{\definetextcolor}[2]{%
		% Define/Change the color which textcolor corresponds to
		\definecolor{textcolor}{#1}{#2}
		% If we're in a XeTeX/LuaTeX engine, use fontspec features to update
		% the font color.
		\ifxeluatex
			\addfontfeatures{Color=textcolor}
		\else
			% TODO: Extensive testing in non-XeTeX/LuaTeX environments.
			%       PDFTEX HAS BEEN COMPLETELY UNTESTED!
		\fi
	}

	% Taken from I.-M.:
	% “
	%   Many of these macro patches are gonna take work, since the color
	%   commands aren’t hooking well into the \rule family of commands. They’re
	%   pretty fragile to start, and occasionally \color commands “bleed” their
	%   color past their local contexts.
	%
	%   According to The LaTeX Companion, TeX’s \hrule command is actually used
	%   in the default LaTeX definition of \footnoterule instead of LaTeX’s
	%   \rule command in order to avoid strange effects of \rule, which
	%   actually starts a paragraph and related calculations. KOMA-Script
	%   modifies the footnote definition, but it still appears to be patchable.
	%
	%   If the below patch fails, the undefined macro \xpatchfailed will be
	%   “executed”, resulting in an undefined control sequence error, and will
	%   stop processing. A successful patch will be silent.
	%
	%   [TO-DO] Create a macro that writes a confirmation of the successful
	%           patch either to the terminal or a logfile during compilation.
	% ”
	\usepackage{xpatch}
	\xpretocmd{\footnoterule}{\color{textcolor}}{}{\xpatchfailed}
	% “
	%   Make table/array rules match the ambient text color...
	% ”
	\usepackage{colortbl}
	\AfterPreamble{\arrayrulecolor{textcolor}}



%% Now include advanced font features from defaults-fonts.tex
	%%
% Package which provides some minimal common elements for use by all defaults-*
% files.
%
% See defaults.tex for more info.
%
% AUTHORS:      Justin Willmert ‹justin@jdjlab.com›
% SOURCES:      http://jmert.jdjlab.com/cgit/latex.git
%
% CONTRIBUTORS:
%   1. Ian-Mathew Hornburg ‹imhornburg@gmail.com›
%      (none — by personal correspondence)
%
%%
\NeedsTeXFormat{LaTeX2e}
\ProvidesPackage{defaults-fonts}
    [2016/04/09 J. Willmert's preample elements]
\RequirePackage{defaults-utilities}


%% Add fancy font features if using an advanced TeX engine.
\ifxeluatex\RequirePackage{fontspec}

    % Set font features which should be true for all fonts loaded by default
    \defaultfontfeatures{
        % Permit using TeX-style ligatures (i.e. --- for em-dash)
        Ligatures = TeX
    }

    % Also allow math to input using Unicode by default
    \RequirePackage{amsmath,amssymb}
    \RequirePackage{unicode-math}

    % Skip resetting the fonts if we're in draft mode
    \ifdraft{\relax}{
        % Use the XITS Math font by default
        \setmathfont[
                % Use italic fonts at all times
                math-style = ISO,
                bold-style = ISO,
                % Force \epsilon,\varepsilon and \phi,\varphi match the unicode
                % character ε,ϵ and φ,ϕ
                %vargreek-shape = unicode
            ]{XITS Math}
        
        % Additionally, use an alternate stylistic set for caligraphic letters
        % so that they are different from the script letters
        \setmathfont[range={\mathcal,\mathbfcal},StylisticSet=1]{XITS Math}
        
        % The author's opinion is that the STIX summation symbol is highly
        % inferior to those in Computer Modern/Latin Modern. Use CM since it
        % provides a symbol which is slightly larger than LM, and therefore
        % matches the rest of STIX a bit better.
        \setmathfont[range={"220F}]{Latin Modern Math}  % Product
        \setmathfont[range={"2210}]{Latin Modern Math}  % Coproduct
        \setmathfont[range={"2211}]{Latin Modern Math}  % Summation
        % Use the Asana Math for \top (Unicode Down Tack) since it's slightly
        % better when used as a transpose symbol.
        \setmathfont[range={"22A4}]{Asana Math}  % Down Tack
        % The beginner of the STIX \ell loop starts far too high up (close to
        % half-way up the character height), so replace it. Again, LM has the
        % right shape, but in this case, it's a bit too light compared to the
        % rest of STIX glyphs. Asana Math is a reasonable compromise.
        \setmathfont[range={"2113}]{Asana Math}  % Small Script L

        % Have the Linux Libertine family of fonts provide the default serif
        % and sans-serif fonts
        \setmainfont{Linux Libertine O}
        \setsansfont{Linux Biolinum O}
    } %\ifdraft

\else %\ifxeluatex

    % Allow pdfLaTeX to parse Unicode without choking. Note, though, that a
    % document *probably* still won't compile since the mapping from character
    % code to font glyph isn't setup by this. Additional work is needed yet
    % before this can happen automatically (if ever).
    %
    % Currently, my use is to allow \ifxeluatex ... \else ... \fi guards to be
    % used around tricky spots and not have pdflatex completely barf as it
    % parses those.
    \RequirePackage[T1]{fontenc}
    \RequirePackage[utf8]{inputenc}

    \ifdraft{\relax}{
        % Also use Linux Libertine fonts in pdflatex mode, but do this through
        % the libertine package.
        \RequirePackage{libertine}
        % Also choose to use use the STIX fonts like the XeLaTeX version
        \RequirePackage[notext]{stix}

        % Similar font substitutions as in the Xe/LuaLaTeX cases above.

        % Summation symbol substitution:
        %     Loads Computer Modern instead of Latin Modern
        \DeclareSymbolFont{cmlargesymbols}{OMX}{cmex}{m}{n}
        %     Map the summation operator to use the CM glyph
        \DeclareMathSymbol{\sumop}{\mathop}{cmlargesymbols}{"50}

        % Small Script L substitution:
        %     Loads Palatino instead of Asana Math
        \DeclareSymbolFont{plletters}{OML}{zplm}{m}{it}
        \DeclareMathSymbol{\ell}{\mathalpha}{plletters}{"60}

        % Inconsolata is sized much more appropriately for the other chosen
        % fonts
        \RequirePackage{inconsolata}
    }

\fi %\ifxeluatex

\endinput





%% Taken from I.-M.:
%
% Default glue settings under XeTeX for some fonts sometimes creates very
% cramped text.
%
% From http://tex.stackexchange.com/questions/49298/what-settings-for-xspaceskip-would-you-suggest-for-linux-libertine?rq=1
% Bringhurst suggests these values for body text: m/3 (even better if you can
% keep it at m/4) for interword space, m/2 for maximum space, and m/5 for
% minimum space. He also feels strongly against using the extra space after
% sentences, so assume \frenchspacing.
\appto\selectfont{%
		\fontdimen2\font=0.250em % M/4
		\fontdimen3\font=\dimexpr0.500em-\fontdimen2\font % M/2
		\fontdimen4\font=\dimexpr\fontdimen2\font-0.200em%
	}
    \AfterPreamble{\frenchspacing}



%% Make all parentheses stretchy
%
% By default, all parentheses are just normal parentheses in math mode. This
% to me doesn't seem to make sense since I almost always want them to be
% stretchy.
%
% Adapted from http://tex.stackexchange.com/questions/7359/how-to-make-a-real-backslash-escape-character/7372#7372
	% Start by defining a macro to quickly reset the default behavior if we
	% decide that a given environment shouldn't use the active definition. We
	% need the \edef so that the \the\mathcode is immediately expanded before
	% we change things.
	\edef\resetparens{%
			\mathcode`(=\the\mathcode`(
			\mathcode`)=\the\mathcode`)\relax
		}
	% Now we actually do the magic like this:
	%   1) Start a local group so that we can temporarily define a lowercase
	%      tilde to be the left paranetheses and not affect anything later in
	%      the document.
	%      a) We use the tilde because it is already defined to be an active
	%         character. This saves a step since to define a character to
	%         execute a macro sequence, it first needs to be active. The
	%         \lowercase command does for us since the lowercase tilde (i.e.
	%         paren) is made to have the same category code as its uppercase
	%         counterpart.
	%   2) \lccode assigns the lowercase tilde to be the left paren
	%   3) Then use \lowercase to cause us to define the macro of active left
	%      paren (without explicitly changing category codes).
	%      a) The \endgroup ends the local group we just started, but not
	%         before lowercase has already learned how to convert the tilde
	%         This lets the next \edef to be at the current global scope,
	%         avoiding use of \global
	%   4) Then define the active left paren to be the sequence \left( so that
	%      we invoke the stretchy code
	%   5) Finally change the mathcode of the left paren to be active within
	%      math environments, meaning we don't have to worry about the
	%      character being active in normal text mode.
	\begingroup%
			\lccode`~=`(%
			\lowercase{\endgroup\edef~}{\left(}
		\mathcode`(="8000
	% Do it again for the right paren
	\begingroup%
			\lccode`~=`)%
			\lowercase{\endgroup\edef~}{\right)}
		\mathcode`)="8000
	% Finally, this change of math codes breaks some changes the amsmath
	% package makes to the definitions of \left( and \right), so we fix this
	% by patching the internal amsmath macro which breaks to be prepended with
	% the \resetparens command
	\makeatletter
		\preto{\resetMathstrut@}{\resetparens}
		\makeatother



% Enable the microtypographic extensions
% · Requires that the 2.5 (beta-08) version of microtype be installed. This can
%   be included by loading the tlcontrib repo with tlmgr and upgrading
%   the TeXLive version of microtype
\usepackage[
    protrusion=true,
]{microtype}



%% Enable all the PDF features
	% First patch the author and title macros to let us access their values
	% more easily
	\let\dflttitle\title%
	\let\dfltauthor\author%
	\renewcommand{\title}[1]{\def\dfltthetitle{#1}\dflttitle{#1}}
	\renewcommand{\author}[1]{\def\dflttheauthor{#1}\dfltauthor{#1}}
	% Then make the inclusion of hyperref and hypcap the last things done
	% before the end of the preamble.
	%
	% IMPORTANT NOTE!
	%   It took much debugging to figure out that biblatex makes use of the
	%   \AtEndPreamble command to finalize much of its setup, especially with
	%   respect to knowing if hyperref has been loaded or not. For this reason,
	%   loading hyperref with \AtEndPreamble as well *after* we've already
	%   loaded biblatex means the the code below is executed after code that
	%   biblatex has loaded into the hook.
	%
	%   To guard against such mistakes (and assumptions that hyperref was
	%   loaded in the user-written preamble which will definitely occur
	%   before the hook), we can instead *prepend* our code to the hook that
	%   \AtEndPreamble normally appends to. We do this by patching a copy of
	%   the command to use \gpreto instead of \gappto.
	%
	\let\PreAtEndPreamble\AtEndPreamble%
	\patchcmd{\PreAtEndPreamble}{\gappto}{\gpreto}{}{%
		\PackageError{defaults}{%
			Prepending to the \verb|\AtEndPreamble| hook failed. Could not
			setup hyperref correctly.
		}{%
			Loading hyperref is a tricky business, and our attempt to have
			hyperref load at the correct time when also using biblatex failed.
			A guess would be that the format of the \verb|\AtEndPreamble|
			command has changed, which means our attempt to patch a copy for
			our purposes has failed.
			\MessageBreak\MessageBreak
			See comments in the source for umnthesis.cls for more information.`
		}
	}
	% Now prepend to the \AtEndPreamble hook.
	\PreAtEndPreamble{
		\usepackage[bookmarks,pagebackref=false]{hyperref}
			\usepackage[all]{hypcap}
			\hypersetup{
				,bookmarksopen  = false%
				,pdfborderstyle = {/S/U/W 1}%
				,ocgcolorlinks  = true%
				,colorlinks     = false%
				,setpagesize    = false%
				,pdfstartview   = {XYZ null null 1}%
				,pdfprintscaling= None%
				,pdfpagelayout  = TwoPageLeft%
				,verbose        = true%
			}

		% Taken from I.M.:
		% “
		%   When default colors are set by fontspec, hyperref isn’t able to
		%   color the links at all. The following hack allows hyperref colors
		%   to make it to the top of the stack.
		% ”
		\def\HyColor@@@@UseColor#1\@nil{\addfontfeatures{Color=#1}}

		% Check to see if the author and title macros have been set in the
		% preamble. If they have, then set the PDF attributes accordingly
		\ifdefined\dfltthetitle
			\hypersetup{ pdftitle = {\dfltthetitle} }
		\fi
		\ifdefined\dflttheauthor%
			% Sanitize the author string for use in the PDF properties:
			\begingroup
				% Separate authors with ampersands
				\def\and{ \& }
				\hypersetup{ pdfauthor = {\dflttheauthor} }%
			\endgroup
		\else
			\hypersetup{ pdfauthor = {Justin Willmert} }%
		\fi
	}

%% Hide internal macros again
\makeatletter

%%
% Collection of LaTeX input files which provide many convenient changes to the
% default environment.
%
% See defaults.tex for more info.
%
% AUTHORS:      Justin Willmert ‹justin@jdjlab.com›
% SOURCES:      http://www.jdjlab.com/hg/latex/
%
% CONTRIBUTORS:
%   1. Ian-Mathew Hornburg ‹imhornburg@gmail.com›
%      (none — by personal correspondence)
%
%==============================================================================
%
% ********************
% CONFIGURABLE OPTIONS
% ********************
%
% 1. To enable the tightpages environment, add the line
%        \PassOptionsToPackage{active}{preview}
%    before loading this file.
%
%%

%% Provide a package-like environment where internal macros are accessible
\makeatletter

%% Load TikZ and pgfplots libraries
\usepackage{pgfplots}
	\usetikzlibrary{calc}
	% Use the command \tikzexternalize to enable externalization
	\usetikzlibrary{external}
	% Then set the external call to match the currently running invocation of
	% TeX. Note that the extra spaces are critical since LaTeX gobbles the
	% separating space between \externlatex and \tikzexternalcheckshellescape,
	% so we include it in the command so that the system call works.
	\ifxetex
		\def\externlatex{xelatex }
	\else
		\ifluatex
			\def\externlatex{lualatex }
		\else
			\def\externlatex{pdflatex }
		\fi
	\fi
	% Now set the external command to use the appropriate LaTeX engine
	\tikzset{external/system call={\externlatex \tikzexternalcheckshellescape
		-halt-on-error -interaction=batchmode -jobname "\image" "\texsource"}}

	% It is assumed that defaults.tex is loaded before this file, so also
	% patch tikzpicture environments to respect the change in font colors
	%
	% Taken from I.-M.:
	\AtBeginEnvironment{tikzpicture}{\color{textcolor}}


%% Enable use of the preview package to isolate environments as PDF pages
\usepackage[
		tightpages
	]{preview}

%% Hide internal macros again
\makeatletter

\usetikzlibrary{intersections}

\title{University of Minnesota Physics GWE Solutions}
\author{%
	Justin Willmert%
}

% Extend the capabilities of enumerations and itemized lists
	\usepackage{enumerate}
% Multicolumn pages
	\usepackage{multicol}
% Circuit drawing
	\usepackage[american,siunitx]{circuitikz}
% Include more unit types
	\DeclareSIUnit{\cal}{cal}
	\DeclareSIUnit{\minute}{min}
	\DeclareSIUnit{\year}{yr}
% Allow subfigures
	\usepackage{subcaption}

% Generate an index of all the GWE problems
\usepackage{makeidx}
	\makeindex
	% Change the index printing so that multicol can balance the columns
	\let\origtheindex\theindex
	\let\origendtheindex\endtheindex
	\def\theindex{%
		\def\twocolumn{\begin{multicols}{2}}%
		\def\onecolumn{}%
		%~ \clearpage
		\origtheindex
	}
	\def\endtheindex{%
		\end{multicols}%
		\origendtheindex
	}

\usepackage[parfill]{parskip}

%% Setup page styles
	% Fancy headers
	\usepackage{fancyhdr}
	\fancypagestyle{plain}{%
		\fancyhf{}	% Clear all headers and footers
		\renewcommand{\headrulewidth}{0pt}
		\renewcommand{\footrulewidth}{0pt}
		\fancyhfoffset{0.25in}
		% Add the page number on the outside of each page and the current
		% exam (section) on the inner header
		\fancyhead[LE,RO]{\sffamily\textbf\thepage}
		\fancyhead[RE,LO]{\sffamily\textbf\leftmark}
		% Also add some information about where to find the source for these
		% solutions in the footer
		\cfoot{%
			\color{black!30!white}
			These solutions are compiled in a \texttt{git} repository available
			at \href{https://github.com/jmert/umn_phys_gwe}
			{https://github.com/jmert/umn\_phys\_gwe}.
		}
	}
	\pagestyle{plain}
	% Since we've added headers and footers which aren't accounted for in the
	% default layout for geometry, setup the new page styles
	\newgeometry{
		twoside     = true,
		includehead = true,
		includefoot = true,
		top         = 0.5in,
		bottom      = 0.5in,
		inner       = 1in,
		outer       = 1in
	}
	% Don't number the sections explicitly, but still use \section, etc
	% (non-starred) so that the TOC is still generated
	\setcounter{secnumdepth}{0}
	% Then have the TOC only show the year and problem numbers without also
	% listing the Question and Answer sections
	\setcounter{tocdepth}{2}

%% Change the heading styles
	\usepackage[sf,bf,compact]{titlesec}
	% Allow use of unstarred sectioning commands without worrying about them
	% adding an unnecessary label before the title. Do this so that the entries
	% are added to the table of contents automatically.
	\titlelabel{}
	\titlespacing*{\section}%
		{-0.25in}		% Create a hanging indent
		{0pt}			% No extra space above the section title
		{0.75em}		% Small amount of space below
	\titlespacing*{\subsection}%
		{-0.125in}		% Create a hanging indent
		{0pt}			% No extra space above the section title
		{0.75em}		% Small amount of space below

%% Make some GWE-specific heading commands
	\makeatletter
	% First, create the requisite counters
	\newcounter{gwe@year}
	\newcounter{gwe@part}
	\newcounter{gwe@prob}[gwe@part]
	% Also an if that signals the first problem in a new exam
	\newif\ifgwe@firstproblem
	% Then add an internal macro which will make the GWE Exam headings
	\newcommand{\gwe@examname}{\relax}
	\newcommand{\gwe@examsymb}{\relax}
	\newcommand{\gwe@exam}[4]{%
		\renewcommand{\gwe@examname}{#1}%
		\renewcommand{\gwe@examsymb}{#2}%
		\setcounter{gwe@year}{#3}%
		\setcounter{gwe@part}{#4}%
		\setcounter{gwe@prob}{0}%
		{%
			% Each exam starts on its own page and is its own paragraph
			\clearpage%
			% Generate the title for the command
			\def\gwe@title{%
				\gwe@examname{}~\arabic{gwe@year} Part~\Roman{gwe@part}%
			}%
			% Update the marks so that the headers show which exam we're in,
			% and add ourselves to the TOC
			\markboth{\scshape\gwe@title}{}%
			\section{\gwe@title}
			% Automatically generate a label for every exam of the form
			%     exam:SYYYYP
			% where S is either S or F for spring and fall, respectively,
			%       YYYY is the exam year
			%       P is the part number (I or II)
			\edef\@currentlabelname{\gwe@title}%
			\edef\gwe@label{\gwe@examsymb\arabic{gwe@year}\Roman{gwe@part}}%
			\label{exam:\gwe@label}%
		}%
		% Signal to \problem that it is the first after a new exam
		\gwe@firstproblemtrue%
	}%
	% Next, add convenience wrappers to quickly define the Spring and Fall
	% exams.
	\newcommand{\fallexam}[2]{\gwe@exam{Fall}{F}{#1}{#2}}%
	\newcommand{\springexam}[2]{\gwe@exam{Spring}{F}{#1}{#2}}%
	% Finally, provide a heading for each problem
	\newcommand{\problem}[1]{%
		\setcounter{gwe@prob}{#1}%
		% Reset the equation numbering counter
		\setcounter{equation}{0}%
		{%
			% New problems also start on their own pages if it's not the first
			% in the exam (still want to share the page with exam title)
			\ifgwe@firstproblem%
				\global\gwe@firstproblemfalse%
			\else%
				\clearpage%
			\fi%
			% Generate the title for the command
			\def\gwe@title{%
				Problem~\arabic{gwe@prob}%
			}%
			\def\gwe@fulltitle{%
				\gwe@examname{}~\arabic{gwe@year} Part~\Roman{gwe@part}, %
				\gwe@title%
			}%
			\subsection{\gwe@title}
			% Automatically generate a label for every problem of the form
			%     prob:SYYYYPNN
			% where S is either S or F for spring and fall, respectively,
			%       YYYY is the exam year
			%       P is the part number (I or II)
			%       NN is the problem number including leading zero
			\edef\nn{\ifnum\value{gwe@prob}<10 0\fi\thegwe@prob}%
			\edef\@currentlabelname{\gwe@fulltitle}%
			\edef\gwe@label{\gwe@examsymb\arabic{gwe@year}\Roman{gwe@part}\nn}%
			\label{prob:\gwe@label}%
		}%
	}
	% Set the equation numbering system:
	\renewcommand{\theequation}{%
		\gwe@examsymb\arabic{gwe@year} \Roman{gwe@part} \arabic{gwe@prob}.\arabic{equation}%
	}
	\makeatother

%% Convenience macros
	% Script L used for Lagrangian
	\newcommand{\sL}{\ensuremath{\mathcal L}}
	% Derivative shortcut
	\newcommand{\dd}{\ensuremath{\,\mathrm{d}}}
	% Quantum bra-ket notation can be annoying, so make that more convenient
	% to type
	% · First, have macros for lone bras and kets
	\newcommand{\bra}[1]{\ensuremath{\left\langle #1 \right|}}
	\newcommand{\ket}[1]{\ensuremath{\left| #1 \right\rangle}}
	% · Next, make a bra-ket pair so that all three < | > are the same sizes
	\newcommand{\braket}[2]{\ensuremath{%
		\left\langle #1 \middle| #2 \right\rangle%
	}}
	% · Also have one that can sandwich an operator expression in the middle.
	\newcommand{\braopket}[3]{\ensuremath{%
		\left\langle #1 \middle| #2 \middle| #3 \right\rangle
	}}
	% · Finally, expectation values
	\newcommand{\expect}[1]{\ensuremath{\left\langle #1 \right\rangle}}

\begin{document}

\begingroup
	% Use smaller fonts for the TOC and index
	\small
	% Patch the table of contents to use double columns. Needs to be done here
	% because something else redefines \@starttoc when the document begins
	% (probably hyperref...)
	\makeatletter
	\let\origstarttoc\@starttoc
	\renewcommand*{\@starttoc}[1]{%
		\begin{multicols}{2}%
			\origstarttoc{#1}%
		\end{multicols}%
	}
	\makeatother
	\tableofcontents

	% Print the Index
	\printindex
	\clearpage
\endgroup

\fallexam{2000}{1}
%%%%%%%%%%%%%%%%%%%%%%%%%%%%%%%%%%%%%%%%%%%%%%%%%%%%%%%%%%%%%%%%%%%%%%%%%%%%%%%
%%%% Problem 8
%%%%%%%%%%%%%%%%%%%%%%%%%%%%%%%%%%%%%%%%%%%%%%%%%%%%%%%%%%%%%%%%%%%%%%%%%%%%%%%
\problem{8}
\subsubsection{Question}
% Keywords
	\index{thermodynamics!Nitrogen velocity}

Estimate (a) the average speed (in \si{\m\per\s}) and (b) the mean free path
(in \si{\m}) of a nitrogen molecule in this room.

\subsubsection{Answer}

\begin{enumerate}[(a)]
	\item
		We relate the kinetic energy of an $N_2$ molecule with the thermal
		energy by the equipartition theorem. Since there are 3 translational
		degrees of freedom,
		\begin{align*}
			\frac 12 mv² &= \frac 32 k_B T \\
			v &= \sqrt{\frac{3 k_B T}{m}}
		\end{align*}
		The mass of the molecule is twice that of a single nitrogen atom
		which is itself about 14 proton masses. Therefore
		\begin{empheq}[box=\fbox]{align}
			v &≈ \sqrt{\frac{3 k_B T}{28 m_p}} \\
			v &≈ \SI{515}{\m\per\s}
		\end{empheq}
	\item
		Two particles collide if they come within $2r₀$ of each other where
		$r₀$ is the typical radius of the particle. For diatomic nitrogen,
		we assume $r₀ ≈ 2a₀$ where $a₀$ is the Bohr radius. Then in the time
		$τ$ that the particle is moving at velocity $⟨v⟩$, the particle can
		collide with any other particle within the swept-out volume
		\begin{align*}
			\mathcal V &= π(2r₀)² ⋅ ⟨v⟩τ
		\end{align*}
		Since there are $n$ particles per unit volume, there are $\mathcal N$
		atoms to collide with:
		\begin{align*}
			\mathcal N &= n\mathcal V = 4πn{r₀}² ⟨v⟩τ
		\end{align*}
		On average then, there are $\mathcal N$ collisions per length $⟨v⟩τ$
		traversed, or in its reciprocal form, the mean free path $λ$ is
		\begin{align*}
			λ &= \frac{1}{4πn{r₀}²}
		\end{align*}
		To estimate the particle density, consider the ideal gas law
		$PV=Nk_BT$. We can assume atmospheric pressure at room temperature,
		so the density is
		\begin{align*}
			n &= \frac N V = \frac{P}{k_B T}
		\end{align*}
		Putting it all together,
		\begin{empheq}[box=\fbox]{align}
			λ &≈ \frac{k_B T}{4π{r₀}²P} \\
			{}&≈ \SI{2.90e-7}{\m}
		\end{empheq}
\end{enumerate}


\springexam{2000}{1}
%%%%%%%%%%%%%%%%%%%%%%%%%%%%%%%%%%%%%%%%%%%%%%%%%%%%%%%%%%%%%%%%%%%%%%%%%%%%%%%
%%%% Problem 1
%%%%%%%%%%%%%%%%%%%%%%%%%%%%%%%%%%%%%%%%%%%%%%%%%%%%%%%%%%%%%%%%%%%%%%%%%%%%%%%
\problem{1}
\subsubsection{Question}
% Keywords
	\index{mechanics!Energy loss in orbital descent}

A satellite of mass $m = \SI{500}{\kg}$ is in a circular orbit at an
altitude $h = \SI{150}{\km}$ above the Earth's surface. As a result of air
friction, the satellite's orbit degrades. Protected by a heat shield, the
satellite eventually impacts with a velocity of $\SI{2}{\km\per\s}$. How
much energy (in Joules) was released as heat in the process?

\subsubsection{Answer}

The solution method will be a simple energy balance equation. In its initial
state, the satellite had gravitation potential energy that contributed to
its energy equal to
\begin{align*}
    V_0 &= \frac{GM_E m}{R_E + h}
\end{align*}
where $M_E$ and $R_E$ are the mass and radius of the Earth. The kinetic energy
can be determined by making use of simple circle relations. The centripetal
force must be provided by the gravitation force, so
\begin{align*}
    \frac{{v_0}^2}{R_E + h} &= \frac{GM_E}{(R_E + h)^2} \\
    v_0 &= \sqrt{\frac{GM_E}{R_E + h}}
\end{align*}
making the initial kinetic energy
\begin{align*}
    T_0 &= \frac{GM_E m}{2(R_E + h)}
\end{align*}

In it's final state, the satellite consists of it's final given velocity's
kinetic energy and more gravitational potential energy (with respect to the
center of the Earth). They are given by
\begin{align*}
    T_f &= \frac 12 m{v_f}^2 \\
    V_f &= \frac{GM_E m}{R_E}
\end{align*}
By conservation of energy, any energy different must result from the loss
of energy between the initial and final states, so the lost energy
$E_\mathrm{loss}$ is
\begin{align*}
    E_\mathrm{loss} &= T_0 + V_0 - T_f - V_f \\
    {} &= \frac{GM_E m}{R_E(R_E + h)}(R_E - 2h) - \frac 12 m{v_f}^2
\end{align*}
Plugging in all given quantities and constants, we have that
\begin{align}
    \boxed{
    E_\mathrm{loss} = \SI{2.816e10}{\J} = \SI{28.16}{\giga\J}
    }
\end{align}

%%%%%%%%%%%%%%%%%%%%%%%%%%%%%%%%%%%%%%%%%%%%%%%%%%%%%%%%%%%%%%%%%%%%%%%%%%%%%%%
%%%% Problem 2
%%%%%%%%%%%%%%%%%%%%%%%%%%%%%%%%%%%%%%%%%%%%%%%%%%%%%%%%%%%%%%%%%%%%%%%%%%%%%%%
\problem{2}
\subsubsection{Question}
% Keywords
    \index{mechanics!Recoiling iron}

What is the velocity of recoil of an ${}^{57}\mathrm{Fe}$ nucleus that emits
a \SI{100}{\keV} photon, both in units of the speed of light in vacuum and in
meters per second.

\subsubsection{Answer}

In the initial state observed from the iron atom's rest frame, the momentum
is zero. After the emission, the photon has a momentum of $p_\gamma = E_\gamma/c$ and
consequently, the nucleus must recoil with momentum $p_{Fe} = -p_\gamma$. Knowing
that the energy is non-relativistic, then
\begin{align*}
    m_{Fe}v_{Fe} &= \frac{E_\gamma }{c} \\
    v_{Fe} &= \frac{E_\gamma }{57m_p c}
\end{align*}
where we've approximate the mass of the iron atom by a multiple of the proton
mass. Plugging in the numbers
\begin{align}
    \boxed{ v_{Fe} = \SI{561.4}{\m\per\s} }
\end{align}
or in units of $c$
\begin{align}
    \boxed{ v_{Fe} = \SI{1.87e-6}{c} }
\end{align}

%%%%%%%%%%%%%%%%%%%%%%%%%%%%%%%%%%%%%%%%%%%%%%%%%%%%%%%%%%%%%%%%%%%%%%%%%%%%%%%
%%%% Problem 3
%%%%%%%%%%%%%%%%%%%%%%%%%%%%%%%%%%%%%%%%%%%%%%%%%%%%%%%%%%%%%%%%%%%%%%%%%%%%%%%
\problem{3}
\subsubsection{Question}
% Keywords
    \index{electrodynamics!Voltage and phase in an RL circuit}
    \index{circuits!Voltage and phase in an RL circuit}

A resistance $R$ and an inductance $L$ are connected in series, and an
alternating voltage $V_0\cos {\omega} t$ is impressed across the combination. The
resulting steady state voltage across the resistance can be written as $V_R
\cos({\omega} t + \beta)$. Find $V_R$ and $\beta$, assuming both $V_0$ and $V_R$ to be
positive.

\subsubsection{Answer}

The problem is simplified by constucting the solution using complex voltages
and currents and recovering the correct component at the end. Since the
source voltage is a cosine, the real component will be kept at the end.

The complex impedance for a resistor is simply the resitance itself, so $Z_R
= R$. For the inductor, it is $Z_L = i{\omega} L$. We then use the complex
impedances together with Ohm's Law and the first Kirchoff rule to solve for
the complex current $I$.
\begin{align*}
    V_0e^{i{\omega} t} &= IR + i{\omega} LI \\
    I &= \frac{V_0}{R+i{\omega} L} e^{i{\omega} t}
\end{align*}
Then by inserting this solution into the voltage law for inductors, we can
solve for the complex voltage across the inductor.
\begin{align*}
    V_L &= L\frac{dI}{dt} \\
    V_L &= V_0 \frac{i{\omega} L}{R+i{\omega} L} e^{i{\omega} t}
\end{align*}

The voltage drops across the resistor and inductor must equal the impressed
voltage, so we can solve for the unknown voltage across the resistor.
\begin{align*}
    V_0e^{i{\omega} t} &= V + V_L \\
    V &= V_0e^{i{\omega} t} - V_0\frac{i{\omega} L}{R+i{\omega} L}e^{i{\omega} t} \\
    V &= V_0 (\frac{R^2 - i{\omega} RL}{R^2 + {\omega}^2L^2}) e^{i{\omega} t}
\end{align*}
In order to make taking the real component simpler, we put the term in
parentheses in complex exponential form according to the relation $z =
|z|e^{i \arg z}$.
\begin{align*}
    \left| \frac{R^2 - i{\omega} RL}{R^2 + {\omega}^2L^2} \right| &=
	\frac{R\sqrt{R^2 + {\omega}^2L^2}}{R^2 + {\omega}^2L^2}
    \\
    \arg (\frac{R^2 - i{\omega} RL}{R^2 + {\omega}^2L^2}) &= \arctan(-\frac{{\omega} L}{R})
\end{align*}
Therefore the voltage across the resistor has the form
\begin{align*}
    V &= V_0 \frac{R\sqrt{R^2 + {\omega}^2L^2}}{R^2 + {\omega}^2L^2} e^{i{\omega} t + \arctan(-{\omega} L/R)} \\
    V &= V_R e^{i{\omega} t + i\beta} \\
    \Re\{V\} &= V_R \cos({\omega} t + \beta)
\end{align*}
where
\begin{align}
    \boxed{ V_R = V_0 \frac{R\sqrt{R^2 + {\omega}^2L^2}}{R^2 + {\omega}^2L^2} }
    \quad\quad
    \boxed{ \beta = \arctan(-\frac{{\omega} L}{R}) \vphantom{\frac{\sqrt{R}}{R}} }
\end{align}

%%%%%%%%%%%%%%%%%%%%%%%%%%%%%%%%%%%%%%%%%%%%%%%%%%%%%%%%%%%%%%%%%%%%%%%%%%%%%%%
%%%% Problem 4
%%%%%%%%%%%%%%%%%%%%%%%%%%%%%%%%%%%%%%%%%%%%%%%%%%%%%%%%%%%%%%%%%%%%%%%%%%%%%%%
\problem{4}
\subsubsection{Question}
% Keywords
    \index{electrodynamics!Flux through a loop}

Find the magnetic flux through a square loop of side $a$ due to current $I$ in
a long straight wire. The geometry is as follows: the wire is coplanar with the
loop and runs parallel to the loop's closest side, at a distance $b$ away.
Write your result as a formula in SI units.

\subsubsection{Answer}

By Ampère's law in integral form, the magnetic field at a radial distance $r$
away from the wire is given by
\begin{align*}
    \vec B = \frac{{\mu}_0I}{2{\pi} r} \hat \phi 
\end{align*}
The flux is then the total magnetic field through the loop. Integrating by
lines of length $a$,
\begin{align*}
    \phi  = a \int_b^{a+b} \frac{{\mu}_0I}{2{\pi} r} dr \\
    \boxed{\phi  = \frac{{\mu}_0Ia}{2{\pi}}\ln(\frac{a+b}{a})}
\end{align*}

%%%%%%%%%%%%%%%%%%%%%%%%%%%%%%%%%%%%%%%%%%%%%%%%%%%%%%%%%%%%%%%%%%%%%%%%%%%%%%%
%%%% Problem 6
%%%%%%%%%%%%%%%%%%%%%%%%%%%%%%%%%%%%%%%%%%%%%%%%%%%%%%%%%%%%%%%%%%%%%%%%%%%%%%%
\problem{6}
\subsubsection{Question}
% Keywords
    \index{mathematics!Complex contour integration}

By actually evaluating the integral, show that
\begin{align*}
    \int_0^\infty  \frac{\cos x}{1 + x^2}\,dx &= \frac{{\pi}}{2e}
\end{align*}

\subsubsection{Answer}

Note that the integrand, so start by changing the limits of integration
\begin{align*}
    \int_0^\infty  \frac{\cos x}{1 + x^2}\,dx &=
	\frac 12 \int_{-\infty }^\infty  \frac{\cos x}{1 + x^2}\,dx
\intertext{Then expand the cosine into is complex exponential definition}
    {} &= \frac 14 (\int_{-\infty }^\infty  \frac{e^{ix}}{1 + x^2}\,dx +
	\int_{-\infty }^\infty  \frac{e^{-ix}}{1 + x^2}\,dx) \\
    {} &= \frac 14 (\int_{-\infty }^\infty  \frac{e^{ix}}{(x+i)(x-i)}\,dx +
	\int_{-\infty }^\infty  \frac{e^{-ix}}{(x+i)(x-i)}\,dx)
\intertext{Complex contour integration lets us evaluate the integrals as
limits of the coefficient of a pole as it approaches the pole, so}
    {} &= \frac 14 (2{\pi} i \cdot  \lim_{z\rightarrow i}(\frac{e^{iz}}{z+i}) - 2{\pi} i  \cdot 
	\lim_{z\rightarrow -i}(\frac{e^{-iz}}{z-i}))
\end{align*}
After simplifying,
\begin{align}
    \boxed{
    \int_0^\infty  \frac{\cos x}{1 + x^2}\,dx = \frac{{\pi}}{2e}
    }
\end{align}

%%%%%%%%%%%%%%%%%%%%%%%%%%%%%%%%%%%%%%%%%%%%%%%%%%%%%%%%%%%%%%%%%%%%%%%%%%%%%%%
%%%% Problem 7
%%%%%%%%%%%%%%%%%%%%%%%%%%%%%%%%%%%%%%%%%%%%%%%%%%%%%%%%%%%%%%%%%%%%%%%%%%%%%%%
\problem{7}
\subsubsection{Question}
% Keywords
    \index{dimensional analysis!High-velocity drag force}

The drag force on a very high speed object of area $A$, passing through a gas
of density $\rho$ at a velocity $v$ is expected to be of the form
\begin{align*}
    \text{Force} \sim A^r \rho^s v^t
\end{align*}
Determine the value of the exponents $r$, $s$, and $t$.

\subsubsection{Answer}

Force needs to have a unit of inverse time squared, therefore $v$ as the only
variable with a time unit sets $t = 2$. Similarly, $\rho $ is the only one with
a mass term, so we also immediately know that $s = 1$. That leaves
\begin{align*}
    \left[ \frac{\text{mass}\cdot \text{distance}}{\text{time}^2} \right]
	&= \left[ \frac{\text{mass}}{\text{distance}\cdot \text{time}^2} \right]
	\left[ \text{distance} \right]^r
\end{align*}
Therefore to have the two side have compatible units, $r = 2$.
\begin{align*}
    \boxed{ r = 2,\quad s = 1,\quad t = 2}
\end{align*}

%%%%%%%%%%%%%%%%%%%%%%%%%%%%%%%%%%%%%%%%%%%%%%%%%%%%%%%%%%%%%%%%%%%%%%%%%%%%%%%
%%%% Problem 9
%%%%%%%%%%%%%%%%%%%%%%%%%%%%%%%%%%%%%%%%%%%%%%%%%%%%%%%%%%%%%%%%%%%%%%%%%%%%%%%
\problem{9}
\subsubsection{Question}
% Keywords
    \index{mechanics!Shallow-water wave group velocity}
    \index{waves!Shallow-water wave group velocity}

For waves in shallow water, the relation between frequency $\nu $ and wavelength
${\lambda}$ is
\begin{align*}
    \nu  &= (\frac{2{\pi} T}{\rho {\lambda}^3 })^{1/2}
\end{align*}
where $\rho $ and $T$ are the density and surface tension of water. What is the
group velocity of these waves?

\subsubsection{Answer}

Transforming relation given to use the angular frequency ${\omega} = 2{\pi}\nu $ and
wave number $k = 2{\pi}/{\lambda}$,
\begin{align*}
    {\omega} &= \sqrt{ \frac{k^3 T}{\rho } }
\end{align*}
Then the group velocity is simply the partial derivative with respect to $k$:
\begin{align*}
    v_g &= \frac{\partial {\omega}}{\partial k} = \frac 32 \sqrt{\frac{kT}{\rho }}
\end{align*}
Putting this back into the form which involves only $\nu $ and ${\lambda}$, we arrive at
the answer
\begin{align}
    \boxed{ v_g = \frac 32 \sqrt{\frac{2{\pi} T}{\rho {\lambda}}} }
\end{align}

%%%%%%%%%%%%%%%%%%%%%%%%%%%%%%%%%%%%%%%%%%%%%%%%%%%%%%%%%%%%%%%%%%%%%%%%%%%%%%%
%%%% Problem 10
%%%%%%%%%%%%%%%%%%%%%%%%%%%%%%%%%%%%%%%%%%%%%%%%%%%%%%%%%%%%%%%%%%%%%%%%%%%%%%%
\problem{10}
\subsubsection{Question}
% Keywords
    \index{mathematics!Eigenvalues and eigenvectors}

Find the eigenvalues and corresponding eigenvectors (which need \emph{not} be
normalized) of the following matrix:
\begin{align*}
    M = \begin{bmatrix}
	1 & 0 & -i \\
	0 & 2 & 0 \\
	i & 0 & -1
    \end{bmatrix}
\end{align*}

\subsubsection{Answer}

The eigenvalue equation is found from the standard procedure of adding a
parameter ${\lambda}$ into the matrix and taking the determinant equal to zero:
\begin{align*}
    \det{\begin{bmatrix}
	1-{\lambda} & 0   & -i   \\
	0   & 2-{\lambda} & 0    \\
	i   & 0   & -1-{\lambda}
    \end{bmatrix}} = 0 &= (1-{\lambda})(2-{\lambda})(-1-{\lambda}) - i(-i)(2-{\lambda}) \\
    0 &= -(1-{\lambda})(1+{\lambda})(2-{\lambda}) - (2-{\lambda}) \\
    0 &= -(2-{\lambda})\qty[ (1-{\lambda})(1+{\lambda}) + 1]
\end{align*}
Solving the two equations $0 = (2 - {\lambda})$ and $(1-{\lambda})(1+{\lambda}) = -1$ gives the three
eigenvalues
\begin{align}
    \boxed{
    {\lambda} = \left\{ -\sqrt 2, \sqrt 2, 2 \right\}
    }
\end{align}

Starting with the eigenvalue ${\lambda}=2$, we solve for its eigenvector using the
usual Gaussian elimination approach:
\begin{align*}
    \det{\begin{bmatrix}
	-1  & 0   & -i   \\
	0   & 0   & 0    \\
	i   & 0   & -3
    \end{bmatrix}}
\end{align*}
The second variable is completely unconstrained, so we can set that component
of the vector to 1. The first and last rows are incompatible, so that means
that both the first and third variables must be zero. This gives us the
eigenvector
\begin{align*}
    v_3 &= \begin{bmatrix} 0 \\ 1 \\ 0 \end{bmatrix}
\end{align*}
In a similar manner for the eigenvalues ${\lambda} = \pm\sqrt 2$, we perform Gaussian
elimination. When all free parameters have been set arbitrarily to 1 and other
constraints considered, we end up with the three eigenvectors:
\begin{align}
    \boxed{ {\lambda} = -\sqrt 2 \quad\rightarrow\quad v_1 =
	\begin{bmatrix} 1 \\ 0 \\ -i(1+\sqrt 2) \end{bmatrix} }
\end{align}
\begin{align}
    \boxed{ {\lambda} = \sqrt2 \quad\rightarrow\quad v_2 =
	\begin{bmatrix} 1 \\ 0 \\ -i(1-\sqrt 2) \end{bmatrix} }
\end{align}
\begin{align}
    \boxed{ {\lambda} = 2 \quad\rightarrow\quad v_3 =
	\begin{bmatrix} 0 \\ 1 \\ 0 \end{bmatrix} }
\end{align}

%%%%%%%%%%%%%%%%%%%%%%%%%%%%%%%%%%%%%%%%%%%%%%%%%%%%%%%%%%%%%%%%%%%%%%%%%%%%%%%
%%%% Problem 12
%%%%%%%%%%%%%%%%%%%%%%%%%%%%%%%%%%%%%%%%%%%%%%%%%%%%%%%%%%%%%%%%%%%%%%%%%%%%%%%
\problem{12}
\subsubsection{Question}
% Keywords
    \index{quantum!Spin-$\frac 32$ electron}

Suppose the electron were to have spin $\frac 32$ instead of spin $\frac 12$.
What would then be the atomic numbers $Z$ of the \emph{three} lowest-mass
noble gases, i.e.~the equivalents of helium, neon, and argon?

\subsubsection{Answer}

In a spin $\frac 32$ particle, there are 4 possible spin configurations
corresponding to the spin projections $s_z = \{ -\frac 32, -\frac 12, \frac 12,
\frac 32\}$. Because of this, each projection of the orbital angular momentum
can hold 4 electrons instead of just two. This means we can make use of
spectroscopic notation to easily count up to the atoms of interest.
\begin{align*}
    \ell  &= 0 \rightarrow  \text{1 $\ell_z$ state}	& & 1{s^4} \\
    \ell  &= 1 \rightarrow  \text{3 $\ell_z$ states} & & 1{s^4}2{p^{12}}2s^4 \\
    \ell  &= 2 \rightarrow  \text{5 $\ell_z$ states} & & 1{s^4}2{p^{12}}2{s^4}3d^{20}3{p^{12}}3s^4
\end{align*}

Since all shells are filled at each level, we just have to sum the number of
electrons in each line above to get the $Z$ number of the new noble gases.
\begin{align}
    \boxed{ Z = \{4, 20, 56 \} }
\end{align}


\springexam{2002}{1}
%%%%%%%%%%%%%%%%%%%%%%%%%%%%%%%%%%%%%%%%%%%%%%%%%%%%%%%%%%%%%%%%%%%%%%%%%%%%%%%
%%%% Problem 1
%%%%%%%%%%%%%%%%%%%%%%%%%%%%%%%%%%%%%%%%%%%%%%%%%%%%%%%%%%%%%%%%%%%%%%%%%%%%%%%
\problem{1}
\subsubsection{Question}
% Keywords
	\index{mechanics!Vehicle's resonant bouncing}

A \SI{1000}{\kg} automobile has ground clearance of \SI{18}{\cm} but when
loaded with an extra \SI{500}{\kg} from its 4 passengers it only clears the
ground by \SI{12}{\cm}. The car's shock absorbers are ineffective. At what
speed (in miles per hour) will the car bounce in resonance when it travels
along a smooth road containing transverse tar patches every \SI{15}{\m}?
Assume that the front and rear suspensions have the same bouncing frequency.

\subsubsection{Answer}

To find the natural resonant frequency of the vehicle, we use the two data
points about it's clearance to obtain the spring constant $k$. We know that
a constant force on a spring system does not change the dynamics, so it's
only the change in mass and distance which are relevant. Therefore by simple
equilibrium requirements,
\begin{align*}
    k \Delta L &= mg \\
    k &= \frac{mg}{\Delta L}
\end{align*}
where $m$ and $M$ are the masses of the passengers and empty car respectively
and $\Delta L$ is the difference in the clearance between the unloaded and loaded
car.

Next, we know that the resonant frequency of a simple harmonic oscillator is
given by ${\omega}  = \sqrt{k/m_\mathrm{tot}}$, so
\begin{align*}
    {\omega}  &= \sqrt{\frac{mg}{(M+m)\Delta L}} \\
    f &= \frac{1}{2{\pi} } \sqrt{\frac{mg}{(M+m)\Delta L}}
\end{align*}
where the second line has converted from angular to linear frequency. We then
simply relate that to how often the car hits a tar patch and solve for the
velocity. Letting $v$ be the velocity of the car and $d$ be the separation
distance between tar patches,
\begin{align*}
    \frac{v}{d} &= f \\
    v &= \frac{d}{2{\pi} } \sqrt{\frac{mg}{(M+m)\Delta L}}
\end{align*}
Plugging in the numbers,
\begin{align}
    \boxed{
    v = \SI{17.62}{\m\per\s} = \SI{39.42}{mph}
    }
\end{align}

%%%%%%%%%%%%%%%%%%%%%%%%%%%%%%%%%%%%%%%%%%%%%%%%%%%%%%%%%%%%%%%%%%%%%%%%%%%%%%%
%%%% Problem 2
%%%%%%%%%%%%%%%%%%%%%%%%%%%%%%%%%%%%%%%%%%%%%%%%%%%%%%%%%%%%%%%%%%%%%%%%%%%%%%%
\problem{2}
\subsubsection{Question}
% Keywords
	\index{quantum!Bound state energy threshold}

Suppose a particle of mass $m$ moves in a 1-dimensional square potential
well of width $L$ and depth $V$. What is the minimum depth of the well such
that the particle will have two bound states?

\subsubsection{Answer}

Divide the problem into three regions according to the diagram below:
\begin{figure}[H]
    \centering
    \begin{tikzpicture}
	% Draw the potential and the axes
	\draw (0,0) -- (3,0) -- (3,-2) -- (5,-2) -- (5,0) -- (8,0);
	\draw [dashed] (4,-3) -- (4,1);
	\draw [dashed] (0,-2) -- (8,-2);

	% Then add a few labels
	\node at (3,0) [anchor=west] {$V$};
	\node at (3,-2) [anchor=north] {$-\frac L2$};
	\node at (5,-2) [anchor=north] {$\frac L2$};

	% Now add the region labels
	\node at (1.5,-1) {\Large I};
	\node at (4.0,-1) {\Large II};
	\node at (6.5,-1) {\Large III};
    \end{tikzpicture}
\end{figure}

From the Schrödinger equation, we know that there will be two different types
of solutions. Within region II, no potential exists, so the solution has
the form
\begin{align*}
    \psi  &= A\cos (kx) + B\sin (kx) \\
    k^2 &= \frac{2mE}{{\hbar}^2}
\end{align*}
For regions I and III, we assume that the energy $E < V$ since we are only
interested in the bound solutions and not ones where the particle is free.
This gives the following exponential solution form:
\begin{align*}
    \psi  &= Ae^{\kappa x} + Be^{-\kappa x} \\
    \kappa^2 &= \frac{2m(V-E)}{{\hbar}^2}
\end{align*}
We immediately know that for regions I and III, the wavefunction must go to
zero at $\pm\infty $, so that immediately removes two terms in the solution. This
leaves us with the complete set of solutions
\begin{align*}
    \psi_{I}   &= Ae^{\kappa x} \\
    \psi_{II}  &= B\cos(kx) + C\sin(kx) \\
    \psi_{III} &= De^{-\kappa x}
\end{align*}

We make a further simplification by noting that in this situation where there
is a symmetric potential, the solution can also be divided into symmetric
and antisymmetric solutions, corresponding to keeping either the $\sin$ or
the $\cos$ solutions in region II. To solve the problem, we want to know the
threshold potential for $V$ that will maintain a second bound state. The first
bound state is the symmetric case (which heuristically is true by its virtue
of having only a single ``hump'' within region II), so the first excited
state/second state is the antisymmetric case (heuristically expected since
the sine solution will have two ``humps'' within region II). This lets us
simplify the problem and immediately set $B = 0$ to isolate just the
antisymmetric solution.

To continue, we make use of the fact that the wavefunction must be continuous
and differentiably continuous at the boundaries between regions I-II and
II-III. This means we find that:
\begin{align*}
    Ae^{-\kappa L/2} &= B\sin(-\frac{kL}{2})
    & B\sin(\frac{kL}{2}) &= De^{-\kappa L/2}
    \\
    A\kappa e^{-\kappa L/2} &= Bk\cos(-\frac{kL}{2})
    & Bk\cos(\frac{kL}{2}) &= -D\kappa e^{-\kappa L/2}
\end{align*}
Dividing the lower equation by the upper equation in both cases leads to the
condition
\begin{align*}
    \kappa  &= -k\cot(\frac{kL}{2})
\end{align*}

If we perform a variable substitution where $v = kL/2$ and $u = \kappa L/2$, the
equation above takes the slightly simpler form
\begin{align}
    u = -v\cot v
	\label{eqn:sp2002p1.2:transcendental}
\end{align}
This transformation comes in more useful when we consider the energy equations
that defined $k$ and $\kappa $. Substituting into the equation for $k$ and solving
for $E$ we have that
\begin{align*}
    E &= \frac{2v^2{\hbar}^2}{mL^2}
\end{align*}
And doing the same for $\kappa $,
\begin{align*}
    V - E &= \frac{2u^2{\hbar}^2}{mL^2}
\end{align*}
which combined gives the equation of a circle:
\begin{align*}
    u^2 + v^2 &= \frac{mL^2}{2{\hbar}^2}V
	\label{eqn:sp2002p1.2:circle}
\end{align*}
Therefore, the only valid solutions occur when both
(\ref{eqn:sp2002p1.2:transcendental}) and (\ref{eqn:sp2002p1.2:circle}) are
satisfied. The minimum value for a given cotangent line occurs at the roots
which occur when $v = (2n+1){\pi} /2$ for any integer $n$. The first excited state
for $n=1$ then occurs at $v = {\pi} /2$ and consequently $u = 0$. Substiting this
into the equation above and solving for $V$, we find that the threshold energy
for a second bound state corresponds to a potential depth of
\begin{align}
    \boxed{ V = \frac{{\pi} ^2{\hbar}^2}{2mL^2} }
\end{align}

%%%%%%%%%%%%%%%%%%%%%%%%%%%%%%%%%%%%%%%%%%%%%%%%%%%%%%%%%%%%%%%%%%%%%%%%%%%%%%%
%%%% Problem 3
%%%%%%%%%%%%%%%%%%%%%%%%%%%%%%%%%%%%%%%%%%%%%%%%%%%%%%%%%%%%%%%%%%%%%%%%%%%%%%%
\problem{3}
\subsubsection{Question}
% Keywords
	\index{thermodynamics!Mean Free Path of $H_2$}
	\index{statistical mechanics!Mean Free Path of $H_2$}

The cross-section for collisions between helium atoms is about \SI{e-16}{
\cm\squared}. Estimate the mean free path of helium atoms in helium gas at
atmospheric pressure and temperature.

\subsubsection{Answer}

Consider the path traced out by a helium atom as it travels a path length $L$,
colliding with other helium atoms along the way. Given that the cross section
of helium is $\sigma$, than we can estimate the volume that contains probable
interactions with our atom of interest as $\mathcal V = {\sigma}L$. To get the number
of interactions, we make use of the fact that we're treating the gas as an
ideal gas. From the ideal gas law,
\begin{align*}
    PV &= NkT
\intertext{so that solving for the number density}
    n = \frac{N}{V} &= \frac{P}{k_B T}
\end{align*}
Combining the density with the volume, we get the number of other [point
particle] helium atoms that are contained within the given helium atom's
interaction volume. If we then assume that the atom interacts with all other
atoms within the volume, and that the collisions are spaced out equally in
time, we just have to normalize the value by the trajectory's path length to
get an estimate of the mean free path of helium in a helium gas:
\begin{align*}
    {\lambda}^{-1} &= \frac{n\mathcal V}{L} = \frac{{\sigma}P}{k_B T}
\end{align*}
Plugging in ${\sigma} = \SI{e-16}{\cm\squared}$, $P = \SI{1.013e5}{\Pa}$, $k_B =
\SI{1.38e-23}{\J\per\K}$, and $T = \SI{298}{\K}$, we get
\begin{align}
    \boxed{
    {\lambda}^{-1} = \SI{4.06}{\micro\m}
    }
\end{align}

%%%%%%%%%%%%%%%%%%%%%%%%%%%%%%%%%%%%%%%%%%%%%%%%%%%%%%%%%%%%%%%%%%%%%%%%%%%%%%%
%%%% Problem 4
%%%%%%%%%%%%%%%%%%%%%%%%%%%%%%%%%%%%%%%%%%%%%%%%%%%%%%%%%%%%%%%%%%%%%%%%%%%%%%%
\problem{4}
\subsubsection{Question}
% Keywords
	\index{unsolved!Spring 2002 I.P4}
	\index{electrostatics!Electric field of a Coaxial Capacitor}
Consider an infinitely long cylindrical coaxial capacitor. The outer conductor has radius $R$ and the applied voltage is $V$. For what radius of the inner conductor will the field strength at its surface be a minimum?
\subsubsection{Answer}
%%%%%%%%%%%%%%%%%%%%%%%%%%%%%%%%%%%%%%%%%%%%%%%%%%%%%%%%%%%%%%%%%%%%%%%%%%%%%%%
%%%% Problem 5
%%%%%%%%%%%%%%%%%%%%%%%%%%%%%%%%%%%%%%%%%%%%%%%%%%%%%%%%%%%%%%%%%%%%%%%%%%%%%%%
\problem{5}
\subsubsection{Question}
% Keywords
	\index{unsolved!Spring 2002 I.P5}
	\index{statistical mechanics!Blackbody Radiation}
	\index{thermodynamics!Blackbody Radiation}
A rigid container is filled with a classical gas of molecular mass $m$. The temperature inside the container is $T$ and the pressure is $P$. Outside there is a vacuum. Suppose that a small hole of area $A$ is punched in the outer wall. At what rate (molecules per unit time per unit area) will gas leave the container? Write the result as a function of $m, P,\text{ and }T$. Some possibly useful information 
\begin{equation*}
	\int_0^\infty x^ne^{-\lambda x^2}\dd x = \frac{1}{2}\frac{\Gamma(\frac{n+1}{2})}{\lambda^{(n+1)/2}};\ \Gamma\qty(\frac{1}{2}) = \pi^{1/2}
\end{equation*}
\subsubsection{Answer}

%%%%%%%%%%%%%%%%%%%%%%%%%%%%%%%%%%%%%%%%%%%%%%%%%%%%%%%%%%%%%%%%%%%%%%%%%%%%%%%
%%%% Problem 6
%%%%%%%%%%%%%%%%%%%%%%%%%%%%%%%%%%%%%%%%%%%%%%%%%%%%%%%%%%%%%%%%%%%%%%%%%%%%%%%
\problem{6}
\subsubsection{Question}
% Keywords
	\index{mechanics!Tension in a supported rope}

A single closed loop of chain with mass $m$ and length $L$ rests
horizontally on a smooth frictionless cone with half-angle ${\alpha}$. What is the
tension in the chain?

\subsubsection{Answer}

\begin{figure}[H]
    \centering
    \begin{subfigure}[b]{0.49\textwidth}
	\centering
	\begin{tikzpicture}
	    % Draw the triangle
	    \draw (0,0) -- (0,6) -- (4,0) -- cycle;

	    % Add the arc and label to the given angle
	    \draw (0,5.2) arc (-90:-56.31:0.8) coordinate (angle end);
	    \node [anchor=north west] at ($(0,5.2)!0.25!(angle end)$) {${\alpha}$};

	    % Pic an arbitrary point along the surface of the cone to represent
	    % the chain length element.
	    \coordinate (point) at ($(0,6)!0.6!(4,0)$);
	    \draw (point) node [fill,black,circle,inner sep=2pt] {};

	    % Draw the gravitational vector from the point
	    \draw [thick,->] (point) -- ++(0,-1)
		node [anchor=north] {$\vec F_g$};

	    % And the normal vector
	    \path [name path=Ny proj] ($(point)+(0,1)$) -- ($(point)+(2,1)$);
	    \path [name path=N vec] (point) -- ($(point)!1!90:(4,0)$);
	    \draw [name intersections={of=Ny proj and N vec, by=x}]
		[thick,->] (point) -- (x)
		node [anchor=south west] {$\vec N$}
		coordinate (norm end);

	    % Draw projections of the normal vector
	    \draw [thick,dashed] (point) -- ++(0,1) -- (norm end);
	\end{tikzpicture}
	\caption{Side View}
    \end{subfigure}
    \hfil
    \begin{subfigure}[b]{0.49\textwidth}
	\centering
	\begin{tikzpicture}
	    % Setup some coordinates:
	    \def\alpha{20}
	    \def\r{5cm}
	    \def\dr{0.2cm}
	    \def\Nr{1cm}
	    
	    \coordinate (center left)  at ($(90+\alpha:\r)$);
	    \coordinate (center mid)   at (90:\r);
	    \coordinate (center right) at ($(90-\alpha:\r)$);

	    \coordinate (outer left)   at ($(90+\alpha:\r+\dr)$);
	    \coordinate (outer mid)    at ($(90:\r+\dr)$);
	    \coordinate (outer right)  at ($(90-\alpha:\r+\dr)$);

	    \coordinate (inner left)   at ($(90+\alpha:\r-\dr)$);
	    \coordinate (inner mid)    at ($(90:\r-\dr)$);
	    \coordinate (inner right)  at ($(90-\alpha:\r-\dr)$);
	    
	    % Draw the radii
	    \draw [dashed] (0,0) -- (center left);
	    \draw [dashed] (0,0) -- (center mid);
	    \draw [dashed] (0,0) -- (center right);
	    % And label them
	    \node [anchor=west] at ($(0,0)!0.5!(center mid)$)   {$r$};
	    \node [anchor=west] at ($(0,0)!0.5!(center right)$) {$r$};

	    % Draw the chain
	    \draw [thick] (inner left) -- (outer left)
		arc (90+\alpha:90-\alpha:\r+\dr) -- (inner right)
		arc (90-\alpha:90+\alpha:\r-\dr);

	    % Draw the normal force
	    \draw (center mid) node [fill,black,circle,inner sep=2pt] {}
		node [anchor=west] {$ds$}
		[very thick,->] -- ++(0,\Nr)
		node [anchor=south] {$\vec N_r$};

	    % Finally, draw the tension vectors
	    \path [name path=Nr] ($(center mid)+(-4,-\Nr)$) -- ++(8,0);
	    \path [name path=Tl]
		(center left) -- ($(center left)!2.5cm!-90:(0,0)$);
	    \path [name path=Tr]
		(center right) -- ($(center right)!2.5cm!90:(0,0)$);

	    \draw [name intersections={of=Nr and Tl, by=x}]
		[very thick,->] (center left) -- (x)
		coordinate (left T end)
		node [anchor=north east] {$\vec T$};
	    \draw [name intersections={of=Nr and Tr, by=x}]
		[very thick,->] (center right) -- (x)
		coordinate (right T end)
		node [anchor=north west] {$\vec T$};
	    \draw [dashed] (left T end) -- (right T end);

	    % Add the known angles
	    \draw ($(0,0)!1cm!(center left)$) arc (90+\alpha:90-\alpha:1cm);
	    \node [anchor=south] at (0,1) {$d\phi $};
	    \draw ($(left T end)!1cm!(center left)$) arc ({\alpha}:0:1cm)
		node [anchor=north] {$\frac 12 d\phi $};
	\end{tikzpicture}
	\caption{Top View}
    \end{subfigure}
\end{figure}

Consider just a small element of the chain of arc length $ds$. It will have
a corresponding mass $dm = {\lambda}\,ds$ where ${\lambda} = m/L$. Knowing that it's a statics
problem, we can easily determine the normal force by balancing the vertical
component with that of gravity.
\begin{align*}
    F_g &= N \sin {\alpha} \\
    N &= \frac{{\lambda} g\,ds}{\sin {\alpha}}
\end{align*}
This leaves the horizontal component of the normal force to be balanced with
the tension within the chain.

Now switching to the top view, we consider the short chain segment $ds$, shown
above with an exagerated curvature. We note that the radial part of the
normal force must be opposed by the sum of the radial components of the two
tensions $T$ acting on the end of the chain segment. By geometry, we know that
the angle with respect to the midpoint's tangent is one half the differential
angle change $d\phi  = ds/r$. This means we balance the forces as
\begin{align*}
    2T\sin(\frac 12 d\phi ) &= N \cos {\alpha} \\
    2T\sin(\frac 12 d\phi ) &= {\lambda} g\,ds \cot {\alpha}
\end{align*}
By the small angle approximation, $\sin (\frac 12 d\phi ) \approx \frac 12 d\phi $, so after
substituting for the fact that $dr = L/2{\pi} $ and ${\lambda} = M/L$,
\begin{align}
    \boxed{
    T = \frac{Mg}{2{\pi} } \cot {\alpha}
    }
\end{align}

%%%%%%%%%%%%%%%%%%%%%%%%%%%%%%%%%%%%%%%%%%%%%%%%%%%%%%%%%%%%%%%%%%%%%%%%%%%%%%%
%%%% Problem 7
%%%%%%%%%%%%%%%%%%%%%%%%%%%%%%%%%%%%%%%%%%%%%%%%%%%%%%%%%%%%%%%%%%%%%%%%%%%%%%%
\problem{7}
\subsubsection{Question}
% Keywords
	\index{mechanics!Relativistic collision of electron and photon}

A laser beam (photon energy \SI{1}{\eV}) collides head-on with a \SI{50}{\GeV}
ultra-relativistic electron beam. What is the energy of the photons reflected
backwards in the collision?

\subsubsection{Answer}

We'll be solving the problem using conservation of 4-momentum, so we define
the following momenta with the assumption that the electron beam is moving
to the right, and the photons are initially moving to the left. Let the
unprimed and primed $q^{\mu}$ be the photon's 4-momentum before and after the
collision respectively, and define the electron's momenta $k^{\mu}$ similarly.
Then in terms of the energies $E$ and 3-momenta $p$ (actually taken to be 1D
without loss of generality) for each of the photon $\gamma $ and electron $e$:
\begin{align*}
    q^{\mu} &= \begin{pmatrix} E_\gamma /c \\ -E_\gamma /c \end{pmatrix} &
	q'^{\mu} &= \begin{pmatrix} E'_\gamma /c \\ E'_\gamma /c \end{pmatrix}
    \\
    k^{\mu} &= \begin{pmatrix} E_e/c \\ p_e \end{pmatrix} &
	k'^{\mu} &= \begin{pmatrix} E'_e/c \\ p'_e \end{pmatrix}    
\end{align*}
By conservation of momentum,
\begin{align*}
    q'^{\mu} + k'^{\mu} &= q^{\mu} + k^{\mu} \\
    k'^{\mu} &= q^{\mu} - q'^{\mu} + k^{\mu}
\end{align*}
Solving for the unknown electron momentum after the collision lets us eliminate
it from the equation; when we square the equation, the squared quantities are
Lorentz invariant, and therefore the product can be evaluated in any frame. A
convenient choice is the rest frame where the electron evaluates to its mass
energy and photons vanish.
\begin{align*}
    \underbrace{k'^{\mu} k'_{\mu}}_{m_e c^2} &=
	\underbrace{q^{\mu} q_{\mu}}_{0}
	- \underbrace{q'^{\mu} q'_{\mu}}_{0}
	+ \underbrace{k^{\mu} k_{\mu}}_{m_e c^2}
	- q'^{\mu} q_{\mu} + q'^{\mu} k_{\mu} - q^{\mu} k_{\mu}
\end{align*}
This greatly simplifies the rest of the problem to
\begin{align*}
    q'^{\mu} q_{\mu} - q'^{\mu} k_{\mu} &= - q^{\mu} k_{\mu}
\end{align*}
Inserting the energy and 3-momentum components and performing the inner
products,
\begin{align*}
    2\frac{E_\gamma  E'_\gamma }{c^2} - \frac{E'_\gamma  E_e}{c^2} + \frac{E'_\gamma  p_e}{c} &=
	-\frac{E_\gamma  E_e}{c^2} - \frac{E_\gamma  p_e}{c}
\end{align*}
Isolating $E'_\gamma $ on the left and $E_\gamma $ on the right,
\begin{align*}
    -E'_\gamma  (E_e - p_e c - 2E_\gamma ) &= - E_\gamma (E_e - p_e c)
\end{align*}
which solving for the unknown photon energy gives
\begin{align}
    \boxed{ E'_\gamma  = E_\gamma  (1 - \frac{2E_\gamma }{E_e - p_e c})^{-1} }
	\label{eqn:2002sp1.7:analytic_soln}
\end{align}

The solution is formally complete, but actually calculating a numerical answer
can prove difficult because $E_e \approx p_e c$. Therefore, we will expand the
solution. Beginning with the definition of the momentum from Einstein's energy
relation,
\begin{align*}
    p_e c &= \sqrt{{E_e}^2 - {m_e}^2c^4}
\intertext{we can subtract it from $E_e$, leading to the useful form}
    E_e - p_e c &= E_e ( 1 - \sqrt{1 - \frac{{m_e}^2c^4}{{E_e}^2}} )
\intertext{Expanding the root to first order in its argument,}
    E_e - p_e c &= E_e \cdot  \frac 12 \frac{{m_e}^2c^4}{{E_e}^2} \\
    {} &= \frac 12 \frac{{m_e}^2c^4}{E_e}
\end{align*}

Plugging this into the solution (\ref{eqn:2002sp1.7:analytic_soln}), the
photon energy is then approximately given by
\begin{align}
    \boxed{ E'_\gamma  \approx E_\gamma  ( 1 - 4 \frac{E_\gamma  E_e}{{m_e}^2c^4} )^{-1} }
	\label{eqn:2002sp1.7:approx_soln}
\end{align}
The numerics are much easier to calculate in this case, and we find that the
final energy of the reflected photon is
\begin{align}
    \boxed{ E'_\gamma  = \SI{4.273}{\eV} }
\end{align}

%%%%%%%%%%%%%%%%%%%%%%%%%%%%%%%%%%%%%%%%%%%%%%%%%%%%%%%%%%%%%%%%%%%%%%%%%%%%%%%
%%%% Problem 8
%%%%%%%%%%%%%%%%%%%%%%%%%%%%%%%%%%%%%%%%%%%%%%%%%%%%%%%%%%%%%%%%%%%%%%%%%%%%%%%
\problem{8}
\subsubsection{Question}
% Keywords
	\index{electrodynamics!LR circuit}

In the figure below, $\mathcal E = \SI{100}{\V}$, $R_1 = \SI{5}{\ohm}$,
$R_2 = \SI{10}{\ohm}$, $R_3 = \SI{15}{\ohm}$, and $L = \SI{1.0}{\henry}$. Find
the values of the currents $I_1$ and $I_2$
\begin{enumerate}[a)]
    \item immediately after the switch $S$ is closed,
    \item a long time later,
    \item immediately after switch $S$ is opened again,
    \item and then how long must you wait, after the switch is opened, before
	$I_2$ falls by a factor of $e$?
\end{enumerate}

\begin{center}
    \vspace{\baselineskip}
    \begin{circuitikz}
	\draw
	    (0,0)
		to[battery,l_=$\mathcal E$]
	    ++(0,3)
		to[closing switch,l_=$S$]
	    ++(3,0)
		to[resistor,l_=$R_1$]
	    ++(3,0)
		coordinate (I2 I3 break)
		to[resistor,l_=$R_3$,i=$I_3$]
	    ++(3,0)
		to[inductor,l_=$L$]
	    ++(0,-3)
		--
	    ++(-3,0)
		coordinate (I2 I3 combine)
		to[short,i=$I_1$]
	    (0,0)
	;
	\draw
	    (I2 I3 break)
		to[resistor,l_=$R_2$,i=$I_2$]
	    (I2 I3 combine)
	;
    \end{circuitikz}
    \vspace{\baselineskip}
\end{center}

\subsubsection{Answer}

Start by applying Kirchoff's rules to the circuit: current is conserved and
the voltage changes must sum to zero around each loop, so
\begin{align}
    I_1 &= I_2 + I_3
	\label{eqn:sp2002p1.8:kirchoff_current} \\
    0 &= \mathcal E - I_1R_1 - I_2R_2
	\label{eqn:sp2002p1.8:kirchoff_leftloop} \\
    0 &= -I_3R_3 - L \frac{dI_3}{dt} + I_2R_2
	\label{eqn:sp2002p1.8:kirchoff_rightloop}
\end{align}

Since only (\ref{eqn:sp2002p1.8:kirchoff_rightloop}) has a term involving
a time derivative, we choose to first solve for the current $I_3$. By solving
for $I_2R_2$ in (\ref{eqn:sp2002p1.8:kirchoff_leftloop}) and substituting,
we eliminate $I_2$ and have
\begin{align}
    0 &= -I_3R_3 - L\frac{dI_3}{dt} + \mathcal E - I_1R_1
    \label{eqn:sp2002p1.8:right_noI2}
\end{align}
Furthermore, by also substituting the value of $I_2$ into
(\ref{eqn:sp2002p1.8:kirchoff_current}):
\begin{align}
    I_1 &= \frac{\mathcal E}{R_1+R_2} + \frac{R_2}{R_1+R_2} I_3
    \label{eqn:sp2002p1.8:current_noI2}
\end{align}
Then by combining (\ref{eqn:sp2002p1.8:right_noI2}) and
(\ref{eqn:sp2002p1.8:current_noI2}), we can produce a differential equation
for $I_3$:
\begin{align*}
    -I_3R_3 - L\frac{dI_3}{dt} + \mathcal E - \frac{R_1}{R_1+R_2}\mathcal E -
	\frac{R_1R_2}{R_1+R_2}I_3 = 0
    \\
    -\underbrace{\frac{R_1R_2 + R_1R_3 + R_2R_3}{R_1+R_2}}_{R'}I_3 = L\frac{dI_3}{dt}
	- \frac{R_2}{R_1+R_2} \mathcal E
\end{align*}
\begin{align}
    \frac{dI_3}{dt} = -\frac{R'}{L} I_3 + \frac{1}{L}\frac{R_2}{R_1+R_2}\mathcal E
	\label{eqn:sp2002p1.8:diffeq_I3}
\end{align}
Considering just the homogeneous part, we easily solve it to find the standard
exponential solution
\begin{align*}
    I_{3h}(t) &= I_{30} e^{-R't/L}
\end{align*}
And using the ansatz $I_{3p}(t) = At + B$ for the inhomogeneous part,
\begin{align*}
    A &= -\frac{R'}{L}At - \frac{R'}{L}B + \frac 1L\frac{R_2}{R_1+R_2}\mathcal E
    \\
    A &= 0 \\
    B &= \frac{1}{R'}\frac{R_2}{R_1+R_2}\mathcal E
\end{align*}
At $t = 0$, the inductor has no current passing through it, so when the switch
is closed, the current must remain continuous. This gives us the initial
condition necessary to solve for the unknown $I_{30}$, and after doing so
and simplifying, the total solution is
\begin{align}
    I_3(t) &= \frac{\mathcal E}{R'}\frac{R_2}{R_1+R_2}(1 - e^{-R't/L})
	\label{eqn:sp2002p1.8:I3_charging}
\end{align}

Then by substituting this solution back into (\ref{eqn:sp2002p1.8:current_noI2})
we get the solution for $I_1$:
\begin{align}
    I_1(t) &= \frac{\mathcal E}{R_1+R_2} \left[ 1 + \frac{1}{R'}
	\frac{{R_2}^2}{R_1+R_2} (1 - e^{-R't/L}) \right]
	\label{eqn:sp2002p1.8:I1_charging}
\end{align}
Finally, combining both inserting both solutions for $I_1$ and $I_3$ into
(\ref{eqn:sp2002p1.8:kirchoff_current}), the solution for $I_2$ is
\begin{align}
    I_2(t) &= \frac{\mathcal E}{R_1+R_2} \left[ 1 - \frac{1}{R'}
	\frac{R_1R_2}{R_1+R_2} (1 - e^{-R't/L}) \right]
	\label{eqn:sp2002p1.8:I2_charging}
\end{align}

Plugging in all of the given values, we find that the currents at the instant
the switch is closed are
\begin{align}
    \boxed{I_1(0) = \SI{6.66}{\A}\quad\quad\text{$S$ is closed}} \\
    \boxed{I_2(0) = \SI{6.66}{\A}\quad\quad\text{$S$ is closed}}
\end{align}
For a long time later, we can let $t \rightarrow \infty $ and find that
\begin{align}
    \boxed{I_1(\infty ) = \SI{9.09}{\A}\quad\quad\text{$S$ is closed}} \\
    \boxed{I_2(\infty ) = \SI{5.54}{\A}\quad\quad\text{$S$ is closed}}
\end{align}

Right after the switch is opened, the left loop is taken out of the circuit,
so we immediately know that the value of $I_1$ is zero.
\begin{align}
    \boxed{I_1(0) = 0\quad\quad\text{$S$ is open	}}
\end{align}
For the right loop, we start by noting that the steady state current through
the inductor will be needed. Taking the limit of (\ref
{eqn:sp2002p1.8:I3_charging}), we have that the new initial condition is
\begin{align*}
    I_3(0) &= \frac{\mathcal E}{R'}\frac{R_2}{R_1+R_2}
\end{align*}
$I_2$ now is equal to $-I_3$ since there is no other path for the current to
traverse. This loop's differential equation is then
\begin{align*}
    -(R_2+R_3)I_3 - L\frac{dI_3}{dt} = 0
\end{align*}
Solving for the exponential and using the initial condition above, the time
solution is
\begin{align*}
    I_3(t) &= \frac{\mathcal E}{R'}\frac{R_2}{R_1+R_2} e^{-(R_2+R_3)t/L} \\
    I_2(t) &= -\frac{\mathcal E}{R'}\frac{R_2}{R_1+R_2} e^{-(R_2+R_3)t/L}    
\end{align*}
Therefore the current in $I_2$ just after the switch is opened reverses
direction
\begin{align}
    \boxed{I_2(0) = \SI{-3.64}{\A}}
\end{align}
Then by simple exponential relations, we know that the time time to decay by
a factor of $e$ is given by the reciprocal of the coefficient of $t$, so
inserting the appropriate numbers
\begin{align}
    \boxed{t_\mathrm{decay} = \SI{0.04}{\s}}
\end{align}

%%%%%%%%%%%%%%%%%%%%%%%%%%%%%%%%%%%%%%%%%%%%%%%%%%%%%%%%%%%%%%%%%%%%%%%%%%%%%%%
%%%% Problem 9
%%%%%%%%%%%%%%%%%%%%%%%%%%%%%%%%%%%%%%%%%%%%%%%%%%%%%%%%%%%%%%%%%%%%%%%%%%%%%%%
\problem{9}
\subsubsection{Question}
% Keywords
	\index{thermodynamics!Arbitrary engine efficiency}

An engine using \SI{1}{\mol} of an ideal diatomic gas performs the cycle $A
\rightarrow B \rightarrow C \rightarrow A$ as shown in the diagram below. $A
\rightarrow B$ is an adiabatic expansion, $B \rightarrow C$ occurs at
constant pressure, and $C \rightarrow A$ takes place at constant volume.
What is the efficiency of the cycle?

\begin{center}
    \begin{tikzpicture}[
	>=stealth
    ]
    % Draw the axes
	\draw [->] (0,0) -- (0,3) node [anchor=south] { P [\si{atm}] };
	\draw [->] (0,0) -- (5,0) node [anchor=west] {V [\si{\L}]};

	% Then draw the engine cycle
	\draw [very thick,dashed,->] (1,1)    -- (1, 2.64)
	    node [anchor=south] {A};
	\draw [very thick,dashed,->] (1,2.64) parabola[bend at end] (4,1)
	    node [anchor=west] {B};
	\draw [very thick,dashed,->] (4,1)    -- (1,1)
	    node [anchor=north east] {C};
	;

	% Then draw in the labels that give the absolute numbers
	\draw [dashed] (1,2.64) -- (0,2.64) node [anchor=east] {2.64};
	\draw [dashed] (1,1) -- (0,1) node [anchor=east] {1.00};
	\draw [dashed] (1,1) -- (1,0) node [anchor=north] {10};
	\draw [dashed] (4,1) -- (4,0) node [anchor=north] {20};
    \end{tikzpicture}
\end{center}

\subsubsection{Answer}

Since we want to find the efficiency of the cycle, we only care about the
heat exchanged during each stage of the cycle. Because the path $A
\rightarrow B$ is adiabatic, we immediately know that $Q = 0$. Then
proceeding to look at the stage $C \rightarrow A$, we know that the work
done during this cycle is identically zero since there is no area under the
curve. That means we are left simply with the equation
\begin{align*}
    dU = dQ
\end{align*}
Because this is an ideal [diatomic] classical gas, we combine the equations
\begin{align*}
    U &= \frac 52 nRT
\intertext{and}
    PV &= nRT
\end{align*}
to get that the difference in energy across the path is
\begin{align*}
    Q_{CA} &= U = \frac 52 nR(T_A - T_C) \\
    {}&= \frac 52 V_1 (P_2 - P_1)
\end{align*}

For the remaining stage $B \rightarrow C$, we use the full thermodynamic
identity:
\begin{align*}
    dU &= dQ - P\,dV
\end{align*}
The pressure $P_1$ is constant, so both integration of $dU$ and $dV$ are simply
the differences in each quantity. Again substituting for the temperature in
$U$ with the ideal gas law,
\begin{align*}
    \frac 52 nR(T_C - T_B) &= Q_{BC} - P_1(V_1 - V_2) \\
    \frac 52 P_1(V_1 - 2V_1) &= Q_{BC} + P(V_1 - 2V_1) \\
    Q_{BC} &= -\frac 72 P_1V_1
\end{align*}

We've accounted for all the heat flow in the system. $Q_{BC}$ is negative, so
this is the heat flow out of the system, while $Q_{CA}$ is positive and is the
heat flow into the system. By definition then, the efficiency ${\eta}$ of the system
is
\begin{align*}
    {\eta} &= 1 - \frac{Q_{out}}{Q_{in}} \\
    {}&= 1 - \frac{\frac 72 P_1 V_1}{\frac 52 V_1 (P_2 - P_1)} \\
    {}&= 1 - \frac 57 \frac{P_1}{P_2 - P_1}
\end{align*}
Plugging in the given values, we find the efficiency to be
\begin{align}
    \boxed{
    {\eta} = 0.146 = \SI{14.6}{\percent}
    }
\end{align}

%%%%%%%%%%%%%%%%%%%%%%%%%%%%%%%%%%%%%%%%%%%%%%%%%%%%%%%%%%%%%%%%%%%%%%%%%%%%%%%
%%%% Problem 10
%%%%%%%%%%%%%%%%%%%%%%%%%%%%%%%%%%%%%%%%%%%%%%%%%%%%%%%%%%%%%%%%%%%%%%%%%%%%%%%
\problem{10}
\subsubsection{Question}
% Keywords
	\index{mechanics!Friction and a rolling hoop}

A thin circular hoop rolls down an inclined plane under the influence of
gravity. What minimum coefficient of friction is required to ensure that it
rolls rather than slides?

\subsubsection{Answer}

Begin first by finding the motion that describes the rolling without slipping
state. We do this by solving the system's Lagrangian:
\begin{align*}
    \mathcal L &= (\frac 12 m{\dot x}^2 + \frac 12 I{\dot \theta }^2) - (mgx\sin {\alpha})
\end{align*}
where $x$ is the length along the ramp with $x=0$ at the bottom, $I$ is the
moment of inertia of the hoop, $m$ is its mass, $\theta $ is the angle of rotation
of the hoop about its center, and ${\alpha}$ is the angle of the incline plane. By
noting that rolling without slipping requires that $r\dot \theta  = \dot x$, we
can reduce the problem to the single variable $x$. The result is the following
differential equation, where $I = mR^2$ has been substituted in:
\begin{align*}
    2m \ddot x &= -mg\sin {\alpha} \\
    \ddot x &= -\frac 12 g\sin {\alpha}
\end{align*}
Therefore we know the linear acceleration will be $a = -\frac 12 g\sin {\alpha}$ in
the non-slipping case.

To find what coefficient of friction produces this motion, we consider the
forces acting on the hoop with the coordinate system still oriented along and
perpendicular to the plane. In the perpendicular direction, the normal force
$N$ is canceled by the perpendicular component of gravity, so
\begin{align*}
    N &= mg\cos {\alpha}
\end{align*}
In the parallel direction, the frictional force and the parallel component of
gravity must sum to give the requisite force, namely $ma$.
\begin{align*}
    {\mu}N - mg\sin {\alpha} &= ma = -\frac 12 mg\sin {\alpha} \\
    {\mu}mg\cos {\alpha} &= \frac 12 mg\sin {\alpha} \\
    {\mu} &= \frac 12 \tan {\alpha}
\end{align*}

\fbox{
Therefore we find that the coefficient of friction must be equal to half of
the tangent of the inclined plane's angle.
}

%%%%%%%%%%%%%%%%%%%%%%%%%%%%%%%%%%%%%%%%%%%%%%%%%%%%%%%%%%%%%%%%%%%%%%%%%%%%%%%
%%%% Problem 11
%%%%%%%%%%%%%%%%%%%%%%%%%%%%%%%%%%%%%%%%%%%%%%%%%%%%%%%%%%%%%%%%%%%%%%%%%%%%%%%
\problem{11}
\subsubsection{Question}
% Keywords
	\index{quantum!Probability to stay in ground state}

A particle is confined within a cubical box with sides of length $L$ and is
initially in the ground state. If the length of one side of the box (along
the $x$-direction) is abruptly increased to a length $2L$, what is the
probability that the particle remains in the ground state?

\subsubsection{Answer}

We start by recalling the solution for a particle in a box. In a 1D box with is
left edge at the origin, the properly normalized wavefunction is given by
\begin{align*}
    \psi (x) &= \sqrt{\frac{2}{L}} \sin (\frac{n{\pi} x}{L})
\end{align*}
where $L$ is the size of the box. The Cartesian extension into 3D is simple
and is respectively for the $L\times L\times L$ and $2L\times L\times L$ boxes:
\begin{align*}
    \psi (x,y,z) &= (\frac{2}{L})^{3/2} \sin(\frac{n_x {\pi} x}{L})
	\sin(\frac{n_y {\pi} y}{L}) \sin(\frac{n_y {\pi} y}{L}) \\
    \psi '(x,y,z) &= \frac{1}{\sqrt 2}(\frac{2}{L})^{3/2} \sin(\frac{n_x {\pi} x}{2L})
	\sin(\frac{n_y {\pi} y}{L}) \sin(\frac{n_y {\pi} y}{L})
\end{align*}

To find the probability of remaining in the ground state, we simply must
take the inner product of both wavefunctions in the ground state over an
appropriate domain; this means that the initial, unexanded box's wavefunction
is 0 within the new region.
\begin{align*}
    \mathscr{P} &= \braket{\psi _{111}}{\psi '_{111}} \\
    {} &= \frac{1}{\sqrt{2}} (\frac{2}{L})^3  (\int_0^L \sin(\frac{{\pi} x}{L})
	\sin(\frac{{\pi} x}{2L}) \,dx) (\int_0^L \sin(\frac{{\pi} y}{L})
	\sin(\frac{{\pi} y}{L}) \,dy) (\int_0^L \sin(\frac{{\pi} z}{L})
	\sin(\frac{{\pi} z}{L}))
\end{align*}
The integrals over $y$ and $z$ are simple and simply evaluate to $L/2$ as
we'd expect from the normalization factor. To evaluate the integral over $x$,
use the trigonometric identity $\sin 2\theta  = 2\sin \theta  \cos \theta $ and a change of
variables with $u = \sin({\pi} x/2L)$ to arrive at the integral
\begin{align*}
    \mathscr{P} &= \frac{2\sqrt 2}{L} \int_0^1 u^2 \cdot  \frac{2L}{{\pi} }\,du
\end{align*}
Evaluating this, we find the probability of remaining the ground state after
the box is expanded suddenly to be
\begin{align}
    \boxed{\mathscr{P} = \frac{4\sqrt 2}{3{\pi} } \approx 0.60}
\end{align}

%%%%%%%%%%%%%%%%%%%%%%%%%%%%%%%%%%%%%%%%%%%%%%%%%%%%%%%%%%%%%%%%%%%%%%%%%%%%%%%
%%%% Problem 12
%%%%%%%%%%%%%%%%%%%%%%%%%%%%%%%%%%%%%%%%%%%%%%%%%%%%%%%%%%%%%%%%%%%%%%%%%%%%%%%
\problem{12}
\subsubsection{Question}
% Keywords
	\index{mechanics!Deep-water gravity waves}
	\index{waves!Deep-water gravity waves}

The frequency $f$ of a deep water gravity wave (i.e. an ordinary ocean wave)
is given by
\begin{align*}
    f =\sqrt{\frac{1}{2{\pi} }} \rho ^a g^b {\lambda}^c
\end{align*}
where $\rho $, $g$, and ${\lambda}$ are the water density, gravitational acceleration, and
wavelength of the wave, respectively. What are the values of the exponents
$a$, $b$, and $c$, and what is the ratio of the wave group velocity to phase
velocity?

\subsubsection{Answer}

We proceed by dimensional analysis. Immediately we know that $a = 0$ since a
frequency does not have a mass component, and neither $g$ nor ${\lambda}$ have a
mass term to cancel the one in $\rho $. Furthermore, $g$ is the only one with a
time term, so it's exponent must then by $b = \frac 12$ in order to give $f$
its $[\si{\s^{-1}}]$ unit. That leaves $c = \frac 12$ in order to cancel
the $\sqrt{\si{\m}}$ dimension left over from $g$.

\begin{align*}
    \boxed{
    f = \sqrt{\frac{g{\lambda}}{2{\pi} }}
	\quad\quad\text{ with }\quad a = 0,\, b = \frac 12,\, c = \frac 12
    }
\end{align*}

The phase velocity can be derived from the frequency given by noting that
$v_p = {\omega} /k$ together with $k^{-1} = 2{\pi} {\lambda}$ and ${\omega}  = 2{\pi} f$. Put together, this
gives
\begin{align*}
    v_p &= \frac{1}{k}\sqrt{\frac{g}{k}}
\end{align*}
The group velocity is given by $v_g = d{\omega} /dk$, so
\begin{align*}
    v_g &= -\frac{1}{2k}\sqrt{\frac{g}{k}}
\end{align*}
Taking only the absolute values and finding the ratio
\begin{align}
    \boxed{\frac{v_g}{v_p} = \frac 12}
\end{align}


\springexam{2002}{2}
%%%%%%%%%%%%%%%%%%%%%%%%%%%%%%%%%%%%%%%%%%%%%%%%%%%%%%%%%%%%%%%%%%%%%%%%%%%%%%%
%%%% Problem 2
%%%%%%%%%%%%%%%%%%%%%%%%%%%%%%%%%%%%%%%%%%%%%%%%%%%%%%%%%%%%%%%%%%%%%%%%%%%%%%%
\problem{2}
\subsubsection{Question}
% Keywords
	\index{thermodynamics!Zipper partition function}
    \index{statistical mechanics!Zipper partition function}

A zipper has $N$ links; each link has a closed state with zero energy and an
open state with energy $\varepsilon $. We require, however, that the zipper can only
unzip from the left end, and that the link number $s$ can only open if all
links to the left (i.e.\ $1, 2, \ldots s-1$) are already open.
\begin{enumerate}[a.]
    \item
        Find an explicit expression for the partition function by doing the
        appropriate summation.
    \item
        In the limit $\varepsilon \gg k_B T$ find the average number of open links. This
        model is a very simplified model of the unwinding of two-stranded
        DNA molecules.
\end{enumerate}

\subsubsection{Answer}

\begin{enumerate}[a.]
    \item
        Create the partition function by induction; start by assuming there is
        only a single link. Then the partition function is a simple two-state
        system:
        \begin{align*}
            Z_1 &= e^0 + e^{-\varepsilon /k_BT} = 1 + e^{-\varepsilon /k_BT}
        \end{align*}
        Adding a second link,
        \begin{align*}
            Z_2 &= \underbrace{e^{0+0}}_{\text{both closed}} +
                \underbrace{e^{(-\varepsilon +0)/k_BT}}_{\text{1 open, 1 closed}} +
                \underbrace{e^{(-\varepsilon -\varepsilon )/k_BT}}_{\text{both open}}
            \\
            {} &= 1 + e^{-\varepsilon /k_BT} + e^{-2\varepsilon /k_BT}
        \end{align*}
        Following, for three links:
        \begin{align*}
            Z_3 &= \underbrace{e^{0+0+0}}_{\text{all closed}} +
                \underbrace{e^{(-\varepsilon +0+0)/k_BT}}_{\text{1 open, 2 closed}} +
                \underbrace{e^{(-\varepsilon -\varepsilon +0)/k_BT}}_{\text{2 open, 1 closed}} +
                \underbrace{e^{(-\varepsilon -\varepsilon -\varepsilon )/k_BT}}_{\text{all open}}
            \\
            {} &= 1 + e^{-\varepsilon /k_BT} + e^{-2\varepsilon /k_BT} + e^{-3\varepsilon /k_BT}
        \end{align*}
        By induction, we see that the maximum coefficient in the series of
        exponential factors is just the number of links, so by induction we
        conclude that
        \begin{align*}
            Z &= \sum_{s=0}^{N} e^{-s\varepsilon /k_BT}
        \end{align*}
        Applying the results of a finite geometric series, the closed-form
        solution for the partition function of the links is
        \begin{align}
            \boxed{
            Z = \frac{1 - e^{-(N+1)\varepsilon /k_BT}}{1 - e^{-\varepsilon /k_BT}}
            }
        \end{align}
    \item
        To get the average number of open links, we use the standard
        procedure for finding expvalation values.
        \begin{align*}
            \expval{s} &= \frac{1}{Z} \sum_{s=0}^N s e^{-s\varepsilon /k_BT}
        \end{align*}
        By making use of differentiation under the summation trick, we can
        find the closed-form solution:
        \begin{align*}
            \expval{s} &= \frac{1}{Z} \sum_{s=0}^N
                \frac{d}{d(\frac{\varepsilon }{k_BT})} \Big[ -e^{-s\varepsilon /k_BT} \Big] \\
            {} &= -\frac{1}{Z} \frac{\partial }{\partial (\frac{\varepsilon }{k_BT})}
                \sum_{s=0}^N e^{-s\varepsilon /k_BT} \\
        \intertext{
        Noting that the summation is the same as above,
        }
            \expval{s} &= -\frac{1}{Z} \frac{\partial Z}{\partial (\frac{\varepsilon }{k_BT})}
        \end{align*}
        First considering just the derivative part:
        \begin{align*}
            \frac{\partial Z}{\partial (\frac{\varepsilon }{k_BT})} &=
                \frac{(N+1)e^{-(N+1)\varepsilon /k_BT}}{1 - e^{-\varepsilon /k_BT}} -
                \frac{1 - e^{-(N+1)\varepsilon /k_BT}}{(1 - e^{-\varepsilon /k_BT})^2} e^{-\varepsilon /k_BT}
        \intertext{
        which when combined with the factor $-1/Z$ simplifies to
        }
            -\frac{1}{Z} \frac{\partial Z}{\partial (\frac{\varepsilon }{k_BT})} &=
                -(N+1)\frac{e^{-(N+1)\varepsilon /k_BT}}{1 - e^{-(N+1)\varepsilon /k_BT}} +
                \frac{e^{-\varepsilon /k_BT}}{1 - e^{-\varepsilon /k_BT}} \\
            &= \frac{1}{e^{\varepsilon /k_BT} - 1} - \frac{N+1}{e^{(N+1)\varepsilon /k_BT} - 1}
        \end{align*}
        Therefore the analytic solution is
        \begin{align}
            \boxed{
            \expval{s} = \frac{1}{e^{\varepsilon /k_BT}-1} - \frac{N+1}{e^{(N+1)\varepsilon /k_BT}-1}
            }
        \end{align}
        In the limit that $\varepsilon \gg k_BT$, though, the exponentials in the denominator
        are very large in comparison to 1, so we ignore the unity factors
        and make the approximation that
        \begin{align*}
            \expval{s} &= e^{-\varepsilon /k_BT} - (N+1)e^{-(N+1)\varepsilon /k_BT}
        \intertext{
        Collecting like terms,
        }
            {} &= \left[1 - (N+1)e^{-N\varepsilon /k_BT} \right] e^{-\varepsilon /k_BT}
        \intertext{
            The second term in the brackets approximate zero, so
        }
            {} &= e^{-\varepsilon /k_BT}
        \end{align*}
        Therefore in the low temperature limit where the thermal energy is
        much less than the energy of the open state,
        \begin{align}
            \boxed{
            \expval{s} = e^{-\varepsilon /k_BT}
            }
        \end{align}
\end{enumerate}


\fallexam{2002}{1}
%%%%%%%%%%%%%%%%%%%%%%%%%%%%%%%%%%%%%%%%%%%%%%%%%%%%%%%%%%%%%%%%%%%%%%%%%%%%%%%
%%%% Problem 5
%%%%%%%%%%%%%%%%%%%%%%%%%%%%%%%%%%%%%%%%%%%%%%%%%%%%%%%%%%%%%%%%%%%%%%%%%%%%%%%
\problem{5}
\subsubsection{Question}
% Keywords
	\index{mechanics!Elastic collision on spring-connected blocks}
	\index{Lagrangian!Elastic collision on spring-connected blocks}

Blocks of mass $m$ and $2m$ are free to slide without friction on a
horizontal wire. They are connected by a massless spring of equilibrium
length $L$ and force constant $k$. A projectile of mass $m$ is fired with
velocity $v$ into the block with mass $m$ and sticks to it. If the blocks
are initially at rest, what is the maximum displacement between them in the
subsequent motion?

\subsubsection{Answer}

Take time $t=0$ to be the moment the projectile collides with the mass $m$,
and let the subsequent transfer of momentum be instantaneous. In this case,
the initial conditions of the problem are then:
\begin{align*}
    x₁(0) &= 0			& \dot x₁(0) &= u \\
    x₂(0) &= L			& \dot x₂(0) &= 0
\end{align*}
where $u$ is the initial velocity of the combined project-mass system. We get
$u$ from conservation of mometum:
\begin{align*}
    2mu &= mv + 0 \\
    u &= \frac 12 v
\end{align*}

Now solve the mechanics problem using the Lagrangian approach. Both masses have
kinetic energy, and the spring stores potential energy, so
\begin{align*}
    T &= m{\dot x₁}² + m{\dot x₂}² \\
    V &= \frac 12 m (x₂ - x₁)² \\
    L &= m ({\dot x₁}² + {\dot x₂}²) - \frac 12 k({x₁}² + {x₂}² + 2x₁x₂)
\end{align*}
Setting up the differential equation, we get
\begin{align*}
    \frac{∂L}{∂x₁} &= -kx₁ + kx₂	& \frac{d}{dt}
	\left[\frac{∂L}{∂\dot x₁}\right] &= 2m \ddot x₁ \\
    \frac{∂L}{∂x₂} &=  kx₁ - kx₂	& \frac{d}{dt}
	\left[\frac{∂L}{∂\dot x₁}\right] &= 2m \ddot x₂ \\
\end{align*}
Leading to the system of equations where $ω² = k/2m$,
\begin{align*}
    \begin{bmatrix} \ddot x₁ \\ \ddot x₂ \end{bmatrix} &=
	\begin{bmatrix} -ω² & ω² \\ ω² & -ω² \end{bmatrix}
	\begin{bmatrix} x₁ \\ x₂ \end{bmatrix}
\end{align*}
Solving the eigensystem, we find the eigenfrequencies to be $λ = \{0, -2ω²\}$.
Letting ${ω'}² = 2ω²$, the eigenfunction equations are then
\begin{align*}
    \ddot ψ₁ &= 0
	& \rightarrow&&
	ψ₁ &= A₁t + B₁ \\
    \ddot ψ₂ &= -2ω² ψ₂
	& \rightarrow&&
	ψ₂ &= A₂\cos(ω't) + B₂\sin(ω't)
\end{align*}
From the eigenvectors, we express the solutions of $x₁$ and $x₂$ in terms of
$ψ₁$ and $ψ₂$:
\begin{align*}
    \begin{bmatrix} x₁ \\ x₂ \end{bmatrix} &=
	\begin{bmatrix} 1 & 1 \\ 1 & -1 \end{bmatrix}
	\begin{bmatrix} ψ₁ \\ ψ₂ \end{bmatrix}
\end{align*}
\begin{align*}
    x₁ &= A₁t + B₁ + A₂\cos(ω't) + B₂\sin(ω't) \\
    x₂ &= A₁t + B₁ - A₂\cos(ω't) - B₂\sin(ω't)
\end{align*}
Applying the boundary conditions, we find that
\begin{align*}
    x₁(t) &= \frac 14 vt + \frac 12 L - \frac 12 L\cos(ω't) +
	\frac{v}{4ω'}\sin(ω't) \\
    x₂(t) &= \frac 14 vt + \frac 12 L + \frac 12 L\cos(ω't) -
	\frac{v}{4ω'}\sin(ω't)
\end{align*}
The distance $ℓ(t) = x₂(t) - x₁(t)$ between the two masses maximizes when
\begin{align*}
    \frac{dℓ}{dt} = 0 &= \frac{d}{dt} \left[ L\cos(ω't) -
	\frac{v}{2ω'}\sin(ω't) \right] \\
    t &= -\frac{1}{ω'} \arctan (\frac{v}{2Lω'})
\end{align*}
Plugging back into the function $ℓ(t)$,
\begin{align*}
    ℓ &= L\cos \left[ -\arctan (\frac{v}{2Lω'}) \right] - \frac{v}{2ω'}
	\sin \left[ -\arctan (\frac{v}{2Lω'}) \right] \\
    ℓ &= L \frac{2Lω'}{\sqrt{v² + 4L² {ω'}²}} + \frac{v}{2ω'}
	\frac{v}{\sqrt{v² + 4L² {ω'}²}} \\
    ℓ &= \frac{\sqrt{v² + 4L² {ω'}²}}{2ω'}
\end{align*}
Finally, substituting back in $ω' = \sqrt{2k/m}$ and simplifying, we get the
final solution that maximum distance between the two masses is
\begin{empheq}[box=\fbox]{align}
    ℓ &= \sqrt{L² + \frac{\frac 12 mv²}{8k}}
\end{empheq}
which agrees qualitatively with the fact that a larger spring constant should
stiffen the system and decrease the maximum displacement, while launching the
projectile with a greater velocity would increase it.

%%%%%%%%%%%%%%%%%%%%%%%%%%%%%%%%%%%%%%%%%%%%%%%%%%%%%%%%%%%%%%%%%%%%%%%%%%%%%%%
%%%% Problem 10
%%%%%%%%%%%%%%%%%%%%%%%%%%%%%%%%%%%%%%%%%%%%%%%%%%%%%%%%%%%%%%%%%%%%%%%%%%%%%%%
\problem{10}
\subsubsection{Question}
% Keywords
	\index{dimensional analysis!Freezing ice}
	\index{thermodynamics!Freezing ice}

Ice on a pond is \SI{10}{\cm} thick and the water temperature just below the
ice is \SI{0}{\celsius}. If the air temperature is \SI{-20}{\celsius}, by
how much will the ice thickness increase in 1 hour? Assuming that the air
temperature stays the same over a long period, how will the ice thickness
increase with time? Comment on any approximation that you make in your
calculation.

Density of ice ${}= \SI{0.9}{\g\per\cm\cubed}$

Thermal conductivity of ice ${}= \SI{0.0005}{\cal\per\cm\per\s\per\celsius}$

Latent heat of fusion of water ${}= \SI{80}{\cal\per\g}$

\subsubsection{Answer}

Since the thermal heat flow is a one dimensional problem, immediately
consider everything with respect to a small area element with its normal
perpendicular to the ice-water interface $dA$. Then we want to know how much
ice is generated on the surface of the ice. This small ice element's mass is
simply
\begin{align*}
    dm &= ρ\,dA\,dz
\end{align*}
where $dz$ is the thickness of the new ice layer. To generate this ice, the
latent heat of fusion must be conducted away, so the energy released is,
\begin{align*}
    dE_f &= L_f\,dm \\
    {} &= L_f ρ \,dA\,dz
\end{align*}

The energy flow is through the ice, and we expect this to increase with the
temperature differential across the ice sheet, suggesting that the thermal
conductivity $κ$ be multiplied by the temperature difference $ΔT$. Furthermore,
the ice will decrease the rate of heat flow as it becomes thicker, so the
quantity should also be divided by the thickness $z$. This gives
\begin{align*}
    \frac{κ ΔT}{z} &= \left[ \si{\cal\per\cm\squared\per\s}
	\right]
\end{align*}
This energy is flowing through a surface element $dA$, giving the power flow
due to heat as
\begin{align*}
    \frac{κΔT\,dA}{z} &= \left[ \si{\cal\per\s} \right]
\end{align*}

This power can be matched in units with the energy released from the ice
calculated above by taking the time derivative of $dE_f$, so equating the two
we have
\begin{align*}
    L_f ρ \,dA\frac{dz}{dt} &= \frac{κΔT\,dA}{z} \\
    ∫_{z₀}^{z₀+δz} z\,dz &= ∫_0^t \frac{κΔT}{L_f ρ}\,dt \\
    2z₀ δz + (δz)² &= \frac{κΔT}{L_f ρ}t
\end{align*}
Solving for the length the ice grows $δz$,
\begin{align*}
    δz &= \frac{-2z₀ ± \sqrt{4{z₀}² - 4(\frac{κΔT}{L_f ρ})t} }{2} \\
    δz &= z₀(1 ± \sqrt{1 - \frac{κΔT}{L_f ρ {z₀}²} t})
\end{align*}
The two roots give solutions $δz = \{ \SI{0.0501}{\cm}, \SI{19.950}{\cm} \}$.
Since the second root is unrealistic, we know that the solution must then be
\begin{empheq}[box=\fbox]{align}
    δz &= \SI{0.0501}{\cm} \quad\text{in an hour}
\end{empheq}

%%%%%%%%%%%%%%%%%%%%%%%%%%%%%%%%%%%%%%%%%%%%%%%%%%%%%%%%%%%%%%%%%%%%%%%%%%%%%%%
%%%% Problem 11
%%%%%%%%%%%%%%%%%%%%%%%%%%%%%%%%%%%%%%%%%%%%%%%%%%%%%%%%%%%%%%%%%%%%%%%%%%%%%%%
\problem{11}
\subsubsection{Question}
% Keywords
	\index{statistical mechanics!Carbon-14 dating}

Carbon-14 is produced by cosmic rays interacting with the nitrogen in the
Earth's atmosphere. It is eventaully incorporated into all living things,
and since it has a half-life of \SI{5730(40)}{\year}, it is useful for
dating archaeological specimens up to several tens of thousands of years
old. The radioactivity of a particular specimen of wood containing \SI{3}{\g}
of carbon was measured with a counter whose efficiency was
\SI{18}{\percent}; a count rate of \SI{12.8(1)}{\minute^{-1}} was measured.
It is known that in \SI{1}{\g} of living wood, there are
\SI{16.1}{\minute^{-1}} radioactive carbon-14 decays. What is the age of
this specimen, and its uncertainity? (Where errors are not quoted, they can
be assumed to be negligible).

\subsubsection{Answer}
The rate $N$ after a given time is given by the exponential decay formula
\begin{align*}
    N(t) &+= N₀ e^{-t/τ}
\end{align*}
Since we have the half-life $t_{1/2}$ instead of the decay constant $τ$, we
use the relation $t_{1/2} = τ\ln 2$ to simplify the expression instead to
\begin{align*}
    N(t) &= N₀ (\frac 12)^{t/t_{1/2}}
\end{align*}

The counter use has an efficiency of $ε = 0.18$, so the measured counting
rate $N_m$ must be corrected for that. Furthermore, the sample has a mass of
\SI{3}{\g} whereas we know the rate for a one gram sample, so we also
normalize the count rate by the mass of the sample. Plugging this all into
the exponential decay function above gives
\begin{align*}
    \frac{N_m}{3ε} &= N₀ (\frac 12)^{t/t_{1/2}}
\end{align*}
The only unknown left in the equation is the time, so solving for it,
\begin{align*}
    t &= t_{1/2} \log_{1/2} (\frac{N_m}{3εN₀}) \\
    t &= t_{1/2} \frac{\ln (\frac{N_m}{3εN₀})}{\ln 2} \\
    t &= \frac{t_{1/2}}{\ln 2} \ln (\frac{N_m}{3εN₀})
\end{align*}

To find the uncertainty, we note that only the quantities $N_m$ and $t_{1/2}$
have non-negligible uncertainties, so we propagate the errors only over
these two terms:
\begin{align*}
    {σ_t}² &= (-\frac{t_{1/2}}{N_m \ln 2})² {σ_{N_m}}² +
	(\frac{1}{\ln 2} \ln(\frac{3εN₀}{N_m}))² {σ_{t_{1/2}}}² \\
    σ_t &= \frac{t_{1/2}}{\ln 2} \sqrt{ (\frac{σ_{N_m}}{N_m})² +
	(\frac{σ_{t_{1/2}}}{t_{1/2}})² \left[ \ln(\frac{3εN₀}{N_m}) \right]²}
\end{align*}
Plugging in all the numbers, we get $t = \SI{4248.8435}{\year}$ and $σ_t =
\SI{161.717}{\year}$. The given uncertainties have a single significant digit,
so adding an extra significant figure to the uncertainity and matching decimal
places in the answer, we conclude that the sample has an age of
\begin{empheq}[box=\fbox]{align}
    t &= \SI{4250(160)}{\year}
\end{empheq}


\fallexam{2008}{1}
%%%%%%%%%%%%%%%%%%%%%%%%%%%%%%%%%%%%%%%%%%%%%%%%%%%%%%%%%%%%%%%%%%%%%%%%%%%%%%%
%%%% Problem 2
%%%%%%%%%%%%%%%%%%%%%%%%%%%%%%%%%%%%%%%%%%%%%%%%%%%%%%%%%%%%%%%%%%%%%%%%%%%%%%%
\problem{2}
\subsubsection{Question}
% Keywords
	\index{mechanics!Impulse on a rod}

If an impulse is delivered to the end of a uniform rod of length $ℓ$, lying on
a frictionless plane, how far will it travel while making one revolution? The
impulse is in the plane of the table and perpendicular to the rod.

\subsubsection{Answer}

For a given impulse $\vec J$, the change in the motion is $\vec J = Δ\vec p$.
If the rod start at rest, then the final momentum must be $\vec p = \vec J$.
This means the rod is moving laterally with a velocity
\begin{align*}
    V = \frac 1m \vec J
\end{align*}
which when integrated over a time $t$ gives the distance it has moved $\vec x$.
\begin{align*}
    \vec x = \frac 1m \vec J t
\end{align*}

The impulse also imparts a rotation on the rod because the force was not
applied at the rod's center of mass. The torque $\vec τ$ relates the force
to the angular momentum $\vec L$ by
\begin{align*}
    \vec r × \vec F &= \vec τ = \dot{\vec L}
\end{align*}
Integrating both sides of the equation, we can write the equation in terms of
the given impulse:
\begin{align*}
    \vec r × \int \vec F \,dt &= \int \dot{\vec L} \,dt \\
    \vec r × \vec J &= Δ\vec L
\end{align*}
Again, since the rod starts at rest, we know that the final angular momentum
must be
\begin{align*}
    \vec L = \vec r × \vec J
\end{align*}
The rotation about the rod's center of mass  occurs at a rate $\vec ω$
dependent on the moment of inertia $I = \frac{1}{12} mℓ²$, so
\begin{align*}
    \vec ω = \frac{12}{mℓ²} \vec r × \vec J
\end{align*}
We know that the impulse is applied perpendicular to the rod, so we can easily
integrate the expression in time and solve for the time it takes to revolve
$2π$ radians:
\begin{align*}
    θ = 2π &= \frac{12}{mℓ²} rJt \\
    t &= \frac{πmℓ²}{6 rJ}
\end{align*}

Plugging this back into the linear motion equation, the rod travels
\begin{align*}
    \vec x = \frac{1}{m} \vec J ⋅ \frac{πmℓ²}{6 rJ}
\end{align*}
where we can set $r = \frac 12 ℓ$ and therefore simplifies to
\begin{align}
    \boxed{
    \vec x = \frac{πℓ}{3} \hat J
    }
\end{align}
where $\hat J$ is the direction of the applied impulse.

%%%%%%%%%%%%%%%%%%%%%%%%%%%%%%%%%%%%%%%%%%%%%%%%%%%%%%%%%%%%%%%%%%%%%%%%%%%%%%%
%%%% Problem 3
%%%%%%%%%%%%%%%%%%%%%%%%%%%%%%%%%%%%%%%%%%%%%%%%%%%%%%%%%%%%%%%%%%%%%%%%%%%%%%%
\problem{3}
\subsubsection{Question}
% Keywords
	\index{electrodynamics!Properties of a magnetic field}

A time-indpendent magnetic field is given by $\vec B = 2bxy \,\hat ı +
ay² \,\hat ȷ$.
\begin{enumerate}[a)]
    \item
        What is the relationship between the constants $a$ and $b$?
    \item
        Determine the steady current density $J$ that gives rise to this field.
\end{enumerate}

\subsubsection{Answer}
For part (a), we realize that all magnetic fields must be divergenceless.
Therefore we can find the requirements on the constants $a$ and $b$ by
constraining the divergence to be zero.
\begin{align*}
    \vec ∇ ⋅ \vec B = 0 &= \frac{∂}{∂x}(2bxy) + \frac{∂}{∂y}(ay²) \\
    0 &= 2by + 2ay \\
    b &= -a
\end{align*}
Therefore the relation between the constants is that
\begin{align}
    \boxed {b = -a}
\end{align}

For the second part, we make use of Maxwell's equations. Assuming that none of
the field is due to a time-varying electric field, we make use of
\begin{align*}
    \vec ∇ × \vec B &= μ₀ \vec J
\end{align*}
to calculate the current that generates the field. Doing so, we find that the
solution is
\begin{align}
    \boxed{ \vec J = \frac{2a}{μ₀} x \,\hat k }
\end{align}

%%%%%%%%%%%%%%%%%%%%%%%%%%%%%%%%%%%%%%%%%%%%%%%%%%%%%%%%%%%%%%%%%%%%%%%%%%%%%%%
%%%% Problem 4
%%%%%%%%%%%%%%%%%%%%%%%%%%%%%%%%%%%%%%%%%%%%%%%%%%%%%%%%%%%%%%%%%%%%%%%%%%%%%%%
\problem{4}
\subsubsection{Question}
% Keywords
	\index{electrodynamics!Charges from multipole moments}

A set of four point charges $q₁$, $q₂$, $q₃$, and $q₄$ are arranged
collinearly along the $z$-axis at $z₁ = 0$, $z₂ = a$, $z₃ = 2a$, $z₄ = 4a$,
respectively and the resulting electric field at a distant point $\vec r$ ($r
≫ a$) decays \emph{faster} than $1/r³$. Determine the values of $q₁$ and $q₄$
which $q₂ = +2$ and $q₃ = +4$. Units for all charges are Coulombs.

\subsubsection{Answer}

Given that the electric field must fall off faster than $1/r³$, this
corresponds to a potential which drops off faster than $1/r²$. We know that
the monopole moment drops off like $1/r$ and the dipole like $1/r²$, so we
conclude that the first configuration which could satisfy the given
requirement is that of a quadrupole moment.

Making use of the fact that he monopole and dipole moments are vanishing, we
can use them to generate constraint equations for what the charges must be:
we have two unknown charges and the two equations will allow us to solve them.

For the monopole, the sum of all charges must simply equal zero. Therefore
we immediately know that
\begin{align*}
    0 &= q₁ + q₄ + 6 \\
    -6 &= q₁ + q₄
\end{align*}

The dipole moment (where we take the dipole considered at the origin) is given
by
\begin{align*}
    \vec p = \sum_i \vec{r_i} q_i
\end{align*}
This gives us the equation
\begin{align*}
    0 &= 10a + 4aq₄ \\
    q₄ &= -\frac 52
\end{align*}
The charge $q₁$ does not show up in the equation since it is located at the
origin. This lets us very simply then solve for $q₁$ as
\begin{align*}
    -6 &= q₁ - \frac 52
\end{align*}
Therefore, the solution is that the charges have values of
\begin{align}
    \boxed{ q₁ = -\frac 72 } \\
    \boxed{ q₄ = -\frac 52 }
\end{align}

%%%%%%%%%%%%%%%%%%%%%%%%%%%%%%%%%%%%%%%%%%%%%%%%%%%%%%%%%%%%%%%%%%%%%%%%%%%%%%%
%%%% Problem 5
%%%%%%%%%%%%%%%%%%%%%%%%%%%%%%%%%%%%%%%%%%%%%%%%%%%%%%%%%%%%%%%%%%%%%%%%%%%%%%%
\problem{5}
\subsubsection{Question}
% Keywords
	\index{quantum!Spectral emission line width}

The Lyman-α transition in atomic hydrogen has a wavelength $λ =
\SI{121.5}{\nm}$, and a transition rate of \SI{0.6e9}{\s^{-1}}. Estimate the
minimum value of $Δλ/λ$.

\subsubsection{Answer}

We can make an estimate of the spread $Δλ$ by making use of the Heisenberg
uncertainty relation for energy-time. Starting with the variation in
wavelength,
\begin{align*}
    Δλ &= λ - λ' \\
    {} &= \frac{hc}{E} - \frac{hc}{E'} \\
    {} &= \frac{hc(E' - E)}{E E'}
\intertext{Making use of the approximation that $E ≈ E'$,}
    {} &= \frac{hcΔE}{E²}
\end{align*}
Dividing by the frequency and substituting in the uncertainty relation $ΔEΔt =
\frac{ℏ}{2}$,
\begin{align*}
    \frac{Δλ}{λ} &= \frac{hc}{λ} ⋅ \frac{1}{E²}\frac{ℏ}{2Δt} \\
    {} &= \frac{λ}{4πcΔt}
\end{align*}
For the time, we estimate the transition rate is occuring as fast as it can
within the limits of the uncertainty relation, so we can let $Δt ≈ \SI{0.6e9}
{\s^{-1}}$. Plugging in the other values, we find the fractional line width
to be estimated as
\begin{align}
    \boxed{ \frac{Δλ}{λ} ≈ \num{1.935e-8} ≈ \text{1 part in 50 million} }
\end{align}

%%%%%%%%%%%%%%%%%%%%%%%%%%%%%%%%%%%%%%%%%%%%%%%%%%%%%%%%%%%%%%%%%%%%%%%%%%%%%%%
%%%% Problem 11
%%%%%%%%%%%%%%%%%%%%%%%%%%%%%%%%%%%%%%%%%%%%%%%%%%%%%%%%%%%%%%%%%%%%%%%%%%%%%%%
\problem{11}
\subsubsection{Question}
% Keywords
	\index{statistical mechanics!Radiometric dating from mass ratios}

A rock is found to contain \SI{4.20}{\mg} of ${}^{238}U$ and \SI{2.00}{\mg}
of ${}^{206}Pb$. Assume tha the rock contained no lead at the time of its
formation, so that all the lead now present is due to th decay of the
uranium orignally present in the rock. Find the age of the rock given that
the half-life of ${}^{238}U$ is \SI{4.47e9}{\year}. The decay times of all
intermediate elements are negligibly short and ignore any differences in the
binding energies.

\subsubsection{Answer}

From decay processes, we know that the uranium atom count will decrease as an
exponential according to
\begin{align*}
    N_U = N_{U0}e^{-t/τ}
\end{align*}
where $τ = t_{1/2}/\ln 2$. Likewise, the number of lead atoms will increase
according to
\begin{align*}
    N_{Pb} = N_{U0} (1 - e^{-t/τ})
\end{align*}
Solveing for $N_{U0}$ in the first equation and substituting it into the
second, we can solve for the time required to generate a specific number of
uranium and lead atoms in a sample.
\begin{align*}
    N_{Pb} &= N_U e^{t/τ} (1 - e^{-t/τ}) \\
    t &= τ \ln(\frac{N_{Pb}}{N_U} + 1) \\
    t &= \frac{t_{1/2}}{\ln 2} \ln(\frac{N_{Pb}}{N_U} + 1)
\end{align*}
We were only given the masses, though, so we approximate the mass of each
atom by the number of nucleons in the nucleus; each uranium atom has a mass
of $m_U = 238m_N$ making the $N_U$ atoms have a mass of $M_U = 238 N_U m_N$,
and similar for the lead. This gives us the final equation
\begin{align*}
    t &= \frac{t_{1/2}}{\ln 2} \ln(\frac{238}{206} \frac{M_{Pb}}{M_U} + 1)
\end{align*}
Plugging in all the numbers,
\begin{align}
    \boxed{ t = \SI{2.83e9}{\year} }
\end{align}

%%%%%%%%%%%%%%%%%%%%%%%%%%%%%%%%%%%%%%%%%%%%%%%%%%%%%%%%%%%%%%%%%%%%%%%%%%%%%%%
%%%% Problem 12
%%%%%%%%%%%%%%%%%%%%%%%%%%%%%%%%%%%%%%%%%%%%%%%%%%%%%%%%%%%%%%%%%%%%%%%%%%%%%%%
\problem{12}
\subsubsection{Question}
% Keywords
	\index{circuits!Current amplitude and phase in LRC circuit}

The applied AC voltage in the circuit is given by $V(t) = V₀ \sin ωt$, with 
a frequency fixed at $ω = 1/(LC)^{1/2}$. Determine the steady state 
amplitude and phase of the current through the resistor $R$. Express your 
answer in terms of the amplitude $V₀$ of the applied voltage and the other 
circuit parameters.

\begin{center}
	\vspace{\baselineskip}
	\begin{circuitikz}
		\resetparens
		\draw (0,-2)
		to [sV,l=$V(t)$] ++(0,4)
			-- ++(3,0)
		to [L,l=$L$] ++(0,-2)
			coordinate (split)
			-- ++(-1,0)
		to [C,l=$C$] ++(0,-2)
			-- (0,-2)
			(split) -- ++(1,0)
		to [R,l=$R$] ++(0,-2)
			-- (0,-2)
		;
	\end{circuitikz}
	\vspace{\baselineskip}
\end{center}

\subsubsection{Answer}

AC problems are simplified by using complex impedances, so we first convert 
the given voltage into a complex one:
\begin{align*}
	\tilde V(t) &= V₀ e^{iωt}
\end{align*}
where the physical solution can be recovered by keeping the imaginary 
component of the complex solution. Then to solve the problem, we realize 
that there is another complimentary circuit diagram which is helpful: the 
one with the resistor and capacitor replaced by an effective resistor 
(impedance). The circuit looks like
\begin{center}
	\vspace{\baselineskip}
	\begin{circuitikz}
		\resetparens
		\draw (0,-2)
		to [sV,l=$V(t)$] ++(0,4)
			-- ++(3,0)
		to [R,l=$Z_L$] ++(0,-2)
		to [R,l=$Z_{eff}$] ++(0,-2)
			-- (0,-2)
		;
	\end{circuitikz}
	\vspace{\baselineskip}
\end{center}
The inductor has been been replaced by an effective resistor with impedance 
$Z_L = iωL$. The effective resistor that replaced the capacitor and resistor 
is a complex impedance that is calculated the same as for traditional 
resistors in parallel:
\begin{align*}
	Z_{eff} &= ( \frac{1}{Z_C} + \frac{1}{Z_R} )^{-1} \\
		&= ( iωC + \frac{1}{R} )^{-1} \\
		&= \frac{R}{iωCR + 1}
\end{align*}
Now making use of Kirchoff's loop rule on this simplified circuit where the 
total current passing through the voltage source is labeled $\tilde I₀$,
\begin{align*}
	0 &= \tilde V - \tilde I₀ (Z_L + Z_{eff}) \\
	\tilde V &= (iωL + \frac{R}{iωCR + 1}) \tilde I₀ \\
	\tilde V &= \frac{R(1 - ω²LC) + iωL}{iωRC + 1} \tilde I₀
\intertext{The first term in the numerator goes to zero since $ω² = 1/LC$,
leaving}
	\tilde I₀ &= \frac{iωRC + 1}{iωL} V₀ e^{iωt}
\end{align*}

To isolate the current passing through the resistor, we return to the 
original unsimplified circuit diagram and apply Kirchoff's loop rule to only 
the inner loop. If we define the current through capacitor to be $I₁$ and
through the resistor to be $I₂$, we get
\begin{align*}
	0 &= -\tilde I₂ Z_R + \tilde I₁ Z_C \\
	\tilde I₁ &= \frac{Z_R}{Z_C} I₂ \\
	\tilde I₁ &= iωRC I₂ \\	
\end{align*}
Remembering the the current passing into a junction must be conserved, we know
that $I₀ = I₁ + I₂$ and therefore,
\begin{align*}
	\tilde I₀ &= iωRC \tilde I₂ + \tilde I₂ \\
	\tilde I₂ &= \frac{1}{iωRC + 1} \tilde I₀
\end{align*}
Inserting the solution for $I₀$ from the previous part leaves
\begin{align*}
	\tilde I₂ &= \frac{V₀}{iωL} e^{iωt}
\end{align*}
To prepare for finding the physical solution, we transform the coefficient
complex polar form.
\begin{align*}
	\tilde I₂ &= \left|-\frac{iV₀}{ωL}\right| e^{i\arg(-iV₀/ωL)} e^{iωt} \\
		&= \frac{V₀}{ωL} e^{-iπ/2} e^{iωt}
\end{align*}
Therefore taking the imaginary part of the solution,
\begin{align}
	\boxed{
	I_R(t) = V₀ \sqrt{\frac{C}{L}} \sin(ωt - \frac π2)
	}
\end{align}
The current's amplitude is $V₀\sqrt{C/L}$ and has a phase of $-π/2$ with
respect to the voltage.


\fallexam{2011}{1}
%%%%%%%%%%%%%%%%%%%%%%%%%%%%%%%%%%%%%%%%%%%%%%%%%%%%%%%%%%%%%%%%%%%%%%%%%%%%%%%
%%%% Problem 1
%%%%%%%%%%%%%%%%%%%%%%%%%%%%%%%%%%%%%%%%%%%%%%%%%%%%%%%%%%%%%%%%%%%%%%%%%%%%%%%
%\subsection{Problem 1}
\problem{1}
\subsubsection{Question}
% Keywords
	\index{mechanics!Pendulum in Elevator}
	\index{pendulum!Pendulum in Elevator}

An elevator operator in a skyscraper, being a very meticulous person, put a
pendulum clock on the wall of the elevator to make sure that he spends exactly 8
hours a day at his work place. Over the course of his work day, he records that
the time during which the elevator has acceleration $a$ is exactly equal to the
time during which it has acceleration $-a$. Does the elevator operator work, in
actual time, (1) more than 8 hours, (2) exactly 8 hours, or (3) less than 8
hours? Why?

\subsubsection{Answer}
The nominal period of a pendulum is
\begin{align*}
	T_{nom} &= 2π \sqrt\frac{ℓ}{g}
\end{align*}
but within the elevator, the acceleration $g$ is not going to be constant and
will rather depend on the acceleration of the elevator. Therefore,
\begin{align*}
	T_↑ &= 2π\sqrt\frac{ℓ}{g+a} & T_↓ &= 2π\sqrt\frac{ℓ}{g-a}
\end{align*}
for the upward and downward cases, respectively.

Since the elevator operator observed that equal time was spent going up as was
spent going down, so he must have observed $N$ oscillations in both cases. In
order to compare to the actual time, we simply compare the elevator's total time
measurement with that of a stationary clock.
\begin{align*}
	NT_↑ + NT_↓ \stackrel{?}{=} 2NT_{nom}
\end{align*}

\begin{align*}
	2πN\sqrt\frac{ℓ}{g+a} + 2πN\sqrt\frac{ℓ}{g-a}
		&\stackrel{?}{=} 4πN\sqrt\frac{ℓ}{g}
		\\
	\sqrt\frac{1}{g+a} + \sqrt\frac{1}{g-a} &\stackrel{?}{=} 2\sqrt\frac{1}{g}\\
	\sqrt\frac{g}{g+a} + \sqrt\frac{g}{g-a} &\stackrel{?}{=} 2 \\
\end{align*}
Use the test value $a=5$ for comparison (with $g = 10$)
\begin{align}
	\boxed{
	2.23 > 2
	}
\end{align}
Therefore the elevator operator actually spends more than 8 hours in the
elevator during his shift.

%%%%%%%%%%%%%%%%%%%%%%%%%%%%%%%%%%%%%%%%%%%%%%%%%%%%%%%%%%%%%%%%%%%%%%%%%%%%%%%
%%%% Problem 2
%%%%%%%%%%%%%%%%%%%%%%%%%%%%%%%%%%%%%%%%%%%%%%%%%%%%%%%%%%%%%%%%%%%%%%%%%%%%%%%
\problem{2}
\subsubsection{Question}
% Keywords
	\index{mechanics!Central Forces}
	\index{Lagrangian!Central Forces}
	\index{orbits!Central Forces}

A classical particle is subject to an attractive central force proportional to
$r^α$, where $r$ is the radius and $α$ is a constant. Show by perturbation
analysis what is required of $α$ in order for the particle to have a stable
circular orbit.

\subsubsection{Answer}
Construct the Lagrangian for the system in order to determine the equations of
motion for the given central force (noting that we were given the \emph{force}
so we need to make an appropriate potential).
\begin{align*}
	T &= \frac{1}{2}m ( \dot r² + r²\dot θ² )
		& V &= \frac{k}{α+1}r^{α+1}
\end{align*}
\begin{align*}
	\sL &= \frac{1}{2}m\dot r² + \frac{1}{2}mr²\dot θ² - \frac{k}{α+1}r^{α+1}
\end{align*}
Conservation of angular momentum is a consequence of the $θ$ and $\dot θ$
coordinates:
\begin{align*}
	0 &= \frac{∂\sL}{∂θ} - \frac{d}{dt} \left[ \frac{∂\sL}{∂\dot θ} \right] \\
	0 &= \frac{d}{dt} \left[ mr²\dot θ \right] \\
\intertext{Nothing that}
	ℓ &= \left|\frac{\vec r × \vec p}{m}\right| = r²\dot θ
\intertext{we can say that}
	\dot θ &= \frac{ℓ}{r²}
\end{align*}

Then returning to the $r$ and $\dot r$ coordinates in the Lagrangian,
\begin{align*}
	\frac{∂\sL}{∂r} &= mr\dot θ² - kr^α &
		\frac{∂\sL}{∂\dot r} &= m\dot r
	\\
	{}&{}&
	\frac{d}{dt}\left[ \frac{∂\sL}{∂\dot φ} \right]
		&= m \ddot r
\end{align*}
Putting the differential equation together and substituting for the angular
momentum per unit mass gives
\begin{align}
	m\ddot r &= \frac{mℓ²}{r³} - kr^α
\end{align}

In the case that the orbit is circular, $r$ must be a constant, so let $r = a$
and note that $\ddot r = 0$ necessarily.
\begin{align*}
	\frac{mℓ²}{a³} &= ka^α
\end{align*}

Returning to the differential equation, let the actual distance $r$ be a
perturbation from a circular orbit, and Taylor expand in $x$ where $x = r - a$.
\begin{align*}
	m\ddot x &= \frac{mℓ²}{a³} (1 + \frac{x}{a} )^{-3} -
		ka^α \left(1 + \frac{x}{a} \right)^α \\
	m\ddot x &≈ \frac{mℓ²}{a³} (1 - 3\frac{x}{a} + \ldots ) -
		ka^α (1 + α\frac{x}{a} + \ldots ) \\
	m\ddot x &≈ ka^α (1 - 3\frac{x}{a} ) -
		ka^α (1 + α\frac{x}{a} ) \\
	m\ddot x &≈ -3ka^α \frac{x}{a} - αka^α\frac{x}{a} \\
	m\ddot x &≈ -ka^{α-1} (3+α)x \\
\end{align*}

To form a stable orbit, the coefficient on $x$ must be negative, giving a simple
harmonic solution. Therefore $3+α > 0$ to keep the coefficient negative and
\begin{align}
	\boxed{
	a > -3
	}
\end{align}

%%%%%%%%%%%%%%%%%%%%%%%%%%%%%%%%%%%%%%%%%%%%%%%%%%%%%%%%%%%%%%%%%%%%%%%%%%%%%%%
%%%% Problem 3
%%%%%%%%%%%%%%%%%%%%%%%%%%%%%%%%%%%%%%%%%%%%%%%%%%%%%%%%%%%%%%%%%%%%%%%%%%%%%%%
\problem{3}
\subsubsection{Question}
% Keywords
	\index{electrostatics!Charges in Conductor Cavities}
	\index{Gauss' Law!Charges in Conductor Cavities}

A neutral conductor A with a spherical outer surface of radius $R$ contains
three cavities B, C, and D, but is solid otherwise. B and C are spherical, and
D is hemispherical. Without touching A, positive charges $q_B$ and $q_C$ are
introduced at the centers of B and C, respectively.
\begin{enumerate}
	\item
		Give the amount and the distribution of the induced charges on the
		surfaces of A, B, C, and D.
	\item
		Now another positive charge $q_E$ is introduced at a distance $r > R$
		from the center of A. Describe qualitatively the distribution of
		induced charges on the surfaces of A, B, C, and D.
	\item
		Give the amount of the induced charges on the surfaces of A, B, C, and
		D for the situation in (2).
\end{enumerate}

\subsubsection{Answer (1)}
An ideal conductor will not support an electric field inside the solid, so each
of cavities B and C will have a surface charge to cancel the electric fields
emminating from $q_B$ and $q_C$ respectively.
\begin{itemize}
	\item
		Cavity B will have a uniform surface charge density of $-q_B/4πr²_B$,
		where $r_B$ is the radius of cavity B, with total induced charge $-q_B$
		(because of symmetry and use of a Gaussian surface).
	\item
		Cavity C will have a uniform surface charge density of $-q_C/4πr²_C$,
		where $r_C$ is the radius of cavity B, with total induced charge $-q_C$
		(because of symmetry and use of a Gaussian surface).
\end{itemize}
Cavity D will not have a surface charge since a Guassian surface coincident with
its boundary contains no charge.

The surface A will have total charge $q_B + q_C$ with uniform surface charge
density of $(q_B + q_C) / 4πR²$ in accordance with the symmetry of a Gaussian
surface containing the sphere as well as properties of an ideal conductor.

\subsubsection{Answer (2)}
The surfaces B, C, and D will remain unaffected since the surrounding conductor
shields the cavities from electric fields produced by charge $q_E$. The
distribution on surface A will shift so that the negative charge concentration
is greatest on the side nearest to $q_E$ with an increasingly positive
distribution towards the opposite side.

\subsubsection{Answer (3)}
The surface of A will still contain the same total charge $q_B + q_C$ since only
a redistribution of induced charges occurred along the surface. Similarly,
because surface B, C, and D are shielded from the electric field of $q_E$ by
conductor A, the total charges along their surfaces remains unchanged as well.

%%%%%%%%%%%%%%%%%%%%%%%%%%%%%%%%%%%%%%%%%%%%%%%%%%%%%%%%%%%%%%%%%%%%%%%%%%%%%%%
%%%% Problem 4
%%%%%%%%%%%%%%%%%%%%%%%%%%%%%%%%%%%%%%%%%%%%%%%%%%%%%%%%%%%%%%%%%%%%%%%%%%%%%%%
\problem{4}
\subsubsection{Question}
% Keywords
	\index{electrostatics!Dielectric Breakdown of Air}

The dielectric strength of air at standard temperature and pressure is
\SI{3e6}{\V\per\m}. What is the maximum intensity in units of \si{\W\per\m^2}
for a monocromatic laser that can be used in the laboratory?

\subsubsection{Answer}
Failure of a dielectric occurs when the energy density in the dielectric is
great enough to overcome the ionization energy of the constituent atoms. This
suggests that an electric field of greater than \SI{3e6}{\V\per\m} would cause
this ionization to occur.

Starting here, We can calculate the energy density of the electric field at any
point in space by
\begin{align*}
	U_{em} = \frac{ε₀}{2}E²
\end{align*}
(where we've used the vacuum energy density since air differs very little from
the vacuum permittivity).

Then the power transmitted by the laser is $P = cU_{em}$, so plugging in the
numbers,
\begin{align*}
	P &= \frac{1}{2}
		\left( \SI[per-mode=fraction]{8.854e-12}
			{\coulomb\squared\per\N\per\m\squared} \right)
		\left( \SI{3e8}{\m\per\s} \vphantom{\frac{V}{V}} \right)
		\left( \SI{3e6}{\V\per\m} \right)² \\
	P &= \SI{1.19e10}{\W\per\m\squared}
\end{align*}

\begin{center}
	\fbox{The maximum power of a laser usable in the lab is \SI{1.19e10}
	{\W\per\m\squared}.}
\end{center}

%%%%%%%%%%%%%%%%%%%%%%%%%%%%%%%%%%%%%%%%%%%%%%%%%%%%%%%%%%%%%%%%%%%%%%%%%%%%%%%
%%%% Problem 5
%%%%%%%%%%%%%%%%%%%%%%%%%%%%%%%%%%%%%%%%%%%%%%%%%%%%%%%%%%%%%%%%%%%%%%%%%%%%%%%
\problem{5}
\subsubsection{Question}
% Keywords
	\index{particle!Proton Collision}
	\index{relativity!Proton Collision}

What is the minimum energy of the projectile proton required to induce the
reaction $p + p \rightarrow p + p + p + \bar p$ if the target proton is at rest?

\subsubsection{Answer}
Energy and momentum must be conserved. At the minimum allowed energy, the
resultant 4 proton/anti-protons will be colinear with no relative momentum with
respect to one another, so the momenum equation in the lab frame is simply
\begin{align}
	p_i = 4p_f
\end{align}
Similarly, the resultant (anti-)protons are indistinguishable, so they will
all have equivalent energy $E_f$. The initial protons have different energies
since one is at rest in the lab frame while the other is moving, leading to
the energy equation
\begin{align*}
	\sqrt{p²_i c² + m²_p c⁴} + m_p c² &= 4\sqrt{p²_f c² + m²_p c⁴}
\end{align*}
Substituting the momentum relation into the equation, squaring, and simplifying,
\begin{align*}
	\sqrt{16p²_f c² + m²_p c⁴} + m_p c² &= 4\sqrt{p²_f c² + m²_p c⁴} \\
	16p²_f c² + m²_p c⁴ + m²_p c⁴ + 2\sqrt{m²_p c⁴(16p²_f c² + m²_p c⁴)}
		&= 16p²_f c² + 16m²_p c⁴ \\
	2\sqrt{m²_p c⁴(16p²_f c² + m²_p c⁴)} &= 14m²_p c⁴ \\
	16p²_f c² + m²_p c⁴ &= 49m²_p c⁴ \\
	p²_f &= 3m²_p c⁴
\end{align*}
Therefore,
\begin{align*}
	p²_i &= 48m²_p c⁴ \\
\intertext{and}
	E₁ &= \sqrt{49m²_p c⁴}
\end{align*}
\begin{align}
	\boxed{
	E₁ ≈ \SI{6.567}{\GeV\per c\squared}
	}
\end{align}

%%%%%%%%%%%%%%%%%%%%%%%%%%%%%%%%%%%%%%%%%%%%%%%%%%%%%%%%%%%%%%%%%%%%%%%%%%%%%%%
%%%% Problem 8
%%%%%%%%%%%%%%%%%%%%%%%%%%%%%%%%%%%%%%%%%%%%%%%%%%%%%%%%%%%%%%%%%%%%%%%%%%%%%%%
\problem{8}
\subsubsection{Question}
% Keywords
	\index{thermodynamics!Atmospheric Scale Height (Pressure)}

Assume that the atmosphere near the earth's surface is in approximate
hydrostatic equilibrium, where any movement of air parcels is gentle and
adiabatic. Find an expression for the pressure $P$ of the atmosphere as a
function of the height $z$.

\subsubsection{Answer}
Note that the pressure at a given point is due to the mass of air above the
given point. Then by moving an infinitesimal distance vertically, the total
mass is changed by the density of the air (which is affected by the
gravitational force). This leads to the differential equation
\begin{align*}
	\frac{dP}{dz} &= ρg
\end{align*}
Then using the ideal gas equation
\begin{align*}
	PV &= N k_B T \\
\intertext{multiply and divide by the average molecular mass $m$ of the air (in
\si{\kg}) which combined with the number of molecules $N$ gives the total mass}
	PV &= (Nm) \frac{1}{m} k_B T \\
\intertext{and then divide by the volume to get the ideal gas equation in terms
of the mass density}
	P &= \frac{Nm}{V} \frac{1}{m} k_B T \\
	P &= ρ \frac{k_B T}{m} \\
	ρ &= \frac{P m}{k_B T}
\end{align*}
Finally, substitute this into the differential equation above and solve to
get the atmospheric scale height equation.
\begin{align*}
	\frac{dP}{dz} ={}& \frac{P m}{k_B T}g \\
	\frac{dP}{P} ={}& \frac{mg}{k_B T} dz
\end{align*}
\begin{align}
	\boxed{
	P(z) = P₀ e^{z/ξ}
		\quad\quad\text{where }ξ = \frac{k_B T}{mg}
	}
\end{align}


\fallexam{2011}{2}
%%%%%%%%%%%%%%%%%%%%%%%%%%%%%%%%%%%%%%%%%%%%%%%%%%%%%%%%%%%%%%%%%%%%%%%%%%%%%%%
%%%% Problem 1
%%%%%%%%%%%%%%%%%%%%%%%%%%%%%%%%%%%%%%%%%%%%%%%%%%%%%%%%%%%%%%%%%%%%%%%%%%%%%%%
\subsection{Problem 1}

\subsubsection{Question}
Mass $m₁$ moves freely along a fixed, long, horizontal rod. The position of
$m₁$ on the rod is $x$. A massless string of length $ℓ$ is attached to $m₁$
at the end and to mass $m₂$ at the other. Mass $m₂$ executes pendulum motion
in the vertical plane containing the rod.
\begin{enumerate}
	\item
		Find the Lagrangian of the system.
	\item
		Derive the equations of motion and the corresponding conservation laws.
	\item
		Assume that $x(0)=x₀$, $\dot x(0)=0$, $φ(0)=φ₀$ $(|φ₀| ≪ 1)$,
		and $\dot φ(0)=0$. Find $x(t)$ and $φ(t)$ for $t > 0$.
\end{enumerate}


\subsubsection{Answer (1)}
For the sliding support mass $m₁$:
\begin{align*}
	T₁ &= \frac{1}{2}m₁ \dot{x}² \\
	V₁ &= 0
\end{align*}
For the pendulum mass $m₂$:
\begin{align*}
	T₂ &= \frac{1}{2}m₂ \dot{y}² + \frac{1}{2}m₂(\dot x + \dot x₂)² \\
	V₂ &= -m₂gy₂ \\
\intertext{Then using $x₂ = ℓ\sin φ$ and $y₂ = -ℓ\cos φ$,}
	T₂ &= \frac{1}{2}m₂ \left( ℓ²\dot φ² + \dot x² + 2ℓ\dot φ\dot x \cos φ
		\right) \\
	V₂ &= -m₂ g y₂
\end{align*}
Putting the Lagrangian together equals the first line. Applying the small
angle approximation gives the second line where the kinetic energy term
involving $\cos φ$ can be simply expanded as $\cos φ ≈ 1$, but the potential
energy term must be expanded to second order so that $\cos φ ≈ 1 -
\frac{1}{2}φ²$.

\begin{empheq}[box=\fbox]{align}
	\sL &= \frac{1}{2} (m₁ + m₂)\dot x² + \frac{1}{2} m₂ \left(
		ℓ²\dot φ² + 2ℓ\dot φ\dot x\cos φ \right) + m₂gℓ\cos φ
	\\
	\sL &≈ \frac{1}{2} (m₁ + m₂)\dot x² + \frac{1}{2} m₂ \left(
		ℓ²\dot φ² + 2ℓ\dot φ\dot x \right) + m₂gℓ - \frac{1}{2}m₂gℓφ²
\end{empheq}

\subsubsection{Answer (2)}
Constructing the Euler-Lagrange equations for $x$ and $\dot x$:
\begin{align*}
	\frac{∂\sL}{∂x} &= 0 &
		\frac{∂\sL}{∂\dot x} &= (m₁ + m₂)\dot x + m₂ℓ\dot φ
	\\
	{}&{}&
	\frac{d}{dt}\left[ \frac{∂\sL}{∂\dot x} \right]
		&= (m₁ + m₂)\ddot x + m₂ℓ\ddot φ
\end{align*}
\begin{align}
	(m₁ + m₂)\ddot x + m₂ℓ\ddot φ = 0
\end{align}
and for $φ$ and $\dot φ$:
\begin{align*}
	\frac{∂\sL}{∂φ} &= -m₂gℓφ &
		\frac{∂\sL}{∂\dot φ} &= m₂ℓ²\dot φ + m₂ℓ\dot x
	\\
	{}&{}&
	\frac{d}{dt}\left[ \frac{∂\sL}{∂\dot φ} \right]
		&= m₂ℓ²\ddot φ + m₂ℓ\ddot x
\end{align*}
\begin{align}
	-m₂gℓφ - m₂ℓ²\ddot φ + m₂ℓ\ddot x = 0
\end{align}

The equations of motion are:
\begin{empheq}[box=\fbox]{align}
	\ddot x + \frac{m₂}{m₁+m₂} ℓ \ddot φ &= 0 \\
	\ddot φ + \frac{1}{ℓ}\ddot x + \frac{g}{ℓ}φ &= 0
\end{empheq}

\subsubsection{Answer (3)}
Solve for $\ddot x$ and substitute into the other differential equation
\begin{align}
	\ddot φ - \frac{1}{ℓ}\frac{m₂}{m₁+m₂} ℓ \ddot φ + \frac{g}{ℓ}φ &= 0\nonumber
	\\
	\frac{m₁}{m₁+m₂} \ddot φ + \frac{g}{ℓ}φ &= 0\nonumber
	\\
	\ddot φ + \frac{g}{ℓ}\frac{m₁+m₂}{m₁} φ &= 0
\end{align}
This is just the differential equation for a simple harmonic oscillator, so
considering the given boundary conditions,
\begin{empheq}[box=\fbox]{align}
	\begin{split}
		φ(t) ={}& φ₀\cos(ωt) \\
		{}&\text{where } ω² = \frac{g}{ℓ}\frac{m₁+m₂}{m₁}
	\end{split}
\end{empheq}

Then differentiating $φ(t)$ twice and substituting into the first equation,
\begin{align*}
	\ddot x &= ℓφ₀ω²\frac{m₂}{m₁+m₂}\cos(ωt)
\end{align*}
Then integrating twice and applying the boundary conditions,
\begin{empheq}[box=\fbox]{align}
	x(t) &= x₀ - \frac{g}{ω²}\frac{m₂}{m₁}\cos(ωt)
\end{empheq}



\springexam{2012}{1}
%%%%%%%%%%%%%%%%%%%%%%%%%%%%%%%%%%%%%%%%%%%%%%%%%%%%%%%%%%%%%%%%%%%%%%%%%%%%%%%
%%%% Problem 1
%%%%%%%%%%%%%%%%%%%%%%%%%%%%%%%%%%%%%%%%%%%%%%%%%%%%%%%%%%%%%%%%%%%%%%%%%%%%%%%
\problem{1}
\subsubsection{Question}
% Keywords
	\index{quantum!Significance in limits}

For a many particle system of weekly interacting particles, will quantum
effects be more important for (a) high densities or low densities and (b)
high temperatures or low temperatures for a system. Explain your answers in
terms of the de Broglie wavelength $λ$ defined as $λ² ≡ h²⁄(3mk_BT)$ where
$m$ is the mass of the particles and $k_b$ Boltzmann's constant.

\subsubsection{Answer}
\renewcommand{\labelenumi}{(\alph{enumi})}
\begin{enumerate}
	\item
		High density — The de Broglie wavelength gives a ``size'' of the
		particle, and in the high density limit, the wavefunctions overlap
		significantly so quantum effects and interactions are critical to
		the behavior of the system.
	\item
		Low temperature — Since $λ² \propto T^{-1}$, as $T → 0$, $λ$ increases
		so that again the wavefunctions overlap and quantum effects are
		significant.
\end{enumerate}

%%%%%%%%%%%%%%%%%%%%%%%%%%%%%%%%%%%%%%%%%%%%%%%%%%%%%%%%%%%%%%%%%%%%%%%%%%%%%%%
%%%% Problem 2
%%%%%%%%%%%%%%%%%%%%%%%%%%%%%%%%%%%%%%%%%%%%%%%%%%%%%%%%%%%%%%%%%%%%%%%%%%%%%%%
\problem{2}
\subsubsection{Question}
% Keywords
	\index{quantum!Helium ionization}

The ground state energy of Helium is \SI{-79}{\eV}. What is its ionization
energy, which is the energy required to remove just one electron?

\subsubsection{Answer}
Using the Hydrogen solution with modifications for single-electron atoms of
higher $Z$, we know that the ground state energy of singly ionized Helium is
\begin{align*}
	E_{He}^{1} = 2² (\SI{-13.6}{\eV}) = \SI{-54.4}{\eV}
\end{align*}
Therefore, the difference between the singly-ionized and neutral ground state
energies gives the first ionization energy of the Helium atom.
\begin{align}
	\boxed{
	E_i = \SI{-24.6}{\eV}
	}
\end{align}

%%%%%%%%%%%%%%%%%%%%%%%%%%%%%%%%%%%%%%%%%%%%%%%%%%%%%%%%%%%%%%%%%%%%%%%%%%%%%%%
%%%% Problem 3
%%%%%%%%%%%%%%%%%%%%%%%%%%%%%%%%%%%%%%%%%%%%%%%%%%%%%%%%%%%%%%%%%%%%%%%%%%%%%%%
\problem{3}
\subsubsection{Question}
% Keywords
	\index{dimensional analysis!Vacuum (Casimir) force}

It is known that the force per unit area ($F/A$) between two neutral
conducting plates due to polarization fluctuations of the vacuum (namely,
the Casimir force) is a function of $h$ (Planck's constant), $c$ (speed of
light), and $z$ (distance between the plates) only. Using only dimensional
analysis, obtain $F/A$ as a function of $h$, $c$, and $z$.

\subsubsection{Answer}
The units of $F/A$ are
\begin{align*}
	\frac{F}{A} &= \left[ \frac{\si{\kg}}{\si{\m\s\squared}} \right]
\end{align*}
The \si{\kg} suggests a factor proportional to $h$, making the equation
\begin{align*}
	\frac{F}{A} &\sim \left[ \frac{1}{\si{\m\cubed\s}} \right] h \\
\intertext{Accounting for the factor of seconds requires a $c$:}
	\frac{F}{A} &\sim \left[ \frac{1}{\si{\m\tothe{4}}} \right] hc \\
\intertext{Finally, account for all the factors of distance:}
	\frac{F}{A} &\sim \frac{hc}{z⁴} \\
\end{align*}
Therefore,
\begin{align}
	\boxed{
	\frac{F}{A} \sim \frac{hc}{z⁴}
	}
\end{align}

%%%%%%%%%%%%%%%%%%%%%%%%%%%%%%%%%%%%%%%%%%%%%%%%%%%%%%%%%%%%%%%%%%%%%%%%%%%%%%%
%%%% Problem 4
%%%%%%%%%%%%%%%%%%%%%%%%%%%%%%%%%%%%%%%%%%%%%%%%%%%%%%%%%%%%%%%%%%%%%%%%%%%%%%%
\problem{4}
\subsubsection{Question}
% Keywords
	\index{circuits!Parallel capacitors with switches}

In the circuit diagram opposite, initially the two identical capacitors with
capacitance $C$ are uncharged. The connections between the components are
all made with short copper wires. The battery is an ideal EMF and supplies a
voltage $V$.
\begin{enumerate}
	\item
		At first Switch A is closed and Switch B is kept open. What is the
		final sotred energy on capacitor $C_a$?
	\item
		Switch A is opened and afterwards Switch B is closed. What is the
		final energy stored in both capacitors?
	\item
		Provide a physical explanation for any difference between the
		results of parts (a) and (b), if there is one.
\end{enumerate}
\begin{center}
	\begin{circuitikz}
		\draw
			% Draw the battery
			(0,-1) to [battery=$V$] ++(0,2)
			% and then the switches
			to [cspst=$A$] ++(2,0)
				coordinate (between)
			to [cspst=$B$] ++(2,0)
			% then down through capacitor B
			to [capacitor=$C_b$] ++(0,-2)
			% And complete the outer loop
			to [short] (0,-1);
		;
		% Go back and draw capacitor A
		\draw (between) to [capacitor=$C_a$] ++(0,-2);
	\end{circuitikz}
\end{center}

\subsubsection{Answer}
\begin{enumerate}
	\item
		Initially, the right side of the circuit with $C_b$ can be ignored,
		so the total energy is simply the energy stored within $C_a$.
		\begin{align}
			\boxed{
			E = \frac 12 CV²
			}
		\end{align}
	\item
		The system is now effectively just the two capacitors on the right.
		Because the voltage difference is supported across both capacitors,
		the system can be modeled as an effective capacitor in parallel
		\begin{align*}
			C_{eff} &= 2C
		\end{align*}
		The total charge stored by the capacitors must remain the same when
		switching from Switch A being closed to Switch B. Initially,
		\begin{align*}
			Q &= CV_i
		\end{align*}
		and afterwards it is
		\begin{align*}
			Q &= C_{eff}V = 2CV_f
		\end{align*}
		so the final voltage across the capacitors is
		\begin{align*}
			V_f &= \frac 12 V_i
		\end{align*}
		This means the total energy is
		\begin{align*}
			E &= \frac 12 C_{eff} {V_f}²
		\end{align*}
		\begin{align}
			\boxed{
			E = \frac 14 CV²
			}
		\end{align}
	\item
		The energy is dissipated (heat, fields, etc).
\end{enumerate}

%%%%%%%%%%%%%%%%%%%%%%%%%%%%%%%%%%%%%%%%%%%%%%%%%%%%%%%%%%%%%%%%%%%%%%%%%%%%%%%
%%%% Problem 5
%%%%%%%%%%%%%%%%%%%%%%%%%%%%%%%%%%%%%%%%%%%%%%%%%%%%%%%%%%%%%%%%%%%%%%%%%%%%%%%
\problem{5}
\subsubsection{Question}
% Keywords
	\index{orbits!Angular momentum of a planet}
	\index{mechanics!Angular momentum of a planet}

A planet of mass $m$ moves around the sun, mass $M$, in an elliptical orbit
with minimum and maximum distances of $r₁$ and $r₂$, respectively. Find the
angular momentum of the planet relative to the center of the sun in terms of
these quantities and the gravitational constant $G$.

\subsubsection{Answer}
We solve the problem using conservation of energy since we know that stable
elliptical orbits have constant energy. The generic equation is
\begin{align*}
	E &= \frac{L²}{2I} - \frac{GMm}{r}
\end{align*}
where $L$ is the angular momentum and $I$ the moment of inertia. Substituting
for the values at both $r₁$ and $r₂$ and equating,
\begin{align*}
	\frac{L²}{2m{r₁}²} - \frac{GMm}{r₁} &= \frac{L²}{2m{r₂}²} - \frac{GMm}{r₂}\\
	\frac{L²}{2m}(\frac{1}{{r₁}²} - \frac{1}{{r₂}²}) &=
		GMm(\frac{1}{r₁} - \frac{1}{r₂})
\end{align*}
which leads to the solution
\begin{align}
	\boxed{
	L = \sqrt{ \frac{2GMm² r₁ r₂}{r₁ + r₂} }
	}
\end{align}

%%%%%%%%%%%%%%%%%%%%%%%%%%%%%%%%%%%%%%%%%%%%%%%%%%%%%%%%%%%%%%%%%%%%%%%%%%%%%%%
%%%% Problem 6
%%%%%%%%%%%%%%%%%%%%%%%%%%%%%%%%%%%%%%%%%%%%%%%%%%%%%%%%%%%%%%%%%%%%%%%%%%%%%%%
\problem{6}
\subsubsection{Question}
% Keywords
	\index{mechanics!Central Forces}
	\index{Lagrangian!Central Forces}
	\index{orbits!Central Forces}

A particle moves in a circular orbit under the influence of a central force
that varies as the $n$-th power of the distance. Show that this motion is
unstable if $n < -3$. (Hint: Consider the centrifugal potential.)

\subsubsection{Answer}
See solution for \nameref{prob:F2011I02} with the condition inverted so
that \emph{in}stability is $n < -3$ rather than stability requiring $n > -3$.

%%%%%%%%%%%%%%%%%%%%%%%%%%%%%%%%%%%%%%%%%%%%%%%%%%%%%%%%%%%%%%%%%%%%%%%%%%%%%%%
%%%% Problem 7
%%%%%%%%%%%%%%%%%%%%%%%%%%%%%%%%%%%%%%%%%%%%%%%%%%%%%%%%%%%%%%%%%%%%%%%%%%%%%%%
\problem{7}
\subsubsection{Question}
% Keywords
	\index{thermodynamics!Isentropic compression}

A classical, ideal, monatomic gas of $N$ particles is reversibly compressed
\emph{isentropically}, i.e.~with the entropy kept constant, from an initial
temperature $T₀$ and pressure $P$ to a pressure $2P$. Find (a) the work done
on the system, and (b) the net change in entropy of the system and its
surroundings.

\begin{enumerate}
	\item
		An isentropic process is the same as an adiabatic process since no
		heat can be exchanged ($T\dd S = Q = 0$), so we begin with the relation
		that $PV^{γ}$ is a constant. Combining this with the ideal gas law,
		we can determine that
		\begin{align*}
			P^{1-γ}T^{γ} = \mathrm{const}
		\end{align*}
		where $γ = C_p/C_v$ is the ratio of heat capacities with $C_p =
		\frac 52 Nk_B$ and $C_v = \frac 32 Nk_B$ for a monatomic ideal gas.
		Using this, we solve for the final temperature of the system after
		compressions as
		\begin{align*}
			T_f &= 2^{2/5} T₀ ≈ 1.32T₀
		\end{align*}
		Combining both of
		\begin{align*}
			ΔU &= C_v ΔT \\
			ΔU &= Q + W
		\end{align*}
		where $Q = 0$, we get that
		\begin{align}
			\boxed{
			W = \frac 32 Nk_B T₀ (2^{2/5} - 1)
			}
		\end{align}
	\item
		Because the compression is done reversibly, by definition, $ΔS = 0$.
\end{enumerate}

%%%%%%%%%%%%%%%%%%%%%%%%%%%%%%%%%%%%%%%%%%%%%%%%%%%%%%%%%%%%%%%%%%%%%%%%%%%%%%%
%%%% Problem 8
%%%%%%%%%%%%%%%%%%%%%%%%%%%%%%%%%%%%%%%%%%%%%%%%%%%%%%%%%%%%%%%%%%%%%%%%%%%%%%%
\problem{8}
\subsubsection{Question}
% Keywords
	\index{thermodynamics!Fermi gas properties}
	\index{statistical mechanics!Fermi gas properties}

For an idea Fermi gas of $N$ neutral spin-$\frac 12$ particles in a volume
$V$ at $T = 0$, calculate the following:
\begin{enumerate}
	\item The chemical potential
	\item The average energy per particle
	\item The pressure
\end{enumerate}

\subsubsection{Answer}
\begin{enumerate}
	\item
		At $T = 0$, the particles are all in the lowest state allowed by
		Fermi-Dirac statistics, so the chemical potential, defined by the
		energy required to add another particle to the system, is equal to the
		Fermi energy. For a particle contained within a box $V$, the energy
		per particle is
		\begin{align*}
			ε_n &= \frac{π²ℏ²}{2mV^{2/3}} n²
		\end{align*}
		Given a Fermi energy $ε_F$, the maximum occupied state is
		\begin{align*}
			n_F &= \sqrt{\frac{2mV^{2/3}}{π²ℏ²}} \sqrt{ε_F}
		\end{align*}
		Equally we know that all $N$ particles must exist within the
		eighth-sphere of $n$ space, where the extra factor of 2 is because
		there are two spin states per $n$:
		\begin{align*}
			N &= 2·\frac 18 · \frac 43 π{n_F}³ \\
			N &= \frac 13 π ( \frac{2m}{π²ℏ²} )^{3/2} V {ε_F}^{3/2} \\
			ε_F &= \frac{ℏ²}{2m} (\frac{3π²N}{V})^{2/3}
		\end{align*}
		Therefore $μ = ε_F$,
		\begin{align}
			\boxed{
			μ = \frac{ℏ²}{2m} (\frac{3π²N}{V})^{2/3}
			}
		\end{align}
	\item
		To get the total energy, we can imagine filling all $N$ particles one
		at a time, so that at each step, there are $N'$ total particles:
		\begin{align*}
			U &= ∫_0^N ε_F \dd N' \\
			U &= \frac{ℏ²}{2m} (\frac{3π²}{V})^{2/3} ∫_0^N N^{2/3} \dd N' \\
			U &= \frac{ℏ²}{2m} (\frac{3π²}{V})^{2/3} · \frac 35 N^{5/3} \dd N'
		\end{align*}
		Therefore, the average energy per particle is $U/N$ or
		\begin{align}
			\boxed{
			\langle ε \rangle = \frac 35 ε_F
			}
		\end{align}
	\item
		From the thermodynamic relation
		\begin{align*}
			dU &= T\dd S - P\dd V + μ\dd N
		\end{align*}
		we can read off the derivative that defines the pressure $P$ as
		\begin{align*}
			P &= - ( \frac{∂U}{∂V} )_{S,N}
		\end{align*}
		Doing so, we get that
		\begin{align*}
			\frac{∂U}{∂V} &= \frac 35 N · \frac{ℏ²}{2m} (\frac{3π²}{V})^{2/3} ·
				(-\frac{2}{3V})
		\end{align*}
		making the pressure
		\begin{align}
			\boxed{
			P = \frac 25 \frac{N}{V} ε_F
			}
		\end{align}
\end{enumerate}

%%%%%%%%%%%%%%%%%%%%%%%%%%%%%%%%%%%%%%%%%%%%%%%%%%%%%%%%%%%%%%%%%%%%%%%%%%%%%%%
%%%% Problem 10
%%%%%%%%%%%%%%%%%%%%%%%%%%%%%%%%%%%%%%%%%%%%%%%%%%%%%%%%%%%%%%%%%%%%%%%%%%%%%%%
\problem{10}
\subsubsection{Question}
% Keywords
	\index{electrostatics!Hall effect}
	\index{solid state!Hall effect}

A piece of $p$-doped silicon has a carrier density
$n=\SI[per-mode=reciprocal]{e15}{\per\cm\cubed}$ and dimensions of $Δx =
\SI{10}{\mm}$, $Δy = \SI{2}{\mm}$, and $Δz = \SI{1}{\mm}$. A magnetic field
of $B_z = \SI{1}{T}$ is applied in the $z$-direction and a current $I_x =
\SI{1}{\A}$ flows in the $x$-direction, and the voltage $V_y$ is measured.
\begin{enumerate}
	\item
		Express the current density $j_x$ in terms of the carrier density $n$
		and the carrier velocity $v_x$.
	\item
		Write down the equilibrium force condition that determins $V_y$.
	\item
		Find $V_y$ in volts.
\end{enumerate}

\subsubsection{Answer}

\begin{enumerate}
	\item
		The current passing through each thin cross-sectional slice of the
		conductor is dependent on the charge of a carrier, carrier density,
		and velocity of the flow.
		\begin{align*}
			I_x &= enΔyΔzv_x
		\end{align*}
		The current density is just the current passing through each point, so
		\begin{align*}
			j_x &= \frac{I_x}{ΔyΔz}
		\end{align*}
		\begin{align}
			\boxed{
			j_x = nev_x
			}
		\end{align}
	\item
		The positive carriers drift to the edge of the conductor due to the
		magnetic field and the holes accumulate on the opposite edge. An
		electric field is created between the charge separation, so an
		equilibrium is set up between the electric field trying to bring the
		opposite charges together and the magnetic drift separating them.
		\begin{align*}
			0 &= e\vec E + \vec v × \vec B
		\end{align*}
		By the right-hand rule, the positive charges accumulate along $y=0$, so
		$\vec E = E \hat y$. Similarly, $\vec v × \vec B = -v_xB_z\hat y$:
		\begin{align*}
			0 &= eE\hat y - ev_xB_z\hat y
		\end{align*}
		Written in terms of the potential $V_y = EΔy$, the equilibrium
		condition becomes
		\begin{align}
			\boxed{
			V_y = v_xB_zΔy
			}
		\end{align}
	\item
		Substituting in for given quantities
		\begin{align*}
			V_y &= \frac{I_x B_z}{neΔz} \\
			V_y &= \frac{(\SI{1}{\A})(\SI{1}{T})}
				{(\SI[per-mode=reciprocal]{e15}{\per\cm\cubed})
				 (\SI{1.612e-19}{\coulomb})(\SI{1}{\mm})}
		\end{align*}
		\begin{align}
			\boxed{
			V_y = \SI{6.24}{\V}
			}
		\end{align}
\end{enumerate}


\springexam{2012}{2}
%%%%%%%%%%%%%%%%%%%%%%%%%%%%%%%%%%%%%%%%%%%%%%%%%%%%%%%%%%%%%%%%%%%%%%%%%%%%%%%
%%%% Problem 1
%%%%%%%%%%%%%%%%%%%%%%%%%%%%%%%%%%%%%%%%%%%%%%%%%%%%%%%%%%%%%%%%%%%%%%%%%%%%%%%
\problem{1}
\subsubsection{Question}
% Keywords
	\index{quantum!Expectation values}

An electron in a hydrogen atom occupies a state:
\begin{align*}
	\ket{ψ} &= \sqrt{\frac 13}\ket{3,1,0,+} + \sqrt{\frac 23} \ket{2,1,1,-}
\end{align*}
where the properly normalized states are specified by the quantum numbers
$\ket{n,ℓ,m,±}$ and the $±$ specifies whether the spin is up or down.

\makeatletter
\newcommand{\interitemtext}[1]{%
	\begin{list}{}
		{
			\itemindent=0mm\labelsep=0mm
			\labelwidth=0mm\leftmargin=0mm
			\addtolength{\leftmargin}{-\@totalleftmargin}
		}
		\item #1
	\end{list}
}
\makeatother

\renewcommand{\labelenumi}{(\alph{enumi})}
\begin{enumerate}
	\item
		What is the expectation value of the energy in terms of the ground
		state energy?
	\item
		If you meausred the expectation values of the orbital momentum squared
		$\expect{L²}$, the square of the spin $\expect{S²}$, and their
		$z$-components $\expect{L_z}$ and $\expect{S_z}$, what would be the
		result?
	\item
		Show that if you measure the position of the electron, the probability
		density for finding it an an angle specified by $θ$ and $ϕ$
		integrated over all values of $r$ is independent of $θ$ and $ϕ$. Note,
		for this part you will need $Y_1^0 = \sqrt{3/4π}\cos θ$ and $Y_1^1 =
		-\sqrt{3/8π} \sin θ \exp(iϕ)$. You do \emph{not}, however, need to
		know the radial functions, only that they are properly normalized and
		orthogonal to each other.
	\item
		List all additional possible states that are degenerate with the first
		state in the linear combination above. Note: this part can be done
		even if you have not answered the previous parts.
\interitemtext{Assume now that the state $\ket{ψ}$, given above, is the
initial state of an electron in a hydrogen atom.}
	\item
		Write down the elctron's state as a function of time for all $t>0$.
	\item
		go through the results you obtained in parts (a) through (c) and
		determine which of them are time independent.
\end{enumerate}

\subsubsection{Answer}
\begin{enumerate}
	\item
		Calculate the energy by sandwhiching the Hamiltonian between the
		wavefunction:
		\begin{align*}
			\expect{E} &= \braopket{ψ}{H}{ψ} \\
			{} &= (\sqrt{\frac 13}\bra{3,1,0,+} + \sqrt{\frac 23}
				\bra{2,1,1,-})H(\sqrt{\frac 13}\ket{3,1,0,+} + \sqrt{\frac 23}
				\ket{2,1,1,-}) \\
			\begin{split}
			{} &= \frac 13 \braopket{3,1,0,+}{H}{3,1,0,+} + \frac{\sqrt{2}}{3}
				\braopket{2,1,1,-}{H}{3,1,0,+} \\
				&\quad + \frac{\sqrt{2}}{3} \braopket{3,1,0,+}{H}{2,1,1,-} +
				\frac 23 \braopket{2,1,1,-}{H}{2,1,1,-}
			\end{split}\\
		\intertext{For every term, the wavefunctions are eigenstates of the
		Hamiltonian, so we extract the appropriate energy term from every
		bra-ket sandwhich. Then the middle two terms integrate to zero since
		states with different $n$ are orthogonal while the first and last terms
		integrate to unity since they are properly normalized.}
			\expect{E} &= \frac 13 E₃ + 0 + 0 + \frac 23 E₂ \\
		\intertext{Each energy is related to the ground state energy by
		$E_n = E₀/n²$, so}
			{} &= \frac 13 \frac{E₀}{9} + \frac 23 \frac{E₀}{4}
		\end{align*}
		\begin{align}
			\boxed{
			\expect{E} = \frac{11}{54} E₀ ≈ \SI{-2.77}{\eV}
			}
		\end{align}
	\item
		For each of the other expectation values, the process is very
		similar with an appropriate change for eigenvalues; specifically,
		\begin{align*}
			L²\ket{n,ℓ,m,±} &= ℓ(ℓ+1)ℏ²\ket{n,ℓ,m,±} \\
			S²\ket{n,ℓ,m,±} &= \frac 12(\frac 12 + 1)ℏ²\ket{n,ℓ,m,±} \\
			L_z\ket{n,ℓ,m,±} &= ℓℏ\ket{n,ℓ,m,±} \\
			S_z\ket{n,ℓ,m,±} &= ±\frac 12 ℏ\ket{n,ℓ,m,±}
		\end{align*}
		The same restrictions that the middle terms integrate to zero because
		of orthogonality and the first and last terms integrate to unity still
		applies, so we can almost immediately conclude that
		\begin{align}
			\boxed{\expect{L²} = 2ℏ²} \\
			\boxed{\expect{S²} = \frac{3ℏ²}{4}} \\
			\boxed{\expect{L_z} = \frac{2ℏ}{3}} \\
			\boxed{\expect{S_z} = -\frac{ℏ}{6}}
		\end{align}
	\item
		In the $\ket{r,θ,ϕ}$ basis,
		\begin{align*}
			\ket{3,1,0} &= R_{3,1}(r) Y_1^0(θ,ϕ) = R_{31}(r)
				\sqrt{\frac{3}{4π}}\cos θ \\
			\ket{2,1,1} &= R_{2,1}(r) Y_1^1(θ,ϕ) = -R_{21}(r)
				\sqrt{\frac{3}{8π}} e^{iϕ} \sin θ \\
		\end{align*}
		This means that the probability density is
		\begin{align*}
			\braket{ψ}{ψ} &= \frac 13 \braket{3,1,0}{3,1,0} +
				\frac{\sqrt{2}}{3} \braket{2,1,1}{3,1,0} +
				\frac{\sqrt{2}}{3} \braket{3,1,0}{2,1,1} +
				\frac 23 \braket{2,1,1}{2,1,1} \\
			{} &= \frac{1}{4π}\cos² θ R_{31}²(r) - \frac{1}{π} \sin θ \cos θ
				(e^{iϕ} + e^{-iϕ}) R_{21}(r) R_{31}(r) + \frac{1}{4π}
				\sin² θ R_{21}²(r)
		\end{align*}
		Integrating over $r$,
		\begin{align*}
			\begin{split}
				∫_0^∞ \braket{ψ}{ψ} \dd r &=
					∫_0^∞ \frac{1}{4π}\cos² θ R_{31}²(r) -
					\frac{1}{π} \sin θ\cos θ (e^{iϕ} + e^{-iϕ})
					R_{21}(r)R_{31}(r)\\
					&\quad + \frac{1}{4π} \sin² θ R_{21}²(r) \dd r
			\end{split}\\
		\intertext{Integrating over all $r$, we know that $R_{nℓ}R_{n'ℓ'}$
		are orthonormal, so again the first and last terms' $R$ integrates to
		unity and the middle term integrates to zero.}
			{} &= \frac{1}{4π}(\cos² θ + \sin² θ)
		\end{align*}
		Therefore we find that the probability density is constant in $θ$ and
		$ϕ$ when integrated over all $r$.
		\begin{align}
			\boxed{
			∫_0^∞ \braket{ψ}{ψ} \dd r = \frac{1}{4π}
			}
		\end{align}
	\item
		The states degenerate with the first term in $ψ$ are all combinations
		of allowed $ℓ$, $m$, and $±$: $n$ must remain at $n=3$ since it is the
		$n$ quantum number which determines the energy of the state. The
		angular momentum number $ℓ$ has to be in the range $[0, n-1]$, so there
		are at least 3 cases.
		\begin{align*}
			\ket{3,0,m,±} \\
			\ket{3,1,m,±} \\
			\ket{3,2,m,±}
		\end{align*}
		Then for each $ℓ$, the projection $m$ can take a range of values
		$m ∈ [-ℓ,ℓ]$ so using $\{...,-1,0,-1,...\}$ to denote a set of options,
		\begin{align*}
			\ket{3,0,m,±} &\rightarrow \ket{3,0,\{0\},±}
				& \text{2 states} \\
			\ket{3,1,m,±} &\rightarrow \ket{3,1,\{-1,0,1\},±}
				& \text{6 states} \\
			\ket{3,2,m,±} &\rightarrow \ket{3,2,\{-2,-1,0,1,2\},±}
				& \text{10 states}
		\end{align*}
		\begin{center}
			\framebox{In total, there are 18 degenerate states}
		\end{center}
	\item
		To get the time evolution, we simply use the fact that for each basis
		eigenstate, we can add the time evolution component
		\begin{align*}
			\exp (-\frac{iE_nt}{ℏ})
		\end{align*}
		to get (in terms of the ground state energy $E₀$)
		\begin{align}
			\boxed{
			\ket{ψ(t)} = \sqrt{\frac 13}\ket{3,1,0,+} e^{-iE₀t/9ℏ} +
				\sqrt{\frac 23} \ket{2,1,1,-} e^{-iE₀t/4ℏ}
			}
		\end{align}
	\item
		From Ehrenfest's Theorem, we can quickly find the answers to most of
		the question without worrying about the wavefunction. Ehrenfest's
		Theorem is
		\begin{align*}
			\frac{\dd}{\dd t}\expect{E} &= -\frac iℏ \expect{[Ω,H]} +
				\expect{\frac{∂Ω}{∂t}}
		\end{align*}
		None of the operators $L²$, $S²$, $L_z$, and $S_z$ are explicit in
		time, so the second term on the right can be dropped. Then because
		each of these operators commute with the Hamiltonian, the first term
		on the right is also dropped. Therefore, the expectation values are
		constant in time, so
		\begin{align}
			\boxed{\expect{L²} \quad\text{Time independent}} \\
			\boxed{\expect{S²} \quad\text{Time independent}} \\
			\boxed{\expect{L_z} \quad\text{Time independent}} \\
			\boxed{\expect{S_z} \quad\text{Time independent}}
		\end{align}
		For the probabilty density, we return to the integral in part (c) and
		insert the appropriate exponential terms. The first and last terms'
		exponentials cancel each other out, leaving
		\begin{align*}
			\begin{split}
				∫_0^∞ \braket{ψ}{ψ} \dd r &=
					∫_0^∞ \frac{1}{4π}\cos² θ R_{31}²(r) -
					\frac{1}{π} \sin θ\cos θ (e^{iϕ} + e^{-iϕ})
					R_{21}(r)R_{31}(r) \\
					&\quad · \left[ \exp(\frac{i(E₂-E₃)t}{ℏ}) +
					\exp(-\frac{i(E₂-E₃)t}{ℏ}) \right] \\
					&\quad + \frac{1}{4π} \sin² θ R_{21}²(r) \dd r
			\end{split}
		\end{align*}
		The integral is unaffected by the new time factors, though, so
		integrating over $r$, the middle term still goes to zero and we're
		left with the same result previously of $1/4π$, therefore
		\begin{align}
			\boxed{
			∫_0^∞ \braket{ψ}{ψ} \dd r \quad\text{Time independent}
			}
		\end{align}
\end{enumerate}

%%%%%%%%%%%%%%%%%%%%%%%%%%%%%%%%%%%%%%%%%%%%%%%%%%%%%%%%%%%%%%%%%%%%%%%%%%%%%%%
%%%% Problem 5
%%%%%%%%%%%%%%%%%%%%%%%%%%%%%%%%%%%%%%%%%%%%%%%%%%%%%%%%%%%%%%%%%%%%%%%%%%%%%%%
\problem{5}
\subsubsection{Question}
% Keywords
	\index{thermodynamics!Magnetic moments}
	\index{statistical mechanics!Magnetic moments}

Coinsider $N$ non-interacting, stationary particles, each with magnetic
moment $\vec μ$ at temperature $T$ in a uniform external magnetic field
$\vec B$. Their energy is $-\vec μ · \vec B$. Calculate the partition
function $Z$, the internal energy, and magnetization for two distinct cases
(a and b below):
\begin{enumerate}
	\item
		The mgnetic moment of each particle can be oriented only parallel or
		anti-parallel to the magnetic field.
	\item
		The magnetic moment of each particle can rotate freely.
	\item
		Show that, in both cases, the total magnetization $\vec M$ can be
		written as a derivative of the parition function.
	\item
		In each case, calculate the fluctuations of magnetization
		$\expect{(Δ\vec u)²}$.
\end{enumerate}

\subsubsection{Question}
\begin{enumerate}
	\item
		Begin by constructing the partition function for a single particle.
		Since there are only two energy states, the sum is simply over the
		two Boltzmann factors:
		\begin{align*}
			Z₁ &= e^{μB/kT} + e^{-μB/kT}
		\end{align*}
		This can be simplified using trigonometric identities to
		\begin{align*}
			Z₁ &= 2 \cosh (\frac{μB}{kT})
		\end{align*}
		For fixed site particles, the partition function for $N$ particles
		is simply $Z = Z^N$, so
		\begin{align}
			\boxed{
			Z = 2^N \cosh^N (\frac{μB}{kT})
			}
		\end{align}
		The total energy can be calculated either by finding the expectation
		energy per particle $\expect{ε}$ and multiplying by $N$ using the
		Boltzmann factors directly, or by using the thermodynamic identity
		\begin{align*}
			 U &= kT² \frac{∂\ln Z}{∂T}
		\end{align*}
		Doing so,
		\begin{align*}
			U &= kT² \frac{N}{2\cosh(\frac{μB}{kT})} · 2\sinh(\frac{μB}{kT})
				· (-\frac{μB}{kT²})
		\end{align*}
		\begin{align}
			\boxed{
			U = -NμB \tanh (\frac{μB}{kT})
			}
		\end{align}
		To find the Magnetization, we use the Boltzmann factors directly since
		we don't know a thermodynamic relation. Let
		\begin{align*}
			\expect{m} &= \sum_μ μ \frac{e^{-ε_μ/kT}}{Z₁} \\
			{} &= \frac{1}{Z₁}( -μ e^{μB/kT} + μ e^{-μB/kT} ) \\
			{} &= -μ \frac{2\sinh(\frac{μB}{kT})}{2\cosh(\frac{μB}{kT})}
		\end{align*}
		So knowing that $\expect{M} = N\expect{m}$,
		\begin{align}
			\boxed{
			\expect{M} = -Nμ \tanh(\frac{μB}{kT})
			}
		\end{align}
	\item
		In the continuous case, the sum needs to be changed into an integral,
		remembering to keep $Z$ unitless. This requires dividing by the volume
		of the energy state, which in this case is $μB$. (Justification: think
		of the energy vector $\vec μ · \vec B = μB\cos θ$ on the unit circle
		of length $μB$. From geometry, the unitless $\dd θ$ is related to
		$\dd ε$ by the factor $μB$.)
		\begin{align*}
			Z₁ &= ∫_{-μB}^{μB} e^{-ε/kT} \frac{\dd ε}{μB} \\
		\intertext{Letting $u = -\frac{ε}{kT}$,}
			Z₁ &= -\frac{kT}{μB} ∫_{μB/kT}^{-μB/kT} e^u \dd u \\
			{} &= 2 \frac{kT}{μB} \sinh (\frac{μB}{kT})
		\end{align*}
		Therefore the partition function for all $N$ particles is
		\begin{align}
			\boxed{
			Z = (\frac{2kT}{μB})^N \sinh^N(\frac{μB}{kT})
			}
		\end{align}
		The total energy is found in the same way as the previous case, giving
		\begin{align}
			\boxed{
			U = NkT - NμB \coth (\frac{μB}{kT})
			}
		\end{align}
		For the magnetization, we also calculate the expectation value from
		integrating the probability distribution, again making sure to keep
		the correct units. This time we work with the relevant projection of
		the magnetic moment $m = μ\cos θ$ so that when combined with the energy
		$ε = -μB\cos θ$, the magnetization per particle in each state is $m =
		-ε/B$.
		\begin{align*}
			\expect{m} &= ∫_{-μB}^{μB} -\frac{ε}{B} \frac{e^{-ε/kT}}{Z₁}
				\frac{\dd ε}{μB} \\
			{} &= \frac{1}{μZ₁}(\frac{kT}{B})² ∫_{-μB/kT}^{μB/kT} e^u \dd u \\
			{} &= \frac{kT}{B} \frac{2\sinh(\frac{μB}{kT})}
				{2\sinh(\frac{μB}{kT})} \\
			{} &= \frac{kT}{B}
		\end{align*}
		The total magnetization $\expect{M} = N\expect{m}$ is
		\begin{align}
			\boxed{
			\expect{M} = \frac{NkT}{B}
			}
		\end{align}
		Note that this is to be expected for the continuous case limit which
		corresponds to the classical limit. We'd expect the total energy to
		be related to the magnetization by $U = MB$. Rearranging the terms,
		\begin{align*}
			\expect{M}B &= NkT \\
			U &= NkT
		\end{align*}
		which is the expected result from the equipartition theorem for a
		stationary particle with two rotational degrees of freedom.
	\item
		Proving the discrete case only differs from the continuous case
		proof by the obvious substitutions, so only the continuous case will
		be presented here. Begin by writing the first starting integral
		from the previous problem
		\begin{align*}
			\expect{m} &= ∫_{-μB}^{μB} m \frac{e^{-ε/kT}}{Z₁}
				\frac{\dd ε}{μB}
		\end{align*}
		The $Z₁$ can be pulled outside the integral since it is a constant.
		Then note that per our definition $ε = -mB$, it follows that
		\begin{align*}
			\frac{∂ε}{∂B} &= -m
		\end{align*}
		We identify the integral above to be a result of using the chain
		rule, so we undo that and get
		\begin{align*}
			\expect{m} &= \frac{1}{Z₁} ∫_{-μB}^{μB} \frac{∂}{∂B}( -e^{-ε/kT} )
				\frac{\dd ε}{μB}
		\intertext{Changing the order of integration and differentiation,}
			{} &= \frac{1}{Z₁} \frac{∂}{∂B}( ∫_{-μB}^{μB} -e^{-ε/kT}
				\frac{\dd ε}{μB} )
		\intertext{The term within the brackets is simply the definition of
		the partition function, so}
			\expect{m} &= \frac{1}{Z₁} \frac{∂Z₁}{∂B} = \frac{∂\ln Z₁}{∂B}
		\end{align*}
		To then get the total magnetization $\expect{M}$, we use several
		properties of differentiation and logarithms:
		\begin{align*}
			\expect{M} &= N\expect{m} \\
			{} &= N \frac{∂\ln Z₁}{∂B} \\
			{} &= \frac{∂ (N\ln Z₁)}{∂B} \\
			{} &= \frac{∂\ln (Z₁)^N}{∂B}
		\end{align*}
		Giving us the final expression
		\begin{align}
			\boxed{
			\expect{M} = \frac{∂\ln Z}{∂B}
			}
		\end{align}
	\item
		Using the definition
		\begin{align*}
			\expect{(Δμ)²} &= \expect{μ²} - \expect{μ}²
		\end{align*}
		we already know $\expect{μ}²$ for both cases from the previous
		problems, so we must only calculate $\expect{μ²}$.
\end{enumerate}


\fallexam{2012}{1}
%%%%%%%%%%%%%%%%%%%%%%%%%%%%%%%%%%%%%%%%%%%%%%%%%%%%%%%%%%%%%%%%%%%%%%%%%%%%%%%
%%%% Problem 2
%%%%%%%%%%%%%%%%%%%%%%%%%%%%%%%%%%%%%%%%%%%%%%%%%%%%%%%%%%%%%%%%%%%%%%%%%%%%%%%
\problem{2}
\subsubsection{Question}
% Keywords
	\index{quantum!Infinite square-well periodicity}

Show that a particle in a one-dimensional infinite square well initially in a
state $Ψ(x,0)$ will always return to that state after a time $T = 4ma²/π\hbar$
where $a$ is the width of the well.

\subsubsection{Answer}
Use the standard time independent Schödinger equation
\begin{align*}
	Ψ(x,t) &= ψ(x) e^{iEt/\hbar}
\end{align*}
with associated differential equation
\begin{align*}
	-\frac{\hbar²}{2m} \frac{d²ψ}{dx²} + V(x)ψ &= Eψ
\end{align*}

For an infinite square well, the potential has the form
\begin{align*}
	V(x) &=
		\begin{cases}
			0	&	|x| < \frac{a}{2} \\
			∞	&	\text{otherwise}
		\end{cases}
\end{align*}
so that the only region to consider is $-\frac{a}{2} < x < \frac{a}{2}$. In this
region, the differential equation takes the form of a harmonic oscillator
\begin{align*}
	\frac{d²ψ}{dx²} &= -\frac{2mE}{\hbar²}ψ
\end{align*}
leading to solutions
\begin{align*}
	ψ(x) ={}& A\cos kx + B\sin kx \\
	{}& \text{where } k² = \frac{2mE}{\hbar²}
\end{align*}

The boundary conditions $ψ(-\frac{a}{2}) = 0$ and $ψ(\frac{a}{2}) = 0$ impose
\begin{align*}
	ψ\left(-\frac{a}{2}\right) &= 0 = A\cos \frac{ka}{2} - B\sin \frac{ka}{2} \\
	ψ\left(\frac{a}{2}\right)  &= 0 = A\cos \frac{ka}{2} + B\sin \frac{ka}{2}
\end{align*}
so that $B = 0$ and
\begin{align*}
	0 &= 2A \cos \frac{ka}{2} \\
	\frac{(2n+1)π}{2} &= \frac{ka}{2} \\
	k &= \frac{(2n+1)π}{a}
\end{align*}

We already had a relation for $k$ defined, so substitute and solve for the
energies $E_n$.
\begin{align*}
	\frac{(2n+1)²π²}{a²} &= \frac{2mE}{\hbar²} \\
	E_n &= \frac{(2n+1)²π²\hbar²}{2ma²}
\end{align*}

Then considering $Ψ(x,t)$, the complex exponential is periodic in time with
period
\begin{align*}
	T_n &= \frac{2π\hbar}{E}
\end{align*}
where $n = 0$ will be the case with the longest periodicity, so
\begin{align*}
	T &= \frac{2π\hbar · 2ma²}{π²\hbar²} \\
	{} &= \frac{4ma²}{π\hbar}
\end{align*}

Therefore, the function is periodic in time with a periodicity
\begin{empheq}[box=\fbox]{align}
	T &= \frac{4ma²}{π\hbar}
\end{empheq}

%%%%%%%%%%%%%%%%%%%%%%%%%%%%%%%%%%%%%%%%%%%%%%%%%%%%%%%%%%%%%%%%%%%%%%%%%%%%%%%
%%%% Problem 4
%%%%%%%%%%%%%%%%%%%%%%%%%%%%%%%%%%%%%%%%%%%%%%%%%%%%%%%%%%%%%%%%%%%%%%%%%%%%%%%
\problem{4}
\subsubsection{Question}
% Keywords
	\index{particle!Compton scattering}
	\index{relativity!Compton scattering}

A photon collides with a stationary electron. If the photon scatters at an
angle $θ$, show that the resulting wavelength $λ'$ is given in terms of the
original wavelength $λ$ by
\begin{align*}
	λ' &= λ + \frac{h}{mc} (1 - \cos θ)
\end{align*}
where $m$ is the mass of the electron.

\subsubsection{Answer}
Start by considering conservation of momentum for the system. The initial values
are
\begin{align*}
	p_{γx} &= \frac{h}{λ}		&		p'_{γx} &= \frac{h}{λ'} \cos θ \\
	p_{γy} &= 0					&		p'_{γy} &= \frac{h}{λ'} \sin θ \\
	p_{ex} &= 0					&		p'_{ex} &= ? \\
	p_{ey} &= 0					&		p'_{ey} &= ?
\end{align*}
and considering each component in turn:
\begin{align*}
	\frac{h}{λ} + 0 &= \frac{h}{λ'}\cos θ + p'_{ex}
		& 0 &= \frac{h}{λ'}\sin θ + p'_{ey}
	\\
	p'_{ex} &= \frac{h}{λ} - \frac{h}{λ'}\cos θ
		& p'_{ey} &= -\frac{h}{λ'}\sin θ
\end{align*}
The total momentum of the electron is then
\begin{align}
	p²_e &= \left( \frac{h}{λ} - \frac{h}{λ'}\cos θ \right)² +
		\left( -\frac{h}{λ'}\sin θ \right)²
		\nonumber
	\\
	{} &= \frac{h²}{λ²} - \frac{2h²}{λλ'}\cos θ + \frac{h²}{λ'^2}\cos ²θ +
		\frac{h²}{λ'^2}\sin ²θ
		\nonumber
	\\
	p²_e &= h² \left( \frac{1}{λ²} + \frac{1}{λ'^2} \right) -
		\frac{2h²}{λλ'}\cos θ
\end{align}

Then consider energy conservation, with initial values
\begin{align*}
	E_γ &= \frac{hc}{λ}			&		E'_γ &= \frac{hc}{λ'} \\
	E_e &= mc²					&		E'_e &= \frac{{p'_e}²}{2m} + mc²
\end{align*}
leading to the equation
\begin{align*}
	\frac{hc}{λ} + mc² &= \frac{hc}{λ'} + \frac{{p'_e}²}{2m} + mc²
	\\
	\frac{hc}{λ} &= \frac{hc}{λ'} + \frac{h²}{2m} \left( \frac{1}{λ²} +
		\frac{1}{λ'^2} \right) - \frac{2h²}{2mλλ'}\cos θ
	\\
	\frac{hc}{λ} - \frac{hc}{λ'} &= \frac{h²}{2m} \left( \frac{1}{λ²} +
		\frac{1}{λ'^2} \right) - \frac{2h²}{2mλλ'}\cos θ
	\\
	\frac{λ' - λ}{λλ'} &= \frac{h}{2mc} \frac{λ'^2 + λ²}{λ²{λ'}^2} -
		\frac{h}{mcλλ'}\cos θ
	\\
	λ' - λ &= \frac{h}{2mc} \left( \frac{(λ' - λ)² + 2λλ'}{λλ'} \right) -
		\frac{h}{mc}\cos θ
	\\
	λ' - λ &= \frac{h}{2mc} \left( \frac{(λ' - λ)²}{λλ'} + 2 \right) -
		\frac{h}{mc}\cos θ
	\\
	λ' - λ &= \frac{h}{2mc} \frac{(λ' - λ)²}{λλ'} + \frac{h}{mc}(1 - \cos θ)
\end{align*}
The difference in the wavelengths is small, so
\begin{align*}
	\frac{(λ' - λ)²}{λλ'} &≈ 0
\end{align*}
leading to the final Compton scattering equation
\begin{empheq}[box=\fbox]{align}
	λ' &= λ + \frac{h}{mc}(1 - \cos θ)
\end{empheq}


\end{document}
