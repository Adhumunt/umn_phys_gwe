%%%%%%%%%%%%%%%%%%%%%%%%%%%%%%%%%%%%%%%%%%%%%%%%%%%%%%%%%%%%%%%%%%%%%%%%%%%%%%%
%%%% Problem 1
%%%%%%%%%%%%%%%%%%%%%%%%%%%%%%%%%%%%%%%%%%%%%%%%%%%%%%%%%%%%%%%%%%%%%%%%%%%%%%%
%\subsection{Problem 1}
\problem{1}
\subsubsection{Question}
% Keywords
	\index{unsolved!Fall 2017 II.P1}
The Coriolis force is a force that acts on objects that are in motion relative to a rotating reference frame. The magnitude of the Coriolis acceleration of the object is a $\boldsymbol{a}_C = -2m\boldsymbol{\Omega}\times\boldsymbol{v}$, where $\boldsymbol{a}_C$ is the acceleration of the particle in the rotating system, $\boldsymbol{v}$ is the velocity of the particle with respect to the rotating system, and $\boldsymbol{\Omega}$ is the angular velocity vector having magnitude equal to the rotation rate $\boldsymbol{\Omega}$. The effect of the Coriolis force on the pendulum produces a precession, or rotation with time of the plane of oscillation. Describe the motion of this system, know as a Foucault pendulum. Assume that the oscillations have small amplitude with the horizontal excursions small compared with the length of the pendulum.
\subsubsection{Answer}


%%%%%%%%%%%%%%%%%%%%%%%%%%%%%%%%%%%%%%%%%%%%%%%%%%%%%%%%%%%%%%%%%%%%%%%%%%%%%%%
%%%% Problem 2
%%%%%%%%%%%%%%%%%%%%%%%%%%%%%%%%%%%%%%%%%%%%%%%%%%%%%%%%%%%%%%%%%%%%%%%%%%%%%%%
%\subsection{Problem 2}
\problem{2}
\subsubsection{Question}
% Keywords
	\index{unsolved!Fall 2017 II.P2}
Consider a tank that is divided by a barrier into two equal volumes $V$.
\begin{enumerate}
	\item One side of the tank is filled with $N$ molecules of an ideal gas. The other side is empty. What is the change in entropy of the entire system if the barrier is removed and the system is equilibrated? Assume that the tank is thermally insulated.
	\item The barrier is closed again and the gas on one side is instantaneously heated up to a higher temperature $T_{i1}$ while the gas on the other side remains at the initial temperature $T_{i2}$. Both sides of the tank are in thermal contact and allowed to equilibrate. What is the final temperature $T_f$ of the gas?
	\item Calculate the change in entropy of the entire system in b) as a function of the initial temperatures.
	\item Now suppose that the barrier is removed and that the entire equilibrated gas with the heat capacity $C_P$ and $C_V$ undergoes the following quasi-static cycle:
	\begin{enumerate}
		\item $a \to b$ adiabatic expansion from $V_1$ to $V_2$.
		\item $b \to c$ cooling at constant volume from $T_1$ to $T_2$.
		\item $c \to d$ adiabatic compression from $V_2$ to $V_1$.
		\item $d \to a$ heating at constant volume $V_1$.
	\end{enumerate}
	Calculate the thermal efficiency from the ratio of the entire work performed by the gas and the heat taken up in 4).
\end{enumerate}
\subsubsection{Answer}



%%%%%%%%%%%%%%%%%%%%%%%%%%%%%%%%%%%%%%%%%%%%%%%%%%%%%%%%%%%%%%%%%%%%%%%%%%%%%%%
%%%% Problem 3
%%%%%%%%%%%%%%%%%%%%%%%%%%%%%%%%%%%%%%%%%%%%%%%%%%%%%%%%%%%%%%%%%%%%%%%%%%%%%%%
%\subsection{Problem 3}
\problem{3}
\subsubsection{Question}
% Keywords
	\index{unsolved!Fall 2017 II.P3}
An infinite straight wire carries a constant current $I$. A square loop, of total resistance $R$, has sides of length $L$, two of which are parallel to the wire. The closest side of the loop is initially at distance $D_0$ from the wire, and the loop is moving away from the wire, with a constant velocity $v$, under the influence of an external force.

Compute the following
\begin{enumerate}
	\item The flux of the magnetic field through the loop at a generic time $t$;
	\item The induced current in the loop and its direction;
	\item The energy dissipated through the loop per unit time;
	\item The net magnetic force on the loop;
	\item The power input by the external agent.
\end{enumerate}
\subsubsection{Answer}



%%%%%%%%%%%%%%%%%%%%%%%%%%%%%%%%%%%%%%%%%%%%%%%%%%%%%%%%%%%%%%%%%%%%%%%%%%%%%%%
%%%% Problem 4
%%%%%%%%%%%%%%%%%%%%%%%%%%%%%%%%%%%%%%%%%%%%%%%%%%%%%%%%%%%%%%%%%%%%%%%%%%%%%%%
%\subsection{Problem 4}
\problem{4}
\subsubsection{Question}
% Keywords
	\index{unsolved!Fall 2017 II.P4}
A particle of spin $1/2$ has a magnetic moment ${\mu} = {\gamma} \vec{s}$ , where ${\gamma}$ is the gyromagnetic ratio and $\vec{s}$ is the spin operator. The particle is placed in a magnetic field $B(t) = B(t)\hat{\boldsymbol{k}}$ with its spin in the $+x$ direction at time $t = 0$. Here $\hat{\boldsymbol{k}}$ is the unit vector in the $+z$ direction.
\begin{enumerate}
	\item Find the energy eigenvalues of the particle as functions of $t$.
	\item Find the spin wave function of the particle at time $t > 0$.
\end{enumerate}

\subsubsection{Answer}


%%%%%%%%%%%%%%%%%%%%%%%%%%%%%%%%%%%%%%%%%%%%%%%%%%%%%%%%%%%%%%%%%%%%%%%%%%%%%%%
%%%% Problem 5
%%%%%%%%%%%%%%%%%%%%%%%%%%%%%%%%%%%%%%%%%%%%%%%%%%%%%%%%%%%%%%%%%%%%%%%%%%%%%%%
%\subsection{Problem 5}
\problem{5}
\subsubsection{Question}
% Keywords
	\index{unsolved!Fall 2017 II.P5}
Spaceships A, B, and C have the same proper length $\ell$, and according to an observer on Earth, are moving with same relativistic speed $v$ in the $+x$, $-x$, and $-y$ direction, respectively. Find the length of each spaceship as measured by an observer in the Spaceship A.

\subsubsection{Answer}
