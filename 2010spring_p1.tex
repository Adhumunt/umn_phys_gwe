%%%%%%%%%%%%%%%%%%%%%%%%%%%%%%%%%%%%%%%%%%%%%%%%%%%%%%%%%%%%%%%%%%%%%%%%%%%%%%%
%%%% Problem 1
%%%%%%%%%%%%%%%%%%%%%%%%%%%%%%%%%%%%%%%%%%%%%%%%%%%%%%%%%%%%%%%%%%%%%%%%%%%%%%%
%\subsection{Problem 1}
\problem{1}
\subsubsection{Question}
% Keywords
	\index{unsolved!Spring 2010 I.P1}
	\index{mechanics!Small Oscillations}
	\index{Lagrangian!Pendulum}

Consider a thin uniform and rigid rod of mass $m$ and length $L$. A small ball of mass $M$ is attached to one end of the rod. The other end of the rod is suspended from the ceiling and the system is free to oscillate about the suspension point without friction. Compute the period of the small oscillations (in a plane) of this system. Verify that you obtain the expected result when $M \gg m$.

\subsubsection{Answer}


%%%%%%%%%%%%%%%%%%%%%%%%%%%%%%%%%%%%%%%%%%%%%%%%%%%%%%%%%%%%%%%%%%%%%%%%%%%%%%%
%%%% Problem 2
%%%%%%%%%%%%%%%%%%%%%%%%%%%%%%%%%%%%%%%%%%%%%%%%%%%%%%%%%%%%%%%%%%%%%%%%%%%%%%%
%\subsection{Problem 2}
\problem{2}
\subsubsection{Question}
% Keywords
	\index{unsolved!Spring 2010 I.P2}
	\index{quantum!Semi-Infinite Potential Well}
Consider a semi-infinite one dimensional potential well. The potential is infinite at $x = 0$, it is zero for $0 < x < a$, and it has the finite value $V_0 > 0$ for all $x > a$. Compute the minimum value of $a$ for which such a potential can confine a particle of mass $m$.

\subsubsection{Answer}



%%%%%%%%%%%%%%%%%%%%%%%%%%%%%%%%%%%%%%%%%%%%%%%%%%%%%%%%%%%%%%%%%%%%%%%%%%%%%%%
%%%% Problem 3
%%%%%%%%%%%%%%%%%%%%%%%%%%%%%%%%%%%%%%%%%%%%%%%%%%%%%%%%%%%%%%%%%%%%%%%%%%%%%%%
%\subsection{Problem 3}
\problem{3}
\subsubsection{Question}
% Keywords
	\index{unsolved!Spring 2010 I.P3}
	\index{particle!Threshold Energy}

The electron has mass $m_e = 0.511 \text{ MeV}/c^2$. The top quark has mass $m_t = 173 \text{ GeV}/c^2$. A machine produces a beam of electrons, of energy $E_1$ each. A second machine produces a beam of positrons, of energy $E_2$ each. The two beams are made to collide head on. A total amount of energy can be given to the electrons and positrons beams, in whatever ratio; namely $E_1=xE$, $E_2 = (1 - x) E$. (i) For any value of $x$, compute the threshold energy $E$ that allows the production of pairs of real top and anti-top quarks when one electron collides with a positron (you may disregard $m_e$ when compared to $m_t$). (ii) Which choice of $x$ gives the smallest value of $E$?

\subsubsection{Answer}



%%%%%%%%%%%%%%%%%%%%%%%%%%%%%%%%%%%%%%%%%%%%%%%%%%%%%%%%%%%%%%%%%%%%%%%%%%%%%%%
%%%% Problem 4
%%%%%%%%%%%%%%%%%%%%%%%%%%%%%%%%%%%%%%%%%%%%%%%%%%%%%%%%%%%%%%%%%%%%%%%%%%%%%%%
%\subsection{Problem 4}
\problem{4}
\subsubsection{Question}
% Keywords
	\index{unsolved!Spring 2010 I.P4}
	\index{electrodynamics!Solenoid}
A very long linear solenoid is made of n circular loops per unit length. The area of each loop is $A$. The current in this solenoid is increased linearly with time, $I = \alpha t$ , where $\alpha$ is a constant. (i) What is the magnetic field inside this solenoid?

The solenoid is placed perpendicularly to a planar circuit, as shown in Figure 1 (the solenoid extends both inside and outside the page, for a distance much greater than the dimensions of the circuit; the arrows on the figure show the direction of the current in the loops of the solenoid). The circuit shown in the figure consists of two resistors, of resistances $R_1$ and $R_2$ , and two voltmeters. The internal resistances of the two voltmeters are much greater than $R_1$ and $R_2$ . (ii) What are the magnitudes of the potential differences $V_1$ and $V_2$ measured by the two voltmeters?

\subsubsection{Answer}


%%%%%%%%%%%%%%%%%%%%%%%%%%%%%%%%%%%%%%%%%%%%%%%%%%%%%%%%%%%%%%%%%%%%%%%%%%%%%%%
%%%% Problem 5
%%%%%%%%%%%%%%%%%%%%%%%%%%%%%%%%%%%%%%%%%%%%%%%%%%%%%%%%%%%%%%%%%%%%%%%%%%%%%%%
%\subsection{Problem 5}
\problem{5}
\subsubsection{Question}
% Keywords
	\index{unsolved!Spring 2010 I.P5}
	\index{electrostatics!Coaxial Superconducting Loops}
Consider two identical and coaxial superconducting loops. Each loop has self-inductance $L$. Initially, the two loops are very far apart from each other, and a current $I$ flows in each of them; the currents in the two loops have the same direction. Starting from this initial configuration, the two loops are then ``translated'' one on the top of the other, and superimposed (you can assume that they do not touch, although their distance becomes negligible). What is the final current in each of them? What are the initial and final energies of the system?

\subsubsection{Answer}



%%%%%%%%%%%%%%%%%%%%%%%%%%%%%%%%%%%%%%%%%%%%%%%%%%%%%%%%%%%%%%%%%%%%%%%%%%%%%%%
%%%% Problem 6
%%%%%%%%%%%%%%%%%%%%%%%%%%%%%%%%%%%%%%%%%%%%%%%%%%%%%%%%%%%%%%%%%%%%%%%%%%%%%%%
%\subsection{Problem 6}
\problem{6}
\subsubsection{Question}
% Keywords
	\index{unsolved!Spring 2010 I.P6}
	\index{thermodynamics!Equilibrium Heat Flux}
Two parallel perfectly black planes are in a vacuum, and are kept at constant and different temperatures $T_1$ and $T_2$. Denote by $\Phi$ the heat flux between these two planes. If a third perfectly black plane is inserted between these two planes, the system reaches a new steady state, for which the flux between the two external plates is $\Phi^\prime=6\Phi$. Compute the ratio $\Phi^\prime/\Phi$.

\subsubsection{Answer}

%%%%%%%%%%%%%%%%%%%%%%%%%%%%%%%%%%%%%%%%%%%%%%%%%%%%%%%%%%%%%%%%%%%%%%%%%%%%%%%
%%%% Problem 7
%%%%%%%%%%%%%%%%%%%%%%%%%%%%%%%%%%%%%%%%%%%%%%%%%%%%%%%%%%%%%%%%%%%%%%%%%%%%%%%
%\subsection{Problem 7}
\problem{7}
\subsubsection{Question}
% Keywords
	\index{unsolved!Spring 2010 I.P7}
	\index{quantum!1D Harmonic Oscillator}
A nonrelativistic particle of mass $m$ and electric charge $q$ is in the ground state of a one-dimensional simple harmonic oscillator potential $V (x) = m\omega^2x^2/2$. Recall that the normalized wavefunction for this state is
\begin{equation}
	\psi = \qty(\frac{m\omega}{\pi\hbar})^{1/4}\exp(-\frac{m\omega x^2}{2\hbar})
\end{equation}
At some moment, a uniform electric field in the x direction is switched on very quickly (i.e., on a timescale which can be regarded as instantaneous for this problem), and is then kept constant. (i) Show that the new (i.e., with the electric field switched on) potential to which the particle is subject is of the simple harmonic oscillator type. (ii) Compute the probability that the particle is found in the ground state of this new potential.

\subsubsection{Answer}



%%%%%%%%%%%%%%%%%%%%%%%%%%%%%%%%%%%%%%%%%%%%%%%%%%%%%%%%%%%%%%%%%%%%%%%%%%%%%%%
%%%% Problem 8
%%%%%%%%%%%%%%%%%%%%%%%%%%%%%%%%%%%%%%%%%%%%%%%%%%%%%%%%%%%%%%%%%%%%%%%%%%%%%%%
%\subsection{Problem 8}
\problem{8}
\subsubsection{Question}
% Keywords
	\index{unsolved!Spring 2010 I.P8}
	\index{thermodynamics!Latent Heat}
The latent heat of melting for ordinary ice is $334$ J/g. Use this and your own experience on how the volumes of ice and water differ to determine the sign and estimate the slope of the melting curve for water in the $p-T$ (pressure and temperature) plane.
\subsubsection{Answer}



%%%%%%%%%%%%%%%%%%%%%%%%%%%%%%%%%%%%%%%%%%%%%%%%%%%%%%%%%%%%%%%%%%%%%%%%%%%%%%%
%%%% Problem 9
%%%%%%%%%%%%%%%%%%%%%%%%%%%%%%%%%%%%%%%%%%%%%%%%%%%%%%%%%%%%%%%%%%%%%%%%%%%%%%%
%\subsection{Problem 9}
\problem{9}
\subsubsection{Question}
% Keywords
	\index{unsolved!Spring 2010 I.P9}
	\index{thermodynamics!Ideal Gas Cycle}
An engine with 1 mol of an ideal gas starts at $V_1 = 26.9$ liters and performs a cycle consisting of four steps:
\begin{enumerate}
	\item Heating at constant pressure to twice its initial volume, $V_2 = 2 V_1$.
	\item Isothermal expansion at $T_2$ to $V_3 = 3 V_1$.
	\item Cooling at constant volume to $T_1=250$K.
	\item Isothermal compression to its original volume $V_1$.
\end{enumerate}Assume that the molar heat capacity at constant volume for this gas is $C_V = 21$ J/K. (i) Calculate the $P, V, T$ (pressure, volume, temperature) points, and draw the engine cycle on a $P-V$ diagram. (ii) Determine the efficiency of this engine.


\subsubsection{Answer}



%%%%%%%%%%%%%%%%%%%%%%%%%%%%%%%%%%%%%%%%%%%%%%%%%%%%%%%%%%%%%%%%%%%%%%%%%%%%%%%
%%%% Problem 10
%%%%%%%%%%%%%%%%%%%%%%%%%%%%%%%%%%%%%%%%%%%%%%%%%%%%%%%%%%%%%%%%%%%%%%%%%%%%%%%
%\subsection{Problem 10}
\problem{10}
\subsubsection{Question}
% Keywords
	\index{unsolved!Spring 2010 I.P10}
	\index{relativity!Compton Scattering}
	\index{relativity!Half Life}
A particle of mass $M$ is initially moving along the $x-$axis, with constant speed $v$ (as measured in the laboratory frame), which can vary from $0$ to near the speed of light. This particle decays into two identical particles of mass $m$, with isotropic probability in its rest frame. In the following, unprimed quantities are in the laboratory frame, while primed quantities are in the rest frame of the initial particle. Choose the $x$ and $x^\prime$ axis of these two frames to coincide. Denote by $\mathbf{p}$ the momentum of one of the decay products. Choose the axes such that $p_z = p_z^\prime = 0$. Denote by $\theta$ the angle in the laboratory frame between the x-axis and the velocity of this decay product ($\theta=0$ if the decay product moves in the direction of the initial particle). The corresponding angle $\theta^\prime$ in the rest frame of the initial particle is shown in Figure 2. (i) For a given $\theta^\prime$, compute $p_x^\prime$ , $p_y^\prime$, $p_x$, $p_y$, and determine the relation between $\theta$ and $\theta^\prime$. (ii) Consider a beam of many such initial particles, all moving along the same straight line with velocity $v$. Consider the value of $v$ for which, in the laboratory frame, half of the decay products are emitted inside a cone forming an angle $\theta\le\theta_0$ with the direction of the initial beam. Find the relation between $\theta_0$ and $v$, as $v$ varies from $0$ to the speed of light.

\subsubsection{Answer}

%%%%%%%%%%%%%%%%%%%%%%%%%%%%%%%%%%%%%%%%%%%%%%%%%%%%%%%%%%%%%%%%%%%%%%%%%%%%%%%
%%%% Problem 11
%%%%%%%%%%%%%%%%%%%%%%%%%%%%%%%%%%%%%%%%%%%%%%%%%%%%%%%%%%%%%%%%%%%%%%%%%%%%%%%
%\subsection{Problem 11}
\problem{11}
\subsubsection{Question}
% Keywords
	\index{unsolved!Spring 2010 I.P11}
	\index{optics!Radius of Curvature}
A cook has a spherically shaped soup spoon. On looking into the concave side he sees his inverted image 4 cm from the bottom of the spoon (see Figure 3). Without changing his distance to the spoon, he turns it over, and sees an erect image of himself 3 cm from the bottom of the spoon. What is the radius of curvature of the spoon?

\subsubsection{Answer}

%%%%%%%%%%%%%%%%%%%%%%%%%%%%%%%%%%%%%%%%%%%%%%%%%%%%%%%%%%%%%%%%%%%%%%%%%%%%%%%
%%%% Problem 12
%%%%%%%%%%%%%%%%%%%%%%%%%%%%%%%%%%%%%%%%%%%%%%%%%%%%%%%%%%%%%%%%%%%%%%%%%%%%%%%
%\subsection{Problem 12}
\problem{12}
\subsubsection{Question}
% Keywords
	\index{unsolved!Spring 2010 I.P12}
	\index{particle!Neutrino Interactions}
In one day in 1987, the IMB detector observed 8 neutrino interactions. The normal background interaction rate in the detector was two a day. (i) What is the probability of eight background events being detected in one day? (ii) In fact, all those neutrinos occurred in a 10 second period. What is the probability that all those events were due to a background fluctuation?
\subsubsection{Answer}

