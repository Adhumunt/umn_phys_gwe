%%
% Collection of LaTeX input files which provide many convenient changes to the
% default environment.
%
% See also:
%    defaults-beamer.tex
%    defaults-bib.tex
%    defaults-drawing.tex
%    defaults-utilities.sty
%
% AUTHORS:      Justin Willmert ‹justin@jdjlab.com›
% SOURCES:      http://www.jdjlab.com/hg/latex/
%
% CONTRIBUTORS:
%   1. Ian-Mathew Hornburg ‹imhornburg@gmail.com›
%      (none — by personal correspondence)
%
%==============================================================================
%
% ********************
% CONFIGURABLE OPTIONS
% ********************
%
% There are a few configurable options that if set before the file is loaded
% will override defaults set here. In order to override a feature, simply
% declare a macro with the desired value. For example, to define the default
% font color used throughout the document, one should provide a command such
% as
%     \newcommand{\deftextcolor}{0,0,0,0.5}
%
% TODO: Use pgfkeys as a way to get and set options instead? Maybe something
%       like
%           \def\optdefaults{%
%               textcolor      = {0,0,0,1},
%               stretchyparens = 0
%               ...
%            }
%            %%
% Collection of LaTeX input files which provide many convenient changes to the
% default environment.
%
% See also:
%    defaults-beamer.tex
%    defaults-bib.tex
%    defaults-drawing.tex
%    defaults-fonts.tex
%    defaults-utilities.sty
%
% AUTHORS:      Justin Willmert ‹justin@jdjlab.com›
% SOURCES:      http://www.jdjlab.com/hg/latex/
%
% CONTRIBUTORS:
%   1. Ian-Mathew Hornburg ‹imhornburg@gmail.com›
%      (none — by personal correspondence)
%
%==============================================================================
%
% ********************
% CONFIGURABLE OPTIONS
% ********************
%
% There are a few configurable options that if set before the file is loaded
% will override defaults set here. In order to override a feature, simply
% declare a macro with the desired value. For example, to define the default
% font color used throughout the document, one should provide a command such
% as
%     \newcommand{\deftextcolor}{0,0,0,0.5}
%
% TODO: Use pgfkeys as a way to get and set options instead? Maybe something
%       like
%           \def\optdefaults{%
%               textcolor      = {0,0,0,1},
%               stretchyparens = 0
%               ...
%            }
%            %%
% Collection of LaTeX input files which provide many convenient changes to the
% default environment.
%
% See also:
%    defaults-beamer.tex
%    defaults-bib.tex
%    defaults-drawing.tex
%    defaults-fonts.tex
%    defaults-utilities.sty
%
% AUTHORS:      Justin Willmert ‹justin@jdjlab.com›
% SOURCES:      http://www.jdjlab.com/hg/latex/
%
% CONTRIBUTORS:
%   1. Ian-Mathew Hornburg ‹imhornburg@gmail.com›
%      (none — by personal correspondence)
%
%==============================================================================
%
% ********************
% CONFIGURABLE OPTIONS
% ********************
%
% There are a few configurable options that if set before the file is loaded
% will override defaults set here. In order to override a feature, simply
% declare a macro with the desired value. For example, to define the default
% font color used throughout the document, one should provide a command such
% as
%     \newcommand{\deftextcolor}{0,0,0,0.5}
%
% TODO: Use pgfkeys as a way to get and set options instead? Maybe something
%       like
%           \def\optdefaults{%
%               textcolor      = {0,0,0,1},
%               stretchyparens = 0
%               ...
%            }
%            %%
% Collection of LaTeX input files which provide many convenient changes to the
% default environment.
%
% See also:
%    defaults-beamer.tex
%    defaults-bib.tex
%    defaults-drawing.tex
%    defaults-fonts.tex
%    defaults-utilities.sty
%
% AUTHORS:      Justin Willmert ‹justin@jdjlab.com›
% SOURCES:      http://www.jdjlab.com/hg/latex/
%
% CONTRIBUTORS:
%   1. Ian-Mathew Hornburg ‹imhornburg@gmail.com›
%      (none — by personal correspondence)
%
%==============================================================================
%
% ********************
% CONFIGURABLE OPTIONS
% ********************
%
% There are a few configurable options that if set before the file is loaded
% will override defaults set here. In order to override a feature, simply
% declare a macro with the desired value. For example, to define the default
% font color used throughout the document, one should provide a command such
% as
%     \newcommand{\deftextcolor}{0,0,0,0.5}
%
% TODO: Use pgfkeys as a way to get and set options instead? Maybe something
%       like
%           \def\optdefaults{%
%               textcolor      = {0,0,0,1},
%               stretchyparens = 0
%               ...
%            }
%            \input{defaults}
%
% The following options are current recognized and supported:
%
% COMMAND              DEFAULT VALUE        DESCRIPTION
% -----------------------------------------------------------------------------
% \deftextcolor        0,0,0,1              The default text color is any valid
%                                           CMYK color value supported by the
%                                           xcolor package. Currently, no
%                                           support is available for setting
%                                           the default in other color spaces.
%
%
%==============================================================================
%
% *****************
% COMMANDS PROVIDED
% *****************
%
% This section documents some commands which may be necessary due to the
% modifications which are made by default, either overriding a previous change
% locally or restoring the LaTeX default functionality.
%
% COMMAND              DESCRIPTION
% -----------------------------------------------------------------------------
% \definetextcolor     Change the font color of the body text. Attempts are
%                      made to propagate the changes into all environments,
%                      but see the inline documentation for a list of the
%                      exceptions/caveats.
%
% \resetparents        By default, all parentheses in math mode made stretchy,
%                      but this may be undesirable in some context, so this
%                      command restores them to their default, inactive state.
%
%%

%% Provide a package-like environment where internal macros are accessible
\makeatletter

%% Load several commonly-assumed commands or macros to be defined
    \RequirePackage{defaults-utilities}



%% Load all commonly used packages
	\usepackage[
		margin=1in,
		twoside=true
	]{geometry}
	\usepackage{amsmath,amssymb}
	\usepackage[
		usenames,dvipsnames,svgnames,
		table,
		hyperref
	]{xcolor}
	\usepackage{float}
	\usepackage{verbatim}
	\usepackage{array}
	\usepackage{tabularx}
	\usepackage[
		exponent-product     = \cdot,
		load-configurations  = abbreviations,
		per-mode             = symbol-or-fraction,
		separate-uncertainty = true,
		math-micro           = \text{µ},
		text-micro           = µ
	]{siunitx}
	\usepackage[
		font      = footnotesize,
		labelfont = bf
	]{caption}



%% Make some amsmath fixes
%
% As of 2012-12-23 (XeTeX version 3.1415926-2.4-0.9998, TeXLive 2012), XeTeX
% does not properly handle parentheses when using the TeX primitives
%   \overwithdelims
%   \atopwithdelims
%   \abovewithdelims
% Because of this, the macro \binom which internally uses these these commands
% creates a binomial with the wrong size of brackets (too small). Patch around
% this issue by using \left( and \right) around a bracket-less binomial.
% LuaTeX doesn't seem to suffer from this.
\ifxetex
	\renewrobustcmd{\binom} [2]{\left(\genfrac{}{}{0pt}{}{#1}{#2}\right)}
	\renewcommand  {\dbinom}[2]{\left(\genfrac{}{}{0pt}{0}{#1}{#2}\right)}
	\renewcommand  {\tbinom}[2]{\left(\genfrac{}{}{0pt}{1}{#1}{#2}\right)}
\fi



%% Before loading advanced fonts, set up a default color scheme.
	% We choose black by default to correspond to norms, but this default can
	% be overridden by simply defining textcolor before loading this defaults
	% file.
	\providecommand{\deftextcolor}{0,0,0,1}
	\providecolor{textcolor}{cmyk}{\deftextcolor}
	% Now define a command which can be used to change the text colors mid-way
	% through a document. Right now, this doesn't work in all locations (e.g.
	% the body text will change color but section headings don't), but by
	% using a macro, we can potentially update the macro to also affect various
	% text environments.
	%
	% TODO: Colors do not apply to these environments even when the color is
	%       set within the preamble:
	%         · \hrule, \rule
	%
	% TODO: Colors currently will not propagate into the following modes if
	%       this macro is used after font selection/definition has occurred:
	%         · math mode
	%         · section headings
	%
	% \definetextcolor
	%    #1 -> xcolor model-list — See the xcolor documentation for more info
	%    #2 -> xcolor spec-list  — See the xcolor documentation for more info
	\newcommand{\definetextcolor}[2]{%
		% Define/Change the color which textcolor corresponds to
		\definecolor{textcolor}{#1}{#2}
		% If we're in a XeTeX/LuaTeX engine, use fontspec features to update
		% the font color.
		\ifxeluatex
			\addfontfeatures{Color=textcolor}
		\else
			% TODO: Extensive testing in non-XeTeX/LuaTeX environments.
			%       PDFTEX HAS BEEN COMPLETELY UNTESTED!
		\fi
	}

	% Taken from I.-M.:
	% “
	%   Many of these macro patches are gonna take work, since the color
	%   commands aren’t hooking well into the \rule family of commands. They’re
	%   pretty fragile to start, and occasionally \color commands “bleed” their
	%   color past their local contexts.
	%
	%   According to The LaTeX Companion, TeX’s \hrule command is actually used
	%   in the default LaTeX definition of \footnoterule instead of LaTeX’s
	%   \rule command in order to avoid strange effects of \rule, which
	%   actually starts a paragraph and related calculations. KOMA-Script
	%   modifies the footnote definition, but it still appears to be patchable.
	%
	%   If the below patch fails, the undefined macro \xpatchfailed will be
	%   “executed”, resulting in an undefined control sequence error, and will
	%   stop processing. A successful patch will be silent.
	%
	%   [TO-DO] Create a macro that writes a confirmation of the successful
	%           patch either to the terminal or a logfile during compilation.
	% ”
	\usepackage{xpatch}
	\xpretocmd{\footnoterule}{\color{textcolor}}{}{\xpatchfailed}
	% “
	%   Make table/array rules match the ambient text color...
	% ”
	\usepackage{colortbl}
	\AfterPreamble{\arrayrulecolor{textcolor}}



%% Now include advanced font features from defaults-fonts.tex
	\input{defaults-fonts}



%% Taken from I.-M.:
%
% Default glue settings under XeTeX for some fonts sometimes creates very
% cramped text.
%
% From http://tex.stackexchange.com/questions/49298/what-settings-for-xspaceskip-would-you-suggest-for-linux-libertine?rq=1
% Bringhurst suggests these values for body text: m/3 (even better if you can
% keep it at m/4) for interword space, m/2 for maximum space, and m/5 for
% minimum space. He also feels strongly against using the extra space after
% sentences, so assume \frenchspacing.
\appto\selectfont{%
		\fontdimen2\font=0.250em % M/4
		\fontdimen3\font=\dimexpr0.500em-\fontdimen2\font % M/2
		\fontdimen4\font=\dimexpr\fontdimen2\font-0.200em%
	}
    \AfterPreamble{\frenchspacing}



%% Make all parentheses stretchy
%
% By default, all parentheses are just normal parentheses in math mode. This
% to me doesn't seem to make sense since I almost always want them to be
% stretchy.
%
% Adapted from http://tex.stackexchange.com/questions/7359/how-to-make-a-real-backslash-escape-character/7372#7372
	% Start by defining a macro to quickly reset the default behavior if we
	% decide that a given environment shouldn't use the active definition. We
	% need the \edef so that the \the\mathcode is immediately expanded before
	% we change things.
	\edef\resetparens{%
			\mathcode`(=\the\mathcode`(
			\mathcode`)=\the\mathcode`)\relax
		}
	% Now we actually do the magic like this:
	%   1) Start a local group so that we can temporarily define a lowercase
	%      tilde to be the left paranetheses and not affect anything later in
	%      the document.
	%      a) We use the tilde because it is already defined to be an active
	%         character. This saves a step since to define a character to
	%         execute a macro sequence, it first needs to be active. The
	%         \lowercase command does for us since the lowercase tilde (i.e.
	%         paren) is made to have the same category code as its uppercase
	%         counterpart.
	%   2) \lccode assigns the lowercase tilde to be the left paren
	%   3) Then use \lowercase to cause us to define the macro of active left
	%      paren (without explicitly changing category codes).
	%      a) The \endgroup ends the local group we just started, but not
	%         before lowercase has already learned how to convert the tilde
	%         This lets the next \edef to be at the current global scope,
	%         avoiding use of \global
	%   4) Then define the active left paren to be the sequence \left( so that
	%      we invoke the stretchy code
	%   5) Finally change the mathcode of the left paren to be active within
	%      math environments, meaning we don't have to worry about the
	%      character being active in normal text mode.
	\begingroup%
			\lccode`~=`(%
			\lowercase{\endgroup\edef~}{\left(}
		\mathcode`(="8000
	% Do it again for the right paren
	\begingroup%
			\lccode`~=`)%
			\lowercase{\endgroup\edef~}{\right)}
		\mathcode`)="8000
	% Finally, this change of math codes breaks some changes the amsmath
	% package makes to the definitions of \left( and \right), so we fix this
	% by patching the internal amsmath macro which breaks to be prepended with
	% the \resetparens command
	\makeatletter
		\preto{\resetMathstrut@}{\resetparens}
		\makeatother



% Enable the microtypographic extensions
% · Requires that the 2.5 (beta-08) version of microtype be installed. This can
%   be included by loading the tlcontrib repo with tlmgr and upgrading
%   the TeXLive version of microtype
\usepackage[
    protrusion=true,
]{microtype}



%% Enable all the PDF features
	% First patch the author and title macros to let us access their values
	% more easily
	\let\dflttitle\title%
	\let\dfltauthor\author%
	\renewcommand{\title}[1]{\def\dfltthetitle{#1}\dflttitle{#1}}
	\renewcommand{\author}[1]{\def\dflttheauthor{#1}\dfltauthor{#1}}
	% Then make the inclusion of hyperref and hypcap the last things done
	% before the end of the preamble.
	%
	% IMPORTANT NOTE!
	%   It took much debugging to figure out that biblatex makes use of the
	%   \AtEndPreamble command to finalize much of its setup, especially with
	%   respect to knowing if hyperref has been loaded or not. For this reason,
	%   loading hyperref with \AtEndPreamble as well *after* we've already
	%   loaded biblatex means the the code below is executed after code that
	%   biblatex has loaded into the hook.
	%
	%   To guard against such mistakes (and assumptions that hyperref was
	%   loaded in the user-written preamble which will definitely occur
	%   before the hook), we can instead *prepend* our code to the hook that
	%   \AtEndPreamble normally appends to. We do this by patching a copy of
	%   the command to use \gpreto instead of \gappto.
	%
	\let\PreAtEndPreamble\AtEndPreamble%
	\patchcmd{\PreAtEndPreamble}{\gappto}{\gpreto}{}{%
		\PackageError{defaults}{%
			Prepending to the \verb|\AtEndPreamble| hook failed. Could not
			setup hyperref correctly.
		}{%
			Loading hyperref is a tricky business, and our attempt to have
			hyperref load at the correct time when also using biblatex failed.
			A guess would be that the format of the \verb|\AtEndPreamble|
			command has changed, which means our attempt to patch a copy for
			our purposes has failed.
			\MessageBreak\MessageBreak
			See comments in the source for umnthesis.cls for more information.`
		}
	}
	% Now prepend to the \AtEndPreamble hook.
	\PreAtEndPreamble{
		\usepackage[bookmarks,pagebackref=false]{hyperref}
			\usepackage[all]{hypcap}
			\hypersetup{
				,bookmarksopen  = false%
				,pdfborderstyle = {/S/U/W 1}%
				,ocgcolorlinks  = true%
				,colorlinks     = false%
				,setpagesize    = false%
				,pdfstartview   = {XYZ null null 1}%
				,pdfprintscaling= None%
				,pdfpagelayout  = TwoPageLeft%
				,verbose        = true%
			}

		% Taken from I.M.:
		% “
		%   When default colors are set by fontspec, hyperref isn’t able to
		%   color the links at all. The following hack allows hyperref colors
		%   to make it to the top of the stack.
		% ”
		\def\HyColor@@@@UseColor#1\@nil{\addfontfeatures{Color=#1}}

		% Check to see if the author and title macros have been set in the
		% preamble. If they have, then set the PDF attributes accordingly
		\ifdefined\dfltthetitle
			\hypersetup{ pdftitle = {\dfltthetitle} }
		\fi
		\ifdefined\dflttheauthor%
			% Sanitize the author string for use in the PDF properties:
			\begingroup
				% Separate authors with ampersands
				\def\and{ \& }
				\hypersetup{ pdfauthor = {\dflttheauthor} }%
			\endgroup
		\else
			\hypersetup{ pdfauthor = {Justin Willmert} }%
		\fi
	}

%% Hide internal macros again
\makeatletter

%
% The following options are current recognized and supported:
%
% COMMAND              DEFAULT VALUE        DESCRIPTION
% -----------------------------------------------------------------------------
% \deftextcolor        0,0,0,1              The default text color is any valid
%                                           CMYK color value supported by the
%                                           xcolor package. Currently, no
%                                           support is available for setting
%                                           the default in other color spaces.
%
%
%==============================================================================
%
% *****************
% COMMANDS PROVIDED
% *****************
%
% This section documents some commands which may be necessary due to the
% modifications which are made by default, either overriding a previous change
% locally or restoring the LaTeX default functionality.
%
% COMMAND              DESCRIPTION
% -----------------------------------------------------------------------------
% \definetextcolor     Change the font color of the body text. Attempts are
%                      made to propagate the changes into all environments,
%                      but see the inline documentation for a list of the
%                      exceptions/caveats.
%
% \resetparents        By default, all parentheses in math mode made stretchy,
%                      but this may be undesirable in some context, so this
%                      command restores them to their default, inactive state.
%
%%

%% Provide a package-like environment where internal macros are accessible
\makeatletter

%% Load several commonly-assumed commands or macros to be defined
    \RequirePackage{defaults-utilities}



%% Load all commonly used packages
	\usepackage[
		margin=1in,
		twoside=true
	]{geometry}
	\usepackage{amsmath,amssymb}
	\usepackage[
		usenames,dvipsnames,svgnames,
		table,
		hyperref
	]{xcolor}
	\usepackage{float}
	\usepackage{verbatim}
	\usepackage{array}
	\usepackage{tabularx}
	\usepackage[
		exponent-product     = \cdot,
		load-configurations  = abbreviations,
		per-mode             = symbol-or-fraction,
		separate-uncertainty = true,
		math-micro           = \text{µ},
		text-micro           = µ
	]{siunitx}
	\usepackage[
		font      = footnotesize,
		labelfont = bf
	]{caption}



%% Make some amsmath fixes
%
% As of 2012-12-23 (XeTeX version 3.1415926-2.4-0.9998, TeXLive 2012), XeTeX
% does not properly handle parentheses when using the TeX primitives
%   \overwithdelims
%   \atopwithdelims
%   \abovewithdelims
% Because of this, the macro \binom which internally uses these these commands
% creates a binomial with the wrong size of brackets (too small). Patch around
% this issue by using \left( and \right) around a bracket-less binomial.
% LuaTeX doesn't seem to suffer from this.
\ifxetex
	\renewrobustcmd{\binom} [2]{\left(\genfrac{}{}{0pt}{}{#1}{#2}\right)}
	\renewcommand  {\dbinom}[2]{\left(\genfrac{}{}{0pt}{0}{#1}{#2}\right)}
	\renewcommand  {\tbinom}[2]{\left(\genfrac{}{}{0pt}{1}{#1}{#2}\right)}
\fi



%% Before loading advanced fonts, set up a default color scheme.
	% We choose black by default to correspond to norms, but this default can
	% be overridden by simply defining textcolor before loading this defaults
	% file.
	\providecommand{\deftextcolor}{0,0,0,1}
	\providecolor{textcolor}{cmyk}{\deftextcolor}
	% Now define a command which can be used to change the text colors mid-way
	% through a document. Right now, this doesn't work in all locations (e.g.
	% the body text will change color but section headings don't), but by
	% using a macro, we can potentially update the macro to also affect various
	% text environments.
	%
	% TODO: Colors do not apply to these environments even when the color is
	%       set within the preamble:
	%         · \hrule, \rule
	%
	% TODO: Colors currently will not propagate into the following modes if
	%       this macro is used after font selection/definition has occurred:
	%         · math mode
	%         · section headings
	%
	% \definetextcolor
	%    #1 -> xcolor model-list — See the xcolor documentation for more info
	%    #2 -> xcolor spec-list  — See the xcolor documentation for more info
	\newcommand{\definetextcolor}[2]{%
		% Define/Change the color which textcolor corresponds to
		\definecolor{textcolor}{#1}{#2}
		% If we're in a XeTeX/LuaTeX engine, use fontspec features to update
		% the font color.
		\ifxeluatex
			\addfontfeatures{Color=textcolor}
		\else
			% TODO: Extensive testing in non-XeTeX/LuaTeX environments.
			%       PDFTEX HAS BEEN COMPLETELY UNTESTED!
		\fi
	}

	% Taken from I.-M.:
	% “
	%   Many of these macro patches are gonna take work, since the color
	%   commands aren’t hooking well into the \rule family of commands. They’re
	%   pretty fragile to start, and occasionally \color commands “bleed” their
	%   color past their local contexts.
	%
	%   According to The LaTeX Companion, TeX’s \hrule command is actually used
	%   in the default LaTeX definition of \footnoterule instead of LaTeX’s
	%   \rule command in order to avoid strange effects of \rule, which
	%   actually starts a paragraph and related calculations. KOMA-Script
	%   modifies the footnote definition, but it still appears to be patchable.
	%
	%   If the below patch fails, the undefined macro \xpatchfailed will be
	%   “executed”, resulting in an undefined control sequence error, and will
	%   stop processing. A successful patch will be silent.
	%
	%   [TO-DO] Create a macro that writes a confirmation of the successful
	%           patch either to the terminal or a logfile during compilation.
	% ”
	\usepackage{xpatch}
	\xpretocmd{\footnoterule}{\color{textcolor}}{}{\xpatchfailed}
	% “
	%   Make table/array rules match the ambient text color...
	% ”
	\usepackage{colortbl}
	\AfterPreamble{\arrayrulecolor{textcolor}}



%% Now include advanced font features from defaults-fonts.tex
	%%
% Package which provides some minimal common elements for use by all defaults-*
% files.
%
% See defaults.tex for more info.
%
% AUTHORS:      Justin Willmert ‹justin@jdjlab.com›
% SOURCES:      http://jmert.jdjlab.com/cgit/latex.git
%
% CONTRIBUTORS:
%   1. Ian-Mathew Hornburg ‹imhornburg@gmail.com›
%      (none — by personal correspondence)
%
%%
\NeedsTeXFormat{LaTeX2e}
\ProvidesPackage{defaults-fonts}
    [2016/04/09 J. Willmert's preample elements]
\RequirePackage{defaults-utilities}


%% Add fancy font features if using an advanced TeX engine.
\ifxeluatex\RequirePackage{fontspec}

    % Set font features which should be true for all fonts loaded by default
    \defaultfontfeatures{
        % Permit using TeX-style ligatures (i.e. --- for em-dash)
        Ligatures = TeX
    }

    % Also allow math to input using Unicode by default
    \RequirePackage{amsmath,amssymb}
    \RequirePackage{unicode-math}

    % Skip resetting the fonts if we're in draft mode
    \ifdraft{\relax}{
        % Use the XITS Math font by default
        \setmathfont[
                % Use italic fonts at all times
                math-style = ISO,
                bold-style = ISO,
                % Force \epsilon,\varepsilon and \phi,\varphi match the unicode
                % character ε,ϵ and φ,ϕ
                %vargreek-shape = unicode
            ]{XITS Math}
        
        % Additionally, use an alternate stylistic set for caligraphic letters
        % so that they are different from the script letters
        \setmathfont[range={\mathcal,\mathbfcal},StylisticSet=1]{XITS Math}
        
        % The author's opinion is that the STIX summation symbol is highly
        % inferior to those in Computer Modern/Latin Modern. Use CM since it
        % provides a symbol which is slightly larger than LM, and therefore
        % matches the rest of STIX a bit better.
        \setmathfont[range={"220F}]{Latin Modern Math}  % Product
        \setmathfont[range={"2210}]{Latin Modern Math}  % Coproduct
        \setmathfont[range={"2211}]{Latin Modern Math}  % Summation
        % Use the Asana Math for \top (Unicode Down Tack) since it's slightly
        % better when used as a transpose symbol.
        \setmathfont[range={"22A4}]{Asana Math}  % Down Tack
        % The beginner of the STIX \ell loop starts far too high up (close to
        % half-way up the character height), so replace it. Again, LM has the
        % right shape, but in this case, it's a bit too light compared to the
        % rest of STIX glyphs. Asana Math is a reasonable compromise.
        \setmathfont[range={"2113}]{Asana Math}  % Small Script L

        % Have the Linux Libertine family of fonts provide the default serif
        % and sans-serif fonts
        \setmainfont{Linux Libertine O}
        \setsansfont{Linux Biolinum O}
    } %\ifdraft

\else %\ifxeluatex

    % Allow pdfLaTeX to parse Unicode without choking. Note, though, that a
    % document *probably* still won't compile since the mapping from character
    % code to font glyph isn't setup by this. Additional work is needed yet
    % before this can happen automatically (if ever).
    %
    % Currently, my use is to allow \ifxeluatex ... \else ... \fi guards to be
    % used around tricky spots and not have pdflatex completely barf as it
    % parses those.
    \RequirePackage[T1]{fontenc}
    \RequirePackage[utf8]{inputenc}

    \ifdraft{\relax}{
        % Also use Linux Libertine fonts in pdflatex mode, but do this through
        % the libertine package.
        \RequirePackage{libertine}
        % Also choose to use use the STIX fonts like the XeLaTeX version
        \RequirePackage[notext]{stix}

        % Similar font substitutions as in the Xe/LuaLaTeX cases above.

        % Summation symbol substitution:
        %     Loads Computer Modern instead of Latin Modern
        \DeclareSymbolFont{cmlargesymbols}{OMX}{cmex}{m}{n}
        %     Map the summation operator to use the CM glyph
        \DeclareMathSymbol{\sumop}{\mathop}{cmlargesymbols}{"50}

        % Small Script L substitution:
        %     Loads Palatino instead of Asana Math
        \DeclareSymbolFont{plletters}{OML}{zplm}{m}{it}
        \DeclareMathSymbol{\ell}{\mathalpha}{plletters}{"60}

        % Inconsolata is sized much more appropriately for the other chosen
        % fonts
        \RequirePackage{inconsolata}
    }

\fi %\ifxeluatex

\endinput





%% Taken from I.-M.:
%
% Default glue settings under XeTeX for some fonts sometimes creates very
% cramped text.
%
% From http://tex.stackexchange.com/questions/49298/what-settings-for-xspaceskip-would-you-suggest-for-linux-libertine?rq=1
% Bringhurst suggests these values for body text: m/3 (even better if you can
% keep it at m/4) for interword space, m/2 for maximum space, and m/5 for
% minimum space. He also feels strongly against using the extra space after
% sentences, so assume \frenchspacing.
\appto\selectfont{%
		\fontdimen2\font=0.250em % M/4
		\fontdimen3\font=\dimexpr0.500em-\fontdimen2\font % M/2
		\fontdimen4\font=\dimexpr\fontdimen2\font-0.200em%
	}
    \AfterPreamble{\frenchspacing}



%% Make all parentheses stretchy
%
% By default, all parentheses are just normal parentheses in math mode. This
% to me doesn't seem to make sense since I almost always want them to be
% stretchy.
%
% Adapted from http://tex.stackexchange.com/questions/7359/how-to-make-a-real-backslash-escape-character/7372#7372
	% Start by defining a macro to quickly reset the default behavior if we
	% decide that a given environment shouldn't use the active definition. We
	% need the \edef so that the \the\mathcode is immediately expanded before
	% we change things.
	\edef\resetparens{%
			\mathcode`(=\the\mathcode`(
			\mathcode`)=\the\mathcode`)\relax
		}
	% Now we actually do the magic like this:
	%   1) Start a local group so that we can temporarily define a lowercase
	%      tilde to be the left paranetheses and not affect anything later in
	%      the document.
	%      a) We use the tilde because it is already defined to be an active
	%         character. This saves a step since to define a character to
	%         execute a macro sequence, it first needs to be active. The
	%         \lowercase command does for us since the lowercase tilde (i.e.
	%         paren) is made to have the same category code as its uppercase
	%         counterpart.
	%   2) \lccode assigns the lowercase tilde to be the left paren
	%   3) Then use \lowercase to cause us to define the macro of active left
	%      paren (without explicitly changing category codes).
	%      a) The \endgroup ends the local group we just started, but not
	%         before lowercase has already learned how to convert the tilde
	%         This lets the next \edef to be at the current global scope,
	%         avoiding use of \global
	%   4) Then define the active left paren to be the sequence \left( so that
	%      we invoke the stretchy code
	%   5) Finally change the mathcode of the left paren to be active within
	%      math environments, meaning we don't have to worry about the
	%      character being active in normal text mode.
	\begingroup%
			\lccode`~=`(%
			\lowercase{\endgroup\edef~}{\left(}
		\mathcode`(="8000
	% Do it again for the right paren
	\begingroup%
			\lccode`~=`)%
			\lowercase{\endgroup\edef~}{\right)}
		\mathcode`)="8000
	% Finally, this change of math codes breaks some changes the amsmath
	% package makes to the definitions of \left( and \right), so we fix this
	% by patching the internal amsmath macro which breaks to be prepended with
	% the \resetparens command
	\makeatletter
		\preto{\resetMathstrut@}{\resetparens}
		\makeatother



% Enable the microtypographic extensions
% · Requires that the 2.5 (beta-08) version of microtype be installed. This can
%   be included by loading the tlcontrib repo with tlmgr and upgrading
%   the TeXLive version of microtype
\usepackage[
    protrusion=true,
]{microtype}



%% Enable all the PDF features
	% First patch the author and title macros to let us access their values
	% more easily
	\let\dflttitle\title%
	\let\dfltauthor\author%
	\renewcommand{\title}[1]{\def\dfltthetitle{#1}\dflttitle{#1}}
	\renewcommand{\author}[1]{\def\dflttheauthor{#1}\dfltauthor{#1}}
	% Then make the inclusion of hyperref and hypcap the last things done
	% before the end of the preamble.
	%
	% IMPORTANT NOTE!
	%   It took much debugging to figure out that biblatex makes use of the
	%   \AtEndPreamble command to finalize much of its setup, especially with
	%   respect to knowing if hyperref has been loaded or not. For this reason,
	%   loading hyperref with \AtEndPreamble as well *after* we've already
	%   loaded biblatex means the the code below is executed after code that
	%   biblatex has loaded into the hook.
	%
	%   To guard against such mistakes (and assumptions that hyperref was
	%   loaded in the user-written preamble which will definitely occur
	%   before the hook), we can instead *prepend* our code to the hook that
	%   \AtEndPreamble normally appends to. We do this by patching a copy of
	%   the command to use \gpreto instead of \gappto.
	%
	\let\PreAtEndPreamble\AtEndPreamble%
	\patchcmd{\PreAtEndPreamble}{\gappto}{\gpreto}{}{%
		\PackageError{defaults}{%
			Prepending to the \verb|\AtEndPreamble| hook failed. Could not
			setup hyperref correctly.
		}{%
			Loading hyperref is a tricky business, and our attempt to have
			hyperref load at the correct time when also using biblatex failed.
			A guess would be that the format of the \verb|\AtEndPreamble|
			command has changed, which means our attempt to patch a copy for
			our purposes has failed.
			\MessageBreak\MessageBreak
			See comments in the source for umnthesis.cls for more information.`
		}
	}
	% Now prepend to the \AtEndPreamble hook.
	\PreAtEndPreamble{
		\usepackage[bookmarks,pagebackref=false]{hyperref}
			\usepackage[all]{hypcap}
			\hypersetup{
				,bookmarksopen  = false%
				,pdfborderstyle = {/S/U/W 1}%
				,ocgcolorlinks  = true%
				,colorlinks     = false%
				,setpagesize    = false%
				,pdfstartview   = {XYZ null null 1}%
				,pdfprintscaling= None%
				,pdfpagelayout  = TwoPageLeft%
				,verbose        = true%
			}

		% Taken from I.M.:
		% “
		%   When default colors are set by fontspec, hyperref isn’t able to
		%   color the links at all. The following hack allows hyperref colors
		%   to make it to the top of the stack.
		% ”
		\def\HyColor@@@@UseColor#1\@nil{\addfontfeatures{Color=#1}}

		% Check to see if the author and title macros have been set in the
		% preamble. If they have, then set the PDF attributes accordingly
		\ifdefined\dfltthetitle
			\hypersetup{ pdftitle = {\dfltthetitle} }
		\fi
		\ifdefined\dflttheauthor%
			% Sanitize the author string for use in the PDF properties:
			\begingroup
				% Separate authors with ampersands
				\def\and{ \& }
				\hypersetup{ pdfauthor = {\dflttheauthor} }%
			\endgroup
		\else
			\hypersetup{ pdfauthor = {Justin Willmert} }%
		\fi
	}

%% Hide internal macros again
\makeatletter

%
% The following options are current recognized and supported:
%
% COMMAND              DEFAULT VALUE        DESCRIPTION
% -----------------------------------------------------------------------------
% \deftextcolor        0,0,0,1              The default text color is any valid
%                                           CMYK color value supported by the
%                                           xcolor package. Currently, no
%                                           support is available for setting
%                                           the default in other color spaces.
%
%
%==============================================================================
%
% *****************
% COMMANDS PROVIDED
% *****************
%
% This section documents some commands which may be necessary due to the
% modifications which are made by default, either overriding a previous change
% locally or restoring the LaTeX default functionality.
%
% COMMAND              DESCRIPTION
% -----------------------------------------------------------------------------
% \definetextcolor     Change the font color of the body text. Attempts are
%                      made to propagate the changes into all environments,
%                      but see the inline documentation for a list of the
%                      exceptions/caveats.
%
% \resetparents        By default, all parentheses in math mode made stretchy,
%                      but this may be undesirable in some context, so this
%                      command restores them to their default, inactive state.
%
%%

%% Provide a package-like environment where internal macros are accessible
\makeatletter

%% Load several commonly-assumed commands or macros to be defined
    \RequirePackage{defaults-utilities}



%% Load all commonly used packages
	\usepackage[
		margin=1in,
		twoside=true
	]{geometry}
	\usepackage{amsmath,amssymb}
	\usepackage[
		usenames,dvipsnames,svgnames,
		table,
		hyperref
	]{xcolor}
	\usepackage{float}
	\usepackage{verbatim}
	\usepackage{array}
	\usepackage{tabularx}
	\usepackage[
		exponent-product     = \cdot,
		load-configurations  = abbreviations,
		per-mode             = symbol-or-fraction,
		separate-uncertainty = true,
		math-micro           = \text{µ},
		text-micro           = µ
	]{siunitx}
	\usepackage[
		font      = footnotesize,
		labelfont = bf
	]{caption}



%% Make some amsmath fixes
%
% As of 2012-12-23 (XeTeX version 3.1415926-2.4-0.9998, TeXLive 2012), XeTeX
% does not properly handle parentheses when using the TeX primitives
%   \overwithdelims
%   \atopwithdelims
%   \abovewithdelims
% Because of this, the macro \binom which internally uses these these commands
% creates a binomial with the wrong size of brackets (too small). Patch around
% this issue by using \left( and \right) around a bracket-less binomial.
% LuaTeX doesn't seem to suffer from this.
\ifxetex
	\renewrobustcmd{\binom} [2]{\left(\genfrac{}{}{0pt}{}{#1}{#2}\right)}
	\renewcommand  {\dbinom}[2]{\left(\genfrac{}{}{0pt}{0}{#1}{#2}\right)}
	\renewcommand  {\tbinom}[2]{\left(\genfrac{}{}{0pt}{1}{#1}{#2}\right)}
\fi



%% Before loading advanced fonts, set up a default color scheme.
	% We choose black by default to correspond to norms, but this default can
	% be overridden by simply defining textcolor before loading this defaults
	% file.
	\providecommand{\deftextcolor}{0,0,0,1}
	\providecolor{textcolor}{cmyk}{\deftextcolor}
	% Now define a command which can be used to change the text colors mid-way
	% through a document. Right now, this doesn't work in all locations (e.g.
	% the body text will change color but section headings don't), but by
	% using a macro, we can potentially update the macro to also affect various
	% text environments.
	%
	% TODO: Colors do not apply to these environments even when the color is
	%       set within the preamble:
	%         · \hrule, \rule
	%
	% TODO: Colors currently will not propagate into the following modes if
	%       this macro is used after font selection/definition has occurred:
	%         · math mode
	%         · section headings
	%
	% \definetextcolor
	%    #1 -> xcolor model-list — See the xcolor documentation for more info
	%    #2 -> xcolor spec-list  — See the xcolor documentation for more info
	\newcommand{\definetextcolor}[2]{%
		% Define/Change the color which textcolor corresponds to
		\definecolor{textcolor}{#1}{#2}
		% If we're in a XeTeX/LuaTeX engine, use fontspec features to update
		% the font color.
		\ifxeluatex
			\addfontfeatures{Color=textcolor}
		\else
			% TODO: Extensive testing in non-XeTeX/LuaTeX environments.
			%       PDFTEX HAS BEEN COMPLETELY UNTESTED!
		\fi
	}

	% Taken from I.-M.:
	% “
	%   Many of these macro patches are gonna take work, since the color
	%   commands aren’t hooking well into the \rule family of commands. They’re
	%   pretty fragile to start, and occasionally \color commands “bleed” their
	%   color past their local contexts.
	%
	%   According to The LaTeX Companion, TeX’s \hrule command is actually used
	%   in the default LaTeX definition of \footnoterule instead of LaTeX’s
	%   \rule command in order to avoid strange effects of \rule, which
	%   actually starts a paragraph and related calculations. KOMA-Script
	%   modifies the footnote definition, but it still appears to be patchable.
	%
	%   If the below patch fails, the undefined macro \xpatchfailed will be
	%   “executed”, resulting in an undefined control sequence error, and will
	%   stop processing. A successful patch will be silent.
	%
	%   [TO-DO] Create a macro that writes a confirmation of the successful
	%           patch either to the terminal or a logfile during compilation.
	% ”
	\usepackage{xpatch}
	\xpretocmd{\footnoterule}{\color{textcolor}}{}{\xpatchfailed}
	% “
	%   Make table/array rules match the ambient text color...
	% ”
	\usepackage{colortbl}
	\AfterPreamble{\arrayrulecolor{textcolor}}



%% Now include advanced font features from defaults-fonts.tex
	%%
% Package which provides some minimal common elements for use by all defaults-*
% files.
%
% See defaults.tex for more info.
%
% AUTHORS:      Justin Willmert ‹justin@jdjlab.com›
% SOURCES:      http://jmert.jdjlab.com/cgit/latex.git
%
% CONTRIBUTORS:
%   1. Ian-Mathew Hornburg ‹imhornburg@gmail.com›
%      (none — by personal correspondence)
%
%%
\NeedsTeXFormat{LaTeX2e}
\ProvidesPackage{defaults-fonts}
    [2016/04/09 J. Willmert's preample elements]
\RequirePackage{defaults-utilities}


%% Add fancy font features if using an advanced TeX engine.
\ifxeluatex\RequirePackage{fontspec}

    % Set font features which should be true for all fonts loaded by default
    \defaultfontfeatures{
        % Permit using TeX-style ligatures (i.e. --- for em-dash)
        Ligatures = TeX
    }

    % Also allow math to input using Unicode by default
    \RequirePackage{amsmath,amssymb}
    \RequirePackage{unicode-math}

    % Skip resetting the fonts if we're in draft mode
    \ifdraft{\relax}{
        % Use the XITS Math font by default
        \setmathfont[
                % Use italic fonts at all times
                math-style = ISO,
                bold-style = ISO,
                % Force \epsilon,\varepsilon and \phi,\varphi match the unicode
                % character ε,ϵ and φ,ϕ
                %vargreek-shape = unicode
            ]{XITS Math}
        
        % Additionally, use an alternate stylistic set for caligraphic letters
        % so that they are different from the script letters
        \setmathfont[range={\mathcal,\mathbfcal},StylisticSet=1]{XITS Math}
        
        % The author's opinion is that the STIX summation symbol is highly
        % inferior to those in Computer Modern/Latin Modern. Use CM since it
        % provides a symbol which is slightly larger than LM, and therefore
        % matches the rest of STIX a bit better.
        \setmathfont[range={"220F}]{Latin Modern Math}  % Product
        \setmathfont[range={"2210}]{Latin Modern Math}  % Coproduct
        \setmathfont[range={"2211}]{Latin Modern Math}  % Summation
        % Use the Asana Math for \top (Unicode Down Tack) since it's slightly
        % better when used as a transpose symbol.
        \setmathfont[range={"22A4}]{Asana Math}  % Down Tack
        % The beginner of the STIX \ell loop starts far too high up (close to
        % half-way up the character height), so replace it. Again, LM has the
        % right shape, but in this case, it's a bit too light compared to the
        % rest of STIX glyphs. Asana Math is a reasonable compromise.
        \setmathfont[range={"2113}]{Asana Math}  % Small Script L

        % Have the Linux Libertine family of fonts provide the default serif
        % and sans-serif fonts
        \setmainfont{Linux Libertine O}
        \setsansfont{Linux Biolinum O}
    } %\ifdraft

\else %\ifxeluatex

    % Allow pdfLaTeX to parse Unicode without choking. Note, though, that a
    % document *probably* still won't compile since the mapping from character
    % code to font glyph isn't setup by this. Additional work is needed yet
    % before this can happen automatically (if ever).
    %
    % Currently, my use is to allow \ifxeluatex ... \else ... \fi guards to be
    % used around tricky spots and not have pdflatex completely barf as it
    % parses those.
    \RequirePackage[T1]{fontenc}
    \RequirePackage[utf8]{inputenc}

    \ifdraft{\relax}{
        % Also use Linux Libertine fonts in pdflatex mode, but do this through
        % the libertine package.
        \RequirePackage{libertine}
        % Also choose to use use the STIX fonts like the XeLaTeX version
        \RequirePackage[notext]{stix}

        % Similar font substitutions as in the Xe/LuaLaTeX cases above.

        % Summation symbol substitution:
        %     Loads Computer Modern instead of Latin Modern
        \DeclareSymbolFont{cmlargesymbols}{OMX}{cmex}{m}{n}
        %     Map the summation operator to use the CM glyph
        \DeclareMathSymbol{\sumop}{\mathop}{cmlargesymbols}{"50}

        % Small Script L substitution:
        %     Loads Palatino instead of Asana Math
        \DeclareSymbolFont{plletters}{OML}{zplm}{m}{it}
        \DeclareMathSymbol{\ell}{\mathalpha}{plletters}{"60}

        % Inconsolata is sized much more appropriately for the other chosen
        % fonts
        \RequirePackage{inconsolata}
    }

\fi %\ifxeluatex

\endinput





%% Taken from I.-M.:
%
% Default glue settings under XeTeX for some fonts sometimes creates very
% cramped text.
%
% From http://tex.stackexchange.com/questions/49298/what-settings-for-xspaceskip-would-you-suggest-for-linux-libertine?rq=1
% Bringhurst suggests these values for body text: m/3 (even better if you can
% keep it at m/4) for interword space, m/2 for maximum space, and m/5 for
% minimum space. He also feels strongly against using the extra space after
% sentences, so assume \frenchspacing.
\appto\selectfont{%
		\fontdimen2\font=0.250em % M/4
		\fontdimen3\font=\dimexpr0.500em-\fontdimen2\font % M/2
		\fontdimen4\font=\dimexpr\fontdimen2\font-0.200em%
	}
    \AfterPreamble{\frenchspacing}



%% Make all parentheses stretchy
%
% By default, all parentheses are just normal parentheses in math mode. This
% to me doesn't seem to make sense since I almost always want them to be
% stretchy.
%
% Adapted from http://tex.stackexchange.com/questions/7359/how-to-make-a-real-backslash-escape-character/7372#7372
	% Start by defining a macro to quickly reset the default behavior if we
	% decide that a given environment shouldn't use the active definition. We
	% need the \edef so that the \the\mathcode is immediately expanded before
	% we change things.
	\edef\resetparens{%
			\mathcode`(=\the\mathcode`(
			\mathcode`)=\the\mathcode`)\relax
		}
	% Now we actually do the magic like this:
	%   1) Start a local group so that we can temporarily define a lowercase
	%      tilde to be the left paranetheses and not affect anything later in
	%      the document.
	%      a) We use the tilde because it is already defined to be an active
	%         character. This saves a step since to define a character to
	%         execute a macro sequence, it first needs to be active. The
	%         \lowercase command does for us since the lowercase tilde (i.e.
	%         paren) is made to have the same category code as its uppercase
	%         counterpart.
	%   2) \lccode assigns the lowercase tilde to be the left paren
	%   3) Then use \lowercase to cause us to define the macro of active left
	%      paren (without explicitly changing category codes).
	%      a) The \endgroup ends the local group we just started, but not
	%         before lowercase has already learned how to convert the tilde
	%         This lets the next \edef to be at the current global scope,
	%         avoiding use of \global
	%   4) Then define the active left paren to be the sequence \left( so that
	%      we invoke the stretchy code
	%   5) Finally change the mathcode of the left paren to be active within
	%      math environments, meaning we don't have to worry about the
	%      character being active in normal text mode.
	\begingroup%
			\lccode`~=`(%
			\lowercase{\endgroup\edef~}{\left(}
		\mathcode`(="8000
	% Do it again for the right paren
	\begingroup%
			\lccode`~=`)%
			\lowercase{\endgroup\edef~}{\right)}
		\mathcode`)="8000
	% Finally, this change of math codes breaks some changes the amsmath
	% package makes to the definitions of \left( and \right), so we fix this
	% by patching the internal amsmath macro which breaks to be prepended with
	% the \resetparens command
	\makeatletter
		\preto{\resetMathstrut@}{\resetparens}
		\makeatother



% Enable the microtypographic extensions
% · Requires that the 2.5 (beta-08) version of microtype be installed. This can
%   be included by loading the tlcontrib repo with tlmgr and upgrading
%   the TeXLive version of microtype
\usepackage[
    protrusion=true,
]{microtype}



%% Enable all the PDF features
	% First patch the author and title macros to let us access their values
	% more easily
	\let\dflttitle\title%
	\let\dfltauthor\author%
	\renewcommand{\title}[1]{\def\dfltthetitle{#1}\dflttitle{#1}}
	\renewcommand{\author}[1]{\def\dflttheauthor{#1}\dfltauthor{#1}}
	% Then make the inclusion of hyperref and hypcap the last things done
	% before the end of the preamble.
	%
	% IMPORTANT NOTE!
	%   It took much debugging to figure out that biblatex makes use of the
	%   \AtEndPreamble command to finalize much of its setup, especially with
	%   respect to knowing if hyperref has been loaded or not. For this reason,
	%   loading hyperref with \AtEndPreamble as well *after* we've already
	%   loaded biblatex means the the code below is executed after code that
	%   biblatex has loaded into the hook.
	%
	%   To guard against such mistakes (and assumptions that hyperref was
	%   loaded in the user-written preamble which will definitely occur
	%   before the hook), we can instead *prepend* our code to the hook that
	%   \AtEndPreamble normally appends to. We do this by patching a copy of
	%   the command to use \gpreto instead of \gappto.
	%
	\let\PreAtEndPreamble\AtEndPreamble%
	\patchcmd{\PreAtEndPreamble}{\gappto}{\gpreto}{}{%
		\PackageError{defaults}{%
			Prepending to the \verb|\AtEndPreamble| hook failed. Could not
			setup hyperref correctly.
		}{%
			Loading hyperref is a tricky business, and our attempt to have
			hyperref load at the correct time when also using biblatex failed.
			A guess would be that the format of the \verb|\AtEndPreamble|
			command has changed, which means our attempt to patch a copy for
			our purposes has failed.
			\MessageBreak\MessageBreak
			See comments in the source for umnthesis.cls for more information.`
		}
	}
	% Now prepend to the \AtEndPreamble hook.
	\PreAtEndPreamble{
		\usepackage[bookmarks,pagebackref=false]{hyperref}
			\usepackage[all]{hypcap}
			\hypersetup{
				,bookmarksopen  = false%
				,pdfborderstyle = {/S/U/W 1}%
				,ocgcolorlinks  = true%
				,colorlinks     = false%
				,setpagesize    = false%
				,pdfstartview   = {XYZ null null 1}%
				,pdfprintscaling= None%
				,pdfpagelayout  = TwoPageLeft%
				,verbose        = true%
			}

		% Taken from I.M.:
		% “
		%   When default colors are set by fontspec, hyperref isn’t able to
		%   color the links at all. The following hack allows hyperref colors
		%   to make it to the top of the stack.
		% ”
		\def\HyColor@@@@UseColor#1\@nil{\addfontfeatures{Color=#1}}

		% Check to see if the author and title macros have been set in the
		% preamble. If they have, then set the PDF attributes accordingly
		\ifdefined\dfltthetitle
			\hypersetup{ pdftitle = {\dfltthetitle} }
		\fi
		\ifdefined\dflttheauthor%
			% Sanitize the author string for use in the PDF properties:
			\begingroup
				% Separate authors with ampersands
				\def\and{ \& }
				\hypersetup{ pdfauthor = {\dflttheauthor} }%
			\endgroup
		\else
			\hypersetup{ pdfauthor = {Justin Willmert} }%
		\fi
	}

%% Hide internal macros again
\makeatletter

%
% The following options are current recognized and supported:
%
% COMMAND              DEFAULT VALUE        DESCRIPTION
% -----------------------------------------------------------------------------
% \deftextcolor        0,0,0,1              The default text color is any valid
%                                           CMYK color value supported by the
%                                           xcolor package. Currently, no
%                                           support is available for setting
%                                           the default in other color spaces.
%
%
%==============================================================================
%
% *****************
% COMMANDS PROVIDED
% *****************
%
% This section documents some commands which may be necessary due to the
% modifications which are made by default, either overriding a previous change
% locally or restoring the LaTeX default functionality.
%
% COMMAND              DESCRIPTION
% -----------------------------------------------------------------------------
% \definetextcolor     Change the font color of the body text. Attempts are
%                      made to propagate the changes into all environments,
%                      but see the inline documentation for a list of the
%                      exceptions/caveats.
%
% \resetparents        By default, all parentheses in math mode made stretchy,
%                      but this may be undesirable in some context, so this
%                      command restores them to their default, inactive state.
%
%%

%% Provide a package-like environment where internal macros are accessible
\makeatletter

%% Load several commonly-assumed commands or macros to be defined
    \RequirePackage{defaults-utilities}



%% Load all commonly used packages
	\usepackage[
		margin=1in,
		twoside=true
	]{geometry}
	\usepackage{amsmath,amssymb}
	\usepackage[
		usenames,dvipsnames,svgnames,
		table,
		hyperref
	]{xcolor}
	\usepackage{float}
	\usepackage{verbatim}
	\usepackage{array}
	\usepackage{tabularx}
	\usepackage[
		exponent-product     = \cdot,
		load-configurations  = abbreviations,
		per-mode             = symbol-or-fraction,
		separate-uncertainty = true,
		math-micro           = \text{µ},
		text-micro           = µ
	]{siunitx}
	\usepackage[
		font      = footnotesize,
		labelfont = bf
	]{caption}



%% Make some amsmath fixes
%
% As of 2012-12-23 (XeTeX version 3.1415926-2.4-0.9998, TeXLive 2012), XeTeX
% does not properly handle parentheses when using the TeX primitives
%   \overwithdelims
%   \atopwithdelims
%   \abovewithdelims
% Because of this, the macro \binom which internally uses these these commands
% creates a binomial with the wrong size of brackets (too small). Patch around
% this issue by using \left( and \right) around a bracket-less binomial.
% LuaTeX doesn't seem to suffer from this.
\ifxetex
	\renewrobustcmd{\binom} [2]{\left(\genfrac{}{}{0pt}{}{#1}{#2}\right)}
	\renewcommand  {\dbinom}[2]{\left(\genfrac{}{}{0pt}{0}{#1}{#2}\right)}
	\renewcommand  {\tbinom}[2]{\left(\genfrac{}{}{0pt}{1}{#1}{#2}\right)}
\fi



%% Before loading advanced fonts, set up a default color scheme.
	% We choose black by default to correspond to norms, but this default can
	% be overridden by simply defining textcolor before loading this defaults
	% file.
	\providecommand{\deftextcolor}{0,0,0,1}
	\providecolor{textcolor}{cmyk}{\deftextcolor}
	% Now define a command which can be used to change the text colors mid-way
	% through a document. Right now, this doesn't work in all locations (e.g.
	% the body text will change color but section headings don't), but by
	% using a macro, we can potentially update the macro to also affect various
	% text environments.
	%
	% TODO: Colors do not apply to these environments even when the color is
	%       set within the preamble:
	%         · \hrule, \rule
	%
	% TODO: Colors currently will not propagate into the following modes if
	%       this macro is used after font selection/definition has occurred:
	%         · math mode
	%         · section headings
	%
	% \definetextcolor
	%    #1 -> xcolor model-list — See the xcolor documentation for more info
	%    #2 -> xcolor spec-list  — See the xcolor documentation for more info
	\newcommand{\definetextcolor}[2]{%
		% Define/Change the color which textcolor corresponds to
		\definecolor{textcolor}{#1}{#2}
		% If we're in a XeTeX/LuaTeX engine, use fontspec features to update
		% the font color.
		\ifxeluatex
			\addfontfeatures{Color=textcolor}
		\else
			% TODO: Extensive testing in non-XeTeX/LuaTeX environments.
			%       PDFTEX HAS BEEN COMPLETELY UNTESTED!
		\fi
	}

	% Taken from I.-M.:
	% “
	%   Many of these macro patches are gonna take work, since the color
	%   commands aren’t hooking well into the \rule family of commands. They’re
	%   pretty fragile to start, and occasionally \color commands “bleed” their
	%   color past their local contexts.
	%
	%   According to The LaTeX Companion, TeX’s \hrule command is actually used
	%   in the default LaTeX definition of \footnoterule instead of LaTeX’s
	%   \rule command in order to avoid strange effects of \rule, which
	%   actually starts a paragraph and related calculations. KOMA-Script
	%   modifies the footnote definition, but it still appears to be patchable.
	%
	%   If the below patch fails, the undefined macro \xpatchfailed will be
	%   “executed”, resulting in an undefined control sequence error, and will
	%   stop processing. A successful patch will be silent.
	%
	%   [TO-DO] Create a macro that writes a confirmation of the successful
	%           patch either to the terminal or a logfile during compilation.
	% ”
	\usepackage{xpatch}
	\xpretocmd{\footnoterule}{\color{textcolor}}{}{\xpatchfailed}
	% “
	%   Make table/array rules match the ambient text color...
	% ”
	\usepackage{colortbl}
	\AfterPreamble{\arrayrulecolor{textcolor}}



%% Add fancy font features if using an advanced TeX engine.
\ifxeluatex\usepackage{fontspec}

	% Set font features which should be true for all fonts loaded by default
	\defaultfontfeatures{
		% Permit using TeX-style ligatures (i.e. --- for em-dash)
		Ligatures = TeX,
		% Also use our defined color by default
		Color     = textcolor,
	}

	% Also allow math to input using Unicode by default
	\usepackage{unicode-math}
	% Use the XITS Math font by default
	\setmathfont[
			% Use italic fonts at all times
			math-style = ISO,
			bold-style = ISO,
			% Force \epsilon,\varepsilon and \phi,\varphi match the unicode
			% character ε,ϵ and φ,ϕ
			vargreek-shape = unicode
		]{XITS Math}
	% Additionally, use an alternate stylistic set for caligraphic letters so
	% that they are different from the script letters
	\setmathfont[range={\mathcal,\mathbfcal},StylisticSet=1]{XITS Math}
	% Finally, request that certain symbols be provided by Latin Modern
	% Math instead of XITS Math since I think the LM ones looks much nicer.
	\setmathfont[range={"220F}]{Latin Modern Math}	% Product
	\setmathfont[range={"2210}]{Latin Modern Math}	% Coproduct
	\setmathfont[range={"2211}]{Latin Modern Math}	% Summation

	% Have the Linux Libertine family of fonts provide the default serif and
	% sans-serif fonts
	\setmainfont{Linux Libertine O}
	\setsansfont{Linux Biolinum O}

\else %\ifxeluatex

	% Also use Linux Libertine fonts in pdflatex mode, but do this through the
	% libertine package.
	\usepackage{libertine}

	% Also choose to use use the STIX fonts like the XeLaTeX version
	\usepackage[notext]{stix}

\fi %\ifxeluatex



%% Taken from I.-M.:
%
% Default glue settings under XeTeX for some fonts sometimes creates very
% cramped text.
%
% From http://tex.stackexchange.com/questions/49298/what-settings-for-xspaceskip-would-you-suggest-for-linux-libertine?rq=1
% Bringhurst suggests these values for body text: m/3 (even better if you can
% keep it at m/4) for interword space, m/2 for maximum space, and m/5 for
% minimum space. He also feels strongly against using the extra space after
% sentences, so assume \frenchspacing.
\appto\selectfont{%
		\fontdimen2\font=0.250em % M/4
		\fontdimen3\font=\dimexpr0.500em-\fontdimen2\font % M/2
		\fontdimen4\font=\dimexpr\fontdimen2\font-0.200em%
	}
    \AfterPreamble{\frenchspacing}



%% Make all parentheses stretchy
%
% By default, all parentheses are just normal parentheses in math mode. This
% to me doesn't seem to make sense since I almost always want them to be
% stretchy.
%
% Adapted from http://tex.stackexchange.com/questions/7359/how-to-make-a-real-backslash-escape-character/7372#7372
	% Start by defining a macro to quickly reset the default behavior if we
	% decide that a given environment shouldn't use the active definition. We
	% need the \edef so that the \the\mathcode is immediately expanded before
	% we change things.
	\edef\resetparens{%
			\mathcode`(=\the\mathcode`(
			\mathcode`)=\the\mathcode`)
		}
	% Now we actually do the magic like this:
	%   1) Start a local group so that we can temporarily define a lowercase
	%      tilde to be the left paranetheses and not affect anything later in
	%      the document.
	%      a) We use the tilde because it is already defined to be an active
	%         character. This saves a step since to define a character to
	%         execute a macro sequence, it first needs to be active. The
	%         \lowercase command does for us since the lowercase tilde (i.e.
	%         paren) is made to have the same category code as its uppercase
	%         counterpart.
	%   2) \lccode assigns the lowercase tilde to be the left paren
	%   3) Then use \lowercase to cause us to define the macro of active left
	%      paren (without explicitly changing category codes).
	%      a) The \endgroup ends the local group we just started, but not
	%         before lowercase has already learned how to convert the tilde
	%         This lets the next \edef to be at the current global scope,
	%         avoiding use of \global
	%   4) Then define the active left paren to be the sequence \left( so that
	%      we invoke the stretchy code
	%   5) Finally change the mathcode of the left paren to be active within
	%      math environments, meaning we don't have to worry about the
	%      character being active in normal text mode.
	\begingroup%
			\lccode`~=`(%
			\lowercase{\endgroup\edef~}{\left(}
		\mathcode`(="8000
	% Do it again for the right paren
	\begingroup%
			\lccode`~=`)%
			\lowercase{\endgroup\edef~}{\right)}
		\mathcode`)="8000
	% Finally, this change of math codes breaks some changes the amsmath
	% package makes to the definitions of \left( and \right), so we fix this
	% by patching the internal amsmath macro which breaks to be prepended with
	% the \resetparens command
	\makeatletter
		\preto{\resetMathstrut@}{\resetparens}
		\makeatother



% Enable the microtypographic extensions
% · Requires that the 2.5 (beta-08) version of microtype be installed. This can
%   be included by loading the tlcontrib repo with tlmgr and upgrading
%   the TeXLive version of microtype
\usepackage[
    protrusion=true,
]{microtype}



%% Enable all the PDF features
	% First patch the author and title macros to let us access their values
	% more easily
	\let\dflttitle\title%
	\let\dfltauthor\author%
	\renewcommand{\title}[1]{\def\dfltthetitle{#1}\dflttitle{#1}}
	\renewcommand{\author}[1]{\def\dflttheauthor{#1}\dfltauthor{#1}}
	% Then make the inclusion of hyperref and hypcap the last things done
	% before the end of the preamble
	\AtEndPreamble{
		\usepackage[bookmarks,pagebackref=false]{hyperref}
			\usepackage[all]{hypcap}
			\hypersetup{
				,bookmarksopen  = false%
				,pdfborderstyle = {/S/U/W 1}%
				,ocgcolorlinks  = true%
				,colorlinks     = false%
				,setpagesize    = false%
				,pdfstartview   = {XYZ null null 1}%
				,pdfprintscaling= None%
				,pdfpagelayout  = TwoPageLeft%
				,verbose        = true%
			}

		% Taken from I.M.:
		% “
		%   When default colors are set by fontspec, hyperref isn’t able to
		%   color the links at all. The following hack allows hyperref colors
		%   to make it to the top of the stack.
		% ”
		\def\HyColor@@@@UseColor#1\@nil{\addfontfeatures{Color=#1}}

		% Check to see if the author and title macros have been set in the
		% preamble. If they have, then set the PDF attributes accordingly
		\ifdefined\dfltthetitle
			\hypersetup{ pdftitle = {\dfltthetitle} }
		\fi
		\ifdefined\dflttheauthor%
			% Sanitize the author string for use in the PDF properties:
			\begingroup
				% Separate authors with ampersands
				\def\and{ \& }
				\hypersetup{ pdfauthor = {\dflttheauthor} }%
			\endgroup
		\else
			\hypersetup{ pdfauthor = {Justin Willmert} }%
		\fi
	}

%% Hide internal macros again
\makeatletter
