%%%%%%%%%%%%%%%%%%%%%%%%%%%%%%%%%%%%%%%%%%%%%%%%%%%%%%%%%%%%%%%%%%%%%%%%%%%%%%%
%%%% Problem 1
%%%%%%%%%%%%%%%%%%%%%%%%%%%%%%%%%%%%%%%%%%%%%%%%%%%%%%%%%%%%%%%%%%%%%%%%%%%%%%%
%\subsection{Problem 1}
\problem{1}
\subsubsection{Question}
% Keywords
	\index{unsolved!Fall 2014 II.P1}
A particle of mass $m$ and electric charge $q$ is constrained to move along a smooth vertical hoop of radius $R$. The hoop is in a laboratory on Earth. At the lowest point of the hoop there is another fixed charge $q$. Find the equilibrium positions of $m$ and the frequency of small oscillations about the equilibrium.
\subsubsection{Answer}


%%%%%%%%%%%%%%%%%%%%%%%%%%%%%%%%%%%%%%%%%%%%%%%%%%%%%%%%%%%%%%%%%%%%%%%%%%%%%%%
%%%% Problem 2
%%%%%%%%%%%%%%%%%%%%%%%%%%%%%%%%%%%%%%%%%%%%%%%%%%%%%%%%%%%%%%%%%%%%%%%%%%%%%%%
%\subsection{Problem 2}
\problem{2}
\subsubsection{Question}
% Keywords
	\index{unsolved!Fall 2014 II.P2}
Consider a long cylindrical co-axial capacitor with an inner conductor of radius $a$, outer conductor of radius $b$, and a linear dielectric in between that has permittivity $\epsilon(\rho)=\epsilon_r(\rho)\epsilon_0$, where $\rho$ is the radial coordinate of a cylindrical coordinate system. The capacitor has a positive surface charge density $\sigma$ on the inner conductor and is charged to a voltage $V$. You can tune the function $\epsilon(\rho)$.
\begin{enumerate}
	\item What is the function $\epsilon(\rho)$ such that the energy density in the capacitor is independent of $\rho$?
	\item Assuming this energy density, calculate the electric field, the electric displacement, and the polarization.
	\item Calculate the bound volume and surface charges.
\end{enumerate}
Note: you may need to solve part B before you can completely determine $\epsilon(\rho)$ in terms of the parameters provided. Once $\epsilon(\rho)$ is determined, use its explicit form in all other expressions (such as fields, polarization, etc).
\subsubsection{Answer}



%%%%%%%%%%%%%%%%%%%%%%%%%%%%%%%%%%%%%%%%%%%%%%%%%%%%%%%%%%%%%%%%%%%%%%%%%%%%%%%
%%%% Problem 3
%%%%%%%%%%%%%%%%%%%%%%%%%%%%%%%%%%%%%%%%%%%%%%%%%%%%%%%%%%%%%%%%%%%%%%%%%%%%%%%
%\subsection{Problem 3}
\problem{3}
\subsubsection{Question}
% Keywords
	\index{unsolved!Fall 2014 II.P3}
Consider a system of $N$ non-interacting atoms in contact with a thermal reservoir at a temperature $T$. Each one of these atoms can be only in one of two states: the ground state, with zero energy, and the excited state, with energy $\epsilon > 0$.
\begin{enumerate}
	\item Find the general expression for the free energy F of the system, and evaluate it in the limiting cases $T \to 0$ and $kT \gg \epsilon$.
	\item Compute the specific heat of the system and determine how it depends on the temperature in the cases $T \to 0$ and $kT \gg \epsilon$.
	\item The energy $\epsilon$ of the excited state of an atom depends on its average distance from the other atoms such that $\epsilon = b/v^\gamma$, where $b$ is a constant, $v = V/N$ is the volume of the system per atom, and $\gamma > 1$ is the so-called Gruneisen parameter. Find the equation of state relating the pressure $P$, the volume $V$, and the total energy $E$ of the system.
\end{enumerate}
\subsubsection{Answer}



%%%%%%%%%%%%%%%%%%%%%%%%%%%%%%%%%%%%%%%%%%%%%%%%%%%%%%%%%%%%%%%%%%%%%%%%%%%%%%%
%%%% Problem 4
%%%%%%%%%%%%%%%%%%%%%%%%%%%%%%%%%%%%%%%%%%%%%%%%%%%%%%%%%%%%%%%%%%%%%%%%%%%%%%%
%\subsection{Problem 4}
\problem{4}
\subsubsection{Question}
% Keywords
	\index{unsolved!Fall 2014 II.P4}
	\index{statistical mechanics!Half Life}
The probability of nuclear decay when an ${\alpha}$-particle (bound state of two protons and two neutrons) is emitted has a very strong dependence on the energy of the ${\alpha}$-particle $E$. Empirically it was formulated in 1911 as the Geiger-Nuttal law:
\begin{equation*}
	\ln t_{1/2} = a_1\frac{Z}{\sqrt{E}} - a_2
\end{equation*}
where $t_{1/2}$ is the half-life (in seconds), $Z$ is the electric charge of the final nucleus, and $a_{1,2}$ are constants. The law was explained in 1928 by George Gamow who used the WKB approximation to calculate the transmission coefficient for the potential
\begin{equation*}
	U(r) = \begin{cases}
		-U_0 \text{ at } r< r_0\\
		Z\alpha/r \text{ at } r>r_0,
	\end{cases}
\end{equation*}
which represents a potential well at $r < r_0$ and a Coulomb repulsion at $r > r_0$.

Use this potential to find the coefficient $a_1$ for $E\ll Z{\alpha}/r_0$, where you can neglect the size of the nucleus, i.e. take the limit $r_0 \to 0$. Determine the numerical value of $a_1$ when the energy $E$ is measured in MeV.

Reminder: the WKB approximation consists of assuming that the phase of the wave function in a weakly inhomogeneous potential $U(x)$ can be written as $\int k(x) dx$ where $k(x)$ is the wave number corresponding to the value of the potential at point x. Hint: an integral of the form $\int_0^1\sqrt{\frac{1}{x}-1} dx$ can be solved by substituting $x =\sin^2(\phi)$.
\subsubsection{Answer}


%%%%%%%%%%%%%%%%%%%%%%%%%%%%%%%%%%%%%%%%%%%%%%%%%%%%%%%%%%%%%%%%%%%%%%%%%%%%%%%
%%%% Problem 5
%%%%%%%%%%%%%%%%%%%%%%%%%%%%%%%%%%%%%%%%%%%%%%%%%%%%%%%%%%%%%%%%%%%%%%%%%%%%%%%
%\subsection{Problem 5}
\problem{5}
\subsubsection{Question}
% Keywords
	\index{unsolved!Fall 2014 II.P5}
A slow neutron hits a hydrogen atom in its ground state and forms a final state containing a deuterium atom and a photon, $n+H\to D+\gamma$.

The nucleus of the deuterium atom, a deuteron, is the bound state of a neutron and proton with a binding energy of $2.23$ MeV.

What is the velocity of the deuterium in the final state in the limit of zero kinetic energy of the initial neutron? In this limit, what is the total probability to find the electron in any excited state of the deuterium atom?

Hints: (1) You can treat the deuterium motion as non-relativistic, $v \ll c$; (2) transition to a coordinate system moving with velocity $\mathbf{v}$ leads to multiplication of the wave function $\phi(\mathbf{r})$ by a factor $\exp(-im\mathbf{v}\mathbf{r}/\hbar)$; (3) the hydrogen wave function for the ground state is $\psi_0 = (1/\sqrt{\pi a^3})\exp(-r/a),$ where $a$ is the Bohr radius of the atom: $a = \hbar/(mc{\alpha})$ with ${\alpha} = 1/137.$
\subsubsection{Answer}

