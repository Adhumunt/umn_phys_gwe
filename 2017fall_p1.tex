%%%%%%%%%%%%%%%%%%%%%%%%%%%%%%%%%%%%%%%%%%%%%%%%%%%%%%%%%%%%%%%%%%%%%%%%%%%%%%%
%%%% Problem 1
%%%%%%%%%%%%%%%%%%%%%%%%%%%%%%%%%%%%%%%%%%%%%%%%%%%%%%%%%%%%%%%%%%%%%%%%%%%%%%%
%\subsection{Problem 1}
\problem{1}
\subsubsection{Question}
% Keywords
	\index{unsolved!Fall 2017 I.P1}
Assume that the atmosphere near the earth’s surface is in approximate hydrostatic equilibrium, where any movement of air parcels is isothermal. Derive an expression for the pressure $P$ of the atmosphere as a function of the height $z$ using the ideal gas law.

\subsubsection{Answer}


%%%%%%%%%%%%%%%%%%%%%%%%%%%%%%%%%%%%%%%%%%%%%%%%%%%%%%%%%%%%%%%%%%%%%%%%%%%%%%%
%%%% Problem 2
%%%%%%%%%%%%%%%%%%%%%%%%%%%%%%%%%%%%%%%%%%%%%%%%%%%%%%%%%%%%%%%%%%%%%%%%%%%%%%%
%\subsection{Problem 2}
\problem{2}
\subsubsection{Question}
% Keywords
	\index{unsolved!Fall 2017 I.P2}
You are asked to design a yo-yo that accelerates, as it drops, with a value tenth that of the acceleration due to gravity $g$. Yo-yo's are formed by two disks of radius $R$ and connected by a spindle, of a smaller radius $r$, around which a string is wound (the string goes down vertically from your hand to the spindle). The spindle and the string have negligible mass, and the combined mass of the two disks is m. What should be the ratio $R/r$ needed to achieve this acceleration as the yo-yo drops from a stationary hand holding onto the yo-yo string?

\subsubsection{Answer}



%%%%%%%%%%%%%%%%%%%%%%%%%%%%%%%%%%%%%%%%%%%%%%%%%%%%%%%%%%%%%%%%%%%%%%%%%%%%%%%
%%%% Problem 3
%%%%%%%%%%%%%%%%%%%%%%%%%%%%%%%%%%%%%%%%%%%%%%%%%%%%%%%%%%%%%%%%%%%%%%%%%%%%%%%
%\subsection{Problem 3}
\problem{3}
\subsubsection{Question}
% Keywords
	\index{unsolved!Fall 2017 I.P3}

An electric potential is given by the expression, $V(\boldsymbol{r}) = A e^{-\alpha r}/r$. Here $A$ and $\alpha$ are constant. Find
\begin{enumerate}
	\item The corresponding electric field $\boldsymbol{E}(\boldsymbol{r}).$
	\item The charge density $\rho(\boldsymbol{r})$ that generates this potential.
\end{enumerate}
\subsubsection{Answer}



%%%%%%%%%%%%%%%%%%%%%%%%%%%%%%%%%%%%%%%%%%%%%%%%%%%%%%%%%%%%%%%%%%%%%%%%%%%%%%%
%%%% Problem 4
%%%%%%%%%%%%%%%%%%%%%%%%%%%%%%%%%%%%%%%%%%%%%%%%%%%%%%%%%%%%%%%%%%%%%%%%%%%%%%%
%\subsection{Problem 4}
\problem{4}
\subsubsection{Question}
% Keywords
	\index{unsolved!Fall 2017 I.P4}
The flux of solar radiation arriving at the earth is $1.4$ kW/m$^2$. Estimate the total number flux of solar neutrinos arriving at the earth.

\subsubsection{Answer}


%%%%%%%%%%%%%%%%%%%%%%%%%%%%%%%%%%%%%%%%%%%%%%%%%%%%%%%%%%%%%%%%%%%%%%%%%%%%%%%
%%%% Problem 5
%%%%%%%%%%%%%%%%%%%%%%%%%%%%%%%%%%%%%%%%%%%%%%%%%%%%%%%%%%%%%%%%%%%%%%%%%%%%%%%
%\subsection{Problem 5}
\problem{5}
\subsubsection{Question}
% Keywords
	\index{unsolved!Fall 2017 I.P5}
In the state with given angular momentum $\ell$ and its projection on the $z$-axis $m_z$ (i.e., with the wavefunction $\psi_{\ell,m_z}$) find the average values of $\ell^2_x$ and $\ell^2_y$.
\subsubsection{Answer}



%%%%%%%%%%%%%%%%%%%%%%%%%%%%%%%%%%%%%%%%%%%%%%%%%%%%%%%%%%%%%%%%%%%%%%%%%%%%%%%
%%%% Problem 6
%%%%%%%%%%%%%%%%%%%%%%%%%%%%%%%%%%%%%%%%%%%%%%%%%%%%%%%%%%%%%%%%%%%%%%%%%%%%%%%
%\subsection{Problem 6}
\problem{6}
\subsubsection{Question}
% Keywords
	\index{unsolved!Fall 2017 I.P6}
A plank of length $L$ and height $W$ is in equilibrium, balanced on a cylinder of diameter $D$. What condition must be satisfied by $L$, $W$, and $D$ if this equilibrium is to be stable? Hint: Consider the change in height of the mass center of the plank as it rotates through a small angle $\theta$.
\subsubsection{Answer}

%%%%%%%%%%%%%%%%%%%%%%%%%%%%%%%%%%%%%%%%%%%%%%%%%%%%%%%%%%%%%%%%%%%%%%%%%%%%%%%
%%%% Problem 7
%%%%%%%%%%%%%%%%%%%%%%%%%%%%%%%%%%%%%%%%%%%%%%%%%%%%%%%%%%%%%%%%%%%%%%%%%%%%%%%
%\subsection{Problem 7}
\problem{7}
\subsubsection{Question}
% Keywords
	\index{unsolved!Fall 2017 I.P7}
Find the energy eigenvalues and normalized wave functions of the bound state in the potential
\begin{equation*}
	V(x) = -\alpha\delta(x)
\end{equation*}
Here $\alpha$ is a positive constant and $\delta$ is the Dirac delta function. Find the average value of kinetic and potential energies in the bound state.
\subsubsection{Answer}



%%%%%%%%%%%%%%%%%%%%%%%%%%%%%%%%%%%%%%%%%%%%%%%%%%%%%%%%%%%%%%%%%%%%%%%%%%%%%%%
%%%% Problem 8
%%%%%%%%%%%%%%%%%%%%%%%%%%%%%%%%%%%%%%%%%%%%%%%%%%%%%%%%%%%%%%%%%%%%%%%%%%%%%%%
%\subsection{Problem 8}
\problem{8}
\subsubsection{Question}
% Keywords
	\index{unsolved!Fall 2017 I.P8}
Let us consider two identical linear harmonic oscillators 1 and 2 separated by $R$. Each oscillator bears charges $\pm e$ with separation $x_1$ and $x_2$, as shown in Fig. The particles oscillate along the $x$ axis. Let $p_1$ and $p_2$ denote the momenta. The force constant is $C$. All particles have mass $m$. Then the Hamiltonian of the non-interacting system is
\begin{equation*}
	H_0 = \frac{1}{2m}p_1^2 + \frac{1}{2}Cx_1^2+ \frac{1}{2m}p_2^2+ \frac{1}{2}Cx_2^2
\end{equation*}
\begin{enumerate}
	\item Write down the Coulomb interaction energy $(H_1)$ between all the charged particles. Then take the approximation, $\abs{x_{1,2}}\ll R$ and obtain the leading term.
	\item By applying the normal mode transformation to the total Hamiltonian $(H_0+H_1)$,show the characteristic frequencies of these coupled oscillators are 
	\begin{equation*}
		\omega = \qty[\qty(C\pm \frac{2e^2}{R^3})/m]^{1/2}
	\end{equation*}
\end{enumerate}
\subsubsection{Answer}


%%%%%%%%%%%%%%%%%%%%%%%%%%%%%%%%%%%%%%%%%%%%%%%%%%%%%%%%%%%%%%%%%%%%%%%%%%%%%%%
%%%% Problem 9
%%%%%%%%%%%%%%%%%%%%%%%%%%%%%%%%%%%%%%%%%%%%%%%%%%%%%%%%%%%%%%%%%%%%%%%%%%%%%%%
%\subsection{Problem 9}
\problem{9}
\subsubsection{Question}
% Keywords
	\index{unsolved!Fall 2017 I.P9}
Consider a system of $N$ non-interacting particles each with a spin $S = 1$ and fixed position in an external magnetic field $H$. The particles have magnetic moment $\mu_B$. If no magnetic field is present all spin projections $S_z$ of a single particle are degenerate with an energy $E = 0$.

a) Plot the energy for all spin projections as a function of $H$.

b) Calculate the partition function of a single spin and then of the system as a function of the temperature and magnetic field $H$.

c) Calculate the average energy $\expval{E}$ of the system and find its form for $T \to 0$.
\subsubsection{Answer}


%%%%%%%%%%%%%%%%%%%%%%%%%%%%%%%%%%%%%%%%%%%%%%%%%%%%%%%%%%%%%%%%%%%%%%%%%%%%%%%
%%%% Problem 10
%%%%%%%%%%%%%%%%%%%%%%%%%%%%%%%%%%%%%%%%%%%%%%%%%%%%%%%%%%%%%%%%%%%%%%%%%%%%%%%
%\subsection{Problem 10}
\problem{10}
\subsubsection{Question}
% Keywords
	\index{unsolved!Fall 2017 I.P10}
A particle of mass $m_0$ decays at rest in the lab frame, producing particle 1 of mass $m_1$ and particle 2 of mass $m_2$. Find the energy of the particle 1 in the lab frame.
\subsubsection{Answer}

