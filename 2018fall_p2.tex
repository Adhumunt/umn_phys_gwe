%%%%%%%%%%%%%%%%%%%%%%%%%%%%%%%%%%%%%%%%%%%%%%%%%%%%%%%%%%%%%%%%%%%%%%%%%%%%%%%
%%%% Problem 1
%%%%%%%%%%%%%%%%%%%%%%%%%%%%%%%%%%%%%%%%%%%%%%%%%%%%%%%%%%%%%%%%%%%%%%%%%%%%%%%
%\subsection{Problem 1}
\problem{1}
\subsubsection{Question}
% Keywords
	\index{unsolved!Fall 2018 II.P1}
Recall that for a charged particle in a magnetic field the quantum Hamiltonian is:
\begin{equation*}
	H = \frac{1}{2m}\qty(\boldsymbol{p} - q\boldsymbol{A})^2
\end{equation*}
where $\mathbf{B} = \grad\cross \boldsymbol{A}$ and $\boldsymbol{p}\to - i\hbar\grad$. Show that
\begin{equation*}
	\dv{t}\expval{\boldsymbol{r}} = \expval{(\boldsymbol{p} - r\boldsymbol{A})/m}
\end{equation*}
Is this result consistent with classical mechanics?
\subsubsection{Answer}


%%%%%%%%%%%%%%%%%%%%%%%%%%%%%%%%%%%%%%%%%%%%%%%%%%%%%%%%%%%%%%%%%%%%%%%%%%%%%%%
%%%% Problem 2
%%%%%%%%%%%%%%%%%%%%%%%%%%%%%%%%%%%%%%%%%%%%%%%%%%%%%%%%%%%%%%%%%%%%%%%%%%%%%%%
%\subsection{Problem 2}
\problem{2}
\subsubsection{Question}
% Keywords
	\index{unsolved!Fall 2018 II.P2}
A simple pendulum (mass $m$) is suspended from a cart (mass $M$) which moves without friction along a horizontal track.
\begin{enumerate}
	\item Taking the position of the cart on the tract as $X$ and the angle of the pendulum as $\theta$, find the equations of motion.
	\item Show how one of these reduces to the expected special case solutions if we $(i)$ fix the position of the cart or $(ii)$ imagine the pendulum. mass is stationary with respect to the cart. 
\end{enumerate}
\subsubsection{Answer}



%%%%%%%%%%%%%%%%%%%%%%%%%%%%%%%%%%%%%%%%%%%%%%%%%%%%%%%%%%%%%%%%%%%%%%%%%%%%%%%
%%%% Problem 3
%%%%%%%%%%%%%%%%%%%%%%%%%%%%%%%%%%%%%%%%%%%%%%%%%%%%%%%%%%%%%%%%%%%%%%%%%%%%%%%
%\subsection{Problem 3}
\problem{3}
\subsubsection{Question}
% Keywords
	\index{unsolved!Fall 2018 II.P3}
Tritium ions are confined inside a thin wall spherical vessel with a radius of $R$. (Tritium ions are positively charged isotopes of Hydrogen with one proton and two neutrons, and you can assume that they have the mass of three protons). When the hot gas of tritium ions is stored inside the vessel, it forms layers with different concentrations such that the charge density is ${\rho} = a_0 r$ where $a_0$ is a positive constant and $r$ is the radial position from the center of the vessel. The goal is to inject additional Tritium ions into the container so that they can reach the center of the vessel. To know the requirements for your injection system, you need to map out the potential created by the ionic gas.
\begin{enumerate}
	\item What is the potential as a function of radial distance $r$ from the center of the vessel? Make sure you determine the function for all regions of space and assume $V = 0$ at infinity. Give your answer in terms of $R$ and $a_0$.
	\item Sketch a graph of the potential from part a).
	\item Assume the vessel has a radius of $1.00$m and $a_0 = 1.88 {\times} 10^{-6}$ C/m$^4$ . If you are a long distance away from the vessel (you can use the approximation that $r = {\infty}$), at what minimum speed would you need to shoot a Tritium ion towards the vessel if you want it to reach the center of the vessel?
\end{enumerate}
\subsubsection{Answer}



%%%%%%%%%%%%%%%%%%%%%%%%%%%%%%%%%%%%%%%%%%%%%%%%%%%%%%%%%%%%%%%%%%%%%%%%%%%%%%%
%%%% Problem 4
%%%%%%%%%%%%%%%%%%%%%%%%%%%%%%%%%%%%%%%%%%%%%%%%%%%%%%%%%%%%%%%%%%%%%%%%%%%%%%%
%\subsection{Problem 4}
\problem{4}
\subsubsection{Question}
% Keywords
	\index{unsolved!Fall 2018 II.P4}
The energy spectrum of a system consists of two orbitals with one-particle energies ${\epsilon}_1 = -{\epsilon}$ and ${\epsilon}_2 = {\epsilon}$. The system is in thermal and diffusive contact with a reservoir at temperature ${\tau}$ and chemical potential ${\mu}$. Assuming that each orbital can be occupied by no more than one particle and that ${\mu} = 0$, find:
\begin{enumerate}
	\item All possible states of a system, $({\epsilon}, N)$, and the grand partition function $\mathcal{Z}$.
	\item The probability that the system is in a state with zero energy.
	\item The probability for the system to be occupied by one particle.
	\item The average number of particles in the system, $\expval{N}$, and average energy, $\expval{U}$.
	\item $\expval{N}$ and $\expval{U}$ in the limits ${\tau} {\to} 0$ and ${\tau} {\to} {\infty}$.
\end{enumerate}

\subsubsection{Answer}


%%%%%%%%%%%%%%%%%%%%%%%%%%%%%%%%%%%%%%%%%%%%%%%%%%%%%%%%%%%%%%%%%%%%%%%%%%%%%%%
%%%% Problem 5
%%%%%%%%%%%%%%%%%%%%%%%%%%%%%%%%%%%%%%%%%%%%%%%%%%%%%%%%%%%%%%%%%%%%%%%%%%%%%%%
%\subsection{Problem 5}
\problem{5}
\subsubsection{Question}
% Keywords
	\index{unsolved!Fall 2018 II.P5}
A spaceship travels at a constant velocity $v = 0.8c$ with respect to the Earth. Denote the spaceship frame coordinates by a prime $( ^\prime )$. At time $t = t^\prime = 0$ by Earth and spaceship clocks, respectively, a light signal is sent from the tail (back end) of the spaceship towards the nose (front end) of the spaceship, just as the tail of the spaceship (at $x^\prime = 0$ in the spaceship frame) passes the Earth (at $x = 0$ in the Earth frame). The length of the spaceship, measured in a frame in which it is at rest, is $L$.
\begin{enumerate}
	\item At what time, by spaceship clocks, does the light signal reach the nose of the spaceship?
	\item At what time, by Earth clocks, does the light signal reach the nose of the spaceship?
\end{enumerate}
Now suppose there is a mirror at the nose of the spaceship which immediately reflects the light signal back to the tail of the spaceship.
\begin{enumerate}
	\item At what time, by spaceship clocks, does the light signal finally return to the tail of the spaceship?
	\item At what time, by earth clocks, does the light signal finally return to the tail of the spaceship? All answers should be expressed in terms of $L$ and $c$.
\end{enumerate}

\subsubsection{Answer}
