%%%%%%%%%%%%%%%%%%%%%%%%%%%%%%%%%%%%%%%%%%%%%%%%%%%%%%%%%%%%%%%%%%%%%%%%%%%%%%%
%%%% Problem 1
%%%%%%%%%%%%%%%%%%%%%%%%%%%%%%%%%%%%%%%%%%%%%%%%%%%%%%%%%%%%%%%%%%%%%%%%%%%%%%%
\problem{1}
\subsubsection{Question}
% Keywords
	\index{quantum!Significance in limits}

For a many particle system of weekly interacting particles, will quantum
effects be more important for (a) high densities or low densities and (b)
high temperatures or low temperatures for a system. Explain your answers in
terms of the de Broglie wavelength $λ$ defined as $λ² ≡ h²⁄(3mk_BT)$ where
$m$ is the mass of the particles and $k_b$ Boltzmann's constant.

\subsubsection{Answer}
\renewcommand{\labelenumi}{(\alph{enumi})}
\begin{enumerate}
	\item
		High density — The de Broglie wavelength gives a ``size'' of the
		particle, and in the high density limit, the wavefunctions overlap
		significantly so quantum effects and interactions are critical to
		the behavior of the system.
	\item
		Low temperature — Since $λ² \propto T^{-1}$, as $T → 0$, $λ$ increases
		so that again the wavefunctions overlap and quantum effects are
		significant.
\end{enumerate}

%%%%%%%%%%%%%%%%%%%%%%%%%%%%%%%%%%%%%%%%%%%%%%%%%%%%%%%%%%%%%%%%%%%%%%%%%%%%%%%
%%%% Problem 2
%%%%%%%%%%%%%%%%%%%%%%%%%%%%%%%%%%%%%%%%%%%%%%%%%%%%%%%%%%%%%%%%%%%%%%%%%%%%%%%
\problem{2}
\subsubsection{Question}
% Keywords
	\index{quantum!Helium ionization}

The ground state energy of Helium is \SI{-79}{\eV}. What is its ionization
energy, which is the energy required to remove just one electron?

\subsubsection{Answer}
Using the Hydrogen solution with modifications for single-electron atoms of
higher $Z$, we know that the ground state energy of singly ionized Helium is
\begin{align*}
	E_{He}^{1} = 2² (\SI{-13.6}{\eV}) = \SI{-54.4}{\eV}
\end{align*}
Therefore, the difference between the singly-ionized and neutral ground state
energies gives the first ionization energy of the Helium atom.
\begin{align}
	\boxed{
	E_i = \SI{-24.6}{\eV}
	}
\end{align}

%%%%%%%%%%%%%%%%%%%%%%%%%%%%%%%%%%%%%%%%%%%%%%%%%%%%%%%%%%%%%%%%%%%%%%%%%%%%%%%
%%%% Problem 3
%%%%%%%%%%%%%%%%%%%%%%%%%%%%%%%%%%%%%%%%%%%%%%%%%%%%%%%%%%%%%%%%%%%%%%%%%%%%%%%
\problem{3}
\subsubsection{Question}
% Keywords
	\index{dimensional analysis!Vacuum (Casimir) force}

It is known that the force per unit area ($F/A$) between two neutral
conducting plates due to polarization fluctuations of the vacuum (namely,
the Casimir force) is a function of $h$ (Planck's constant), $c$ (speed of
light), and $z$ (distance between the plates) only. Using only dimensional
analysis, obtain $F/A$ as a function of $h$, $c$, and $z$.

\subsubsection{Answer}
The units of $F/A$ are
\begin{align*}
	\frac{F}{A} &= \left[ \frac{\si{\kg}}{\si{\m\s\squared}} \right]
\end{align*}
The \si{\kg} suggests a factor proportional to $h$, making the equation
\begin{align*}
	\frac{F}{A} &\sim \left[ \frac{1}{\si{\m\cubed\s}} \right] h \\
\intertext{Accounting for the factor of seconds requires a $c$:}
	\frac{F}{A} &\sim \left[ \frac{1}{\si{\m\tothe{4}}} \right] hc \\
\intertext{Finally, account for all the factors of distance:}
	\frac{F}{A} &\sim \frac{hc}{z⁴} \\
\end{align*}
Therefore,
\begin{align}
	\boxed{
	\frac{F}{A} \sim \frac{hc}{z⁴}
	}
\end{align}

%%%%%%%%%%%%%%%%%%%%%%%%%%%%%%%%%%%%%%%%%%%%%%%%%%%%%%%%%%%%%%%%%%%%%%%%%%%%%%%
%%%% Problem 4
%%%%%%%%%%%%%%%%%%%%%%%%%%%%%%%%%%%%%%%%%%%%%%%%%%%%%%%%%%%%%%%%%%%%%%%%%%%%%%%
\problem{4}
\subsubsection{Question}
% Keywords
	\index{circuits!Parallel capacitors with switches}

In the circuit diagram opposite, initially the two identical capacitors with
capacitance $C$ are uncharged. The connections between the components are
all made with short copper wires. The battery is an ideal EMF and supplies a
voltage $V$.
\begin{enumerate}
	\item
		At first Switch A is closed and Switch B is kept open. What is the
		final sotred energy on capacitor $C_a$?
	\item
		Switch A is opened and afterwards Switch B is closed. What is the
		final energy stored in both capacitors?
	\item
		Provide a physical explanation for any difference between the
		results of parts (a) and (b), if there is one.
\end{enumerate}
\begin{center}
	\begin{circuitikz}
		\draw
			% Draw the battery
			(0,-1) to [battery=$V$] ++(0,2)
			% and then the switches
			to [cspst=$A$] ++(2,0)
				coordinate (between)
			to [cspst=$B$] ++(2,0)
			% then down through capacitor B
			to [capacitor=$C_b$] ++(0,-2)
			% And complete the outer loop
			to [short] (0,-1);
		;
		% Go back and draw capacitor A
		\draw (between) to [capacitor=$C_a$] ++(0,-2);
	\end{circuitikz}
\end{center}

\subsubsection{Answer}
\begin{enumerate}
	\item
		Initially, the right side of the circuit with $C_b$ can be ignored,
		so the total energy is simply the energy stored within $C_a$.
		\begin{align}
			\boxed{
			E = \frac 12 CV²
			}
		\end{align}
	\item
		The system is now effectively just the two capacitors on the right.
		Because the voltage difference is supported across both capacitors,
		the system can be modeled as an effective capacitor in parallel
		\begin{align*}
			C_{eff} &= 2C
		\end{align*}
		The total charge stored by the capacitors must remain the same when
		switching from Switch A being closed to Switch B. Initially,
		\begin{align*}
			Q &= CV_i
		\end{align*}
		and afterwards it is
		\begin{align*}
			Q &= C_{eff}V = 2CV_f
		\end{align*}
		so the final voltage across the capacitors is
		\begin{align*}
			V_f &= \frac 12 V_i
		\end{align*}
		This means the total energy is
		\begin{align*}
			E &= \frac 12 C_{eff} {V_f}²
		\end{align*}
		\begin{align}
			\boxed{
			E = \frac 14 CV²
			}
		\end{align}
	\item
		The energy is dissipated (heat, fields, etc).
\end{enumerate}

%%%%%%%%%%%%%%%%%%%%%%%%%%%%%%%%%%%%%%%%%%%%%%%%%%%%%%%%%%%%%%%%%%%%%%%%%%%%%%%
%%%% Problem 5
%%%%%%%%%%%%%%%%%%%%%%%%%%%%%%%%%%%%%%%%%%%%%%%%%%%%%%%%%%%%%%%%%%%%%%%%%%%%%%%
\problem{5}
\subsubsection{Question}
% Keywords
	\index{orbits!Angular momentum of a planet}
	\index{mechanics!Angular momentum of a planet}

A planet of mass $m$ moves around the sun, mass $M$, in an elliptical orbit
with minimum and maximum distances of $r₁$ and $r₂$, respectively. Find the
angular momentum of the planet relative to the center of the sun in terms of
these quantities and the gravitational constant $G$.

\subsubsection{Answer}
We solve the problem using conservation of energy since we know that stable
elliptical orbits have constant energy. The generic equation is
\begin{align*}
	E &= \frac{L²}{2I} - \frac{GMm}{r}
\end{align*}
where $L$ is the angular momentum and $I$ the moment of inertia. Substituting
for the values at both $r₁$ and $r₂$ and equating,
\begin{align*}
	\frac{L²}{2m{r₁}²} - \frac{GMm}{r₁} &= \frac{L²}{2m{r₂}²} - \frac{GMm}{r₂}\\
	\frac{L²}{2m}(\frac{1}{{r₁}²} - \frac{1}{{r₂}²}) &=
		GMm(\frac{1}{r₁} - \frac{1}{r₂})
\end{align*}
which leads to the solution
\begin{align}
	\boxed{
	L = \sqrt{ \frac{2GMm² r₁ r₂}{r₁ + r₂} }
	}
\end{align}

%%%%%%%%%%%%%%%%%%%%%%%%%%%%%%%%%%%%%%%%%%%%%%%%%%%%%%%%%%%%%%%%%%%%%%%%%%%%%%%
%%%% Problem 6
%%%%%%%%%%%%%%%%%%%%%%%%%%%%%%%%%%%%%%%%%%%%%%%%%%%%%%%%%%%%%%%%%%%%%%%%%%%%%%%
\problem{6}
\subsubsection{Question}
% Keywords
	\index{mechanics!Central Forces}
	\index{Lagrangian!Central Forces}
	\index{orbits!Central Forces}

A particle moves in a circular orbit under the influence of a central force
that varies as the $n$-th power of the distance. Show that this motion is
unstable if $n < -3$. (Hint: Consider the centrifugal potential.)

\subsubsection{Answer}
See solution for \nameref{prob:F2011I02} with the condition inverted so
that \emph{in}stability is $n < -3$ rather than stability requiring $n > -3$.

%%%%%%%%%%%%%%%%%%%%%%%%%%%%%%%%%%%%%%%%%%%%%%%%%%%%%%%%%%%%%%%%%%%%%%%%%%%%%%%
%%%% Problem 7
%%%%%%%%%%%%%%%%%%%%%%%%%%%%%%%%%%%%%%%%%%%%%%%%%%%%%%%%%%%%%%%%%%%%%%%%%%%%%%%
\problem{7}
\subsubsection{Question}
% Keywords
	\index{thermodynamics!Isentropic compression}

A classical, ideal, monatomic gas of $N$ particles is reversibly compressed
\emph{isentropically}, i.e.~with the entropy kept constant, from an initial
temperature $T₀$ and pressure $P$ to a pressure $2P$. Find (a) the work done
on the system, and (b) the net change in entropy of the system and its
surroundings.

\begin{enumerate}
	\item
		An isentropic process is the same as an adiabatic process since no
		heat can be exchanged ($T\dd S = Q = 0$), so we begin with the relation
		that $PV^{γ}$ is a constant. Combining this with the ideal gas law,
		we can determine that
		\begin{align*}
			P^{1-γ}T^{γ} = \mathrm{const}
		\end{align*}
		where $γ = C_p/C_v$ is the ratio of heat capacities with $C_p =
		\frac 52 Nk_B$ and $C_v = \frac 32 Nk_B$ for a monatomic ideal gas.
		Using this, we solve for the final temperature of the system after
		compressions as
		\begin{align*}
			T_f &= 2^{2/5} T₀ ≈ 1.32T₀
		\end{align*}
		Combining both of
		\begin{align*}
			ΔU &= C_v ΔT \\
			ΔU &= Q + W
		\end{align*}
		where $Q = 0$, we get that
		\begin{align}
			\boxed{
			W = \frac 32 Nk_B T₀ (2^{2/5} - 1)
			}
		\end{align}
	\item
		Because the compression is done reversibly, by definition, $ΔS = 0$.
\end{enumerate}

%%%%%%%%%%%%%%%%%%%%%%%%%%%%%%%%%%%%%%%%%%%%%%%%%%%%%%%%%%%%%%%%%%%%%%%%%%%%%%%
%%%% Problem 8
%%%%%%%%%%%%%%%%%%%%%%%%%%%%%%%%%%%%%%%%%%%%%%%%%%%%%%%%%%%%%%%%%%%%%%%%%%%%%%%
\problem{8}
\subsubsection{Question}
% Keywords
	\index{thermodynamics!Fermi gas properties}
	\index{statistical mechanics!Fermi gas properties}

For an idea Fermi gas of $N$ neutral spin-$\frac 12$ particles in a volume
$V$ at $T = 0$, calculate the following:
\begin{enumerate}
	\item The chemical potential
	\item The average energy per particle
	\item The pressure
\end{enumerate}

\subsubsection{Answer}
\begin{enumerate}
	\item
		At $T = 0$, the particles are all in the lowest state allowed by
		Fermi-Dirac statistics, so the chemical potential, defined by the
		energy required to add another particle to the system, is equal to the
		Fermi energy. For a particle contained within a box $V$, the energy
		per particle is
		\begin{align*}
			ε_n &= \frac{π²ℏ²}{2mV^{2/3}} n²
		\end{align*}
		Given a Fermi energy $ε_F$, the maximum occupied state is
		\begin{align*}
			n_F &= \sqrt{\frac{2mV^{2/3}}{π²ℏ²}} \sqrt{ε_F}
		\end{align*}
		Equally we know that all $N$ particles must exist within the
		eighth-sphere of $n$ space, where the extra factor of 2 is because
		there are two spin states per $n$:
		\begin{align*}
			N &= 2·\frac 18 · \frac 43 π{n_F}³ \\
			N &= \frac 13 π ( \frac{2m}{π²ℏ²} )^{3/2} V {ε_F}^{3/2} \\
			ε_F &= \frac{ℏ²}{2m} (\frac{3π²N}{V})^{2/3}
		\end{align*}
		Therefore $μ = ε_F$,
		\begin{align}
			\boxed{
			μ = \frac{ℏ²}{2m} (\frac{3π²N}{V})^{2/3}
			}
		\end{align}
	\item
		To get the total energy, we can imagine filling all $N$ particles one
		at a time, so that at each step, there are $N'$ total particles:
		\begin{align*}
			U &= ∫_0^N ε_F \dd N' \\
			U &= \frac{ℏ²}{2m} (\frac{3π²}{V})^{2/3} ∫_0^N N^{2/3} \dd N' \\
			U &= \frac{ℏ²}{2m} (\frac{3π²}{V})^{2/3} · \frac 35 N^{5/3} \dd N'
		\end{align*}
		Therefore, the average energy per particle is $U/N$ or
		\begin{align}
			\boxed{
			\langle ε \rangle = \frac 35 ε_F
			}
		\end{align}
	\item
		From the thermodynamic relation
		\begin{align*}
			dU &= T\dd S - P\dd V + μ\dd N
		\end{align*}
		we can read off the derivative that defines the pressure $P$ as
		\begin{align*}
			P &= - ( \frac{∂U}{∂V} )_{S,N}
		\end{align*}
		Doing so, we get that
		\begin{align*}
			\frac{∂U}{∂V} &= \frac 35 N · \frac{ℏ²}{2m} (\frac{3π²}{V})^{2/3} ·
				(-\frac{2}{3V})
		\end{align*}
		making the pressure
		\begin{align}
			\boxed{
			P = \frac 25 \frac{N}{V} ε_F
			}
		\end{align}
\end{enumerate}

%%%%%%%%%%%%%%%%%%%%%%%%%%%%%%%%%%%%%%%%%%%%%%%%%%%%%%%%%%%%%%%%%%%%%%%%%%%%%%%
%%%% Problem 10
%%%%%%%%%%%%%%%%%%%%%%%%%%%%%%%%%%%%%%%%%%%%%%%%%%%%%%%%%%%%%%%%%%%%%%%%%%%%%%%
\problem{10}
\subsubsection{Question}
% Keywords
	\index{electrostatics!Hall effect}
	\index{solid state!Hall effect}

A piece of $p$-doped silicon has a carrier density
$n=\SI[per-mode=reciprocal]{e15}{\per\cm\cubed}$ and dimensions of $Δx =
\SI{10}{\mm}$, $Δy = \SI{2}{\mm}$, and $Δz = \SI{1}{\mm}$. A magnetic field
of $B_z = \SI{1}{T}$ is applied in the $z$-direction and a current $I_x =
\SI{1}{\A}$ flows in the $x$-direction, and the voltage $V_y$ is measured.
\begin{enumerate}
	\item
		Express the current density $j_x$ in terms of the carrier density $n$
		and the carrier velocity $v_x$.
	\item
		Write down the equilibrium force condition that determins $V_y$.
	\item
		Find $V_y$ in volts.
\end{enumerate}

\subsubsection{Answer}

\begin{enumerate}
	\item
		The current passing through each thin cross-sectional slice of the
		conductor is dependent on the charge of a carrier, carrier density,
		and velocity of the flow.
		\begin{align*}
			I_x &= enΔyΔzv_x
		\end{align*}
		The current density is just the current passing through each point, so
		\begin{align*}
			j_x &= \frac{I_x}{ΔyΔz}
		\end{align*}
		\begin{align}
			\boxed{
			j_x = nev_x
			}
		\end{align}
	\item
		The positive carriers drift to the edge of the conductor due to the
		magnetic field and the holes accumulate on the opposite edge. An
		electric field is created between the charge separation, so an
		equilibrium is set up between the electric field trying to bring the
		opposite charges together and the magnetic drift separating them.
		\begin{align*}
			0 &= e\vec E + \vec v × \vec B
		\end{align*}
		By the right-hand rule, the positive charges accumulate along $y=0$, so
		$\vec E = E \hat y$. Similarly, $\vec v × \vec B = -v_xB_z\hat y$:
		\begin{align*}
			0 &= eE\hat y - ev_xB_z\hat y
		\end{align*}
		Written in terms of the potential $V_y = EΔy$, the equilibrium
		condition becomes
		\begin{align}
			\boxed{
			V_y = v_xB_zΔy
			}
		\end{align}
	\item
		Substituting in for given quantities
		\begin{align*}
			V_y &= \frac{I_x B_z}{neΔz} \\
			V_y &= \frac{(\SI{1}{\A})(\SI{1}{T})}
				{(\SI[per-mode=reciprocal]{e15}{\per\cm\cubed})
				 (\SI{1.612e-19}{\coulomb})(\SI{1}{\mm})}
		\end{align*}
		\begin{align}
			\boxed{
			V_y = \SI{6.24}{\V}
			}
		\end{align}
\end{enumerate}
