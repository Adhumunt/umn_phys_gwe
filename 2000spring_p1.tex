%%%%%%%%%%%%%%%%%%%%%%%%%%%%%%%%%%%%%%%%%%%%%%%%%%%%%%%%%%%%%%%%%%%%%%%%%%%%%%%
%%%% Problem 1
%%%%%%%%%%%%%%%%%%%%%%%%%%%%%%%%%%%%%%%%%%%%%%%%%%%%%%%%%%%%%%%%%%%%%%%%%%%%%%%
\problem{1}
\subsubsection{Question}
% Keywords
	\index{mechanics!Energy loss in orbital descent}

A satellite of mass $m = \SI{500}{\kg}$ is in a circular orbit at an
altitude $h = \SI{150}{\km}$ above the Earth's surface. As a result of air
friction, the satellite's orbit degrades. Protected by a heat shield, the
satellite eventually impacts with a velocity of $\SI{2}{\km\per\s}$. How
much energy (in Joules) was released as heat in the process?

\subsubsection{Answer}

The solution method will be a simple energy balance equation. In its initial
state, the satellite had gravitation potential energy that contributed to
its energy equal to
\begin{align*}
    V_0 &= \frac{GM_E m}{R_E + h}
\end{align*}
where $M_E$ and $R_E$ are the mass and radius of the Earth. The kinetic energy
can be determined by making use of simple circle relations. The centripetal
force must be provided by the gravitation force, so
\begin{align*}
    \frac{{v_0}^2}{R_E + h} &= \frac{GM_E}{(R_E + h)^2} \\
    v_0 &= \sqrt{\frac{GM_E}{R_E + h}}
\end{align*}
making the initial kinetic energy
\begin{align*}
    T_0 &= \frac{GM_E m}{2(R_E + h)}
\end{align*}

In it's final state, the satellite consists of it's final given velocity's
kinetic energy and more gravitational potential energy (with respect to the
center of the Earth). They are given by
\begin{align*}
    T_f &= \frac 12 m{v_f}^2 \\
    V_f &= \frac{GM_E m}{R_E}
\end{align*}
By conservation of energy, any energy different must result from the loss
of energy between the initial and final states, so the lost energy
$E_\mathrm{loss}$ is
\begin{align*}
    E_\mathrm{loss} &= T_0 + V_0 - T_f - V_f \\
    {} &= \frac{GM_E m}{R_E(R_E + h)}(R_E - 2h) - \frac 12 m{v_f}^2
\end{align*}
Plugging in all given quantities and constants, we have that
\begin{align}
    \boxed{
    E_\mathrm{loss} = \SI{2.816e10}{\J} = \SI{28.16}{\giga\J}
    }
\end{align}

%%%%%%%%%%%%%%%%%%%%%%%%%%%%%%%%%%%%%%%%%%%%%%%%%%%%%%%%%%%%%%%%%%%%%%%%%%%%%%%
%%%% Problem 2
%%%%%%%%%%%%%%%%%%%%%%%%%%%%%%%%%%%%%%%%%%%%%%%%%%%%%%%%%%%%%%%%%%%%%%%%%%%%%%%
\problem{2}
\subsubsection{Question}
% Keywords
    \index{mechanics!Recoiling iron}

What is the velocity of recoil of an ${}^{57}\mathrm{Fe}$ nucleus that emits
a \SI{100}{\keV} photon, both in units of the speed of light in vacuum and in
meters per second.

\subsubsection{Answer}

In the initial state observed from the iron atom's rest frame, the momentum
is zero. After the emission, the photon has a momentum of $p_\gamma = E_\gamma/c$ and
consequently, the nucleus must recoil with momentum $p_{Fe} = -p_\gamma$. Knowing
that the energy is non-relativistic, then
\begin{align*}
    m_{Fe}v_{Fe} &= \frac{E_\gamma }{c} \\
    v_{Fe} &= \frac{E_\gamma }{57m_p c}
\end{align*}
where we've approximate the mass of the iron atom by a multiple of the proton
mass. Plugging in the numbers
\begin{align}
    \boxed{ v_{Fe} = \SI{561.4}{\m\per\s} }
\end{align}
or in units of $c$
\begin{align}
    \boxed{ v_{Fe} = \SI{1.87e-6}{c} }
\end{align}

%%%%%%%%%%%%%%%%%%%%%%%%%%%%%%%%%%%%%%%%%%%%%%%%%%%%%%%%%%%%%%%%%%%%%%%%%%%%%%%
%%%% Problem 3
%%%%%%%%%%%%%%%%%%%%%%%%%%%%%%%%%%%%%%%%%%%%%%%%%%%%%%%%%%%%%%%%%%%%%%%%%%%%%%%
\problem{3}
\subsubsection{Question}
% Keywords
    \index{electrodynamics!Voltage and phase in an RL circuit}
    \index{circuits!Voltage and phase in an RL circuit}

A resistance $R$ and an inductance $L$ are connected in series, and an
alternating voltage $V_0\cos {\omega} t$ is impressed across the combination. The
resulting steady state voltage across the resistance can be written as $V_R
\cos({\omega} t + \beta)$. Find $V_R$ and $\beta$, assuming both $V_0$ and $V_R$ to be
positive.

\subsubsection{Answer}

The problem is simplified by constucting the solution using complex voltages
and currents and recovering the correct component at the end. Since the
source voltage is a cosine, the real component will be kept at the end.

The complex impedance for a resistor is simply the resitance itself, so $Z_R
= R$. For the inductor, it is $Z_L = i{\omega} L$. We then use the complex
impedances together with Ohm's Law and the first Kirchoff rule to solve for
the complex current $I$.
\begin{align*}
    V_0e^{i{\omega} t} &= IR + i{\omega} LI \\
    I &= \frac{V_0}{R+i{\omega} L} e^{i{\omega} t}
\end{align*}
Then by inserting this solution into the voltage law for inductors, we can
solve for the complex voltage across the inductor.
\begin{align*}
    V_L &= L\frac{dI}{dt} \\
    V_L &= V_0 \frac{i{\omega} L}{R+i{\omega} L} e^{i{\omega} t}
\end{align*}

The voltage drops across the resistor and inductor must equal the impressed
voltage, so we can solve for the unknown voltage across the resistor.
\begin{align*}
    V_0e^{i{\omega} t} &= V + V_L \\
    V &= V_0e^{i{\omega} t} - V_0\frac{i{\omega} L}{R+i{\omega} L}e^{i{\omega} t} \\
    V &= V_0 (\frac{R^2 - i{\omega} RL}{R^2 + {\omega}^2L^2}) e^{i{\omega} t}
\end{align*}
In order to make taking the real component simpler, we put the term in
parentheses in complex exponential form according to the relation $z =
|z|e^{i \arg z}$.
\begin{align*}
    \left| \frac{R^2 - i{\omega} RL}{R^2 + {\omega}^2L^2} \right| &=
	\frac{R\sqrt{R^2 + {\omega}^2L^2}}{R^2 + {\omega}^2L^2}
    \\
    \arg (\frac{R^2 - i{\omega} RL}{R^2 + {\omega}^2L^2}) &= \arctan(-\frac{{\omega} L}{R})
\end{align*}
Therefore the voltage across the resistor has the form
\begin{align*}
    V &= V_0 \frac{R\sqrt{R^2 + {\omega}^2L^2}}{R^2 + {\omega}^2L^2} e^{i{\omega} t + \arctan(-{\omega} L/R)} \\
    V &= V_R e^{i{\omega} t + i\beta} \\
    \Re\{V\} &= V_R \cos({\omega} t + \beta)
\end{align*}
where
\begin{align}
    \boxed{ V_R = V_0 \frac{R\sqrt{R^2 + {\omega}^2L^2}}{R^2 + {\omega}^2L^2} }
    \quad\quad
    \boxed{ \beta = \arctan(-\frac{{\omega} L}{R}) \vphantom{\frac{\sqrt{R}}{R}} }
\end{align}

%%%%%%%%%%%%%%%%%%%%%%%%%%%%%%%%%%%%%%%%%%%%%%%%%%%%%%%%%%%%%%%%%%%%%%%%%%%%%%%
%%%% Problem 4
%%%%%%%%%%%%%%%%%%%%%%%%%%%%%%%%%%%%%%%%%%%%%%%%%%%%%%%%%%%%%%%%%%%%%%%%%%%%%%%
\problem{4}
\subsubsection{Question}
% Keywords
    \index{electrodynamics!Flux through a loop}

Find the magnetic flux through a square loop of side $a$ due to current $I$ in
a long straight wire. The geometry is as follows: the wire is coplanar with the
loop and runs parallel to the loop's closest side, at a distance $b$ away.
Write your result as a formula in SI units.

\subsubsection{Answer}

By Ampère's law in integral form, the magnetic field at a radial distance $r$
away from the wire is given by
\begin{align*}
    \vec B = \frac{{\mu}_0I}{2{\pi} r} \hat \phi 
\end{align*}
The flux is then the total magnetic field through the loop. Integrating by
lines of length $a$,
\begin{align*}
    \phi  = a \int_b^{a+b} \frac{{\mu}_0I}{2{\pi} r} dr \\
    \boxed{\phi  = \frac{{\mu}_0Ia}{2{\pi}}\ln(\frac{a+b}{a})}
\end{align*}

%%%%%%%%%%%%%%%%%%%%%%%%%%%%%%%%%%%%%%%%%%%%%%%%%%%%%%%%%%%%%%%%%%%%%%%%%%%%%%%
%%%% Problem 6
%%%%%%%%%%%%%%%%%%%%%%%%%%%%%%%%%%%%%%%%%%%%%%%%%%%%%%%%%%%%%%%%%%%%%%%%%%%%%%%
\problem{6}
\subsubsection{Question}
% Keywords
    \index{mathematics!Complex contour integration}

By actually evaluating the integral, show that
\begin{align*}
    \int_0^\infty  \frac{\cos x}{1 + x^2}\,dx &= \frac{{\pi}}{2e}
\end{align*}

\subsubsection{Answer}

Note that the integrand, so start by changing the limits of integration
\begin{align*}
    \int_0^\infty  \frac{\cos x}{1 + x^2}\,dx &=
	\frac 12 \int_{-\infty }^\infty  \frac{\cos x}{1 + x^2}\,dx
\intertext{Then expand the cosine into is complex exponential definition}
    {} &= \frac 14 (\int_{-\infty }^\infty  \frac{e^{ix}}{1 + x^2}\,dx +
	\int_{-\infty }^\infty  \frac{e^{-ix}}{1 + x^2}\,dx) \\
    {} &= \frac 14 (\int_{-\infty }^\infty  \frac{e^{ix}}{(x+i)(x-i)}\,dx +
	\int_{-\infty }^\infty  \frac{e^{-ix}}{(x+i)(x-i)}\,dx)
\intertext{Complex contour integration lets us evaluate the integrals as
limits of the coefficient of a pole as it approaches the pole, so}
    {} &= \frac 14 (2{\pi} i \cdot  \lim_{z\rightarrow i}(\frac{e^{iz}}{z+i}) - 2{\pi} i  \cdot 
	\lim_{z\rightarrow -i}(\frac{e^{-iz}}{z-i}))
\end{align*}
After simplifying,
\begin{align}
    \boxed{
    \int_0^\infty  \frac{\cos x}{1 + x^2}\,dx = \frac{{\pi}}{2e}
    }
\end{align}

%%%%%%%%%%%%%%%%%%%%%%%%%%%%%%%%%%%%%%%%%%%%%%%%%%%%%%%%%%%%%%%%%%%%%%%%%%%%%%%
%%%% Problem 7
%%%%%%%%%%%%%%%%%%%%%%%%%%%%%%%%%%%%%%%%%%%%%%%%%%%%%%%%%%%%%%%%%%%%%%%%%%%%%%%
\problem{7}
\subsubsection{Question}
% Keywords
    \index{dimensional analysis!High-velocity drag force}

The drag force on a very high speed object of area $A$, passing through a gas
of density $\rho$ at a velocity $v$ is expected to be of the form
\begin{align*}
    \text{Force} \sim A^r \rho^s v^t
\end{align*}
Determine the value of the exponents $r$, $s$, and $t$.

\subsubsection{Answer}

Force needs to have a unit of inverse time squared, therefore $v$ as the only
variable with a time unit sets $t = 2$. Similarly, $\rho $ is the only one with
a mass term, so we also immediately know that $s = 1$. That leaves
\begin{align*}
    \left[ \frac{\text{mass}\cdot \text{distance}}{\text{time}^2} \right]
	&= \left[ \frac{\text{mass}}{\text{distance}\cdot \text{time}^2} \right]
	\left[ \text{distance} \right]^r
\end{align*}
Therefore to have the two side have compatible units, $r = 2$.
\begin{align*}
    \boxed{ r = 2,\quad s = 1,\quad t = 2}
\end{align*}

%%%%%%%%%%%%%%%%%%%%%%%%%%%%%%%%%%%%%%%%%%%%%%%%%%%%%%%%%%%%%%%%%%%%%%%%%%%%%%%
%%%% Problem 9
%%%%%%%%%%%%%%%%%%%%%%%%%%%%%%%%%%%%%%%%%%%%%%%%%%%%%%%%%%%%%%%%%%%%%%%%%%%%%%%
\problem{9}
\subsubsection{Question}
% Keywords
    \index{mechanics!Shallow-water wave group velocity}
    \index{waves!Shallow-water wave group velocity}

For waves in shallow water, the relation between frequency $\nu $ and wavelength
${\lambda}$ is
\begin{align*}
    \nu  &= (\frac{2{\pi} T}{\rho {\lambda}^3 })^{1/2}
\end{align*}
where $\rho $ and $T$ are the density and surface tension of water. What is the
group velocity of these waves?

\subsubsection{Answer}

Transforming relation given to use the angular frequency ${\omega} = 2{\pi}\nu $ and
wave number $k = 2{\pi}/{\lambda}$,
\begin{align*}
    {\omega} &= \sqrt{ \frac{k^3 T}{\rho } }
\end{align*}
Then the group velocity is simply the partial derivative with respect to $k$:
\begin{align*}
    v_g &= \frac{\partial {\omega}}{\partial k} = \frac 32 \sqrt{\frac{kT}{\rho }}
\end{align*}
Putting this back into the form which involves only $\nu $ and ${\lambda}$, we arrive at
the answer
\begin{align}
    \boxed{ v_g = \frac 32 \sqrt{\frac{2{\pi} T}{\rho {\lambda}}} }
\end{align}

%%%%%%%%%%%%%%%%%%%%%%%%%%%%%%%%%%%%%%%%%%%%%%%%%%%%%%%%%%%%%%%%%%%%%%%%%%%%%%%
%%%% Problem 10
%%%%%%%%%%%%%%%%%%%%%%%%%%%%%%%%%%%%%%%%%%%%%%%%%%%%%%%%%%%%%%%%%%%%%%%%%%%%%%%
\problem{10}
\subsubsection{Question}
% Keywords
    \index{mathematics!Eigenvalues and eigenvectors}

Find the eigenvalues and corresponding eigenvectors (which need \emph{not} be
normalized) of the following matrix:
\begin{align*}
    M = \begin{bmatrix}
	1 & 0 & -i \\
	0 & 2 & 0 \\
	i & 0 & -1
    \end{bmatrix}
\end{align*}

\subsubsection{Answer}

The eigenvalue equation is found from the standard procedure of adding a
parameter ${\lambda}$ into the matrix and taking the determinant equal to zero:
\begin{align*}
    \det{\begin{bmatrix}
	1-{\lambda} & 0   & -i   \\
	0   & 2-{\lambda} & 0    \\
	i   & 0   & -1-{\lambda}
    \end{bmatrix}} = 0 &= (1-{\lambda})(2-{\lambda})(-1-{\lambda}) - i(-i)(2-{\lambda}) \\
    0 &= -(1-{\lambda})(1+{\lambda})(2-{\lambda}) - (2-{\lambda}) \\
    0 &= -(2-{\lambda})\qty[ (1-{\lambda})(1+{\lambda}) + 1]
\end{align*}
Solving the two equations $0 = (2 - {\lambda})$ and $(1-{\lambda})(1+{\lambda}) = -1$ gives the three
eigenvalues
\begin{align}
    \boxed{
    {\lambda} = \left\{ -\sqrt 2, \sqrt 2, 2 \right\}
    }
\end{align}

Starting with the eigenvalue ${\lambda}=2$, we solve for its eigenvector using the
usual Gaussian elimination approach:
\begin{align*}
    \det{\begin{bmatrix}
	-1  & 0   & -i   \\
	0   & 0   & 0    \\
	i   & 0   & -3
    \end{bmatrix}}
\end{align*}
The second variable is completely unconstrained, so we can set that component
of the vector to 1. The first and last rows are incompatible, so that means
that both the first and third variables must be zero. This gives us the
eigenvector
\begin{align*}
    v_3 &= \begin{bmatrix} 0 \\ 1 \\ 0 \end{bmatrix}
\end{align*}
In a similar manner for the eigenvalues ${\lambda} = \pm\sqrt 2$, we perform Gaussian
elimination. When all free parameters have been set arbitrarily to 1 and other
constraints considered, we end up with the three eigenvectors:
\begin{align}
    \boxed{ {\lambda} = -\sqrt 2 \quad\rightarrow\quad v_1 =
	\begin{bmatrix} 1 \\ 0 \\ -i(1+\sqrt 2) \end{bmatrix} }
\end{align}
\begin{align}
    \boxed{ {\lambda} = \sqrt2 \quad\rightarrow\quad v_2 =
	\begin{bmatrix} 1 \\ 0 \\ -i(1-\sqrt 2) \end{bmatrix} }
\end{align}
\begin{align}
    \boxed{ {\lambda} = 2 \quad\rightarrow\quad v_3 =
	\begin{bmatrix} 0 \\ 1 \\ 0 \end{bmatrix} }
\end{align}

%%%%%%%%%%%%%%%%%%%%%%%%%%%%%%%%%%%%%%%%%%%%%%%%%%%%%%%%%%%%%%%%%%%%%%%%%%%%%%%
%%%% Problem 12
%%%%%%%%%%%%%%%%%%%%%%%%%%%%%%%%%%%%%%%%%%%%%%%%%%%%%%%%%%%%%%%%%%%%%%%%%%%%%%%
\problem{12}
\subsubsection{Question}
% Keywords
    \index{quantum!Spin-$\frac 32$ electron}

Suppose the electron were to have spin $\frac 32$ instead of spin $\frac 12$.
What would then be the atomic numbers $Z$ of the \emph{three} lowest-mass
noble gases, i.e.~the equivalents of helium, neon, and argon?

\subsubsection{Answer}

In a spin $\frac 32$ particle, there are 4 possible spin configurations
corresponding to the spin projections $s_z = \{ -\frac 32, -\frac 12, \frac 12,
\frac 32\}$. Because of this, each projection of the orbital angular momentum
can hold 4 electrons instead of just two. This means we can make use of
spectroscopic notation to easily count up to the atoms of interest.
\begin{align*}
    \ell  &= 0 \rightarrow  \text{1 $\ell_z$ state}	& & 1{s^4} \\
    \ell  &= 1 \rightarrow  \text{3 $\ell_z$ states} & & 1{s^4}2{p^{12}}2s^4 \\
    \ell  &= 2 \rightarrow  \text{5 $\ell_z$ states} & & 1{s^4}2{p^{12}}2{s^4}3d^{20}3{p^{12}}3s^4
\end{align*}

Since all shells are filled at each level, we just have to sum the number of
electrons in each line above to get the $Z$ number of the new noble gases.
\begin{align}
    \boxed{ Z = \{4, 20, 56 \} }
\end{align}
