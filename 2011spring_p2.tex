%%%%%%%%%%%%%%%%%%%%%%%%%%%%%%%%%%%%%%%%%%%%%%%%%%%%%%%%%%%%%%%%%%%%%%%%%%%%%%%
%%%% Problem 1
%%%%%%%%%%%%%%%%%%%%%%%%%%%%%%%%%%%%%%%%%%%%%%%%%%%%%%%%%%%%%%%%%%%%%%%%%%%%%%%
%\subsection{Problem 1}
\problem{1}
\subsubsection{Question}
% Keywords
	\index{unsolved!Spring 2011 II.P1}

A pendulum, which consists of a light rod of length $l$ and a bob of mass $m$, is attached to the ceiling. A second identical pendulum is attached to the bob of the first. Denote the angle each rod makes with respect to the vertical as $\theta_1$ and $\theta_2$, respectively (see Figure 1). Assume that both angles are small for all time $t$. Find the relative amplitudes and phases of $\theta_1(t)$ and $\theta_2(t)$ when the system is oscillating in the normal modes, as well as the corresponding frequencies.

\subsubsection{Answer}


%%%%%%%%%%%%%%%%%%%%%%%%%%%%%%%%%%%%%%%%%%%%%%%%%%%%%%%%%%%%%%%%%%%%%%%%%%%%%%%
%%%% Problem 2
%%%%%%%%%%%%%%%%%%%%%%%%%%%%%%%%%%%%%%%%%%%%%%%%%%%%%%%%%%%%%%%%%%%%%%%%%%%%%%%
%\subsection{Problem 2}
\problem{2}
\subsubsection{Question}
% Keywords
	\index{unsolved!Spring 2011 II.P2}
In a region with a uniform vertically upward magnetic field of magnitude $B$, two long straight conducting rails with negligible resistance are set up in a horizontal plane with distance $l$ between them (see Figure 2). Their left ends are connected to a capacitor of capacitance $C$. A metal rod of mass $m$ and resistance $R$ can move on the rails without friction. At time $t = 0$, there is no charge on the capacitor and the rod is released with a velocity $v_0$ to the right.
\begin{enumerate}
	\item Find the charge on the capacitor as a function of $t$.
	\item Find the terminal velocity of the rod.
	\item Show that energy is conserved by considering changes between the initial and terminal states.
\end{enumerate}

\subsubsection{Answer}



%%%%%%%%%%%%%%%%%%%%%%%%%%%%%%%%%%%%%%%%%%%%%%%%%%%%%%%%%%%%%%%%%%%%%%%%%%%%%%%
%%%% Problem 3
%%%%%%%%%%%%%%%%%%%%%%%%%%%%%%%%%%%%%%%%%%%%%%%%%%%%%%%%%%%%%%%%%%%%%%%%%%%%%%%
%\subsection{Problem 3}
\problem{3}
\subsubsection{Question}
% Keywords
	\index{unsolved!Spring 2011 II.P3}
\begin{enumerate}
	\item A non-relativistic particle of mass m is in the ground state of the potential
	\begin{equation*}
		V(x)=\begin{cases}
			0, & 0< x< a\\
			\infty, &\text{elsewhere}.
		\end{cases}
	\end{equation*}
	At some point of time, the potential suddenly changes to
	\begin{equation*}
		V(x)=\begin{cases}
			0, & 0< x< 2a\\
			\infty, &\text{elsewhere}.
		\end{cases}
	\end{equation*}
	Some time later, a measurement of the energy of the particle is made. Find the probability that the particle will be measured to have the ground-state energy for the new potential.
	\item Do the same for the three-dimensional case where the initial potential is
	\begin{equation}
		V(x)=\begin{cases}
			0, & 0< r< a\\
			\infty, &\text{elsewhere}.
		\end{cases}
	\end{equation}
	and the new potential is
	\begin{equation}
		V(x)=\begin{cases}
			0, & 0< r< a\\
			\infty, &\text{elsewhere}.
		\end{cases}
	\end{equation}
	It is suggested that you (a) derive or write down the radial part of the Schr\"odinger equation that needs to be satisfied, and (b) note that the equation only needs to be solved for the ground state.
\end{enumerate}

\subsubsection{Answer}



%%%%%%%%%%%%%%%%%%%%%%%%%%%%%%%%%%%%%%%%%%%%%%%%%%%%%%%%%%%%%%%%%%%%%%%%%%%%%%%
%%%% Problem 4
%%%%%%%%%%%%%%%%%%%%%%%%%%%%%%%%%%%%%%%%%%%%%%%%%%%%%%%%%%%%%%%%%%%%%%%%%%%%%%%
%\subsection{Problem 4}
\problem{4}
\subsubsection{Question}
% Keywords
	\index{unsolved!Spring 2011 II.P4}

Consider a single electron in the Coulomb potential of a nucleus with proton number $Z$. Assume that the electric charge of the nucleus is distributed uniformly within a sphere of radius $R$.
\begin{enumerate}
	\item Calculate the Coulomb potential energy $V (r)$ for the electron as a function of radius $r$ (with the origin at the center of the nucleus).
	\item The ground-state wave function for the electron in a hydrogen-like atom with a pointlike nucleus is $\psi(\mathbf{r})\propto\exp(-r/a)$, where $a$ is a constant. Use this information to make a leading-order estimate of the shift in the ground-state energy due to the Coulomb interaction between the electron and a nucleus of finite $R$ (relative to the case of a point-like nucleus).
	\item Evaluate the energy shift in (2) for $Z = 81$ and $R = 7$ fm.
\end{enumerate}

\subsubsection{Answer}


%%%%%%%%%%%%%%%%%%%%%%%%%%%%%%%%%%%%%%%%%%%%%%%%%%%%%%%%%%%%%%%%%%%%%%%%%%%%%%%
%%%% Problem 5
%%%%%%%%%%%%%%%%%%%%%%%%%%%%%%%%%%%%%%%%%%%%%%%%%%%%%%%%%%%%%%%%%%%%%%%%%%%%%%%
%\subsection{Problem 5}
\problem{5}
\subsubsection{Question}
% Keywords
	\index{unsolved!Spring 2011 II.P5}
	\index{statistical mechanics!Ideal Spinless Gas}
An ideal gas of $N$ spinless atoms has volume $V$, temperature $T$, and partition function $Z_0$ .
\begin{enumerate}
	\item By assuming that the atoms obey Maxwell-Boltzmann statistics, find an expression for $Z_0$ .
	\item Now consider that each atom has two internal energy levels with energy $\epsilon$ and $\epsilon+\Delta$, respectively. Find the new partition function.
	\item Calculate the specific heat at constant volume for the case in (2).
\end{enumerate}

\subsubsection{Answer}

