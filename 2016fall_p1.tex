%%%%%%%%%%%%%%%%%%%%%%%%%%%%%%%%%%%%%%%%%%%%%%%%%%%%%%%%%%%%%%%%%%%%%%%%%%%%%%%
%%%% Problem 1
%%%%%%%%%%%%%%%%%%%%%%%%%%%%%%%%%%%%%%%%%%%%%%%%%%%%%%%%%%%%%%%%%%%%%%%%%%%%%%%
%\subsection{Problem 1}
\problem{1}
\subsubsection{Question}
% Keywords
	\index{unsolved!Fall 2016 I.P1}

A ball of mass $m$ move with no friction along half a circle of center $O$ and radius $R$ (figure below). $AB$ is the diameter of the circle, $O_x$ is the horizontal axis and $O_y$ the vertical axis pointing down. We call $\theta$ the angle between $O_x$ and $OM$. The ball is attached to a spring of elastic constant $k$, attached to $B$ on the other side. The ball moves along the arc of circle.

1. Determine the potential energy $E_p^* = E_p/E_0$ (with $E_0 = mgR$) as a function of $\theta$ and $p$, where $p$ is function of $k$, $R$, $g$ and $m$.

2. Determine the equilibrium positions(s) $\theta_e$ of the system. Draw the plot $E_p^*(\theta)$ as a function of $\theta$ for $p = 2$. Determine $\theta_e$ for $p = 2$.
\subsubsection{Answer}


%%%%%%%%%%%%%%%%%%%%%%%%%%%%%%%%%%%%%%%%%%%%%%%%%%%%%%%%%%%%%%%%%%%%%%%%%%%%%%%
%%%% Problem 2
%%%%%%%%%%%%%%%%%%%%%%%%%%%%%%%%%%%%%%%%%%%%%%%%%%%%%%%%%%%%%%%%%%%%%%%%%%%%%%%
%\subsection{Problem 2}
\problem{2}
\subsubsection{Question}
% Keywords
	\index{unsolved!Fall 2016 I.P2}
A hollow cylinder of mass $m$ and radius $a$ rolls without slipping down a movable wedge of mass $M$. The angle of the wedge relative to the horizontal surface is $\alpha$, and the wedge is free to slide on this smooth horizontal surface. The contact between the cylinder and the wedge is perfectly rough. Find the acceleration of the wedge.
\subsubsection{Answer}



%%%%%%%%%%%%%%%%%%%%%%%%%%%%%%%%%%%%%%%%%%%%%%%%%%%%%%%%%%%%%%%%%%%%%%%%%%%%%%%
%%%% Problem 3
%%%%%%%%%%%%%%%%%%%%%%%%%%%%%%%%%%%%%%%%%%%%%%%%%%%%%%%%%%%%%%%%%%%%%%%%%%%%%%%
%\subsection{Problem 3}
\problem{3}
\subsubsection{Question}
% Keywords
	\index{unsolved!Fall 2016 I.P3}

A cylinder with adiabatic walls (i.e., thermally insulated walls) which is closed at both ends is initially divided into two equal volumes by a frictionless piston that is also thermally insulating. Initially the volume, pressure and temperature of the ideal gas in each side of the cylinder are $V_0, p_0,$ and $T_0$ respectively. A heater in the right-hand volume is used to slowly heat the gas on that side until the pressure there reaches $64p_0/27$. If the heat capacity $C_V$ of the gas is independent of temperature and $C_P/C_V = \gamma = 1.5$, find the following in terms of $V_0, p_0,$ and $T_0$:
\begin{enumerate}
	\item The entropy change of the gas on the left.
	\item The final left-hand volume.
	\item The final left-hand temperature.
\end{enumerate}
\subsubsection{Answer}



%%%%%%%%%%%%%%%%%%%%%%%%%%%%%%%%%%%%%%%%%%%%%%%%%%%%%%%%%%%%%%%%%%%%%%%%%%%%%%%
%%%% Problem 4
%%%%%%%%%%%%%%%%%%%%%%%%%%%%%%%%%%%%%%%%%%%%%%%%%%%%%%%%%%%%%%%%%%%%%%%%%%%%%%%
%\subsection{Problem 4}
\problem{4}
\subsubsection{Question}
% Keywords
	\index{unsolved!Fall 2016 I.P4}
In many cases, graphene can be modeled as a two-dimensional gas of non-interacting electrons with energy $\epsilon(\boldsymbol{k}) = \hbar\nu\abs{\boldsymbol{k}}$, where $\hbar\abs{\boldsymbol{k}}$ is the momentum and $\nu$ is the effective velocity. Each state is fourfold degenerate due to the spin and the so-called valley degrees of freedom. Consider a positive chemical potential corresponding to an electronic density $n$. Calculate the ratio between the average energy per particle and the Fermi energy of the system at zero temperature.
\subsubsection{Answer}


%%%%%%%%%%%%%%%%%%%%%%%%%%%%%%%%%%%%%%%%%%%%%%%%%%%%%%%%%%%%%%%%%%%%%%%%%%%%%%%
%%%% Problem 5
%%%%%%%%%%%%%%%%%%%%%%%%%%%%%%%%%%%%%%%%%%%%%%%%%%%%%%%%%%%%%%%%%%%%%%%%%%%%%%%
%\subsection{Problem 5}
\problem{5}
\subsubsection{Question}
% Keywords
	\index{unsolved!Fall 2016 I.P5}
A very long wire of radius $a$ is suspended a distance $d$ above an infinite conducting plane. The wire is uniformly charged with uniform charge density $\lambda$. In the case that $d\gg a$, find approximate expression for:
\begin{enumerate}
	\item The capacitance per unit length of the wire, conducting plane system.
	\item The surface charge density on the conducting plane as a function of $y$, the distance along the plane lateral to the wire.
\end{enumerate}

\subsubsection{Answer}



%%%%%%%%%%%%%%%%%%%%%%%%%%%%%%%%%%%%%%%%%%%%%%%%%%%%%%%%%%%%%%%%%%%%%%%%%%%%%%%
%%%% Problem 6
%%%%%%%%%%%%%%%%%%%%%%%%%%%%%%%%%%%%%%%%%%%%%%%%%%%%%%%%%%%%%%%%%%%%%%%%%%%%%%%
%\subsection{Problem 6}
\problem{6}
\subsubsection{Question}
% Keywords
	\index{unsolved!Fall 2016 I.P6}
A thin rod of length $a$, mass $m$, and resistance $R$ moves vertically with no friction and closes an electrical circuit with an inductance $L$. In this problem, we consider that the total resistance of the circuit is $R$ and the total inductance of the circuit is $L$. A magnetic field $\boldsymbol{B}$ is applied perpendicularly to the circuit (out of page, the figure below). The rod is dropped at $t = 0$ with no initial velocity.

1. Determine the differential equation of the current in the rod.

2. If $R = 0$, determine the current intensity $i(t)$ and the speed of the rod $v(t)$.
\subsubsection{Answer}

%%%%%%%%%%%%%%%%%%%%%%%%%%%%%%%%%%%%%%%%%%%%%%%%%%%%%%%%%%%%%%%%%%%%%%%%%%%%%%%
%%%% Problem 7
%%%%%%%%%%%%%%%%%%%%%%%%%%%%%%%%%%%%%%%%%%%%%%%%%%%%%%%%%%%%%%%%%%%%%%%%%%%%%%%
%\subsection{Problem 7}
\problem{7}
\subsubsection{Question}
% Keywords
	\index{unsolved!Fall 2016 I.P7}
Consider a thin divergent lens with a focal distance $f^\prime = -30$cm.
\begin{enumerate}
	\item Determine the distance between the lens and the virtual image of a point $A$ located $30$cm in front of the lens.
	\item If a vertical object AB of height $1$mm is placed at point A, what is the height of its virtual image?
\end{enumerate}
\subsubsection{Answer}



%%%%%%%%%%%%%%%%%%%%%%%%%%%%%%%%%%%%%%%%%%%%%%%%%%%%%%%%%%%%%%%%%%%%%%%%%%%%%%%
%%%% Problem 8
%%%%%%%%%%%%%%%%%%%%%%%%%%%%%%%%%%%%%%%%%%%%%%%%%%%%%%%%%%%%%%%%%%%%%%%%%%%%%%%
%\subsection{Problem 8}
\problem{8}
\subsubsection{Question}
% Keywords
	\index{unsolved!Fall 2016 I.P8}
A particle is confined between two planes at $x = 0$ and $x = L$ in an infinite well. The wave function is: $$\psi(x,t) = A\sin(kx) e^{-i\omega t}$$
\begin{enumerate}
	\item Determine the possible values for k as a function of L and a positive integer n.
	\item Find A as a function of L.
	\item Draw $\abs{\psi(x,t)}^2$ as a function of $x$ in the case of $n=1$ and $n=2$.
\end{enumerate}
\subsubsection{Answer}


%%%%%%%%%%%%%%%%%%%%%%%%%%%%%%%%%%%%%%%%%%%%%%%%%%%%%%%%%%%%%%%%%%%%%%%%%%%%%%%
%%%% Problem 9
%%%%%%%%%%%%%%%%%%%%%%%%%%%%%%%%%%%%%%%%%%%%%%%%%%%%%%%%%%%%%%%%%%%%%%%%%%%%%%%
%\subsection{Problem 9}
\problem{9}
\subsubsection{Question}
% Keywords
	\index{unsolved!Fall 2016 I.P9}
A particle of mass $m$ is in the ground state in one-dimensional box of length $L$ with impenetrable walls. Find the distribution of the probability $P$ of the momentum $p$. (This means finding the function $\rho(p)$ such that the probability $\dd P$ of measuring the momentum in the differential interval $\dd p$ is given by $\dd P = \rho(p) \dd p$).
\subsubsection{Answer}


%%%%%%%%%%%%%%%%%%%%%%%%%%%%%%%%%%%%%%%%%%%%%%%%%%%%%%%%%%%%%%%%%%%%%%%%%%%%%%%
%%%% Problem 10
%%%%%%%%%%%%%%%%%%%%%%%%%%%%%%%%%%%%%%%%%%%%%%%%%%%%%%%%%%%%%%%%%%%%%%%%%%%%%%%
%\subsection{Problem 10}
\problem{10}
\subsubsection{Question}
% Keywords
	\index{unsolved!Fall 2016 I.P10}
A $\tau$ lepton decays into muon $(\mu^-)$, muon antineutrino $(\tilde{\mu}_\mu)$ and $\tau$ neutrino $(\nu_\tau)$. What is the maximal possible momentum of the muon (in MeV/c$^2$) in the rest frame of the $\tau$ lepton. The mass of $\tau^-$ is M$ = 1777$MeV/c$^2$, the mass of the muon $m = 106$ MeV/c$^2$ and both neutrinos are very light, so that their masses can be neglected.
\subsubsection{Answer}

