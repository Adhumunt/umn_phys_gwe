%%%%%%%%%%%%%%%%%%%%%%%%%%%%%%%%%%%%%%%%%%%%%%%%%%%%%%%%%%%%%%%%%%%%%%%%%%%%%%%
%%%% Problem 1
%%%%%%%%%%%%%%%%%%%%%%%%%%%%%%%%%%%%%%%%%%%%%%%%%%%%%%%%%%%%%%%%%%%%%%%%%%%%%%%
%\subsection{Problem 1}
\problem{1}
\subsubsection{Question}
% Keywords
	\index{unsolved!Fall 2018 I.P1}
A function $\Psi(x,t)$ is a solution to the Schroedinger equation for a potential $V(x)$:
\begin{equation*}
	i\hbar\pdv{\Psi(x,t)}{t} = \qty[-\frac{\hbar^2}{2m}\pdv[2]{x} + V(x)]\Psi(x,t)
\end{equation*}
Consider now the potential $V^\prime(x) = V(x)+V_0$ where $V_0$ is a constant.
\begin{enumerate}
	\item Is $\Psi(x,t)$ a solution of $(1)$ with the potential $V(x)$ replaced by $V^\prime(x)$? If not, can it be modified so that it is a solution?
	\item Compare with the situation in classical mechanics.
	\item Discuss whether the above results would have experimental consequences. 
\end{enumerate}

\subsubsection{Answer}


%%%%%%%%%%%%%%%%%%%%%%%%%%%%%%%%%%%%%%%%%%%%%%%%%%%%%%%%%%%%%%%%%%%%%%%%%%%%%%%
%%%% Problem 2
%%%%%%%%%%%%%%%%%%%%%%%%%%%%%%%%%%%%%%%%%%%%%%%%%%%%%%%%%%%%%%%%%%%%%%%%%%%%%%%
%\subsection{Problem 2}
\problem{2}
\subsubsection{Question}
% Keywords
	\index{unsolved!Fall 2018 I.P2}
Consider the Earth-Sun system as a quantum gravitational analog of a hydrogen atom.
\begin{enumerate}
	\item What would be, in actual units, the Bohr radius of the system?
	\item By equating the actual energy of the system to the Bohr formula, estimate the quantum number $n$ of the Earth.
\end{enumerate}

\subsubsection{Answer}



%%%%%%%%%%%%%%%%%%%%%%%%%%%%%%%%%%%%%%%%%%%%%%%%%%%%%%%%%%%%%%%%%%%%%%%%%%%%%%%
%%%% Problem 3
%%%%%%%%%%%%%%%%%%%%%%%%%%%%%%%%%%%%%%%%%%%%%%%%%%%%%%%%%%%%%%%%%%%%%%%%%%%%%%%
%\subsection{Problem 3}
\problem{3}
\subsubsection{Question}
% Keywords
	\index{unsolved!Fall 2018 I.P3}
A particle moves in one dimension under the action of a quadratic velocity dependent retarding force. (Assume the initial velocity is $v_0$ , and the motion is horizontal.)

a) Derive the expression for the position as a function of time.

b) Does this particle ever stop (i.e., at $t = \infty$ is the distance traveled finite)?
\subsubsection{Answer}



%%%%%%%%%%%%%%%%%%%%%%%%%%%%%%%%%%%%%%%%%%%%%%%%%%%%%%%%%%%%%%%%%%%%%%%%%%%%%%%
%%%% Problem 4
%%%%%%%%%%%%%%%%%%%%%%%%%%%%%%%%%%%%%%%%%%%%%%%%%%%%%%%%%%%%%%%%%%%%%%%%%%%%%%%
%\subsection{Problem 4}
\problem{4}
\subsubsection{Question}
% Keywords
	\index{unsolved!Fall 2018 I.P4}
A bullet of mass $m$ and velocity $v_0$ strikes a target object of mass $M$, which is resting on a frictionless support. If the bullet emerges with velocity $v_0/2$, find the fraction of the initial kinetic energy which is deposited into the target as frictional heat in terms of the ratio of bullet mass to target mass $\gamma = m/M$.
\subsubsection{Answer}


%%%%%%%%%%%%%%%%%%%%%%%%%%%%%%%%%%%%%%%%%%%%%%%%%%%%%%%%%%%%%%%%%%%%%%%%%%%%%%%
%%%% Problem 5
%%%%%%%%%%%%%%%%%%%%%%%%%%%%%%%%%%%%%%%%%%%%%%%%%%%%%%%%%%%%%%%%%%%%%%%%%%%%%%%
%\subsection{Problem 5}
\problem{5}
\subsubsection{Question}
% Keywords
	\index{unsolved!Fall 2018 I.P5}
Two conducting metal objects are embedded in a weakly conducting material of conductivity $\sigma$.
\begin{enumerate}
	\item Show that the resistance between them is related to the capacitance by $R = \epsilon_0 /\sigma C$.
	\item Suppose that you connect the two objects with a battery so that the potential difference is $V_0$. Show that when you disconnect the battery, the potential decreases exponentially and determine the time constant.
\end{enumerate}
\subsubsection{Answer}



%%%%%%%%%%%%%%%%%%%%%%%%%%%%%%%%%%%%%%%%%%%%%%%%%%%%%%%%%%%%%%%%%%%%%%%%%%%%%%%
%%%% Problem 6
%%%%%%%%%%%%%%%%%%%%%%%%%%%%%%%%%%%%%%%%%%%%%%%%%%%%%%%%%%%%%%%%%%%%%%%%%%%%%%%
%\subsection{Problem 6}
\problem{6}
\subsubsection{Question}
% Keywords
	\index{unsolved!Fall 2018 I.P6}
A satellite in geostationary orbit is used to transmit data via electromagnetic radiation. The satellite is at a height of $35,000$ km above the surface of the earth, and we assume it has an isotropic power output of $1$ kW.

a) What is the amplitude $E_0$ of the electric field vector of the satellite broadcast as measured at the surface of the earth?

b) A receiving dish of (projected) radius $R$ focuses the electromagnetic energy incident from the satellite onto a receiver which has a surface area of $5$ cm$^2$. How large does the radius $R$ need to be to achieve an electric field vector amplitude of $0.1$ mV/m at the receiver?
\subsubsection{Answer}

%%%%%%%%%%%%%%%%%%%%%%%%%%%%%%%%%%%%%%%%%%%%%%%%%%%%%%%%%%%%%%%%%%%%%%%%%%%%%%%
%%%% Problem 7
%%%%%%%%%%%%%%%%%%%%%%%%%%%%%%%%%%%%%%%%%%%%%%%%%%%%%%%%%%%%%%%%%%%%%%%%%%%%%%%
%\subsection{Problem 7}
\problem{7}
\subsubsection{Question}
% Keywords
	\index{unsolved!Fall 2018 I.P7}
Consider a one-particle system capable of three states $(\epsilon_n = n\cdot \Delta, n = 0, 1, 2)$ in thermal contact with a reservoir at temperature ${\tau}$. In the limits ${\tau}/{\Delta} {\to} {\infty}$ and ${\tau}/{\Delta} {\to}$ 0, find the
\begin{enumerate}
	\item Energy.
	\item Free Energy.
	\item Heat Capacity. 
\end{enumerate}
\subsubsection{Answer}



%%%%%%%%%%%%%%%%%%%%%%%%%%%%%%%%%%%%%%%%%%%%%%%%%%%%%%%%%%%%%%%%%%%%%%%%%%%%%%%
%%%% Problem 8
%%%%%%%%%%%%%%%%%%%%%%%%%%%%%%%%%%%%%%%%%%%%%%%%%%%%%%%%%%%%%%%%%%%%%%%%%%%%%%%
%\subsection{Problem 8}
\problem{8}
\subsubsection{Question}
% Keywords
	\index{unsolved!Fall 2018 I.P8}
Consider a non-interacting gas of $N {\gg} 1$ ${}^4He$ atoms at temperature ${\tau}$ in a volume $V$ at pressure $p$. If half of the atoms are in the ground state $({\epsilon} = 0)$, find the chemical potential $\mu$.



\subsubsection{Answer}


%%%%%%%%%%%%%%%%%%%%%%%%%%%%%%%%%%%%%%%%%%%%%%%%%%%%%%%%%%%%%%%%%%%%%%%%%%%%%%%
%%%% Problem 9
%%%%%%%%%%%%%%%%%%%%%%%%%%%%%%%%%%%%%%%%%%%%%%%%%%%%%%%%%%%%%%%%%%%%%%%%%%%%%%%
%\subsection{Problem 9}
\problem{9}
\subsubsection{Question}
% Keywords
	\index{unsolved!Fall 2018 I.P9}
A radioactive nucleus in an excited state decays (at rest) with a lifetime of ${\tau} = 10^{-7}$s to its ground state through the emission of a gamma ray of energy $E = 15$ keV.

a) What is the wavelength of the photon emitted in this decay (in nm)?

b) What is the natural line width of the excited level (in eV)?

c) What is the length of the photon wave train (in meters)?
\subsubsection{Answer}


%%%%%%%%%%%%%%%%%%%%%%%%%%%%%%%%%%%%%%%%%%%%%%%%%%%%%%%%%%%%%%%%%%%%%%%%%%%%%%%
%%%% Problem 10
%%%%%%%%%%%%%%%%%%%%%%%%%%%%%%%%%%%%%%%%%%%%%%%%%%%%%%%%%%%%%%%%%%%%%%%%%%%%%%%
%\subsection{Problem 10}
\problem{10}
\subsubsection{Question}
% Keywords
	\index{unsolved!Fall 2018 I.P10}
The laboratory differential cross section for proton scattering in a certain process is $d{\sigma}/d{\Omega} = a + b\cos^2(\theta)$ with $a$ and $b$ constants with the numerical values $a = 420\mu$b/sterad and $b = 240\mu$b/sterad. Note that $1$ barn $(b) = 10^{-24}$ cm$^2$.

a) What is the total cross section (numerical value)?

b) A counter mounted at $10$ cm from the target at an angle of $60^\circ$ has an effective area of $0.1$ cm$^2$. The target thickness is $10^{-4}$ cm and the number of atoms per cm$^3$ in the target is $10^{22}$. What is the counting rate, when the beam current is $1\mu$A?
\subsubsection{Answer}

