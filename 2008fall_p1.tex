%%%%%%%%%%%%%%%%%%%%%%%%%%%%%%%%%%%%%%%%%%%%%%%%%%%%%%%%%%%%%%%%%%%%%%%%%%%%%%%
%%%% Problem 2
%%%%%%%%%%%%%%%%%%%%%%%%%%%%%%%%%%%%%%%%%%%%%%%%%%%%%%%%%%%%%%%%%%%%%%%%%%%%%%%
\problem{2}
\subsubsection{Question}
% Keywords
	\index{mechanics!Impulse on a rod}

If an impulse is delivered to the end of a uniform rod of length $\ell $, lying on
a frictionless plane, how far will it travel while making one revolution? The
impulse is in the plane of the table and perpendicular to the rod.

\subsubsection{Answer}

For a given impulse $\vec J$, the change in the motion is $\vec J = \Delta \vec p$.
If the rod start at rest, then the final momentum must be $\vec p = \vec J$.
This means the rod is moving laterally with a velocity
\begin{align*}
    V = \frac 1m \vec J
\end{align*}
which when integrated over a time $t$ gives the distance it has moved $\vec x$.
\begin{align*}
    \vec x = \frac 1m \vec J t
\end{align*}

The impulse also imparts a rotation on the rod because the force was not
applied at the rod's center of mass. The torque $\vec \tau$ relates the force
to the angular momentum $\vec L$ by
\begin{align*}
    \vec r \times  \vec F &= \vec \tau = \dot{\vec L}
\end{align*}
Integrating both sides of the equation, we can write the equation in terms of
the given impulse:
\begin{align*}
    \vec r \times  \int \vec F \,dt &= \int \dot{\vec L} \,dt \\
    \vec r \times  \vec J &= \Delta \vec L
\end{align*}
Again, since the rod starts at rest, we know that the final angular momentum
must be
\begin{align*}
    \vec{L} = \vec{r} \times \vec{J}
\end{align*}
The rotation about the rod's center of mass  occurs at a rate $\vec {\omega}$
dependent on the moment of inertia $I = \frac{1}{12} m\ell ^2$, so
\begin{align*}
    \vec {\omega} = \frac{12}{m\ell^2} \vec{r} \times \vec{J}
\end{align*}
We know that the impulse is applied perpendicular to the rod, so we can easily
integrate the expression in time and solve for the time it takes to revolve
$2{\pi}$ radians:
\begin{align*}
    \theta  = 2{\pi} &= \frac{12}{m\ell ^2} rJt \\
    t &= \frac{{\pi} m\ell ^2}{6 rJ}
\end{align*}

Plugging this back into the linear motion equation, the rod travels
\begin{align*}
    \vec x = \frac{1}{m} \vec J \cdot  \frac{{\pi} m\ell ^2}{6 rJ}
\end{align*}
where we can set $r = \frac 12 \ell $ and therefore simplifies to
\begin{align}
    \boxed{
    \vec x = \frac{{\pi}\ell }{3} \hat J
    }
\end{align}
where $\hat J$ is the direction of the applied impulse.

%%%%%%%%%%%%%%%%%%%%%%%%%%%%%%%%%%%%%%%%%%%%%%%%%%%%%%%%%%%%%%%%%%%%%%%%%%%%%%%
%%%% Problem 3
%%%%%%%%%%%%%%%%%%%%%%%%%%%%%%%%%%%%%%%%%%%%%%%%%%%%%%%%%%%%%%%%%%%%%%%%%%%%%%%
\problem{3}
\subsubsection{Question}
% Keywords
	\index{electrodynamics!Properties of a magnetic field}

A time-independent magnetic field is given by $\vec B = 2bxy \,\hat i  +
ay^2 \,\hat j$.
\begin{enumerate}[a)]
    \item What is the relationship between the constants $a$ and $b$?
    \item Determine the steady current density $J$ that gives rise to this field.
\end{enumerate}

\subsubsection{Answer}
For part (a), we realize that all magnetic fields must be divergence-less.
Therefore we can find the requirements on the constants $a$ and $b$ by
constraining the divergence to be zero.
\begin{align*}
    \vec \nabla \cdot \vec B = 0 &= \frac{\partial }{\partial x}(2bxy) + \frac{\partial }{\partial y}(ay^2) \\
    0 &= 2by + 2ay \\
    b &= -a
\end{align*}
Therefore the relation between the constants is that
\begin{align}
    \boxed {b = -a}
\end{align}

For the second part, we make use of Maxwell's equations. Assuming that none of
the field is due to a time-varying electric field, we make use of
\begin{align*}
    \vec \nabla  \times  \vec B &= {\mu}_0 \vec J
\end{align*}
to calculate the current that generates the field. Doing so, we find that the
solution is
\begin{align}
    \boxed{ \vec J = \frac{2a}{{\mu}_0} x \,\hat k }
\end{align}

%%%%%%%%%%%%%%%%%%%%%%%%%%%%%%%%%%%%%%%%%%%%%%%%%%%%%%%%%%%%%%%%%%%%%%%%%%%%%%%
%%%% Problem 4
%%%%%%%%%%%%%%%%%%%%%%%%%%%%%%%%%%%%%%%%%%%%%%%%%%%%%%%%%%%%%%%%%%%%%%%%%%%%%%%
\problem{4}
\subsubsection{Question}
% Keywords
	\index{electrodynamics!Charges from multipole moments}

A set of four point charges $q_1$, $q_2$, $q_3$, and $q_4$ are arranged
collinearly along the $z$-axis at $z_1 = 0$, $z_2 = a$, $z_3 = 2a$, $z_4 = 4a$,
respectively and the resulting electric field at a distant point $\vec r$ ($r
\gg a$) decays \emph{faster} than $1/r^3 $. Determine the values of $q_1$ and $q_4$
which $q_2 = +2$ and $q_3 = +4$. Units for all charges are Coulombs.

\subsubsection{Answer}

Given that the electric field must fall off faster than $1/r^3 $, this
corresponds to a potential which drops off faster than $1/R^2$. We know that
the monopole moment drops off like $1/r$ and the dipole like $1/R^2$, so we
conclude that the first configuration which could satisfy the given
requirement is that of a quadrupole moment.

Making use of the fact that he monopole and dipole moments are vanishing, we
can use them to generate constraint equations for what the charges must be:
we have two unknown charges and the two equations will allow us to solve them.

For the monopole, the sum of all charges must simply equal zero. Therefore
we immediately know that
\begin{align*}
    0 &= q_1 + q_4 + 6 \\
    -6 &= q_1 + q_4
\end{align*}

The dipole moment (where we take the dipole considered at the origin) is given
by
\begin{align*}
    \vec p = \sum_i \vec{r_i} q_i
\end{align*}
This gives us the equation
\begin{align*}
    0 &= 10a + 4aq_4 \\
    q_4 &= -\frac{5}{2}
\end{align*}
The charge $q_1$ does not show up in the equation since it is located at the
origin. This lets us very simply then solve for $q_1$ as
\begin{align*}
    -6 &= q_1 - \frac{5}{2}
\end{align*}
Therefore, the solution is that the charges have values of
\begin{align}
    \boxed{ q_1 = -\frac{7}{2} } \\
    \boxed{ q_4 = -\frac{5}{2} }
\end{align}

%%%%%%%%%%%%%%%%%%%%%%%%%%%%%%%%%%%%%%%%%%%%%%%%%%%%%%%%%%%%%%%%%%%%%%%%%%%%%%%
%%%% Problem 5
%%%%%%%%%%%%%%%%%%%%%%%%%%%%%%%%%%%%%%%%%%%%%%%%%%%%%%%%%%%%%%%%%%%%%%%%%%%%%%%
\problem{5}
\subsubsection{Question}
% Keywords
	\index{quantum!Spectral emission line width}

The Lyman-${\alpha}$ transition in atomic hydrogen has a wavelength ${\lambda} =
\SI{121.5}{\nm}$, and a transition rate of \SI{0.6e9}{\s^{-1}}. Estimate the
minimum value of $\Delta {\lambda}/{\lambda}$.

\subsubsection{Answer}

We can make an estimate of the spread $\Delta {\lambda}$ by making use of the Heisenberg
uncertainty relation for energy-time. Starting with the variation in
wavelength,
\begin{align*}
    \Delta {\lambda} &= {\lambda} - {\lambda}' \\
    {} &= \frac{hc}{E} - \frac{hc}{E'} \\
    {} &= \frac{hc(E' - E)}{E E'}
\intertext{Making use of the approximation that $E \approx E'$,}
    {} &= \frac{hc\Delta E}{E^2}
\end{align*}
Dividing by the frequency and substituting in the uncertainty relation $\Delta E\Delta t =
\frac{{\hbar}}{2}$,
\begin{align*}
    \frac{\Delta {\lambda}}{{\lambda}} &= \frac{hc}{{\lambda}} \cdot  \frac{1}{E^2}\frac{{\hbar}}{2\Delta t} \\
    {} &= \frac{{\lambda}}{4{\pi} c\Delta t}
\end{align*}
For the time, we estimate the transition rate is occurring as fast as it can
within the limits of the uncertainty relation, so we can let $\Delta t \approx \SI{0.6e9}
{\s^{-1}}$. Plugging in the other values, we find the fractional line width
to be estimated as
\begin{align}
    \boxed{ \frac{\Delta {\lambda}}{{\lambda}} \approx \num{1.935e-8} \approx \text{1 part in 50 million} }
\end{align}

%%%%%%%%%%%%%%%%%%%%%%%%%%%%%%%%%%%%%%%%%%%%%%%%%%%%%%%%%%%%%%%%%%%%%%%%%%%%%%%
%%%% Problem 11
%%%%%%%%%%%%%%%%%%%%%%%%%%%%%%%%%%%%%%%%%%%%%%%%%%%%%%%%%%%%%%%%%%%%%%%%%%%%%%%
\problem{11}
\subsubsection{Question}
% Keywords
	\index{statistical mechanics!Radiometric dating from mass ratios}
	\index{statistical mechanics!Half Life}
A rock is found to contain \SI{4.20}{\mg} of ${}^{238}U$ and \SI{2.00}{\mg}
of ${}^{206}Pb$. Assume that the rock contained no lead at the time of its
formation, so that all the lead now present is due to the decay of the
Uranium originally present in the rock. Find the age of the rock given that
the half-life of ${}^{238}U$ is \SI{4.47e9}{\year}. The decay times of all
intermediate elements are negligibly short and ignore any differences in the
binding energies.

\subsubsection{Answer}

From decay processes, we know that the uranium atom count will decrease as an
exponential according to
\begin{align*}
    N_U = N_{U0}e^{-t/\tau}
\end{align*}
where $\tau = t_{1/2}/\ln 2$. Likewise, the number of lead atoms will increase
according to
\begin{align*}
    N_{Pb} = N_{U0} (1 - e^{-t/\tau})
\end{align*}
Solving for $N_{U0}$ in the first equation and substituting it into the
second, we can solve for the time required to generate a specific number of
uranium and lead atoms in a sample.
\begin{align*}
    N_{Pb} &= N_U e^{t/\tau} (1 - e^{-t/\tau}) \\
    t &= \tau \ln(\frac{N_{Pb}}{N_U} + 1) \\
    t &= \frac{t_{1/2}}{\ln 2} \ln(\frac{N_{Pb}}{N_U} + 1)
\end{align*}
We were only given the masses, though, so we approximate the mass of each
atom by the number of nucleons in the nucleus; each uranium atom has a mass
of $m_U = 238m_N$ making the $N_U$ atoms have a mass of $M_U = 238 N_U m_N$,
and similar for the lead. This gives us the final equation
\begin{align*}
    t &= \frac{t_{1/2}}{\ln 2} \ln(\frac{238}{206} \frac{M_{Pb}}{M_U} + 1)
\end{align*}
Plugging in all the numbers,
\begin{align}
    \boxed{ t = \SI{2.83e9}{\year} }
\end{align}

%%%%%%%%%%%%%%%%%%%%%%%%%%%%%%%%%%%%%%%%%%%%%%%%%%%%%%%%%%%%%%%%%%%%%%%%%%%%%%%
%%%% Problem 12
%%%%%%%%%%%%%%%%%%%%%%%%%%%%%%%%%%%%%%%%%%%%%%%%%%%%%%%%%%%%%%%%%%%%%%%%%%%%%%%
\problem{12}
\subsubsection{Question}
% Keywords
	\index{circuits!Current amplitude and phase in LRC circuit}

The applied AC voltage in the circuit is given by $V(t) = V_0 \sin {\omega} t$, with 
a frequency fixed at ${\omega} = 1/(LC)^{1/2}$. Determine the steady state 
amplitude and phase of the current through the resistor $R$. Express your 
answer in terms of the amplitude $V_0$ of the applied voltage and the other 
circuit parameters.

\begin{center}
	\vspace{\baselineskip}
	\begin{circuitikz}
		\resetparens
		\draw (0,-2)
		to [sV,l=$V(t)$] ++(0,4)
			-- ++(3,0)
		to [L,l=$L$] ++(0,-2)
			coordinate (split)
			-- ++(-1,0)
		to [C,l=$C$] ++(0,-2)
			-- (0,-2)
			(split) -- ++(1,0)
		to [R,l=$R$] ++(0,-2)
			-- (0,-2)
		;
	\end{circuitikz}
	\vspace{\baselineskip}
\end{center}

\subsubsection{Answer}

AC problems are simplified by using complex impedances, so we first convert 
the given voltage into a complex one:
\begin{align*}
	\tilde V(t) &= V_0 e^{i{\omega} t}
\end{align*}
where the physical solution can be recovered by keeping the imaginary 
component of the complex solution. Then to solve the problem, we realize 
that there is another complimentary circuit diagram which is helpful: the 
one with the resistor and capacitor replaced by an effective resistor 
(impedance). The circuit looks like
\begin{center}
	\vspace{\baselineskip}
	\begin{circuitikz}
		\resetparens
		\draw (0,-2)
		to [sV,l=$V(t)$] ++(0,4)
			-- ++(3,0)
		to [R,l=$Z_L$] ++(0,-2)
		to [R,l=$Z_{eff}$] ++(0,-2)
			-- (0,-2)
		;
	\end{circuitikz}
	\vspace{\baselineskip}
\end{center}
The inductor has been been replaced by an effective resistor with impedance 
$Z_L = i{\omega} L$. The effective resistor that replaced the capacitor and resistor 
is a complex impedance that is calculated the same as for traditional 
resistors in parallel:
\begin{align*}
	Z_{eff} &= ( \frac{1}{Z_C} + \frac{1}{Z_R} )^{-1} \\
		&= ( i{\omega}C + \frac{1}{R} )^{-1} \\
		&= \frac{R}{i{\omega}CR + 1}
\end{align*}
Now making use of Kirchoff's loop rule on this simplified circuit where the 
total current passing through the voltage source is labeled $\tilde I_0$,
\begin{align*}
	0 &= \tilde V - \tilde I_0 (Z_L + Z_{eff}) \\
	\tilde V &= (i{\omega} L + \frac{R}{i{\omega}CR + 1}) \tilde I_0 \\
	\tilde V &= \frac{R(1 - {\omega}^2LC) + i{\omega} L}{i{\omega}RC + 1} \tilde I_0
\intertext{The first term in the numerator goes to zero since ${\omega}^2 = 1/LC$,
leaving}
	\tilde I_0 &= \frac{i{\omega}RC + 1}{i{\omega} L} V_0 e^{i{\omega} t}
\end{align*}

To isolate the current passing through the resistor, we return to the 
original unsimplified circuit diagram and apply Kirchoff's loop rule to only 
the inner loop. If we define the current through capacitor to be $I_1$ and
through the resistor to be $I_2$, we get
\begin{align*}
	0 &= -\tilde I_2 Z_R + \tilde I_1 Z_C \\
	\tilde I_1 &= \frac{Z_R}{Z_C} I_2 \\
	\tilde I_1 &= i{\omega}RC I_2 \\	
\end{align*}
Remembering the the current passing into a junction must be conserved, we know
that $I_0 = I_1 + I_2$ and therefore,
\begin{align*}
	\tilde I_0 &= i{\omega}RC \tilde I_2 + \tilde I_2 \\
	\tilde I_2 &= \frac{1}{i{\omega}RC + 1} \tilde I_0
\end{align*}
Inserting the solution for $I_0$ from the previous part leaves
\begin{align*}
	\tilde I_2 &= \frac{V_0}{i{\omega} L} e^{i{\omega} t}
\end{align*}
To prepare for finding the physical solution, we transform the coefficient
complex polar form.
\begin{align*}
	\tilde I_2 &= \left|-\frac{iV_0}{{\omega} L}\right| e^{i\arg(-iV_0/{\omega} L)} e^{i{\omega} t} \\
		&= \frac{V_0}{{\omega} L} e^{-i{\pi}/2} e^{i{\omega} t}
\end{align*}
Therefore taking the imaginary part of the solution,
\begin{align}
	\boxed{
	I_R(t) = V_0 \sqrt{\frac{C}{L}} \sin({\omega} t - \frac {\pi}2)
	}
\end{align}
The current's amplitude is $V_0\sqrt{C/L}$ and has a phase of $-{\pi}/2$ with
respect to the voltage.
