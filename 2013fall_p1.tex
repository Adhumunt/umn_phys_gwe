%%%%%%%%%%%%%%%%%%%%%%%%%%%%%%%%%%%%%%%%%%%%%%%%%%%%%%%%%%%%%%%%%%%%%%%%%%%%%%%
%%%% Problem 1
%%%%%%%%%%%%%%%%%%%%%%%%%%%%%%%%%%%%%%%%%%%%%%%%%%%%%%%%%%%%%%%%%%%%%%%%%%%%%%%
\problem{1}
\subsubsection{Question}
% Keywords
	\index{quantum!Spinning electron}
	\index{mechanics!Spinning electron}

Assuming the electron to be a classical particle, a sphere of radius 
\SI{e-15}{\m} and of a uniform mass density with an intrinsic angular 
momentum of order $ℏ$, compute the speed of rotation at the electron's 
equator. How does your result compare with the speed of light?

\subsubsection{Answer}

Since we're assuming the electron is a classical sphere, we can equate the 
angular momentum of a spinning sphere to the intrinsic angular momentum $ℏ$. 
Knowing that the moment of inertia of a sphere is $I = \frac 25 m r²$ for a 
sphere of uniform density, that means the momentum equation is,
\begin{align*}
	ℏ &= \frac 25 m R² ω
\end{align*}
Then solving for the frequency and relating it to the velocity of a point on
the equator,
\begin{align*}
	v &= \frac 52 \frac{ℏ}{mR}
\end{align*}
Plugging in the values,
\begin{align}
	\boxed{ v = \SI{2.89e11}{\m\per\s} = \SI{965}{c} }
\end{align}


%%%%%%%%%%%%%%%%%%%%%%%%%%%%%%%%%%%%%%%%%%%%%%%%%%%%%%%%%%%%%%%%%%%%%%%%%%%%%%%
%%%% Problem 2
%%%%%%%%%%%%%%%%%%%%%%%%%%%%%%%%%%%%%%%%%%%%%%%%%%%%%%%%%%%%%%%%%%%%%%%%%%%%%%%
\problem{2}
\subsubsection{Question}
% Keywords
	\index{quantum!Distinguishable electrons}

Two electrons can be considered distinguishable if they are well separated 
in space from each other, that is, their single particle wavefunctions are 
non-overlapping. In that case, for every possible $x₁$ value, either 
$ψ_α(x₁)$ and $ψ_β(x₁)$ is zero. Show that for non-overlapping wavefunctions 
as defined above, the probability density for the total antisymmetric 
wavefunction $ψ^*_Aψ_A$ is equal to the probability density of the total 
symmetric wavefunction $ψ^*_Sψ_S$.

\subsubsection{Answer}

We start by constructing the antisymmetric and symmetric wavefunctions:
\begin{align*}
	ψ_A &= \frac{1}{\sqrt 2}( ψ_α(x₁)ψ_β(x₂) - ψ_α(x₂)ψ_β(x₁) ) \\
	ψ_S &= \frac{1}{\sqrt 2}( ψ_α(x₁)ψ_β(x₂) + ψ_α(x₂)ψ_β(x₁) )
\end{align*}
Then calculating the complex square of both (where the $A/S$ indicates the 
choice of $±$),
\begin{align*}
	\resetparens
	ψ^*_{A/S}ψ_{A/S} = \frac 12 \Big(
		&  ψ^*_α(x₁)ψ^*_β(x₂)ψ_α(x₁)ψ_β(x₂) \\
		&∓ ψ^*_α(x₁)ψ^*_β(x₂)ψ_α(x₂)ψ_β(x₁) \\
		&∓ ψ^*_α(x₂)ψ^*_β(x₁)ψ_α(x₁)ψ_β(x₂) \\
		&+ ψ^*_α(x₂)ψ^*_β(x₁)ψ_α(x₂)ψ_β(x₁)
	\Big)
\end{align*}
The first and fourth lines are simply the complex square of each 
wavefunction. The second and third, though, contain the cross-terms, and 
since we are given that the wavefunctions are non-overlapping, that means 
necessarily the product of different wavefunctions at the same point in 
space must be zero. Therefore both the second and third lines result in a 
value of 0, thus simplifying the expression to,
\begin{align}
	\boxed{
	ψ^*ψ = \frac 12 \Big( |ψ_α(x₁)|² |ψ_β(x₂)|² + |ψ_α(x₂)|² |ψ_β(x₁)|² \Big)
	}
\end{align}

%%%%%%%%%%%%%%%%%%%%%%%%%%%%%%%%%%%%%%%%%%%%%%%%%%%%%%%%%%%%%%%%%%%%%%%%%%%%%%%
%%%% Problem 3
%%%%%%%%%%%%%%%%%%%%%%%%%%%%%%%%%%%%%%%%%%%%%%%%%%%%%%%%%%%%%%%%%%%%%%%%%%%%%%%
\problem{3}
\subsubsection{Question}
% Keywords
	\index{mechanics!Hooke-like Bohr atom}
	\index{quantum!Hooke-like Bohr atom}

A particle of mass $m$ moves in a circular orbit of radius $r$ in a 
hypothetical atom where the force on the particle is in the form of a 
generalized Hooke's alw: $F = -Cr$ directed towards the center of the atom, 
where $C$ is the `spring constant'. Assuming that Bohr's postulates for the 
atom apply in this case, in particular, that the orbital angular momentum 
is quantized with a quantum number $n$, derive:
\begin{enumerate}[(a)]
	\item The radii of the allowed orbits
	\item
		The energies of these orbits in terms of the quantum number $n$. 
		(You may take the potential energy of this ``spring atom'' to be 
		$V(r) = \frac 12 Cr²$.)
\end{enumerate}

\subsubsection{Answer}

With the assumption that the orbits are circular, we know that the Hooke 
force must provide the centripital acceleration. Setting them equal and 
solving for the angular frequency by $v = ωr$,
\begin{align*}
	m \frac{v²}{r} &= Cr \\
	ω &= \sqrt{\frac{C}{m}}
\end{align*}
The point-particle electron then has an associated moment of inertia and 
angular momentum as it orbits the nucleus.
\begin{align*}
	L &= mr² ⋅ \sqrt{\frac{C}{m}} \\
	nℏ &= r² \sqrt{Cm}
\end{align*}
The quantization condition on the orbital radii is
\begin{align}
	\boxed{ r = (\frac{ℏ²}{Cm})^{1/4} \sqrt{n} }
\end{align}

To find the energies, we simply sum the kinetic energy from the orbital 
motion with the potential energy contained in the ``spring''.
\begin{align*}
	E_n &= \frac 12 I ω² + \frac 12 C r² \\
	&= \frac 12 m r² \frac{C}{m} + \frac 12 C r² \\
	&= C r² = C ⋅ \sqrt{\frac{ℏ²}{Cm}} n
\end{align*}
Therefore the associated energy states are
\begin{align}
	\boxed{ E_n = ℏn\sqrt{\frac{C}{m}} }
\end{align}
