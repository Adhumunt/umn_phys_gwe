%%%%%%%%%%%%%%%%%%%%%%%%%%%%%%%%%%%%%%%%%%%%%%%%%%%%%%%%%%%%%%%%%%%%%%%%%%%%%%%
%%%% Problem 1
%%%%%%%%%%%%%%%%%%%%%%%%%%%%%%%%%%%%%%%%%%%%%%%%%%%%%%%%%%%%%%%%%%%%%%%%%%%%%%%
%\subsection{Problem 1}
\problem{1}
\subsubsection{Question}
% Keywords
	\index{unsolved!Spring 2013 II.P1}
	\index{Lagrangian!Cylindrical Drum}
A uniform cylindrical drum of mass $M$ and radius $R$ is free to rotate about its axis, which is horizontal. An elastic cable with spring constant $\kappa$ and negligible mass is wound on the drum. (In real life, $\kappa$ will decrease as the cable unwinds, but ignore this effect and assume that it is a constant.) On its free end, it carries a mass $m$, which is allowed to fall down unwinding the elastic cable.
\begin{enumerate}
	\item Write down a Lagrangian function in terms of the drum rotation angle $\theta$ and the vertical displacement $x$ of the mass $m$. Derive the corresponding equations of motions.
	\item Find proper normal modes and determine oscillation frequency (or frequencies) of the system.
\end{enumerate}
\subsubsection{Answer}


%%%%%%%%%%%%%%%%%%%%%%%%%%%%%%%%%%%%%%%%%%%%%%%%%%%%%%%%%%%%%%%%%%%%%%%%%%%%%%%
%%%% Problem 2
%%%%%%%%%%%%%%%%%%%%%%%%%%%%%%%%%%%%%%%%%%%%%%%%%%%%%%%%%%%%%%%%%%%%%%%%%%%%%%%
%\subsection{Problem 2}
\problem{2}
\subsubsection{Question}
% Keywords
	\index{unsolved!Spring 2013 II.P2}
An electron is oscillating in a simple harmonic oscillator potential with an angular frequency $\omega=10^{15}$ rad/sec and amplitude $x_0 = 10^{–10}$ m.
\begin{enumerate}
	\item Calculate the amount of energy radiated per cycle. If you don’t remember the radiation equation, you may want to think about what quantities should be included, and dimensional analysis may be useful.
	\item What is the ratio of the radiated energy per cycle to the average mechanical energy?
	\item How long will it take the system to radiate away half of its energy?
\end{enumerate}
\subsubsection{Answer}



%%%%%%%%%%%%%%%%%%%%%%%%%%%%%%%%%%%%%%%%%%%%%%%%%%%%%%%%%%%%%%%%%%%%%%%%%%%%%%%
%%%% Problem 3
%%%%%%%%%%%%%%%%%%%%%%%%%%%%%%%%%%%%%%%%%%%%%%%%%%%%%%%%%%%%%%%%%%%%%%%%%%%%%%%
%\subsection{Problem 3}
\problem{3}
\subsubsection{Question}
% Keywords
	\index{unsolved!Spring 2013 II.P3}
An engine, that uses a photon gas as the working substance, operates in accordance with the Carnot cycle. The energy of a photon gas is given by Stefan-Boltzmann law $U=\alpha V\tau^4$, and the entropy, $\sigma$, is given by $4U/3\tau$, where $U$ is the energy, $\alpha$ is the Stefan-Boltzmann constant, $V$ is the volume of the gas, and $\tau$ is the temperature.

Given $\tau_h$, $\tau_l$, the high and low temperatures of the cycle, and starting with isothermal compression at $(V_l, \tau_l)$ calculate the work done by the gas for each stage of the cycle and compute the total work. Use this information to calculate the efficiency of the engine.
\subsubsection{Answer}



%%%%%%%%%%%%%%%%%%%%%%%%%%%%%%%%%%%%%%%%%%%%%%%%%%%%%%%%%%%%%%%%%%%%%%%%%%%%%%%
%%%% Problem 4
%%%%%%%%%%%%%%%%%%%%%%%%%%%%%%%%%%%%%%%%%%%%%%%%%%%%%%%%%%%%%%%%%%%%%%%%%%%%%%%
%\subsection{Problem 4}
\problem{4}
\subsubsection{Question}
% Keywords
	\index{unsolved!Spring 2013 II.P4}
	\index{statistical mechanics!One Particle System (3 Levels)}
	\index{thermodynamics!One Particle System (3 Levels)}
Consider a one-particle system capable of three states $(-\epsilon,0,\epsilon)$ in thermal contact with a reservoir at temperature $T$. Find:
\begin{enumerate}
	\item Partition function
	\item Average Energy
	\item Heat capacity at constant volume
	\item Entropy
	\item Free energy
\end{enumerate}
In addition, find the leading temperature dependence of (b), (c), and (d) when $T\gg \epsilon$ and $T\ll \epsilon$.
\subsubsection{Answer}


%%%%%%%%%%%%%%%%%%%%%%%%%%%%%%%%%%%%%%%%%%%%%%%%%%%%%%%%%%%%%%%%%%%%%%%%%%%%%%%
%%%% Problem 5
%%%%%%%%%%%%%%%%%%%%%%%%%%%%%%%%%%%%%%%%%%%%%%%%%%%%%%%%%%%%%%%%%%%%%%%%%%%%%%%
%\subsection{Problem 5}
\problem{5}
\subsubsection{Question}
% Keywords
	\index{unsolved!Spring 2013 II.P5}
The nucleus of a hydrogen atom isotope of mass 3 is radioactive, and changes suddenly into a helium nucleus of mass 3 with the emission of an electron that escapes the nucleus. If the initial hydrogen atom was in its ground state, what is the probability that the single-electron helium ion formed by this radioactive decay is in the 1s state? Which other state(s) will it be found in, other than the 1s state? Useful information:
\begin{equation*}
	\psi_{nlm} = R_{nl}(r)Y^{m}_{l}(\theta,\varphi);\ R_{10}(r) = 2\qty(\frac{z}{a_0})^{3/2}e^{-Zr/a_0};\ \int_0^\infty e^{-x}x^n\dd x = n!
\end{equation*}
\subsubsection{Answer}

