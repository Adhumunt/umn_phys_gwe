%%%%%%%%%%%%%%%%%%%%%%%%%%%%%%%%%%%%%%%%%%%%%%%%%%%%%%%%%%%%%%%%%%%%%%%%%%%%%%%
%%%% Problem 1
%%%%%%%%%%%%%%%%%%%%%%%%%%%%%%%%%%%%%%%%%%%%%%%%%%%%%%%%%%%%%%%%%%%%%%%%%%%%%%%
\problem{1}
\subsubsection{Question}
% Keywords
	\index{mechanics!Water jet distance}

A cylindrical bucket is placed on the ground and filled with water to a
height of \SI{150}{\cm}. HOw high from the ground should one punch a hole in
the side of the bucket to make a stream of water that strikes the ground at
the greatest distance from the bucket? What is that distance?

\subsubsection{Answer}

To solve and optimize the kinematic equation to determine the maximum range
of the water jet, we first need to determine what the velocity of the water
exiting the hole is as a function of the hole's depth below the surface of the
water. To do this, we make use of Bernoulli's principle.
\begin{align*}
    \frac{v²}{2} + gz + \frac{p₀}{ρ} = \text{constant}
\end{align*}
We begin by determining the value of the constant by evaluating the equation
at the surface of the water where we know all the properties. The water is
at the ambient air pressure, the height is given in the problem statement,
and the velocity can be assumed to be zero in the limit that the hole leaks
at a rate too slowly to change the height of the surface. Doing so,
\begin{align*}
    \text{constant} = gz₀ + \frac{p₀}{ρ}
\end{align*}

Then at some height $z'$ below the surface of the water, we puncture a hole.
Again the water is moving into a body at atmospheric pressure, so we again
use $p₀$. That leaves the velocity we're searching for remaining, so equating
with the surface value,
\begin{align*}
    gz₀ + \frac{p₀}{ρ} &= \frac 12 v² + gz' + \frac{p₀}{ρ} \\
    v² &= 2g(z₀ - z')
\end{align*}

Now we're ready to solve the kinematic equation. Begin as usual by finding the
time of flight from the vertical components.
\begin{align*}
    0 = z' - \frac 12 gt²
	\quad\quad\rightarrow\quad\quad
    t = \sqrt{\frac{2z'}{g}}
\end{align*}
Then solving the horizontal equation:
\begin{align*}
    ℓ &= vt = 2\sqrt{z'(z₀ - z')}
\end{align*}
Then finding the extrema:
\begin{align*}
    \frac{dℓ}{dz'} &= 0 = \frac{z₀ - 2z'}{\sqrt{z'(z₀-z')}} \\
    z' &= \frac 12 z₀
\end{align*}
That height that maximizes distance, and that distance, is
\begin{align}
    \boxed{
    z' = \SI{75}{\cm}
    }
    &&
    \boxed{
    ℓ  = \SI{150}{\cm}
    }
\end{align}

%%%%%%%%%%%%%%%%%%%%%%%%%%%%%%%%%%%%%%%%%%%%%%%%%%%%%%%%%%%%%%%%%%%%%%%%%%%%%%%
%%%% Problem 5
%%%%%%%%%%%%%%%%%%%%%%%%%%%%%%%%%%%%%%%%%%%%%%%%%%%%%%%%%%%%%%%%%%%%%%%%%%%%%%%
\problem{5}
\subsubsection{Question}
% Keywords
	\index{mechanics!Elastic collision on spring-connected blocks}
	\index{Lagrangian!Elastic collision on spring-connected blocks}

Blocks of mass $m$ and $2m$ are free to slide without friction on a
horizontal wire. They are connected by a massless spring of equilibrium
length $L$ and force constant $k$. A projectile of mass $m$ is fired with
velocity $v$ into the block with mass $m$ and sticks to it. If the blocks
are initially at rest, what is the maximum displacement between them in the
subsequent motion?

\subsubsection{Answer}

Take time $t=0$ to be the moment the projectile collides with the mass $m$,
and let the subsequent transfer of momentum be instantaneous. In this case,
the initial conditions of the problem are then:
\begin{align*}
    x₁(0) &= 0			& \dot x₁(0) &= u \\
    x₂(0) &= L			& \dot x₂(0) &= 0
\end{align*}
where $u$ is the initial velocity of the combined project-mass system. We get
$u$ from conservation of mometum:
\begin{align*}
    2mu &= mv + 0 \\
    u &= \frac 12 v
\end{align*}

Now solve the mechanics problem using the Lagrangian approach. Both masses have
kinetic energy, and the spring stores potential energy, so
\begin{align*}
    T &= m{\dot x₁}² + m{\dot x₂}² \\
    V &= \frac 12 m (x₂ - x₁)² \\
    L &= m ({\dot x₁}² + {\dot x₂}²) - \frac 12 k({x₁}² + {x₂}² + 2x₁x₂)
\end{align*}
Setting up the differential equation, we get
\begin{align*}
    \frac{∂L}{∂x₁} &= -kx₁ + kx₂	& \frac{d}{dt}
	\left[\frac{∂L}{∂\dot x₁}\right] &= 2m \ddot x₁ \\
    \frac{∂L}{∂x₂} &=  kx₁ - kx₂	& \frac{d}{dt}
	\left[\frac{∂L}{∂\dot x₁}\right] &= 2m \ddot x₂ \\
\end{align*}
Leading to the system of equations where $ω² = k/2m$,
\begin{align*}
    \begin{bmatrix} \ddot x₁ \\ \ddot x₂ \end{bmatrix} &=
	\begin{bmatrix} -ω² & ω² \\ ω² & -ω² \end{bmatrix}
	\begin{bmatrix} x₁ \\ x₂ \end{bmatrix}
\end{align*}
Solving the eigensystem, we find the eigenfrequencies to be $λ = \{0, -2ω²\}$.
Letting ${ω'}² = 2ω²$, the eigenfunction equations are then
\begin{align*}
    \ddot ψ₁ &= 0
	& \rightarrow&&
	ψ₁ &= A₁t + B₁ \\
    \ddot ψ₂ &= -2ω² ψ₂
	& \rightarrow&&
	ψ₂ &= A₂\cos(ω't) + B₂\sin(ω't)
\end{align*}
From the eigenvectors, we express the solutions of $x₁$ and $x₂$ in terms of
$ψ₁$ and $ψ₂$:
\begin{align*}
    \begin{bmatrix} x₁ \\ x₂ \end{bmatrix} &=
	\begin{bmatrix} 1 & 1 \\ 1 & -1 \end{bmatrix}
	\begin{bmatrix} ψ₁ \\ ψ₂ \end{bmatrix}
\end{align*}
\begin{align*}
    x₁ &= A₁t + B₁ + A₂\cos(ω't) + B₂\sin(ω't) \\
    x₂ &= A₁t + B₁ - A₂\cos(ω't) - B₂\sin(ω't)
\end{align*}
Applying the boundary conditions, we find that
\begin{align*}
    x₁(t) &= \frac 14 vt + \frac 12 L - \frac 12 L\cos(ω't) +
	\frac{v}{4ω'}\sin(ω't) \\
    x₂(t) &= \frac 14 vt + \frac 12 L + \frac 12 L\cos(ω't) -
	\frac{v}{4ω'}\sin(ω't)
\end{align*}
The distance $ℓ(t) = x₂(t) - x₁(t)$ between the two masses maximizes when
\begin{align*}
    \frac{dℓ}{dt} = 0 &= \frac{d}{dt} \left[ L\cos(ω't) -
	\frac{v}{2ω'}\sin(ω't) \right] \\
    t &= -\frac{1}{ω'} \arctan (\frac{v}{2Lω'})
\end{align*}
Plugging back into the function $ℓ(t)$,
\begin{align*}
    ℓ &= L\cos \left[ -\arctan (\frac{v}{2Lω'}) \right] - \frac{v}{2ω'}
	\sin \left[ -\arctan (\frac{v}{2Lω'}) \right] \\
    ℓ &= L \frac{2Lω'}{\sqrt{v² + 4L² {ω'}²}} + \frac{v}{2ω'}
	\frac{v}{\sqrt{v² + 4L² {ω'}²}} \\
    ℓ &= \frac{\sqrt{v² + 4L² {ω'}²}}{2ω'}
\end{align*}
Finally, substituting back in $ω' = \sqrt{2k/m}$ and simplifying, we get the
final solution that maximum distance between the two masses is
\begin{align}
    \boxed{
    ℓ = \sqrt{L² + \frac{\frac 12 mv²}{8k}}
    }
\end{align}
which agrees qualitatively with the fact that a larger spring constant should
stiffen the system and decrease the maximum displacement, while launching the
projectile with a greater velocity would increase it.

%%%%%%%%%%%%%%%%%%%%%%%%%%%%%%%%%%%%%%%%%%%%%%%%%%%%%%%%%%%%%%%%%%%%%%%%%%%%%%%
%%%% Problem 7
%%%%%%%%%%%%%%%%%%%%%%%%%%%%%%%%%%%%%%%%%%%%%%%%%%%%%%%%%%%%%%%%%%%%%%%%%%%%%%%
\problem{7}
\subsubsection{Question}
% Keywords
	\index{quantum!Average $x$ and $y$ momenta}

In the state $ψ_{ℓ,m}$ with angular momentum $ℓ$ and its projection $m$,
determine the average values ${l_x}²$ and ${l_y}²$.

\subsubsection{Answer}

Begin by noting that the angular momentum operators are related as,
\begin{align*}
    L² &= {L_x}² + {L_y}² + {L_z}²
\end{align*}
If we assume that the average $x$ and $y$ components will be equal based on
symmetry where only the $z$ direction is identifiable, then we can let
$L_x = L_y$ and rearrage the equation to get
\begin{align*}
    {L_x}² &= \frac 12 (L² - {L_z}²)
\end{align*}

The expectation value is then found by the standard method:
\begin{align*}
    \braopket{ψ_{ℓ,m}}{L_x²}{ψ_{ℓ,m}} &= \braopket{ψ_{ℓ,m}}
	{\frac 12 (L² - {L_z}²)}{ψ_{ℓ,m}} \\
    {} &= \frac 12 (\braopket{ψ_{ℓ,m}}{L²}{ψ_{ℓ,m}} -
	\braopket{ψ_{ℓ,m}}{{L_z}²}{ψ_{ℓ,m}}) \\
\intertext{Then because we know these quantum numbers}
    \braopket{ψ_{ℓ,m}}{L_x²}{ψ_{ℓ,m}} &= \frac 12 (ℓ - m²)
\end{align*}

Therefore with both q ${ℓ_x}²$ and ${ℓ_y}²$ being assumed equation, we conclude
that
\begin{align}
    \expect{{ℓ_x}²} = \expect{{ℓ_y}²} = \frac 12 (ℓ - m²)
\end{align}

%%%%%%%%%%%%%%%%%%%%%%%%%%%%%%%%%%%%%%%%%%%%%%%%%%%%%%%%%%%%%%%%%%%%%%%%%%%%%%%
%%%% Problem 10
%%%%%%%%%%%%%%%%%%%%%%%%%%%%%%%%%%%%%%%%%%%%%%%%%%%%%%%%%%%%%%%%%%%%%%%%%%%%%%%
\problem{10}
\subsubsection{Question}
% Keywords
	\index{dimensional analysis!Freezing ice}
	\index{thermodynamics!Freezing ice}

Ice on a pond is \SI{10}{\cm} thick and the water temperature just below the
ice is \SI{0}{\celsius}. If the air temperature is \SI{-20}{\celsius}, by
how much will the ice thickness increase in 1 hour? Assuming that the air
temperature stays the same over a long period, how will the ice thickness
increase with time? Comment on any approximation that you make in your
calculation.

Density of ice ${}= \SI{0.9}{\g\per\cm\cubed}$

Thermal conductivity of ice ${}= \SI{0.0005}{\cal\per\cm\per\s\per\celsius}$

Latent heat of fusion of water ${}= \SI{80}{\cal\per\g}$

\subsubsection{Answer}

Since the thermal heat flow is a one dimensional problem, immediately
consider everything with respect to a small area element with its normal
perpendicular to the ice-water interface $dA$. Then we want to know how much
ice is generated on the surface of the ice. This small ice element's mass is
simply
\begin{align*}
    dm &= ρ\,dA\,dz
\end{align*}
where $dz$ is the thickness of the new ice layer. To generate this ice, the
latent heat of fusion must be conducted away, so the energy released is,
\begin{align*}
    dE_f &= L_f\,dm \\
    {} &= L_f ρ \,dA\,dz
\end{align*}

The energy flow is through the ice, and we expect this to increase with the
temperature differential across the ice sheet, suggesting that the thermal
conductivity $κ$ be multiplied by the temperature difference $ΔT$. Furthermore,
the ice will decrease the rate of heat flow as it becomes thicker, so the
quantity should also be divided by the thickness $z$. This gives
\begin{align*}
    \frac{κ ΔT}{z} &= \left[ \si{\cal\per\cm\squared\per\s}
	\right]
\end{align*}
This energy is flowing through a surface element $dA$, giving the power flow
due to heat as
\begin{align*}
    \frac{κΔT\,dA}{z} &= \left[ \si{\cal\per\s} \right]
\end{align*}

This power can be matched in units with the energy released from the ice
calculated above by taking the time derivative of $dE_f$, so equating the two
we have
\begin{align*}
    L_f ρ \,dA\frac{dz}{dt} &= \frac{κΔT\,dA}{z} \\
    ∫_{z₀}^{z₀+δz} z\,dz &= ∫_0^t \frac{κΔT}{L_f ρ}\,dt \\
    2z₀ δz + (δz)² &= \frac{κΔT}{L_f ρ}t
\end{align*}
Solving for the length the ice grows $δz$,
\begin{align*}
    δz &= \frac{-2z₀ ± \sqrt{4{z₀}² - 4(\frac{κΔT}{L_f ρ})t} }{2} \\
    δz &= z₀(1 ± \sqrt{1 - \frac{κΔT}{L_f ρ {z₀}²} t})
\end{align*}
The two roots give solutions $δz = \{ \SI{0.0501}{\cm}, \SI{19.950}{\cm} \}$.
Since the second root is unrealistic, we know that the solution must then be
\begin{align}
    \boxed{
    δz = \SI{0.0501}{\cm} \quad\text{in an hour}
    }
\end{align}

%%%%%%%%%%%%%%%%%%%%%%%%%%%%%%%%%%%%%%%%%%%%%%%%%%%%%%%%%%%%%%%%%%%%%%%%%%%%%%%
%%%% Problem 11
%%%%%%%%%%%%%%%%%%%%%%%%%%%%%%%%%%%%%%%%%%%%%%%%%%%%%%%%%%%%%%%%%%%%%%%%%%%%%%%
\problem{11}
\subsubsection{Question}
% Keywords
	\index{statistical mechanics!Carbon-14 dating}

Carbon-14 is produced by cosmic rays interacting with the nitrogen in the
Earth's atmosphere. It is eventaully incorporated into all living things,
and since it has a half-life of \SI{5730(40)}{\year}, it is useful for
dating archaeological specimens up to several tens of thousands of years
old. The radioactivity of a particular specimen of wood containing \SI{3}{\g}
of carbon was measured with a counter whose efficiency was
\SI{18}{\percent}; a count rate of \SI{12.8(1)}{\minute^{-1}} was measured.
It is known that in \SI{1}{\g} of living wood, there are
\SI{16.1}{\minute^{-1}} radioactive carbon-14 decays. What is the age of
this specimen, and its uncertainity? (Where errors are not quoted, they can
be assumed to be negligible).

\subsubsection{Answer}
The rate $N$ after a given time is given by the exponential decay formula
\begin{align*}
    N(t) &+= N₀ e^{-t/τ}
\end{align*}
Since we have the half-life $t_{1/2}$ instead of the decay constant $τ$, we
use the relation $t_{1/2} = τ\ln 2$ to simplify the expression instead to
\begin{align*}
    N(t) &= N₀ (\frac 12)^{t/t_{1/2}}
\end{align*}

The counter use has an efficiency of $ε = 0.18$, so the measured counting
rate $N_m$ must be corrected for that. Furthermore, the sample has a mass of
\SI{3}{\g} whereas we know the rate for a one gram sample, so we also
normalize the count rate by the mass of the sample. Plugging this all into
the exponential decay function above gives
\begin{align*}
    \frac{N_m}{3ε} &= N₀ (\frac 12)^{t/t_{1/2}}
\end{align*}
The only unknown left in the equation is the time, so solving for it,
\begin{align*}
    t &= t_{1/2} \log_{1/2} (\frac{N_m}{3εN₀}) \\
    t &= t_{1/2} \frac{\ln (\frac{N_m}{3εN₀})}{\ln 2} \\
    t &= \frac{t_{1/2}}{\ln 2} \ln (\frac{N_m}{3εN₀})
\end{align*}

To find the uncertainty, we note that only the quantities $N_m$ and $t_{1/2}$
have non-negligible uncertainties, so we propagate the errors only over
these two terms:
\begin{align*}
    {σ_t}² &= (-\frac{t_{1/2}}{N_m \ln 2})² {σ_{N_m}}² +
	(\frac{1}{\ln 2} \ln(\frac{3εN₀}{N_m}))² {σ_{t_{1/2}}}² \\
    σ_t &= \frac{t_{1/2}}{\ln 2} \sqrt{ (\frac{σ_{N_m}}{N_m})² +
	(\frac{σ_{t_{1/2}}}{t_{1/2}})² \left[ \ln(\frac{3εN₀}{N_m}) \right]²}
\end{align*}
Plugging in all the numbers, we get $t = \SI{4248.8435}{\year}$ and $σ_t =
\SI{161.717}{\year}$. The given uncertainties have a single significant digit,
so adding an extra significant figure to the uncertainity and matching decimal
places in the answer, we conclude that the sample has an age of
\begin{align}
    \boxed{
    t = \SI{4250(160)}{\year}
    }
\end{align}
