%%%%%%%%%%%%%%%%%%%%%%%%%%%%%%%%%%%%%%%%%%%%%%%%%%%%%%%%%%%%%%%%%%%%%%%%%%%%%%%
%%%% Problem 10
%%%%%%%%%%%%%%%%%%%%%%%%%%%%%%%%%%%%%%%%%%%%%%%%%%%%%%%%%%%%%%%%%%%%%%%%%%%%%%%
\problem{10}
\subsubsection{Question}
% Keywords
	\index{thermodynamics!Freezing ice}

Ice on a pond is \SI{10}{\cm} thick and the water temperature just below the
ice is \SI{0}{\celsius}. If the air temperature is \SI{-20}{\celsius}, by
how much will the ice thickness increase in 1 hour? Assuming that the air
temperature stays the same over a long period, how will the ice thickness
increase with time? Comment on any approximation that you make in your
calculation.

Density of ice ${}= \SI{0.9}{\g\per\cm\cubed}$

Thermal conductivity of ice ${}= \SI{0.0005}{\cal\per\cm\per\s\per\celsius}$

Latent heat of fusion of water ${}= \SI{80}{\cal\per\g}$

\subsubsection{Answer}

Since the thermal heat flow is a one dimensional problem, immediately
consider everything with respect to a small area element with its normal
perpendicular to the ice-water interface $dA$. Then we want to know how much
ice is generated on the surface of the ice. This small ice element's mass is
simply
\begin{align*}
    dm &= ρ\,dA\,dz
\end{align*}
where $dz$ is the thickness of the new ice layer. To generate this ice, the
latent heat of fusion must be conducted away, so the energy released is,
\begin{align*}
    dE_f &= L_f\,dm \\
    {} &= L_f ρ \,dA\,dz
\end{align*}

The energy flow is through the ice, and we expect this to increase with the
temperature differential across the ice sheet, suggesting that the thermal
conductivity $κ$ be multiplied by the temperature difference $ΔT$. Furthermore,
the ice will decrease the rate of heat flow as it becomes thicker, so the
quantity should also be divided by the thickness $z$. This gives
\begin{align*}
    \frac{κ ΔT}{z} &= \left[ \si{\cal\per\cm\squared\per\s}
	\right]
\end{align*}
This energy is flowing through a surface element $dA$, giving the power flow
due to heat as
\begin{align*}
    \frac{κΔT\,dA}{z} &= \left[ \si{\cal\per\s} \right]
\end{align*}

This power can be matched in units with the energy released from the ice
calculated above by taking the time derivative of $dE_f$, so equating the two
we have
\begin{align*}
    L_f ρ \,dA\frac{dz}{dt} &= \frac{κΔT\,dA}{z} \\
    ∫_{z₀}^{z₀+δz} z\,dz &= ∫_0^t \frac{κΔT}{L_f ρ}\,dt \\
    2z₀ δz + (δz)² &= \frac{κΔT}{L_f ρ}t
\end{align*}
Solving for the length the ice grows $δz$,
\begin{align*}
    δz &= \frac{-2z₀ ± \sqrt{4{z₀}² - 4(\frac{κΔT}{L_f ρ})t} }{2} \\
    δz &= z₀(1 ± \sqrt{1 - \frac{κΔT}{L_f ρ {z₀}²} t})
\end{align*}
The two roots give solutions $δz = \{ \SI{0.0501}{\cm}, \SI{19.950}{\cm} \}$.
Since the second root is unrealistic, we know that the solution must then be
\begin{empheq}[box=\fbox]{align}
	δz &= \SI{0.0501}{\cm} \quad\text{in an hour}
\end{empheq}

