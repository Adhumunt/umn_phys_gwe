%%%%%%%%%%%%%%%%%%%%%%%%%%%%%%%%%%%%%%%%%%%%%%%%%%%%%%%%%%%%%%%%%%%%%%%%%%%%%%%
%%%% Problem 1
%%%%%%%%%%%%%%%%%%%%%%%%%%%%%%%%%%%%%%%%%%%%%%%%%%%%%%%%%%%%%%%%%%%%%%%%%%%%%%%
%\subsection{Problem 1}
\problem{1}
\subsubsection{Question}
% Keywords
	\index{unsolved!Spring 2010 II.P1}
	\index{mechanics!Coupled Harmonic Oscillator}

Two identical objects $A$ and $B$, of mass $m$ each, are connected by a spring, of spring constant $k$. At $t = 0$ the two objects are at rest, and the spring is in its equilibrium position. For $t > 0$, the object $A$ is subject to an external force $F_{ext}= F \cos(\omega t)$, with $F$ and $\omega$ constant, as shown in Figure 1. Compute the motion of the object $B$ for any $t \ge 0$. Neglect all friction.

\subsubsection{Answer}


%%%%%%%%%%%%%%%%%%%%%%%%%%%%%%%%%%%%%%%%%%%%%%%%%%%%%%%%%%%%%%%%%%%%%%%%%%%%%%%
%%%% Problem 2
%%%%%%%%%%%%%%%%%%%%%%%%%%%%%%%%%%%%%%%%%%%%%%%%%%%%%%%%%%%%%%%%%%%%%%%%%%%%%%%
%\subsection{Problem 2}
\problem{2}
\subsubsection{Question}
% Keywords
	\index{unsolved!Spring 2010 II.P2}
	\index{quantum!3D Neutron}
The neutron has the magnetic dipole moment $$\boldsymbol{\mu} = \gamma \mathbf{S}$$ where $\gamma$ is a constant, and $\mathbf{S}$ is the spin of the neutron.

A nonrelativistic neutron with momentum $k$ is moving in a uniform and constant magnetic field. The interaction between the neutron magnetic dipole moment and the magnetic field gives a term in the Hamiltonian $H$ of this system. (i) Write down the complete Hamiltonian $H$.

(ii) Assume that the magnetic field is $\mathbf{B} = (0, 0, B_z)$. What are the possible energies for the neutron? What are the corresponding normalized wave functions? (To get the normalization, require that there is a probability one that the neutron is at some place in a large volume $L^3$ ).

(iii) Answer the same questions as in part (ii), in the case of a magnetic field $\mathbf{B} = (B_x, 0, B_z) \equiv \abs{\mathbf{B}} (\sin(\theta), 0, \cos(\theta)).$

(iv) Assume now that $B_x$ is very small, and can be treated as a perturbation on the problem solved at point (ii), where $B_x$ was taken to vanish. Starting from the unperturbed solutions obtained in (ii), compute the possible energies of the neutron to first order in $B_x$. Compare with the exact energies obtained in (iii).

\subsubsection{Answer}



%%%%%%%%%%%%%%%%%%%%%%%%%%%%%%%%%%%%%%%%%%%%%%%%%%%%%%%%%%%%%%%%%%%%%%%%%%%%%%%
%%%% Problem 3
%%%%%%%%%%%%%%%%%%%%%%%%%%%%%%%%%%%%%%%%%%%%%%%%%%%%%%%%%%%%%%%%%%%%%%%%%%%%%%%
%\subsection{Problem 3}
\problem{3}
\subsubsection{Question}
% Keywords
	\index{unsolved!Spring 2010 II.P3}
	\index{statistical mechanics!Two Particle Statistics}
Let $Z_1(m)$ be the partition function of a single (quantum) particle of mass $m$ in a volume $L^3$, at the temperature $T$.

(i) Consider a system of two such particles, assuming that they do not interact. Denote by $Z_{2,\text{dist}}(m)$ the partition function of the system assuming that the two particles are distinguishable. Express this quantity in terms of $Z_1(m)$.

(ii) Assume now that the two particles are indistinguishable spin zero bosons. Denote by $Z_{2,\text{bose}}(m)$ the partition function for this system. Express this quantity in terms of $Z_1(m)$ and $Z_1(m/2)$.

(iii) Comparing the cases (i) and (ii), calculate (to lowest order in the quantum effects) the correction to the expectation value of the energy of the two particle system due to Bose statistics. In which regime is the correction negligible?

\subsubsection{Answer}



%%%%%%%%%%%%%%%%%%%%%%%%%%%%%%%%%%%%%%%%%%%%%%%%%%%%%%%%%%%%%%%%%%%%%%%%%%%%%%%
%%%% Problem 4
%%%%%%%%%%%%%%%%%%%%%%%%%%%%%%%%%%%%%%%%%%%%%%%%%%%%%%%%%%%%%%%%%%%%%%%%%%%%%%%
%\subsection{Problem 4}
\problem{4}
\subsubsection{Question}
% Keywords
	\index{unsolved!Spring 2010 II.P4}
	\index{mechanics!Two Body Problem}
Consider a system of two particles, with identical masses, orbiting in a circle around their center of mass. (i) Show that the gravitational potential energy of the system is $-2$ times the total kinetic energy.

(ii) This relation is true, on average, for any system of particles held together by their mutual gravitational attraction: $\bar{U}_{\text{potential}}= -2\bar{U}_{\text{kinetic}}$, where $\bar{U}$'s are the total amount of potential and kinetic energies, averaged over some sufficiently long time. Suppose that you add a small amount of energy to such system, and then you wait until it equilibrates. Will the particles in the system, on average, move faster, or more slowly? Explain.

(iii) Compute the potential energy for a uniform spherical distribution of particles of radius $R$ and total mass $M$.

(iv) Assume that a star can be modeled by an ideal gas of particles obeying classical statistics, at the same temperature $T$, which interact among themselves only gravitationally. Estimate the temperature of a star of mass $M = 2 \times 10^{30}$Kg and radius $R = 7 \times 10^8$m. Assume for simplicity that the star contains only protons and electrons.

\subsubsection{Answer}


%%%%%%%%%%%%%%%%%%%%%%%%%%%%%%%%%%%%%%%%%%%%%%%%%%%%%%%%%%%%%%%%%%%%%%%%%%%%%%%
%%%% Problem 5
%%%%%%%%%%%%%%%%%%%%%%%%%%%%%%%%%%%%%%%%%%%%%%%%%%%%%%%%%%%%%%%%%%%%%%%%%%%%%%%
%\subsection{Problem 5}
\problem{5}
\subsubsection{Question}
% Keywords
	\index{unsolved!Spring 2010 II.P5}
	\index{electrodynamics!Current in a Wire}
Consider a uniform infinitely long cylindrical wire, of cross section area $A$, with a current $I$ flowing through it. Consider a charged object, of charge $q > 0$, moving parallel to the wire, with speed $v$. The object is outside the wire, at the distance $d$ from it $d \gg \sqrt{A}$. The wire is neutral, and the object moves in the direction opposite to the flow of the current in the wire.

(i) Compute the magnitude and direction of the magnetic force $\mathbf{F}$ acting on the charged object.

(ii) Assume the following idealized situation for the wire: the wire is made of only protons and electrons, uniformly distributed within it. The proton and electrons have the same number density $n$ ($n$ has dimension of inverse volume). The protons are at rest, while all the electrons move with the same velocity $\mathbf{v}$. Assume that this velocity is equal (both in magnitude and direction) to that of the outside object. Express the current $I$ in terms of $v$ (and of any other relevant parameter), and insert this expression in the formula for the magnetic force computed in (i).

All the above statements are made by an observer $\mathcal{O}$ at rest with respect to the wire. Consider now the same situation in the rest-frame of the outside charged object.

(iii) Does the object experience a magnetic force in this frame?

(iv) Compute the number densities of protons $(n_+^\prime)$ and electrons $(n_-^\prime)$ inside the wire in this frame (hint 1: the electric charge of any individual particle is the same in both frames; hint 2: notice that, due to the symmetry of the problem, there is a simple relation between the ratio $n_+^\prime/n$ and the ratio $n_-^\prime/n$).

(v) Compute the linear charge density of the wire in this frame (charge per unit length along the wire). Compute the force $\mathbf{F}^\prime$ acting on the outside object in this frame.

(vi) Show that the resulting ratio $\mathbf{F}^\prime/\mathbf{F}$ is only function of the $\gamma$ factor between the two frames, and of no other parameters. Show that this result is the one you would have expected, given that $\mathbf{F} = {\Delta}\mathbf{p}/{\Delta}t$, $\mathbf{F}^\prime= {\Delta}\mathbf{p}^\prime/{\Delta}t^\prime$, and how ${\Delta}\mathbf{p}$ and ${\Delta}t$ are related to ${\Delta}\mathbf{p}^\prime$ and ${\Delta}t^\prime$.	

\subsubsection{Answer}



%%%%%%%%%%%%%%%%%%%%%%%%%%%%%%%%%%%%%%%%%%%%%%%%%%%%%%%%%%%%%%%%%%%%%%%%%%%%%%%
%%%% Problem 6
%%%%%%%%%%%%%%%%%%%%%%%%%%%%%%%%%%%%%%%%%%%%%%%%%%%%%%%%%%%%%%%%%%%%%%%%%%%%%%%
%\subsection{Problem 6}
\problem{6}
\subsubsection{Question}
% Keywords
	\index{unsolved!Spring 2010 II.P6}
	\index{Lagrangian!Particle on a Cone}

A particle of mass m is confined to slide on the surface of an ``upside-down'' cone with semi-angle $\alpha$, as shown in Figure 2, and is subject to the constant gravitational field of the Earth surface. The axis of the cone is on the $z$-axis. Neglect any form of friction for points (i) and (ii).

(i) Write down the Lagrangian for this particle, using the coordinates $r$ and $\theta$, defined by $x = r\cos(\theta)$ and $y = r \sin(\theta)$ (notice that $r$ and $\theta$ completely specify the position of the particle on the surface of the cone). Write down the Euler-Lagrange equations, obtained from this Lagrangian, that describe the motion of the particle.

(ii) For appropriate speed $\abs{\mathbf{v}}$, the particle can move on a horizontal, and therefore circular, trajectory with $z = \bar{z} = $constant. Write down the relation between $\bar{z}$ and the speed. Write down the total energy for the particle in this motion.

(iii) For this part only, assume that the cone is filled by some viscous medium, so that the particle is subject to a dragging force $\mathbf{F}_{\text{drag}}=-b\mathbf{v}$, where $b$ is constant and $\mathbf{v}$ is the velocity of the particle. Assuming that the particle is initially (at $t = 0$) on a circular horizontal orbit, with height $\bar{z}_0$, and that the effect of the drag is small, so that the orbits of the particle can be approximated as circular at all times (with a very slowly decreasing radius, due to the drag), compute the time evolution of the height of the particle $\bar{z}(t)$.


\subsubsection{Answer}

