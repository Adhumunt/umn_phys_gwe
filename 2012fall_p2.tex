%%%%%%%%%%%%%%%%%%%%%%%%%%%%%%%%%%%%%%%%%%%%%%%%%%%%%%%%%%%%%%%%%%%%%%%%%%%%%%%
%%%% Problem 1
%%%%%%%%%%%%%%%%%%%%%%%%%%%%%%%%%%%%%%%%%%%%%%%%%%%%%%%%%%%%%%%%%%%%%%%%%%%%%%%
%\subsection{Problem 1}
\problem{1}
\subsubsection{Question}
% Keywords
	\index{unsolved!Fall 2012 II.P1}

In the middle of an infinite square well extending from $x = 0$ to $x = a$ we put a delta function potential $H = \kappa\delta(x-a/2)$ where $\kappa$ is constant. (a) Find the first order perturbation correction to the energies. (b) For the ground state, find also the second order correction to the energy.

\subsubsection{Answer}


%%%%%%%%%%%%%%%%%%%%%%%%%%%%%%%%%%%%%%%%%%%%%%%%%%%%%%%%%%%%%%%%%%%%%%%%%%%%%%%
%%%% Problem 2
%%%%%%%%%%%%%%%%%%%%%%%%%%%%%%%%%%%%%%%%%%%%%%%%%%%%%%%%%%%%%%%%%%%%%%%%%%%%%%%
%\subsection{Problem 2}
\problem{2}
\subsubsection{Question}
% Keywords
	\index{unsolved!Fall 2012 II.P2}
\begin{enumerate}
	\item Beginning with the Lorentz transformations show that when an object is moving with velocity $v_2$ in a reference frame that is moving with velocity $v_1$ with respect to an observer, the velocity $v$ of the object as seen by the observer is:
	\begin{equation}
		v = \frac{v_1+v_2}{1+v_1v_2/c^2}
	\end{equation}
	\item A spaceship is initially at rest with respect to frame $S$. At a given instant, it starts to accelerate with a constant acceleration, $a$, in the instantaneous rest frame of the spaceship. What is the relative speed of the spaceship in frame $S$ when the spaceship’s clock reads time $t$? (c) The command center, which is stationary in frame $S$, communicates with the spaceship by using a laser with a wavelength of $\lambda_s$. What would be the wavelength of the laser beam observed on board the spaceship at time t?
\end{enumerate}
\subsubsection{Answer}



%%%%%%%%%%%%%%%%%%%%%%%%%%%%%%%%%%%%%%%%%%%%%%%%%%%%%%%%%%%%%%%%%%%%%%%%%%%%%%%
%%%% Problem 3
%%%%%%%%%%%%%%%%%%%%%%%%%%%%%%%%%%%%%%%%%%%%%%%%%%%%%%%%%%%%%%%%%%%%%%%%%%%%%%%
%\subsection{Problem 3}
\problem{3}
\subsubsection{Question}
% Keywords
	\index{unsolved!Fall 2012 II.P3}
Consider a classical model of a $CO_2$ molecule where the masses are connected by springs of spring constant $k$. Assume all motion to be linear along the axis of the molecule. (a) Find the relative frequencies for the two vibrational modes. (b) Find the eigenvectors for the modes of the molecule, including any zero frequency mode.
\subsubsection{Answer}



%%%%%%%%%%%%%%%%%%%%%%%%%%%%%%%%%%%%%%%%%%%%%%%%%%%%%%%%%%%%%%%%%%%%%%%%%%%%%%%
%%%% Problem 4
%%%%%%%%%%%%%%%%%%%%%%%%%%%%%%%%%%%%%%%%%%%%%%%%%%%%%%%%%%%%%%%%%%%%%%%%%%%%%%%
%\subsection{Problem 4}
\problem{4}
\subsubsection{Question}
% Keywords
	\index{statistical mechanics!Free Fermi Gas}
Consider a free Fermi gas of $N$ electrons in two dimensions confined to a square of area $A$. (a) Find the Fermi energy $(\epsilon_F)$ in terms of $N$ and $A$. (b) Derive the formula for the density of states and show that it is independent of energy. (c) Use this to find the chemical potential $\mu$ as a function of $N$ and the temperature. (d) What is the behavior of the system at low temperature?
\subsubsection{(Partial) Answer}
To find the Fermi energy, we only need to solve the Schroedinger equation for an infinite potential well in two dimensions. The energy levels for this system are well known;
\begin{equation}
	E_{n_1n_2} = \frac{\hbar^2}{2m}\qty[\frac{n_x^2}{\ell_x^2} + \frac{n_y^2}{\ell_y^2}] = \frac{\hbar^2}{2m}k^2
\end{equation}
where $k^2$ is the magnitude of the wave vector in $2$D. In order to obtain the Fermi energy, first split the $2$D surface into squares of width $L$. Then center a circle on the corner of a square. Since electrons are fermions, they will fill up one a quadrant of the circle in $k$ space whose radius $k_F$ is determined by the fact that each pair of electrons requires an area $A\sim L^2$. Let $N_q= qN$ denote the number of free electrons, then the Fermi relation says (area of the circle contained in the square) = (number of electrons per unit lattice area)
\begin{align}
	\frac{1}{4}\qty(\pi k_F^2) = \frac{N_e}{2}\qty(\frac{\pi^2}{A}) \implies \boxed{k_F^2 = \frac{2\pi}{A}N_e = 2\pi n_e.}
\end{align}
Solving for the energy leads us to $\epsilon_F = \frac{\hbar^2}{2m}k_F^2 = \frac{\hbar^2}{2m}2\pi n_e.$



%%%%%%%%%%%%%%%%%%%%%%%%%%%%%%%%%%%%%%%%%%%%%%%%%%%%%%%%%%%%%%%%%%%%%%%%%%%%%%%
%%%% Problem 5
%%%%%%%%%%%%%%%%%%%%%%%%%%%%%%%%%%%%%%%%%%%%%%%%%%%%%%%%%%%%%%%%%%%%%%%%%%%%%%%
%\subsection{Problem 5}
\problem{5}
\subsubsection{Question}
% Keywords
	\index{unsolved!Fall 2012 II.P5}
If magnetic charges were found, (a) write down the proper set of four Maxwell equations in vacuum to include the electric and magnetic charges as well as electric and magnetic currents. (b) Suppose a magnetic monopole of strength, $q_m$ , passes through a zero resistance loop of wire and moves far away. If the self-inductance of the loop is $a$, what would be the electric current induced in the loop? (c) Set up the equation of motion of an electrically charged particle (charge $q_e$ , mass $m$) about a fixed magnetic monopole of strength, $q_m$.
\subsubsection{Answer}

