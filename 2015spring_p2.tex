%%%%%%%%%%%%%%%%%%%%%%%%%%%%%%%%%%%%%%%%%%%%%%%%%%%%%%%%%%%%%%%%%%%%%%%%%%%%%%%
%%%% Problem 1
%%%%%%%%%%%%%%%%%%%%%%%%%%%%%%%%%%%%%%%%%%%%%%%%%%%%%%%%%%%%%%%%%%%%%%%%%%%%%%%
%\subsection{Problem 1}
\problem{1}
\subsubsection{Question}
% Keywords
	\index{unsolved!Spring 2014 I.P1}
Consider the one-dimensional motion at positive displacement $x$ of a particle subject to a force $R = - \frac{b_1}{x^3} $ where $b_1$ is a positive constant.
\begin{enumerate}
	\item Find the escape velocity from position $x_0$.
	\item Suppose the force was $F = - b_2/x$, ($b_2$ is also positive). Show that the escape velocity is infinite.
\end{enumerate}
\subsubsection{Answer}


%%%%%%%%%%%%%%%%%%%%%%%%%%%%%%%%%%%%%%%%%%%%%%%%%%%%%%%%%%%%%%%%%%%%%%%%%%%%%%%
%%%% Problem 2
%%%%%%%%%%%%%%%%%%%%%%%%%%%%%%%%%%%%%%%%%%%%%%%%%%%%%%%%%%%%%%%%%%%%%%%%%%%%%%%
%\subsection{Problem 2}
\problem{2}
\subsubsection{Question}
% Keywords
	\index{unsolved!Spring 2014 I.P2}
A classical particle of mass $m$ moves in a closed orbit in the gravitational field of a mass $M \gg m$; $M$ is at the origin and the potential energy is $V = -k/r$. Besides energy and angular momentum there is another conserved quantity, the Laplace-Runge-Lenz vector $\mathbf{A}$
\begin{equation*}
	\mathbf{A} = \mathbf{p}\times\mathbf{L}-\frac{mk}{r}\mathbf{r}
\end{equation*}
where $\mathbf{p}$, $\mathbf{r}$ and $\mathbf{L}$ denote the linear momentum, position, and angular momentum of the particle, respectively.
\begin{enumerate}
	\item In one or two lines argue why $\mathbf{A}$ is in the plane of the orbit.
	\item The magnitude of the vector $\mathbf{A}$ is related to the eccentricity $\epsilon$ of the orbit. Recalling that when the origin coincides with one of the foci the equation of an ellipse can be written as $$\frac{1}{r} = C(1 + \epsilon \cos(\theta))$$, $r$ where $\theta$ is the azimuthal angle measured relative to $\mathbf{A}$ $(\mathbf{A}\cdot\mathbf{r}= Ar\cos\theta)$ and $C$ is a constant, relate the magnitude $A$ to the eccentricity $\epsilon$, the mass $m$, and $k$.
\end{enumerate}
\subsubsection{Answer}



%%%%%%%%%%%%%%%%%%%%%%%%%%%%%%%%%%%%%%%%%%%%%%%%%%%%%%%%%%%%%%%%%%%%%%%%%%%%%%%
%%%% Problem 3
%%%%%%%%%%%%%%%%%%%%%%%%%%%%%%%%%%%%%%%%%%%%%%%%%%%%%%%%%%%%%%%%%%%%%%%%%%%%%%%
%\subsection{Problem 3}
\problem{3}
\subsubsection{Question}
% Keywords
	\index{unsolved!Spring 2014 I.P3}
Find the magnetic flux through a square loop due to a current $I$ in a long straight wire. The loop is coplanar with the wire and has two sides parallel to it. The loop has side length $a$ and its side nearest to the wire is a distance $b$ from the wire.
\subsubsection{Answer}



%%%%%%%%%%%%%%%%%%%%%%%%%%%%%%%%%%%%%%%%%%%%%%%%%%%%%%%%%%%%%%%%%%%%%%%%%%%%%%%
%%%% Problem 4
%%%%%%%%%%%%%%%%%%%%%%%%%%%%%%%%%%%%%%%%%%%%%%%%%%%%%%%%%%%%%%%%%%%%%%%%%%%%%%%
%\subsection{Problem 4}
\problem{4}
\subsubsection{Question}
% Keywords
	\index{unsolved!Spring 2014 I.P4}
A system is described by the wave function $\Psi = A \cos^2(\phi)$, where $A$ is a normalization constant and $\phi$ is the azimuthal angle. You measure the angular momentum of this system along the $z$ axis. Compute which results you can obtain from such a measurement, and their probabilities.
\subsubsection{Answer}


%%%%%%%%%%%%%%%%%%%%%%%%%%%%%%%%%%%%%%%%%%%%%%%%%%%%%%%%%%%%%%%%%%%%%%%%%%%%%%%
%%%% Problem 5
%%%%%%%%%%%%%%%%%%%%%%%%%%%%%%%%%%%%%%%%%%%%%%%%%%%%%%%%%%%%%%%%%%%%%%%%%%%%%%%
%\subsection{Problem 5}
\problem{5}
\subsubsection{Question}
% Keywords
	\index{unsolved!Spring 2014 I.P5}
Derive the average momentum $\expval{\hat{p}}$ for a packet with a normalizable wave function of the form $$\Psi(x) = C\phi(x)e^{ikx}$$ where $C$ is a normalization constant and $\phi(x)$ is a real function.
\subsubsection{Answer}



%%%%%%%%%%%%%%%%%%%%%%%%%%%%%%%%%%%%%%%%%%%%%%%%%%%%%%%%%%%%%%%%%%%%%%%%%%%%%%%
%%%% Problem 6
%%%%%%%%%%%%%%%%%%%%%%%%%%%%%%%%%%%%%%%%%%%%%%%%%%%%%%%%%%%%%%%%%%%%%%%%%%%%%%%
%\subsection{Problem 6}
\problem{6}
\subsubsection{Question}
% Keywords
	\index{unsolved!Spring 2014 I.P6}
Galaxy A moves away from our galaxy at a speed of $0.6c$. Galaxy B moves away at $0.7c$ with a trajectory that is $45$ degrees to A. Write down the transformation of space and time differential coordinates $({\Delta}t, {\Delta}x, {\Delta}y)$ between two objects moving at constant velocity $v_0$ relative to each other (orient the axes as you see fit), derive the velocity transformation, and express the velocity of B as observed by the civilization in A.
\subsubsection{Answer}

%%%%%%%%%%%%%%%%%%%%%%%%%%%%%%%%%%%%%%%%%%%%%%%%%%%%%%%%%%%%%%%%%%%%%%%%%%%%%%%
%%%% Problem 7
%%%%%%%%%%%%%%%%%%%%%%%%%%%%%%%%%%%%%%%%%%%%%%%%%%%%%%%%%%%%%%%%%%%%%%%%%%%%%%%
%\subsection{Problem 7}
\problem{7}
\subsubsection{Question}
% Keywords
	\index{unsolved!Spring 2014 I.P7}
A circular storage ring with an orbit diameter of 10 meters stores protons with 1 GeV kinetic energy.
\begin{enumerate}
	\item How long do they take to complete one orbit?
	\item What magnetic field strength is needed to constrain them in orbit?
\end{enumerate}
\subsubsection{Answer}



%%%%%%%%%%%%%%%%%%%%%%%%%%%%%%%%%%%%%%%%%%%%%%%%%%%%%%%%%%%%%%%%%%%%%%%%%%%%%%%
%%%% Problem 8
%%%%%%%%%%%%%%%%%%%%%%%%%%%%%%%%%%%%%%%%%%%%%%%%%%%%%%%%%%%%%%%%%%%%%%%%%%%%%%%
%\subsection{Problem 8}
\problem{8}
\subsubsection{Question}
% Keywords
	\index{unsolved!Spring 2014 I.P8}
Consider a planar square lattice with $N$ classical spins at each site $i$ represented by the unit vector $\mathbf{S}_i$. Each spin is restricted to point along only four directions: $\pm\hat{\mathbf{x}}$ and $\pm\hat{\mathbf{y}}$. Each spin interacts only with its nearest neighbors according to the Hamiltonian $$H = -J\sum_{\expval{ij}}\mathbf{S}_i\cdot\mathbf{S}_j$$ where $J > 0$ is the exchange interaction and $\expval{ij}$ denotes a pair of nearest neighbor sites. Since we are ultimately interested in the thermodynamic limit $N\to\infty$ the effect of the boundary is negligible.

By computing the internal energy of the ferromagnetic ordered state (in which all spins are perfectly parallel to each other) and the entropy of the paramagnetic state (in which the spins are completely disordered), and by considering the free energy of the system give an estimate for the temperature at which the system undergoes a phase transition from the ferromagnetic to the paramagnetic state in the limit $N\to\infty$.
\subsubsection{Answer}



%%%%%%%%%%%%%%%%%%%%%%%%%%%%%%%%%%%%%%%%%%%%%%%%%%%%%%%%%%%%%%%%%%%%%%%%%%%%%%%
%%%% Problem 9
%%%%%%%%%%%%%%%%%%%%%%%%%%%%%%%%%%%%%%%%%%%%%%%%%%%%%%%%%%%%%%%%%%%%%%%%%%%%%%%
%\subsection{Problem 9}
\problem{9}
\subsubsection{Question}
% Keywords
	\index{unsolved!Spring 2014 I.P9}
Two protons are at large distance from each other. Initially one proton is at rest and the other is moving head-on toward the first with kinetic energy $\mathcal{E}$. The motion is nonrelativistic, $v\ll c$. Find the minimal distance $r_{min}$ between the protons. At what energy will the minimum distance between the protons be $r_{min}= 10^{-13}$ cm?
\subsubsection{Answer}



%%%%%%%%%%%%%%%%%%%%%%%%%%%%%%%%%%%%%%%%%%%%%%%%%%%%%%%%%%%%%%%%%%%%%%%%%%%%%%%
%%%% Problem 10
%%%%%%%%%%%%%%%%%%%%%%%%%%%%%%%%%%%%%%%%%%%%%%%%%%%%%%%%%%%%%%%%%%%%%%%%%%%%%%%
%\subsection{Problem 10}
\problem{10}
\subsubsection{Question}
% Keywords
	\index{unsolved!Spring 2014 I.P10}
	\index{optics!Reflection and Transmission}
	\index{optics!Index of Refraction}
Mirrors 1 and 4 in Figure 1 are `half-silvered' so half the intensity of light incident upon them is transmitted and half is reflected. Ignore multiple reflections at all the mirrors and assume that other reflections and attenuations are negligible.
\begin{enumerate}
	\item Find the dependence of the intensity of transmission of the device on $L$, the indices of refraction $n_1$ and $n_2$, and the wavelength $\lambda$ of the incident monochromatic, phase coherent light.
	\item You are provided with a crude light detector that can only measure the maximum and minimum of intensity with any confidence (but not intermediate light levels). You are starting with both chambers evacuated and you have a pressure gauge. Show how to determine the index of refraction of a gas as a function of pressure by monitoring the transmission intensity as gas is added to one of the tubes.
\end{enumerate}
\subsubsection{Answer}

