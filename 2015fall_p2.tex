%%%%%%%%%%%%%%%%%%%%%%%%%%%%%%%%%%%%%%%%%%%%%%%%%%%%%%%%%%%%%%%%%%%%%%%%%%%%%%%
%%%% Problem 1
%%%%%%%%%%%%%%%%%%%%%%%%%%%%%%%%%%%%%%%%%%%%%%%%%%%%%%%%%%%%%%%%%%%%%%%%%%%%%%%
%\subsection{Problem 1}
\problem{1}
\subsubsection{Question}
% Keywords
	\index{unsolved!Fall 2015 II.P1}
A long linear solenoid has length $\ell,$ radius $r$, and $n$ turns per unit length. The solenoid is part of a circuit having resistor of resistance $R$ and a generator of emf $\mathcal{E}$. The emf of the generator is very slowly increased in such a way that the current $I$ in the windings of the solenoid grows as $I=at$, where $a$ is a small positive constant and $t$ is time. 
\begin{enumerate}
	\item Compute the total magnetic energy inside the solenoid.
	\item Compute the induced electric field inside and outside the solenoid.
	\item Compute the flux of the Poynting vector through the solenoid (evaluate $\mathbf{B}$ just inside the solenoid), and show that is it equal to the rate of increase with time of the magnetic energy inside the solenoid.
	\item When the current reaches the value $I_0$, the generator is switched off. Compute how the current evolves from this time on, and show that energy is conserved. 
\end{enumerate}
\subsubsection{Answer}


%%%%%%%%%%%%%%%%%%%%%%%%%%%%%%%%%%%%%%%%%%%%%%%%%%%%%%%%%%%%%%%%%%%%%%%%%%%%%%%
%%%% Problem 2
%%%%%%%%%%%%%%%%%%%%%%%%%%%%%%%%%%%%%%%%%%%%%%%%%%%%%%%%%%%%%%%%%%%%%%%%%%%%%%%
%\subsection{Problem 2}
\problem{2}
\subsubsection{Question}
% Keywords
	\index{unsolved!Fall 2015 II.P2}
You decide to do a Rutherford-type scattering experiment to find the composition of an unknown material. You shoot a beam of oxygen nuclei ($m=16m_p$, where $m_p$ is the proton mass) at the material, and find that the oxygen nuclei that are scattered by $60^\circ$  have $54\%$ of their initial kinetic energy. What is the mass of the nucleus that scattered the oxygen in units of the proton mass? You can assume that the velocity of the oxygen beam is much less than the speed of light.
\subsubsection{Answer}



%%%%%%%%%%%%%%%%%%%%%%%%%%%%%%%%%%%%%%%%%%%%%%%%%%%%%%%%%%%%%%%%%%%%%%%%%%%%%%%
%%%% Problem 3
%%%%%%%%%%%%%%%%%%%%%%%%%%%%%%%%%%%%%%%%%%%%%%%%%%%%%%%%%%%%%%%%%%%%%%%%%%%%%%%
%\subsection{Problem 3}
\problem{3}
\subsubsection{Question}
% Keywords
	\index{unsolved!Fall 2015 II.P3}
Consider a particle of mass $m$ in an infinite one dimensional potential well of width $2L$ (i.e., $V=0$ for $-L<x<L$ and is infinite otherwise).
\begin{enumerate}
	\item Compute the (properly normalized) eigenfunctions and the corresponding eigenenergies.
	\item Assume that the potential is perturbed by a small potential of height $V_0$ and width $2a$, where $a\le L$. Compute the corrections to the eigenenergies to leading order in $V_0$.
	\item Compute the perturbed energies in the limit of $a=L$, and compare them with exact solutions in this limit. 
\end{enumerate}
\subsubsection{Answer}



%%%%%%%%%%%%%%%%%%%%%%%%%%%%%%%%%%%%%%%%%%%%%%%%%%%%%%%%%%%%%%%%%%%%%%%%%%%%%%%
%%%% Problem 4
%%%%%%%%%%%%%%%%%%%%%%%%%%%%%%%%%%%%%%%%%%%%%%%%%%%%%%%%%%%%%%%%%%%%%%%%%%%%%%%
%\subsection{Problem 4}
\problem{4}
\subsubsection{Question}
% Keywords
	\index{unsolved!Fall 2015 II.P4}
	A metallic ball with radius $R$ is immersed and suspended in a weakly conducting medium with a uniform conductivity $\sigma$ in the middle of a large metallic vessel (e.g., salty water in a metallic bathtub).
	\begin{enumerate}
		\item One wire from the battery is attached t the ball and the second wire is attached to the vessel. Calculate the resistance of the media. You can assume that the current is spherically symmetric around the ball. (This is how a standard plasma probe tests the ionization degree of a plasma).
		\item After the ball is charge to a charge $Q_0$, the batter is disconnected at $t=0$ and the ball discharges with time. Find how the ball charge $Q$ depends on time and the characteristic time of this discharge by writing a simple differential equation for $Q(t)$ (this characteristic time is called the Maxwell time). 
	\end{enumerate}
\subsubsection{Answer}


%%%%%%%%%%%%%%%%%%%%%%%%%%%%%%%%%%%%%%%%%%%%%%%%%%%%%%%%%%%%%%%%%%%%%%%%%%%%%%%
%%%% Problem 5
%%%%%%%%%%%%%%%%%%%%%%%%%%%%%%%%%%%%%%%%%%%%%%%%%%%%%%%%%%%%%%%%%%%%%%%%%%%%%%%
%\subsection{Problem 5}
\problem{5}
\subsubsection{Question}
% Keywords
	\index{unsolved!Fall 2015 II.P5}
	A hemoglobin molecule can bind four $O_2$ molecules. Assume that $\epsilon$ is the energy of each bound $O_2$, relative to $O_2$ at rest at an infinite distance. Let $\lambda=\exp(\mu/kT)$ denote the absolute activity of the free $O_2$, where $\mu$ is the chemical potential.
	\begin{enumerate}
		\item Find the appropriate partition function.
		\item What is the probability that one and only one $O_2$ is adsorbed on a hemoglobin molecule? Carefully sketch the result qualitatively as a function of $\lambda$. 
		\item What is the probability that all four $O_2$ are adsorbed? Sketch this result also as a function of $\lambda$. 
	\end{enumerate}
\subsubsection{Answer}

