%%%%%%%%%%%%%%%%%%%%%%%%%%%%%%%%%%%%%%%%%%%%%%%%%%%%%%%%%%%%%%%%%%%%%%%%%%%%%%%
%%%% Problem 2
%%%%%%%%%%%%%%%%%%%%%%%%%%%%%%%%%%%%%%%%%%%%%%%%%%%%%%%%%%%%%%%%%%%%%%%%%%%%%%%
\problem{2}
\subsubsection{Question}
% Keywords
	\index{thermodynamics!Zipper partition function}
    \index{statistical mechanics!Zipper partition function}

A zipper has $N$ links; each link has a closed state with zero energy and an
open state with energy $\varepsilon $. We require, however, that the zipper can only
unzip from the left end, and that the link number $s$ can only open if all
links to the left (i.e.\ $1, 2, \ldots s-1$) are already open.
\begin{enumerate}[a.]
    \item
        Find an explicit expression for the partition function by doing the
        appropriate summation.
    \item
        In the limit $\varepsilon \gg k_B T$ find the average number of open links. This
        model is a very simplified model of the unwinding of two-stranded
        DNA molecules.
\end{enumerate}

\subsubsection{Answer}

\begin{enumerate}[a.]
    \item
        Create the partition function by induction; start by assuming there is
        only a single link. Then the partition function is a simple two-state
        system:
        \begin{align*}
            Z_1 &= e^0 + e^{-\varepsilon /k_BT} = 1 + e^{-\varepsilon /k_BT}
        \end{align*}
        Adding a second link,
        \begin{align*}
            Z_2 &= \underbrace{e^{0+0}}_{\text{both closed}} +
                \underbrace{e^{(-\varepsilon +0)/k_BT}}_{\text{1 open, 1 closed}} +
                \underbrace{e^{(-\varepsilon -\varepsilon )/k_BT}}_{\text{both open}}
            \\
            {} &= 1 + e^{-\varepsilon /k_BT} + e^{-2\varepsilon /k_BT}
        \end{align*}
        Following, for three links:
        \begin{align*}
            Z_3 &= \underbrace{e^{0+0+0}}_{\text{all closed}} +
                \underbrace{e^{(-\varepsilon +0+0)/k_BT}}_{\text{1 open, 2 closed}} +
                \underbrace{e^{(-\varepsilon -\varepsilon +0)/k_BT}}_{\text{2 open, 1 closed}} +
                \underbrace{e^{(-\varepsilon -\varepsilon -\varepsilon )/k_BT}}_{\text{all open}}
            \\
            {} &= 1 + e^{-\varepsilon /k_BT} + e^{-2\varepsilon /k_BT} + e^{-3\varepsilon /k_BT}
        \end{align*}
        By induction, we see that the maximum coefficient in the series of
        exponential factors is just the number of links, so by induction we
        conclude that
        \begin{align*}
            Z &= \sum_{s=0}^{N} e^{-s\varepsilon /k_BT}
        \end{align*}
        Applying the results of a finite geometric series, the closed-form
        solution for the partition function of the links is
        \begin{align}
            \boxed{
            Z = \frac{1 - e^{-(N+1)\varepsilon /k_BT}}{1 - e^{-\varepsilon /k_BT}}
            }
        \end{align}
    \item
        To get the average number of open links, we use the standard
        procedure for finding expvalation values.
        \begin{align*}
            \expval{s} &= \frac{1}{Z} \sum_{s=0}^N s e^{-s\varepsilon /k_BT}
        \end{align*}
        By making use of differentiation under the summation trick, we can
        find the closed-form solution:
        \begin{align*}
            \expval{s} &= \frac{1}{Z} \sum_{s=0}^N
                \frac{d}{d(\frac{\varepsilon }{k_BT})} \Big[ -e^{-s\varepsilon /k_BT} \Big] \\
            {} &= -\frac{1}{Z} \frac{\partial }{\partial (\frac{\varepsilon }{k_BT})}
                \sum_{s=0}^N e^{-s\varepsilon /k_BT} \\
        \intertext{
        Noting that the summation is the same as above,
        }
            \expval{s} &= -\frac{1}{Z} \frac{\partial Z}{\partial (\frac{\varepsilon }{k_BT})}
        \end{align*}
        First considering just the derivative part:
        \begin{align*}
            \frac{\partial Z}{\partial (\frac{\varepsilon }{k_BT})} &=
                \frac{(N+1)e^{-(N+1)\varepsilon /k_BT}}{1 - e^{-\varepsilon /k_BT}} -
                \frac{1 - e^{-(N+1)\varepsilon /k_BT}}{(1 - e^{-\varepsilon /k_BT})^2} e^{-\varepsilon /k_BT}
        \intertext{
        which when combined with the factor $-1/Z$ simplifies to
        }
            -\frac{1}{Z} \frac{\partial Z}{\partial (\frac{\varepsilon }{k_BT})} &=
                -(N+1)\frac{e^{-(N+1)\varepsilon /k_BT}}{1 - e^{-(N+1)\varepsilon /k_BT}} +
                \frac{e^{-\varepsilon /k_BT}}{1 - e^{-\varepsilon /k_BT}} \\
            &= \frac{1}{e^{\varepsilon /k_BT} - 1} - \frac{N+1}{e^{(N+1)\varepsilon /k_BT} - 1}
        \end{align*}
        Therefore the analytic solution is
        \begin{align}
            \boxed{
            \expval{s} = \frac{1}{e^{\varepsilon /k_BT}-1} - \frac{N+1}{e^{(N+1)\varepsilon /k_BT}-1}
            }
        \end{align}
        In the limit that $\varepsilon \gg k_BT$, though, the exponentials in the denominator
        are very large in comparison to 1, so we ignore the unity factors
        and make the approximation that
        \begin{align*}
            \expval{s} &= e^{-\varepsilon /k_BT} - (N+1)e^{-(N+1)\varepsilon /k_BT}
        \intertext{
        Collecting like terms,
        }
            {} &= \left[1 - (N+1)e^{-N\varepsilon /k_BT} \right] e^{-\varepsilon /k_BT}
        \intertext{
            The second term in the brackets approximate zero, so
        }
            {} &= e^{-\varepsilon /k_BT}
        \end{align*}
        Therefore in the low temperature limit where the thermal energy is
        much less than the energy of the open state,
        \begin{align}
            \boxed{
            \expval{s} = e^{-\varepsilon /k_BT}
            }
        \end{align}
\end{enumerate}
